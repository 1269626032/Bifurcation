\section{普适开折与分岔的余维}
现在,我们把\ref{1.2}后半部分中讨论的一些概念严格化。
初次接触分岔理论的读者可以略过本节的内容,只需承认定理\ref{thm5.13}的结果,而影响对随后章节的继续学习。
\subsection{普适开折的定义}
\begin{defination}
  设向量场\(X,Y\)(或映射\(f,g\))都在\(p \in M\) 的领域内有定义。
  称X与Y(或f与g)在p点有同一芽(germ),如果存在领域U,\(p \in U \subset M\),使$\left. X \right| _ { \mathbf { v } } = \left. Y \right| _ { v }$(或$\left. f \right| _ { v } = g | v$)。
\end{defination}

\begin{note}
  向量场(或映射)在p点的芽,是向量场(或映射)的一个等价类。
  我们把这个等价类中的任一元素称为这个\textbf{芽的表示}。
  在考虑局部问题时,利用芽的说法可以使称述简明。
  附录C中定义的射式空间\(J_x^i(M,N)\) 或 \(J^k(M,N)\) 都可以在映射芽的意义下给出。
\end{note}

现在考虑向量场族
$X _ { \mu } \in \mathscr { X } ^ { \infty } ( M )$.
在局部情形下,不妨设\(M=\mathbb{R}^m,\mu \in \mathbb{R}^k \).
此时,常把\(X_\mu\) 与其主部\(v(x,\mu)\) 等同(参见附录B中附注B.17R),而\(X_\mu\) 的流由微分方程
\[
\frac { \mathrm { d } x } { \mathrm { d } t } = v ( x , \mu )
\]
所决定,其中
$v \in C ^ { \infty } \left(\mathbb{ R} ^ { m } \times\mathbb{ R} ^ { k } ,\mathbb{ R} ^ { m } \right)$.
在上述等同意义下,也把v称为向量场。

\begin{defination}[开折,局部族]
  对于向量族\(v(x,\mu )\), 当把 v 视为参数空间
  \( \mu \in \mathbb{R}^k \) 在原点的小领域到向量场空间的映射时,
  我们把\(x(x,\mu)\) 称为\(v(x,o)\) 的一个k参数 \textbf{开折(unfolding)},或称为\textbf{形变(deformation)};
  当把v视为直积空间 \(\mathbb{R}^m\times \mathbb{R}^k\) 中在\((x_0,\mu_0)\) 点的映射芽时,称它为一个\textbf{局部族},记为\((v;x_0,\mu_0)\).
\end{defination}

\begin{defination}[局部族的等价]
  称两个向量场局部族\((v;x_0,\mu_0)\) 与 \((w;y_0,\mu_0)\) 等价,
  如果存在映射h,\(y=h(x,\mu)\),在\(x_0,\mu_0\)的映射芽,对于每一个固定的
\end{defination}

