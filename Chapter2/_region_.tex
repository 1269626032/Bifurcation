\message{ !name(dierzhang.tex)}
\message{ !name(dierzhang.tex) !offset(-2) }
\chapter{常见的局部与非局部分岔}
本章介绍一些常见的分岔现象,
其中包括奇点分岔、闭轨分岔、Hopf分岔、同宿分岔、Poincare分岔等,
其中前三种为局部分岔闭题,
后两种分别为半局部分岔和全局分岔问题。
除了奇点分岔外,本章的大部分讨论都限制在相空间为二维的情形.
\section{奇点分岔}
考虑一个光滑的依赖于参数并且有奇点的向量场。
当参数变动时,我们关心奇点个数及其附近的轨道结构如何变化。
这种分岔现象称为\textbf{奇点分岔}.
\subsection{一般理论}
\begin{defination}\label{def2.1.1}
  向量场 \(X\)的奇点 \( p \in M \) 称为\textbf{非退化}的,
  如果他在\(p\)点的线性部分算子是非奇异的,
  即它的所有特征根均非零.
  否则称为退化的。
  \label{def2.1.1}
\end{defination}

\begin{theorem}\label{thm2.1.2}
  光滑的依赖于变量和参数的向量场,
  如果它的奇点是非退化的,
  则奇点本身也光滑地依赖于参数。
  \label{thm2.1.2}
\end{theorem}

\begin{proof}
  设向量场由微分方程
  \begin{equation}
    \frac{\mathrm{d} x}{\mathrm{d} t}=v(x, \mu)
    \label{eq1.1}
  \end{equation}
  给出
  其中
  \(v \in C^{r}\left(\mathbf{R}^{n} \times \mathbf{R}^{k}, \mathbf{R}^{n}\right), r \geqslant 1, k \geqslant 1  \).
   设当\(\mu = \mu_{0}\)时,
  \(x=x_{0}\)为 \ref{eq1.1}
  的非退化奇点,
  既
  \(
  v\left(x_{0}, \mu_{0}\right)=0,\left.\frac{\partial v(x, \mu)}{\partial x}\right|_{\left(x_{0}, \mu_{0}\right)}
  \)
  非奇异。
  由隐函数定理,
  在\(x_{0},\mu_{0}\)附近存在光滑函数
  \(x= \Gamma (\mu)\),
  使得 \(\Gamma ( \mu_{0}) = x_{0}\),
  且
  \(
v(\gamma(\mu), \mu) \equiv 0
  \).
  定理得证。
\end{proof}

\begin{remark}
  \label{rmk2.1.3}
  定理 \ref{thm2.1.2}说明,
  当奇点非退化时,
  奇点的个数在微小变化下不变,
  它的位置也光滑的依赖于参数的变化。
  需要注意,奇点的非退化性与双曲性是不同的概念。
  例如,一个向量场在奇点处的线性部分是一对纯虚根时,
  按照定义\ref{def2.1.1},它是非退化的,但它是双曲的。
  此时在扰动下,虽然奇点个数(在小领域内)不发生变化,
  但其附近的轨道结构可能变化,
  出现Hopf分岔,或Poincare分岔,
  我们将在\ref{sec2.3}与\ref{sec2.3}中分别予以讨论。
\end{remark}

\begin{theorem}\label{thm2.1.4}
  设M是n维紧致流形,
  \(r \geqslant 1\),
  则
  \(\mathscr{X}^{r}(M)\)中仅有非退化奇点(它们必是孤立奇点)或无奇点的向量场集合形成一个开稠子集.
\end{theorem}
\begin{proof}
  设$X \in \mathscr{X}^{r}(M)$相应于微分方程
  \begin{equation}
\frac{\mathrm{d} x}{\mathrm{d} t}=f(x), \quad x \in U(\xi),
\end{equation}
其中\(f \in C^{r}(M, M), U(\xi)\)为\(M\)中\(\xi\)点的领域.
如同第一章\ref{1.5}的讨论,考虑投影
\begin{equation*}
j^{\circ} f_{2} \quad \mathscr{X}^{r}(M) \rightarrow J^{\circ}(M, M) : \quad X \mapsto(\xi, \tilde{f}),
\end{equation*}
其中
\(\tilde{f}=f(\xi)\).
具有奇点的向量场集合在空间\(J^{0}(M, M)\)中有表示式
\begin{equation*}
S=\{(\xi, \tilde{f}) | \tilde{f}=0\},
\end{equation*}
它是\(J^{0}(M, M)\)中的光滑闭子流形(因为M是紧空间).
设向量场\(f(x)\)在\(\xi\)点非退化,
既\(\left.\frac{\partial f}{\partial x}\right|_{\xi}\)非奇异,
从而由附录\ref{fuluC}中定理\ref{C.13}知,
\(j {\circ} f\)与子流形S横截相交。
注意,不想交也是横截.
再利用定理\ref{C.15},
得知仅有非退化奇点或无奇点的向量场在\(\mathscr{X}^{r}(M)\)中形成开稠子集。
\end{proof}
\par
这个定理说明,向量场的一个退化奇点可以经过任意小的扰动转化为(多个)非退化奇点,或经扰动使奇点消失.
但如果我们考虑向量场族\(v(x,\mu)\),
则奇点的退化性往往是不可避免的.
事实上,
虽然小扰动可以把对应于\(\mu=\mu_{1}\)的退化奇点\(x=x_{1}\)扰动为非退化的,
但在\(x_{1}\)附近的\(x_{2}\)点,
相应于\(\mu_{1}\)附近的\(\mu_{2}\)却可能是新的退化奇点.
对一个具体的奇点分岔问题,
通常有两种处理方法:
一种是利用中心流形定理,
把问题归结在中心流形上,
见第一章例\ref{exam1.3.12};
另一种称为Liapunov-Schmidt方法,
或称为更替法(alternative method).
为了说明这个方法的基本思想,
我们先看一种特殊情形.
设变量\(x=(y,z)\),
在\(x=0\)附近微分方程具有下列形式
\begin{equation}
  \label{eq2.1.2}
  \frac{\mathrm{d} y}{\mathrm{d} t}=\boldsymbol{A} y+f(y, z, \lambda),
  \quad
  \frac{\mathrm{d} z}{\mathrm{d} t}=\boldsymbol{B} z+g(y, x, \lambda),
  \end{equation}
  其中\(\boldsymbol{A}\)的特征根均为零,
  而\(\boldsymbol{B}\)的特征根均不为零;
  \(f ; g \in C^{r}, r \geqslant 2\);
  \( f(0,0,0)=0, g(0,0,0)=0, f, g=O\left(|y, z|^{2}\right)\).
为了研究奇点的分布,
在\(x=0\)和\(\lambda=0\)附近考虑方程
\begin{equation}
  \label{eq2.1.3}
  \mathbf{A}{y}+f(y, z, \lambda)=0,
  \quad
  \mathbf{B}z+g(y, z, \lambda)=0
\end{equation}
由隐函数定理,
存在\((y, \lambda)=(0,0)\)
的领域U和\(C^{r}\)函数
\(z=\varphi(y,\lambda)\),使得
\begin{equation*}
  B \varphi(y, \lambda)+g(y, \varphi(y, \lambda), \lambda) \equiv 0,
  \quad
  \forall(y, \lambda) \in U .
\end{equation*}
把函数\(z=\varphi(y,\lambda)\)代入
\ref{eq2.1.3}的第一个方程左端,可得\(C^{r}\)函数
\begin{equation}
  \label{eq2.1.4}
  G(y, \lambda) \stackrel{\mathrm{d}}{\longrightarrow} A y+f(y, \varphi(y, \lambda), \lambda)
\end{equation}
记
\begin{equation*}
  S=\{(y, \lambda) \in U | G(y, \lambda)=0\rangle,
  \quad
  S_{\lambda_{0}}=S  \cap\left\{\lambda=\lambda_{0}\right\}
\end{equation*}
则对不同的\(\lambda,|\lambda| \ll 1, S_{\lambda}\)结构的变化反应了奇点个数的变化规律.
这样就把对\ref{eq2.1.3}的讨论转化为对
\( G(y, \lambda)=0  \)的讨论,
使空间维数得到降低,
通常称\eqref{eq2.1.4}为方程\eqref{eq2.1.2}的\textbf{分岔函数}.
为了应用上的便利,下面在更一般的框架下讨论这个问题.
\subsection{Liapunov-Schmidt方法}
设\(X,Z\)和 \(\Lambda\)为实 \(Banach\)空间,
U和W分别为X和\(\Lambda\)中零点的领域.
\(C^{1}\)映射
\(M : U \times W \subset X \times \Lambda \rightarrow Z\),
满足
\(M(0,0)=0\).
我们要研究方程
\begin{equation}
  \label{eq2.1.5}
  M(x ,\lambda)=0
\end{equation}
在\(U \times W\)中 \((0,0)\)点的某领域内解的结构.
为此,
设\( A=D_{x} M(0,0)\),
并记
\(\mathscr{N}(A) \text {和} \mathscr{R}(A)\)
分别为A在X中的零空间和A在Z中的值域空间.
本节的一个基本假设是
\par
(H)\(\mathscr{N}(A)\)在X中存在补空间;
\(\mathscr{R}(A)\)是Z中的闭集,
并且在Z中存在补空间.
(当A为Fredholm算子时,这个假设总是成立的.在下文的应用中,经常是这种情形.)
\par
因此,在X上存在投影P,在P上存在投影Q,使得
\begin{equation}
  \label{eq2.1.6}
  \mathscr{R}(P)=\mathscr{N}(A), \quad \mathscr{R}(Q)=\mathscr{R}(A).
\end{equation}
\(\forall x \in U\),
可写成
\(
x=u+v, \text{其中}  u=P x \in \mathscr{N}(A)=X_{P},
v=(I-P) x \in \mathscr{N}(P)=X_{I-P}.
\)
这里I是恒同映射,
\(X_{p}\)和\(X_{!-P}\)表示投影
P和I-P的值域.
显然,方程\eqref{eq2.1.5}等价于
\begin{align}
  \label{eq2.1.7}
  {Q M(u+v, \lambda)=0}  \\
  {(I-Q) M(u+v, \lambda)=0}
\end{align}
定义映射
\(\psi : X_{P} \times X_{I-P} \times \Lambda \rightarrow \mathscr{R}(A)\),
\begin{equation*}
\psi(u, v, \lambda)=Q M(u+v, \lambda)
\end{equation*}
则\(\phi(0,0,0)=0, \text{且} \mathrm{D}_{v} \psi(0,0,0)=\left.A\right|_{\mathcal{N}(P)}\)是\(\mathscr{N}(P)\)与\(\mathscr{R}(A)\)间的同构.
由隐函数定理,
存在
\(X_{p}\)在原点的领域\( U_{0}\),
\( X_{I-P}\)在原点的领域\(V_{0}\),
\(\Lambda\)在原点的领域\(W_{0}\),
以及\(C^{1}\)映射\(v^{*}: U_{0} \times W_{0} \rightarrow V_{0}\),
使
\(U_{0} \times W_{0} \subset U, W_{0} \subset W\),且
\begin{equation*}
Q M\left(u+v^{*}(u, \lambda), \lambda\right) \equiv 0, \quad \forall(u, \lambda) \in U_{0} \times W_{0},
\end{equation*}
并且
\(  v^{*} .(0,0)=0, D_{u} v^{*}(0,0)=0\).
利用\(v^{*}\),定义
\(C^{1}\)映射\(x^{*}:U_{0}\times W_{0}\rightarrow U\),
和\(C^{1}\)映射\(G : U_{0} \times W_{0} \rightarrow \mathscr{N}(Q)\),
\begin{equation}
x^{*}(u, \lambda)=u+v^{*}(u, \lambda)
\end{equation}
\begin{equation}
G(u, \lambda)=(I-Q) M\left(u+v^{*}(u, \lambda), \lambda\right)
\end{equation}

容易验证,\(x^{*}(0,0)=0, D_{u} x^{*}(0,0)=I_{M(A)},G(0,0)=0,\mathrm{D}_{u} G(0,0)=0\).
总结上面的讨论,我们有下面的结果.
\begin{theorem}
  \label{thm2.1.5}
  如果条件(H)成立,\(U_{0}, V_{0}, W_{0}\)如上.
  则
  \(\forall u \in U_{0}, x \in U_{0} \times V_{0} \subset X, \text{和}
  \lambda \in W_{0} \subset \Lambda\),
  如下两组结论等价
\begin{itemize}
\item \(P x=u, \quad M(x, \lambda)=0\);
\item \(x=x^{*}(u, \lambda), \quad G(u, \lambda)=0\);
\end{itemize}
其中\(x^{*}\)与G分别有\eqref{2.1.8}和eqref{2.1.9}定义.
\end{theorem}

定理\ref{thm2.1.5}说明,
原来的奇点分岔问题
\(M(x, \lambda)=0\)转化为求解
\textbf{分岔方程}\(G(u,\lambda)=0\).
注意\(x \in X, M(x, \lambda) \in Z\),
而
\(u \in X_{P}=\mathscr{N}(A), G(u, \lambda) \in Z_{I-\mathrm{Q}}=\mathscr{N}(Q)\),
因而使问题的定义域及值域都作了显著的约化.
这就是Liapinov-Schmidt方法的核心思想.
\par
现在我们把上面的一般理论用于$\R^{n}$
上的向量场奇点分岔问题.
考虑依赖于参数\(\lambda\)的向量场
\begin{equation}
  \label{eq2.1.10}
  \frac{\mathrm{d} x}{\mathrm{d} t}=f(x, \lambda),
\end{equation}
其中
\(
f \in C^{r}\left(\mathbf{R}^{n} \times \mathbf{R}^{k}, \mathbf{R}^{n}\right), r \geqslant 2 ; f(0,0)=0, \mathrm{D}_{x} f(0,0)=\boldsymbol{A}.
\)
考虑奇点分岔问题,就是要在
\(\mathbf{R}^{n} \times \mathbf{R}^{k}\)的原点附件考察方程
\begin{equation}
  \label{eq2.1.11}
  f(x, \lambda)=\boldsymbol{A} x-N(x, \lambda)=0
\end{equation}
其中
\(
N \in C^{r}\left(\mathbf{R}^{n} \times \mathbf{R}^{k}, \mathbf{R}^{n}\right), N(0,0)=0, \mathrm{D}_{x} N(0,0)=0.
\)
与前面的一般情况对比,
此时有
\(X=Z=\mathbf{R}^{n}, \Lambda=\mathbf{R}^{k}\).
假设又有
\(\operatorname{dim} \mathscr{r}(\boldsymbol{A})=\operatorname{codim} \mathscr{R}(\boldsymbol{A})=1\),
则存在投影
\(P, Q : \mathbf{R}^{n} \rightarrow \mathbf{R}^{n}\),
满足\eqref{eq2.1.6}.
从而存在
\(u_{0} \in \mathscr{N}(A), w_{0} \in \mathscr{N}(Q)\),使
\begin{equation*}
  \mathscr{N}(\boldsymbol{A})=\operatorname{Span}\left\{\boldsymbol{u}_{0}\right\},
  \quad
  \mathscr{N}(\boldsymbol{Q})=\operatorname{Span}\left\{w_{0}\right\rangle.
\end{equation*}
从上面的一般理论知道,
存在\(\delta>0, \sigma>0\)和\(G^{r}\)函数
\(v=v^{*}(a, \lambda) \in X_{I-P}\),
满足\(v^{*}(0,0)=0, D_{a} v^{*}(0,0)=0\),
使当\(|a|<\delta,|\lambda|<\sigma\)时,
\begin{equation*}
\mathbf{Q}f\left(a u_{0}+v^{*}(a, \lambda), \lambda\right) \equiv 0.
\end{equation*}
由定理\ref{thm2.1.5},
\(
x=a u_{0}+v, v \in X_{I-P}
\)
是\eqref{eq2.1.11}的解,
当且仅当
\(v=v^{*}(a, \lambda) \)且
\((a,\lambda)\)满足\textbf{分岔方程}
\begin{equation*}
g(a, \lambda)=0,
\end{equation*}
这里\textbf{分岔函数}g由下式定义:
\begin{equation}
  \label{eq2.1.12}
  g(a, \lambda) w_{0}=(I-Q) f\left(a u_{0}+v^{*}(a, \lambda), \lambda\right)
\end{equation}

\begin{example}
\label{exam2.1.6}  
用Liapunov-Schmidt方法重新考虑第一章例\ref{exam1.3.12}.
我们考虑\(\R ^{2}\)中一类更广泛的微分方程
\begin{equation}
  \label{exam2.1.13}
  \left\{
    \begin{array}{l}
      {\frac{\mathrm{d} x}{\mathrm{d} t}=y}, \\
      {\frac{\mathrm{d} y}{\mathrm{d} t}=\beta y+x^{2}+x y(1+\varphi(x))+y^{2} \Phi(x, y)}
    \end{array}
  \right.
\end{equation}
其中\(\beta \neq 0, \varphi, \Phi \in C^{\infty}, \varphi(0)=0\).
考虑它的奇点\((0,0)\)在扰动下的分岔问题.
此时线性部分矩阵为
\(
A=\left(
  \begin{array}{ll}
    {0} & {1} \\
    {0} & {\beta}
  \end{array}
\right),
\operatorname{dim} \mathscr{N}(\boldsymbol{A})=\operatorname{codim} \mathscr{R}(\boldsymbol{A})=1.
\)
取满足\eqref{eq2.1.6}的投影
\(P, Q \cdot \mathbf{R}^{2} \rightarrow \mathbf{R}^{2}\).
令
\begin{equation*}
u_{0}=\left( \begin{array}{l}{1} \\ {0}\end{array}\right), v_{0}=\left( \begin{array}{l}{0} \\ {1}\end{array}\right), w_{0}=\left( \begin{array}{l}{0} \\ {1}\end{array}\right), s_{0}=\left( \begin{array}{l}{1} \\ {\beta}\end{array}\right)
\end{equation*}
则
\begin{equation*}
\mathscr{N}(\boldsymbol{A})=X_{P}=\operatorname{Span}\left\{u_{0}\right\}, X_{I-P}=\operatorname{Span}\left\{\boldsymbol{v}_{0}\right\}
\end{equation*}
\begin{equation*}
  \mathscr{R}(\boldsymbol{A})=\mathscr{R}(Q)=\operatorname{Span}\left\{s_{0}\right\}.
\end{equation*}
取\(\mathscr{N}(Q)=\operatorname{Span}\left\{w_{0}\right\}\).
函数\(v=v^{*}(a)=\left( \begin{array}{l}{0} \\ {v_{2}}\end{array}\right) \in X_{I-P}\)由方程
\(Q f\left(a u_{0}+v^{*}(a)\right)=0\)确定,既
\begin{equation*}
0=Q\left\{v_{2} s_{0}+\left[a^{2}+a v_{2}(1+\varphi(\alpha))+y^{2} \Phi\left(\alpha, v_{2}\right)\right] w_{0}\right\}=v_{2} s_{0}.
\end{equation*}
因此
\(v=v^{*}(a) \equiv 0\).
把它代入\eqref{eq2.1.12},得到
\begin{equation*}
\begin{array}{c}{g(a) w_{0}=(I-Q) f\left(a u_{0}+v^{*}(a)\right)} \\ {=(I-Q) f \left( \begin{array}{l}{a} \\ {0}\end{array}\right)=(I-Q) \left( \begin{array}{l}{0} \\ {a^{2}}\end{array}\right)=a^{2} w_{0}}\end{array}
\end{equation*}
从而分岔函数\(g(a)=a^{2}\).
\end{example}
\par
如果我们考虑方程\eqref{eq2.1.13}的\(C^{2}\)扰动,
扰动参数为\(\lambda\),
则扰动后方程的分岔函数\(g(a,\lambda)\)
满足\(g(a,0)=a^{2}\).
利用隐函数定理易知,
存在\(\delta>0\)和\(C^{0}\)函数
\(a=a(\lambda)\),
使得\(a(0)=0, D_{a} g(a(\lambda), \lambda) \equiv 0,
\mathrm{D}_{a}^{2} g(a(\lambda), \lambda) \neq 0, \forall|\lambda|<\delta\).
利用Taylor公式可得,
当\(|\lambda| \ll 1,|a-a(\lambda)| \ll 1 \),有
\begin{equation*}
g(a, \lambda)=\mu(\lambda)+D_{a}^{2} g(a(\lambda), \lambda)(a-a(\lambda))^{2}+o\left(|a-a(\lambda)|^{2}\right)
\end{equation*}
其中
\(\mu(\lambda)=g(a(\lambda), \lambda)\).
因此,
在\((a,\lambda)=(0,0)\)附近,
方程\(g(a,\lambda)=0\)当\(\mu(\lambda) \mathrm{D}_{a}^{2}(0,0)<0\)时有两个零点.
应用定理\ref{thm2.1.5}可知,
原系统\eqref{2.1.13}在扰动下发生鞍结点分岔.
注意,当扰动方程为\(C^{\infty}\)时,
最后的讨论可从第一章定理\ref{thm2.2.12}直接得到.
\begin{note}
  本节中讨论的奇点分岔问题,主要着重于奇点个数随参数变动而发生变化的规律.
  实际上,在奇点个数发生变化的同时(甚至在奇点个数不变时,见附注\ref{ruzhu1.3}),
  轨道结构还可能发生其他变化.
  例如闭轨、同宿轨、异宿轨等的产生或消失.
  这些情形在下章中将会看到.
\end{note}

\section{闭轨分岔}
考虑微分方程族
\begin{equation}
  \left(X_{\lambda}\right) : \quad \frac{d x}{d t}=v(x, \lambda)
  \label{eq2.2.1}
\end{equation}
其中\(v \in C^{r}\left(\mathbf{R}^{n} \times \mathbf{R}^{k}, \mathbf{R}^{n}\right), r \geqslant 1, k \geqslant 1\).
设\(X_{0}\)有一条孤立闭轨\(\gamma\).
当\(\lambda \neq 0,|\lambda| \ll 1\)时,
我们关心\(X_{\lambda}\)在\(\gamma\) 的领域内是否还有闭轨?
有几条闭轨?
这就是\textbf{闭轨分岔}问题.
当\(\gamma\)为双曲闭轨时,问题是平凡的(见第一章\ref{1.1}).
因此,我们要找到一些方法,来判别\(\gamma\)的双曲性,
以及当\(\gamma\)非双曲时如何研究闭轨的分岔问题.
至于\(\gamma\)为方程\eqref{eq2.2.1}的非孤立闭轨的情形,我们留待\ref{2.5}中讨论.
\par
从原则上说,可以把闭轨分岔问题转化为它的Poincare映射的不动点的不动点的分岔问题,
从而可利用上节的方法,事实上,任取
\(p\in \gamma\),存在过\(p\)的\(n-1\)维"无切截面"
\(U' \subset U\),
按第一章定义\ref{def1.1.6}所述,
可定义Poincare映射\(P: U^{\prime} \times W \rightarrow U\),
它是\(C^{r}\)的,
其中W是\(\mathbf{R}^{k}\)中原点的领域,
满足\(P(p, 0)=p\).
\( X_{\lambda}\)的闭轨相应于
\(M(x, \lambda)=\underline{\mathrm{d}}P(x, \lambda)-x
\text{在}
(x, \lambda) \in U^{\prime} \times W
\)
内的零点.
因此,\ref{2.1}中的方法都是适用的.
注意,如果把坐标原点平移到\(p\)点,
就会满足\ref{1.1}中\(M(0,0)=0\)的条件.
\par
在解决具体问题时,困难在于如何实施上述原则.
下面,我们就平面向量场的情形作进一步的讨论,
顺便介绍曲线坐标方法和某些重要结论.
\par
考虑平面上的微分方程族
\begin{equation}
  \label{eq2.2.2}
\left(X_{\lambda}\right):\frac{d x}{d t}=v(x, \lambda)
\end{equation}

其中\(
v \in C^{r}\left(\mathbf{R}^{2} \times \mathbf{R}^{k}, \mathbf{R}^{2}\right), r \geqslant 2, k \geqslant 1.
\)
设\(X_{0}\)有闭轨\(\gamma\),
它有如下的参数表示
\begin{equation*}
\gamma: \quad x=\varphi(t)=\left( \begin{array}{l}{\varphi(t)} \\ {\varphi_{2}(t)}\end{array}\right).
\end{equation*}
设\(\gamma\)以T为周期,
并且为负定向,
既当t增大时,\(\varphi(t)\)沿\(\gamma\)顺时针方向旋转.
取\(\gamma\)在\(\varphi(t)\)点沿外法向的单位向量
\begin{equation}
  \label{eq2.2.3}
\zeta(t)=\frac{1}{\left|\varphi^{\prime}(t)\right|} \left( \begin{array}{c}{-\varphi_{2}^{\prime}(t)} \\ {\varphi_{1}^{\prime}(t)}\end{array}\right).
\end{equation}
由\(\zeta(t) \perp \varphi^{\prime}(t) \text{及}|\zeta(t)|=1\)易知,
\(\forall 0 \leqslant t \leqslant T\),
\begin{equation}
  \label{2.2.4}
\left\langle\zeta(t), \varphi^{\prime}(t)\right\rangle \equiv 0, \quad\left\langle\zeta(t), \zeta^{\prime}(t)\right\rangle \equiv 0,
\end{equation}
其中\(\langle \dot ,\dot \rangle\)表示\(\mathbb{R}^{2}\)中的内积.
取坐标变换
\begin{equation}
  \label{eq2.2.5}
  x=\varphi(s)+\xi(s) n
\end{equation}
其中x在\(\gamma\)附近;
\(0 \leqslant s \leqslant T,|n| \ll 1 \).
坐标
\(
(s,n)
\)
可以这样理解:从\(\varphi(0)\)沿\(\gamma\)经过时间s到达\(\varphi(s)\),
再从\(\varphi(s)\)点沿\(\gamma\)的外法向\(\xi(s)\)移动长度n到达x点
(当n<0时,表示向内法向移动),
见图\ref{pic2.2-1(a)}.
注意,\{ n=常数\}与\{s=常数\}在平面上形成蛛网形坐标曲线(见图\ref{pic2.2-1(b)}),称\((s,n)\)为\textbf{曲线坐标}.
我们先把方程\eqref{eq2.2.2}转换成曲线坐标系下的方程,
然后建立Poincare映射.
把\eqref{eq2.2.5}对t求导,
并应用\eqref{2.2.2}得
\begin{equation}
  \label{eq2.2.6}
  v(\varphi(s)+\zeta(s) n, \lambda)=\frac{\mathrm{d} x}{\mathrm{d} t}=\left(\varphi^{\prime}(s)+\xi^{\prime}(s) n\right) \frac{\mathrm{d} s}{\mathrm{d} t}+\zeta(s) \frac{\mathrm{d} n}{\mathrm{d} t}
\end{equation}
分别以\(\zeta(s) \text{及} \varphi^{\prime}(s)\)对上式作内积,
利用\ref{eq2.2.4}及\(|\zeta(s)|=1\)得
\begin{equation*}
\frac{\mathrm{d} n}{\mathrm{d} t}=\langle\zeta(s), v(\varphi(s)+\zeta(s) n, \lambda)\rangle,
\end{equation*}
\begin{equation*}
\frac{d s}{d t}=\frac{\left\langle\varphi^{\prime}(s), v(\varphi(s)+\zeta(s) n, \lambda)\right\rangle}{\left|\varphi^{\prime}(s)\right|^{2}+n\left\langle\varphi^{\prime}(s), \zeta^{\prime}(s)\right\rangle}
\end{equation*}
消去t得
\begin{equation}
  \frac{d n}{d s}=
  \frac{\left(\left|\varphi^{\prime}(s)\right|^{2}+n\left\langle\varphi^{\prime}(s), \zeta^{\prime}(s)\right\rangle\right)\langle\zeta(s), v(\varphi(s)+\xi(s) n, \lambda)\rangle}{\left\langle\varphi^{\prime}(s), v(\varphi(s)+\xi(s) n, \lambda)\right\rangle}
  \stackrel{\mathrm{d}}{=} F(n, s, \lambda),
\end{equation}
由于\(x=\varphi(s)\)为\(X_{0}\)的解,
故
\begin{equation}
  \varphi^{\prime}(s)=v(\varphi(s), 0);
  \label{eq2.2.8}
\end{equation}
利用\eqref{eq2.2.8}和\eqref{eq2.2.4},可从\eqref{eq2.2.7}算得
\begin{equation*}
F(0, s, 0)=0,
\end{equation*}

\begin{equation}
  \label{eq2.2.9}
\left.\frac{\partial F}{\partial n}\right|_{n=0, \lambda=0}=\left\langle\zeta(s), \frac{\partial v}{\partial x}(\varphi(s), 0) \zeta(s)\right\rangle \cdot= H(s)
\end{equation}
从而\eqref{eq2.2.7}可写成
\begin{equation}
  \label{eq2.2.10}
  \frac{d n}{d s}=\left(H(s)+F_{1}(n, s, \lambda)\right) n,
\end{equation}
其中,\(\left.F_{1}\right|_{\lambda=0}=O(|n|)\).
因此,\eqref{eq2.2.10}满足初值条件\(\left.n\right|_{s=0}=a\)的解可表示为
\begin{equation}
  \label{eq2.2.11}
  n(s, a, \lambda)=a\left(\exp \int_{0}^{s}\left[H(t)+F_{1}(n(t, a, \lambda), t, \lambda)\right] \mathrm{d} t\right).
\end{equation}
\par
现在,取\(X_{0}\) 的闭轨\(\gamma\) 上的点
\( x_{0}=\varphi(0)\),
过\(x_{0}\) 以法线\(n_{0}\) 为方向取一截线L,
建立\ref{eq2.2.11}的Poincare映射
(见第一章定义\ref{def1.1.6},但此时与参数\(\lambda\)有关 )
\( P:(a, \lambda) \mapsto n(T, a, \lambda)\),
这里的函数\(n(s,a,\lambda)\)由\ref{eq2.2.11}定义.
显然,\( n(T, 0,0)=0 \).
定义\textbf{后继函数}
\begin{equation}
  \label{eq2.2.12}
G(a, \lambda)=n(T, a, \lambda)-a,
\end{equation}
则对每个\(\lambda,|\lambda| \ll 1, G(a, \lambda)\)关于a的零点与\(X_{\lambda}\)在\(\gamma\)附近的闭轨相对应.
注意\eqref{eq2.2.2}中\(v \in C^{r}, r \geqslant 2, F\)关于
\(n,\lambda\)为\(C^{r}\)的,
关于s为\(C^{r-1}\)的,
故\eqref{eq2.2.11}中的解\(n \in C^{r}\),
从而\eqref{eq2.2.12}中的函数\(G \in C^{r}\).
\begin{defination}
  \label{def2.2.1}
  若存在\(\epsilon>0\),
  使得\( \forall a \in (0,\epsilon)\),
  都有\( G (a,\lambda)<0(>0)\),
  则称\(\gamma\)为\textbf{外侧稳定(外侧不稳定)的极限环}.
  若存在\(\epsilon>0\),
  使\(\forall a \in (-\epsilon ,0)\),
  都有\(G(a, \lambda)>0(<0)\),
  则称\(\gamma\)为\textbf{内侧稳定(内侧不稳定)的极限环}.
  双侧均稳定(均不稳定)的极限环称为\textbf{稳定(不稳定)极限环};
  双侧稳定性不同时,称\(\gamma\)为\textbf{半稳定极限环}.
\end{defination}
从上述定义可知,稳定(不稳定)极限环w必为孤立闭轨.
下面定义中的$\gamma$为非孤立闭轨.

\begin{defination}
  \label{def2.2.2}
  若$\forall \varepsilon>0, \exists a_{1}, a_{2} \in(0, \varepsilon)$
  使得$G(a_{1},\lambda)=0$,
  但$G(a_{2},\lambda)\neq 0$,
  则称$\gamma$为\textbf{外侧复型极限环}.
  若存在$\epsilon>0$,
  使$\forall a \in (0,\epsilon)$,
  都有$G(a,\lambda)=0$,
  则称$\gamma$为\textbf{外侧周期环域}.
\end{defination}
类似可定义内侧复型极限环与内侧周期环域.
\begin{theorem}
  \label{thm2.2.3}
  解析向量场不存在复型极限环.
\end{theorem}

\begin{proof}
  由于在$X_{0}$的闭轨$  \gamma_{1}\{
  x=\varphi(t) | 0 \leqslant t \leqslant T
  \}$
  上无奇点,
  利用方程\eqref{2.2.8}和$\gamma$的紧致性可知存在
  $\delta>0$,使得当
  $|n|<\delta,|\lambda|<\delta$时,
  $\left\langle\varphi^{\prime}(s), v(\varphi(s)+\zeta(s) n, \lambda)\right\rangle$恒正.
  因此,方程\eqref{eq2.2.7}的右端函数$F(n,s,\lambda)$i解析,
  从而$G(a,\lambda)$解析.
  由于解析函数的非孤立零点必存在一个领域,使函数在其中恒为零.因此定理得证.
\end{proof}

\begin{defination}
  若存在$\exists \epsilon >0$和正整数$k,l \leqslant k \leqslant r$,
  使当$|a|< \epsilon$时,
  有
  \begin{equation}
    \label{eq2.2.13}
G(a, 0)=c_{k} a^{k}+o\left(|a|^{k}\right), \quad c_{k} \neq 0
\end{equation}
则称
$\gamma$为$X_{0}$的\textbf{k重极限环}.
当$k=1$时称为\textbf{单重极限环},
当$k>1$时,称为\textbf{多重极限环}.
\end{defination}
显然,当$k$为奇数时,
$c_k<0$表明$\gamma$为稳定的极限环,
而$c_k>0$表明$\gamma$为不稳定的极限环;
当$k$为偶数时,$\gamma$为半稳定的极限环.
注意,这里说得稳定性为轨道稳定性,而不是结构稳定性.
事实上,与\ref{1.1}的定义相对照可知,
单重环是结构稳定(双曲)的,
而多重环都是结构不稳定(非双曲)的.
为了判别$\gamma$是否为单重的,我们记
\begin{equation}
  \label{eq2.2.14}
  \sigma=\int_{0}^{T} \operatorname{tr} \frac{\partial v}{\partial x}(\varphi(s), 0) d s
\end{equation}
\begin{theorem}
  \label{thm:2.2.5}
  若$\sigma \neq 0,$,则$\gamma$为$X_{0}$的单重环.
  当$\sigma < 0$时$\gamma$为稳定的;
  当$\sigma>0$时$\gamma$为不稳定的.
\end{theorem}

\begin{proof}
  由~(\ref{eq2.2.11}),~(\ref{eq2.2.12})得到
  \begin{equation*}
G(a, 0)=a\left(\exp \int_{0}^{T}\left[H(s)+F_{1}(n(s, a, 0), s, 0)\right] \mathrm{d} s-1\right).
\end{equation*}
故由$F_{1}(n, s, 0)=O(|n|)$,和$n(T, 0,0)=0$,可得
\begin{equation}
  \label{eq2.2.15}
G_{a}^{\prime}(0,0)=\lim _{a \rightarrow 0} \frac{G(a, 0)-G(0,0)}{a}=\exp \int_{0}^{T} H(s) \mathrm{d} s-1
\end{equation}
另一方面,
把~(\ref{eq2.2.9})式的内积按分量展开,
并利用~(\ref{eq2.2.3}),~(\ref{eq2.2.8})可得
\begin{equation*}
\begin{aligned} H(s) &=\left\langle\xi(s), \frac{\partial v}{\partial x}(\varphi(s), 0) \zeta(s)\right\rangle \\ &=\operatorname{tr} \frac{\partial v}{\partial x}(\varphi(s), 0)-\frac{1}{\left|\phi^{\prime}(s)\right|^{2}} \Phi \\ &=\operatorname{tr} \frac{\partial v}{\partial x}(\varphi(s), 0)-\frac{1}{\langle v, v\rangle}\left\langle v, \frac{\partial v}{\partial x} v\right\rangle \\ &=\operatorname{tr} \frac{\partial v}{\partial x}(\varphi(s), 0)-\frac{d}{d s} \ln |v| \end{aligned},
\end{equation*}
其中,$v=v(\varphi(s),0)$,
\begin{equation*}
\Phi=\frac{\partial v_{1}}{\partial x_{1}}\left(\varphi^{\prime}\right)^{2}+\frac{\partial v_{1}}{\partial x_{2}} \varphi^{\prime} \varphi_{2}^{\prime}+\frac{\partial v_{2}}{\partial x_{1}} \Phi^{\prime} \varphi_{1}^{\prime}+\frac{\partial v_{2}}{\partial x_{2}}\left(\varphi_{2}^{\prime}\right)^{2}.
\end{equation*}
注意到$\gamma$以T为周期,因此
\begin{equation*}
\int_{0}^{T} H(s) \mathrm{d} s=\int_{0}^{T} \operatorname{tr} \frac{\partial v}{\partial x}(\varphi(s), 0) \mathrm{d} s=\sigma,
\end{equation*}
代入~(\ref{eq2.2.15})得
\begin{equation*}
G_{a}^{\prime}(0,0)=\mathrm{e}^{\sigma}-1
\end{equation*}
当$\sigma \neq 0$时,得到$G(0,0)=0, G_{a}^{\prime}(0,0) \neq 0$.
故,$\gamma$是单重极限环.
它的稳定性与$\sigma $的符号之间的关系是显然的.定理得证.
\end{proof}

\begin{corollary}
  \label{colory 2.6}
  当$X_{0}$的闭轨$\gamma$为多重极限环、复型极限环或$\gamma$附近为周期环域时,必有$\sigma=0$.
\end{corollary}

\begin{theorem}
  \label{thm2.7}
  设$\gamma$为$X_{0}$的k重极限环($k \geqslant 1$),则对于$X_{\lambda}$
  \begin{itemize}
\item 存在$\gamma$的(环形)领域U和正数$\delta$,使只要$|\lambda|<\delta$,$X_{\lambda}$在U内至多有k个极限环.
\item $\forall i, 1 \leqslant i \leqslant k, \forall \delta>0$,对任给的$\gamma$的(环形)领域 $\boldsymbol{V} \subset \boldsymbol{U}$,
  $\exists X_{0}$的扰动系统$X_{\lambda},|\lambda|<\delta$,
  使得$X_{\lambda}$在$ \boldsymbol{V} $内恰有$i$个极限环.
  当$k$为偶数时,上述结论可扩充至$i=0$.
\item 当$k$为奇数时,$\forall V \subset U, \exists \delta>0$,
  使当$|\lambda|<\delta$时,$X_{\lambda}$在$V$内至少有一个极限环.
\end{itemize}
\end{theorem}

\begin{proof}
  当扰动系统$X_{\lambda}$对应的~(\ref{eq2.2.2})中的$v \in C^{\infty}$时,
  可由Malgrange 定理(第一章定理 \ref{sec:thm1.2.12})直接推得以上结论.
  当$v \in C^{r}$时,可利用隐函数定理和中值定理证明.详细推导从略.
\end{proof}

\begin{example}
  \label{exam:2.8}
  考虑 $\R^{2}$上的系统
  $$\frac{\mathrm{d} x}{\mathrm{d} t}=y, \quad \frac{\mathrm{d} y}{\mathrm{d} t}=\lambda_{1}+\lambda_{2} y+a x^{2}+b x y^{2},$$
  其中参数$\lambda_{2} \neq 0$.设存在闭轨 $\gamma$,周期为T,则
  $\sigma=\int_{0}^{T} \operatorname{tr}\left.\frac{\partial v}{\partial(x, y)}\right|_{\gamma} \mathrm{d} t=\int_{0}^{T}\left.\left(\lambda_{2}+2 b x y\right)\right|_{\gamma} \mathrm{d} t=\lambda_{\mathrm{z}} T \neq 0,$
  这里利用系统的第一个方程得到$2 x y \mathrm{d} t=2 x \mathrm{d} x=\mathrm{d}\left(x^{2}\right)$.
  因此,这闭轨也是双曲的.
\end{example}

\begin{corollary}
  \label{corollary:2.9}
  上例表明,在一些具体问题中,利用~(\ref{eq2.2.14})计算$\sigma$时,可以不必知道$\phi(s)$的表达式,这就给利用定理~\ref{thm:2.2.5}判别闭轨的双曲性提供了方便.
  当$\sigma=0$时,需要从~(\ref{eq2.2.12})进一步计算~(\ref{eq2.2.13})式中第一个不为零的$c_k$,以便按定义~\ref{def2.2.4}判断 $\gamma$的重次.
  这时一般来说计算量就大.
\end{corollary}

\section{Hopf 分岔}
\label{sec:2.3}

当向量场在奇点的线性部分矩阵有一对复特征根,
并且随参数变化为穿越虚轴时,
在奇点附近的一个二维中心流形上,
奇点的稳定性发生翻转,
从而在奇点附近产生闭轨的现象,
称为Hopf分岔,第一章~(\ref{exam1.2.9})就是一个典型的实例.
既然Hopf分岔发生在二维中心流形上,
为了简单,
下面讨论二维方程.

\subsection{金典的Hopf分岔定理}
\label{sec:2.3.1}
考虑$C^{\infty}$向量场
\begin{equation}
  \label{eq:2.3.1}
\left(X_{\mu}\right) : \quad \frac{\mathrm{d} x}{\mathrm{d} t}=A(\mu) x+F(x, \mu)
\end{equation}
其中
$x=\left(x_{1}, x_{2}\right) \in \mathbf{R}^{2}, \mu \in \mathbf{R}^{1}, F(0,0)=0, D_{x} F(0,0)=0$.
设线性部分矩阵$\mathbf{A}(\mu)$有特征值$\alpha(\mu) \pm i \beta(\mu)$,
满足条件
\begin{enumerate}
\item\label{item:1}[($H_{1}$)]
 $ \alpha(0)=0, \beta(0)=\beta_{0} \neq 0 $;
\item\label{item:2}
  $\alpha^{\prime}(0) \neq 0 $;
\item\label{item:3}
  $\operatorname{Re} c_{1} \neq 0$;
\end{enumerate}

其中$c_1$为向量场$X_0$的如下复正规型中的系数(见第一章例~\ref{exam:1.4.9}),
\begin{equation}
  \label{eq:2.3.2}
\frac{\mathrm{d} w}{\mathrm{d} t}=\mathrm{i} \beta_{0} w+c_{1}|w|^{2} w+\cdots+c_{k}|w|^{2 k} w+O\left(|w|^{2 k+3}\right)
\end{equation}

\begin{theorem}[Hopf分岔定理]
  \label{thm:2.3.1}
  设条件~\ref{item:1}和~\ref{item:3}成立,
  则$\exists \sigma >0$和 $x=0$的领域$U$,
  使得当$|\mu|<\sigma$时,
  方程~(\ref{eq:2.3.1})在$U$内至多有一个闭轨(从而是极限环).
  如果条件~\ref{item:2}也成立,
  则$\exists \sigma>0$和在 $0< x_1 \leq \sigma$上定义的函数
  $\mu=\mu\left(x_{1}\right)$,满足
  $\mu(0)=0$,而且
  
\begin{enumerate}
\item\label{item:4}
  当$\mu=\mu\left(x_{1}\right), 0<x_{1} \leqslant 0$时,
  系统~(\ref{eq:2.3.1})过点$(x_1,0)$的轨道是它唯一的轨道.
  当$\operatorname{Re} c_{1}<0$时,它是稳定的;
  当$\operatorname{Re} c_{1}>0$时,它是不稳定的;
\item\label{item:5}
  当$\mu\alpha^{\prime}(0) \operatorname{Rec}_{1}<0$时,$\mu^{\prime}\left(x_{1}\right)>0$;
    当$\mu\alpha^{\prime}(0) \operatorname{Rec}_{1}>0$时,$\mu^{\prime}\left(x_{1}\right)<0$;
  \end{enumerate}
\end{theorem}

在下文中,我们先证明一个更广泛的定理,再证明定理~\ref{thm:2.3.1}.

\begin{corollary}
  \label{corollary:2.3.2}
  Hopf分岔定理有多种形式和多种证法.
  例如可参考[FLLL].
  实际应用定理~\ref{thm:2.3.1}时,重要的是计算$\mathbf{R e}c_{1}$.
  对某些常见的方程,导出计算公式将为应用带来方便.
  下面给出一个例子.
\end{corollary}

\begin{example}[GH]
  \label{example:2.3.3}
  如果二维系统$X_{0}$具有如下形式
  $$
\frac{\mathrm{d}}{\mathrm{d} t}\left(\begin{array}{l}{x} \\ {y}\end{array}\right)=\left(\begin{array}{cc}{0} & {-\beta_{0}} \\ {\beta_{0}} & {0}\end{array}\right)\left(\begin{array}{l}{x} \\ {y}\end{array}\right)+\left(\begin{array}{l}{f(x, y)} \\ {g(x, y)}\end{array}\right),
$$
其中$f(0,0)=g(0,0)=0,$ D $f(0,0)=\operatorname{Dg}(0,0)=0, \beta_{0} \neq 0$,
则有如下计算公式:
\begin{equation}
  \label{eq:2.3.3}
  \begin{array}{l}
    \operatorname{Re} c_{1}=\frac{1}{16}\left\{\left(f_{x x x}+f_{x v y}+g_{x x y}+g_{y y y}\right)\right.\\
    {+\frac{1}{\beta_{0}}\left[f_{x y}\left(f_{x x}+f_{y y}\right)-g_{x y}\left(g_{x x}+g_{y y}\right)\right.} \\
    {-f_{x x} g_{x x}+f_{y y} g_{y y} ]\left.\}\right|_{x=y=0}}
  \end{array}
\end{equation}
\end{example}

\subsection{退化Hopf分岔定理}
\label{sec:2.3.2}

当条件~\ref{item:1}与\label{sec:hopf}~\ref{item:3}至少有一个不成立时,仍有可能出现Hopf分岔,这就是所谓的退化Hopf分岔问题.

考虑二维m参数向量场
\begin{equation}
  \label{eq:2.3.4}
\left(X_{\mu}\right) : \quad \frac{\mathrm{d} x}{\mathrm{d} t}=f(x, y, \mu), \quad \frac{\mathrm{d} y}{\mathrm{d} t}=g(x, y, \mu)
\end{equation}
其中$$
x, y \in \mathbf{R}^{1}, \mu \in \mathbf{R}^{m}, f, g \in C^{\infty}, f(0,0,0)=g(0,0,0)=0
$$.

\begin{corollary}
  \label{corollary:2.3.4}
  设在奇点$(x, y)=(0,0)$,
  系统~(\ref{eq:2.3.4})的线性部分矩阵有一对复特征根
  $\alpha(\mu) \pm \mathrm{i} \beta(\mu)$,
  满足条件~\ref{item:1},则对任意自然数$k$,
  存在$\delta>0$和光滑依赖于参数$\mu$的多项式变换,
  当  $|\mu|<\delta$时,
  可以把~(\ref{eq:2.3.4})化为
  \begin{equation}
    \label{eq:2.3.5}
\begin{aligned}
    \frac{\mathrm{d} w}{\mathrm{d} t}= &[\alpha(\mu)+\mathrm{i} \beta(\mu)] w+c_{1}(\mu) w^{2} \overline{w}+\\
 &   \dots+c_{k}(\mu) w^{2+1} \overline{w}^{2}+O\left(|w|^{2 k+3}\right),
\end{aligned}
\end{equation}
其中$\alpha(0)=0, \beta(0)=\beta_{0}, c_{i}(0)=c_{i}, i=1,2, \cdots, k$,
这里$c_i$是把$X_0$化为~(\ref{eq:2.3.2})后的系数.
\end{corollary}

\begin{proof}
  记$\lambda(\mu)=\left(\lambda_{1}(\mu), \lambda_{2}(\mu)\right)$,
  其中$\lambda_{1}(\mu)$和$\lambda_{2}(\mu)$为复特征根
  $\alpha(\mu) \pm \mathrm{i} \beta(\mu);m=\left(m_{1}, m_{2}\right)$,
  其中$m_{1}, m_{2}$为自然数;
  $M_{k} \stackrel{\mathrm{d}}{\longrightarrow} \left\{m | 2 \leqslant m_{1}+m_{2} \leqslant 2 k+2\right\}$;
  并且
  $$
(m, \lambda(\mu))=m_{1} \lambda_{1}(\mu)+m_{2} \lambda_{2}(\mu)
$$.
由条件~\ref{item:1}知,
$\lambda_{1}(0)=(m, \lambda(0))$给出
$\leqslant 2 k+2$阶的共振条件,
当且仅当$m \in M^{*} \stackrel{\mathrm{d}}{\longrightarrow}\left\{m | m_{1}=m_{2}+1, m_{3}=1, \cdots, k\right\} \subset M_{k}$.
由于$\lambda_{j}(\mu)$是光滑函数(j=1,2),故存在$\delta>0,$使当
$|\mu|<\delta$时,
$\lambda_{1}(\mu) \neq(m, \lambda(\mu))$,
当$m \in M_{k} \backslash M^{*}$.
因此,用第一章定理~\ref{thm:1.4.6}和例~(\ref{exam:1.4.9})同样的推理可得引理的结论.
\end{proof}

\begin{defination}
  \label{def:2.3.5}
  称$X_0$以$(x,y)=(0,0)$为 \textbf{k阶细焦点}($k\geq 1$),
  如果条件~\ref{item:1}成立,并且把$X_0$化成正规形~(\ref{eq:2.3.2})后,
  满足条件
  \begin{equation}
    \label{eq:2.3.6}
\operatorname{Re} c_{1}=\dots=\operatorname{Re} c_{k-1}=0, \operatorname{Re}_{c_{k}} \neq 0
\end{equation}
\end{defination}

\begin{theorem}
  \label{thm:2.3.6}
  设向量场$X_0$以$(0,0)$点为$k$阶细焦点,
  则$X_0$在扰动下可发生 \textbf{$k$阶Hopf分岔},
  既
  
\begin{enumerate}
\item\label{item:6}
  对它的任一开折系统$X_{\mu}$,存在$\sigma>0$和$(x,y)=(0,0)$点的领域$U$,
  使得当$|\mu|<\sigma$时,
  $X_{\mu}$在$U$内至多有$k$个极限环;
\item\label{item:7}
  对任意整数$j, 1 \leqslant j \leqslant k$,任意常数$\sigma^{*}$,
  $0<\sigma^{*}<\sigma$,
  以及$(x,y)=(0,0)$的任意领域$U^{*} \subset U$,
  存在一个开折系统$X_{\mu}^{*} $,
  使得$X_{\mu}^{*}$在$U^{*}$内恰有$j$个极限环,
  其中$|\mu|<\sigma^{*}$.
\end{enumerate}
\end{theorem}

\begin{proof}[RS]
  由引理~\ref{corollary:2.3.4},可经光滑依赖于参数的多项式变换化为~(\ref{eq:2.3.5}).
  对~(\ref{eq:2.3.5})及其共轭方程引入极坐标,
  注意$r^{2}=w \overline{w}, \mathrm{e}^{2 \varphi \mathrm{i}}=w \overline{w}^{-1}$,
  可得到
  \begin{equation}
    \label{eq:2.3.7}
    \left\{\begin{array}{l}
             
             {\dot{r}=\alpha(\mu) r+\operatorname{Re}\left(c_{1}(\mu)\right) r^{3}+\cdots+\operatorname{Re}\left(c_{k}(\mu)\right) r^{2 k+1}+O\left(r^{2 k+3}\right)} \\
             {\dot{\varphi}=\beta(\mu)+O\left(r^{2}\right)}
           \end{array}\right.
       \end{equation}
       由于$\beta \neq 0$,
       所以当$|\mu| \ll 1$时,
       可从~(\ref{eq:2.3.7})得到,
\begin{equation}
\label{eq:2.3.8}
\frac{\mathrm{d} r}{\mathrm{d} \varphi}=h_{0}(\mu) r+h_{1}(\varphi, \mu) r^{2}+\cdots+h_{k}(\varphi, \mu) r^{2 k+1}+O\left(r^{2 k+3}\right)
\end{equation}
其中$r \ll 1$,并且
\begin{align*}
  h_{0}(\mu)=\frac{\alpha(\mu)}{\beta(\mu)},\\
  h_{1}(\varphi, \mu)=\frac{\operatorname{Re}\left(c_{1}(\mu)\right)}{\beta(\mu)}+\eta_{1,0} \alpha(\mu),\\
  \cdots\cdots\cdots\cdots\\
  h_{k}(\varphi, \mu)=& \frac{\operatorname{Re}\left(c_{k}(\mu)\right)}{\beta(\mu)}+\eta_{k, k-1} \operatorname{Re}\left(c_{k-1}(\mu)\right)+\\
 \cdots+\eta_{k, 1} \operatorname{Re}\left(c_{1}(\mu)\right)+\eta_{k, 0} \alpha(\mu),(k \geqslant 2),
\end{align*}
这里$\eta_{i, j}=\eta_{i, j}(\varphi, \mu)$
是$\varphi \in[0,2 \pi]$和$\mu$(在0附近)的光滑函数.
当$\mu=0$时,
(\ref{eq:2.3.8})成为
\begin{equation}
  \label{eq:2.3.9}
\frac{\mathrm{d} r}{\mathrm{d} \varphi}=\frac{\operatorname{Re} c_{k}}{\beta_{0}} r^{2 k+1}+O\left(r^{2 k+3}\right)
\end{equation}
在$x_1$轴上建立方程~(\ref{eq:2.3.8})的Poincare映射$P(x_1,\mu)$,并令
\begin{equation}
  \label{eq:2.3.10}
  V\left(x_{1}, \mu\right)=P\left(x_{1}, \mu\right)-x_{1},
\end{equation}
显然,
\begin{equation}
  \label{eq:2.3.11}
V\left(x_{1}, \mu\right)=V(r, \mu),text{当}x_{1} \geqslant 0.
\end{equation}
$V\left(x_{1}, \mu\right)$在$x_{1}>0$的零点个数对应于方程~(\ref{eq:2.3.8})非零周期解的个数.
\par
令函数
$$
R(r, \varphi, \mu)=u_{1}(\varphi, \mu) r+u_{2}(\varphi, \mu) r^{2}+\cdots+u_{2 k+1}(\varphi, \mu) r^{2 k+1}+\cdots
$$
是~(\ref{eq:2.3.8})满足
$
R(r, 0, \mu)=r
$
的解;
而函数$\psi(r, \varphi)$是方程~(\ref{eq:2.3.9})满足$\psi(r, 0)=r$的解,
则有
$$P\left(x_{1}, 0\right)=\psi(r, 2 \pi)=R(r, 2 \pi, 0)$$.
由此可知,当$x_{1} \geqslant 0$时
\begin{align}
  \label{eq:2.3.12}
  \frac{\partial V}{\partial x_{1}}(0,0)=\frac{\partial P}{\partial x_{2}}(0,0)-1=\frac{\partial \psi}{\partial r}(0,2 \pi)-1,\\
  \frac{\partial^{m} V}{\partial x_{1}^{m}}(0,0)=\frac{\partial^{m} P}{\partial x_{1}^{m}}(0,0)=\frac{\partial^{m} \psi}{\partial r^{m}}(0,2 \pi), m>1.
\end{align}
由于$\phi(r, \varphi)$是方程~(\ref{eq:2.3.9})的解,我们有

% \begin{equation*}
% \left.\frac{\partial}{\partial \varphi} \frac{\partial^{m}}{\partial r^{m}} \psi(\kappa, \varphi)\right|_{r=0}
% =
% \left\{
% \begin{aligned}
%   0, &\text{当}1 \leqslant m<2 k+1,\\
%   [(2 k+1) !] \frac{\operatorname{Re} c_{k}}{\beta_{0}},&\text{当}m=2 k+1.
% \end{aligned}
% \right.
% \end{equation*}
?????????????????????????????????
因此
\begin{equation}
\label{eq:2.3.13}
\left.\frac{\partial^{m}}{\partial r^{m}} \psi(r, 2 \pi)\right|_{r=0}
=
\left\{
\begin{aligned}
  &1, &\text{当} m=1,\\
  &0, &\text{当} 1<m<2k+1,\\
  &2 \pi[(2 k+1) !] \frac{\operatorname{Rec}_{k}}{\beta_{0}} \neq 0,
  \text{当} m=2k+1.
\end{aligned}
  \right.
\end{equation}
由~(\ref{eq:2.3.12})和~(\ref{eq:2.3.13})可得
\begin{equation}
\label{eq:2.3.14}
\frac{\partial^{m} V}{\partial x_{1}^{m}}(0,0)
=
\left\{
\begin{aligned}
  &0,&\text{当} 1 \leqslant m<2 k+1\\
  &2 \pi[(2 k+1) !] \frac{\operatorname{Re} c_{k}}{\beta_{0}},
  &当m=2k+1.
\end{aligned}
  \right.
\end{equation}
注意方程~(\ref{eq:2.3.8})与~(\ref{eq:2.3.9})右端的函数当$r=0$时恒为零,
并可以光滑地开拓到$r<0$.
因此$V\left(x_{1}, \mu\right)=P\left(x_{1}, \mu\right)-x_{1}$
在$(x_1,\mu)=(0,0)$的领域内是光滑函数,
由条件~(\ref{eq:2.3.14}),
利用Malgrange定理(第一章定理~(\ref{thm:1.2.12})),在$(x_1,\mu)=(0,0)$附近存在光滑函数
$h\left(x_{1}, \mu\right), h(0,0) \neq 0$,
以及对$x_1$的$2k+1$阶多项式函数$Q(x_1,\mu),$使
$$
V\left(x_{1}, \mu\right)=Q\left(x_{1}, \mu\right) h\left(x_{1}, \mu\right).
$$
另一方面,$V(0, \mu) \equiv 0$,
且$V(x_1,\mu)$对$x_1$的正根和负根成对出现
(这里要用到,当$(x_1,\mu)$在$(0,0)$)附近的一个小领域内,
$V(x_1,\mu)$对$x_1$至多有k个正根,
定理的结论~\ref{item:6}得证.
\par
为了证明结论~\ref{item:7},
我们假设$X_0$以$x=0$为k阶细焦点,
既它具有如下的正规形
$$
\dot{z}=\mathrm{i} \beta_{0} z+c_{k}|z|^{2 k} z+O\left(|z|^{2k+3}\right) \stackrel{d}{\longrightarrow} F(z), \quad \operatorname{Re} c_{k} \neq 0
$$
取它的扰动系统
\begin{equation}
  \label{eq:2.3.15}
\dot{z}=F(z)+\mu_{k-j}|z|^{2(k-j)} z+\cdots+\mu_{k-1}|z|^{2(k-1)} z,
\end{equation}
其中$\mu_{0} \in \mathbf{R}, k-j \leqslant m \leqslant k-1, j$
固定$(1 \leqslant j \leqslant k)$.
化为极坐标方程
\begin{align}
  \dot{r}=\mu_{k-j} r^{2(k-j)+1}+\cdots+\mu_{k-1} r^{2 k-1}+\operatorname{Re} c_{k} r^{2 i+1}+O\left(r^{2 k+3}\right)\\
  \stackrel{\mathrm{d}}{\longrightarrow} G\left(\mu_{k-j}, \cdots, \mu_{k-1} ; \mathrm{r}\right).
\end{align}
为了使系统~(\ref{eq:2.3.15})存在j个闭轨,
我们按下述方式依次选取
$\mu_{k-1}, \cdots, \mu_{k-j}$,
设$\operatorname{Re} c_{k}>0$
(当$\operatorname{Re} c_{k}<0$时,讨论是类似的),
则可选取$0<r_{k}<1$使
$$ G\left(0, \cdots, 0 ; r_{k}\right)>0 .$$
选$\mu_{k-1}<0,\left|\mu_{k-1}\right| \ll \operatorname{Re} c_{k}$,
及$r_{k-1} \in\left(0, r_{k}\right)$,使得
$$
G\left(0, \cdots, 0, \mu_{k-1} ; r_{k}\right)>0, \quad G\left(0, \cdots, 0, \mu_{k-1}, r_{k-1}\right)<0.
$$
类似地,可选$\mu_{k-2}, r_{k-2}, \cdots, \mu_{k-j}, r_{k-j}$,
使得$\operatorname{Re} c_{k}, \mu_{k-1}, \cdots, \mu_{k-j}$具有交替的符号,
并且
$0<\left|\mu_{k-j}\right| \ll \cdots \ll\left|\mu_{k-1}\right| \ll\left|\operatorname{Rec}_{k}\right|, 0<r_{k-j}<\dots<r_{k-1}<r_{k},$
使得
\begin{equation*}
  \begin{aligned}
    {\dot{r}>0} & \text{当}r=r_{k}, r_{k-2}, \cdots \\
    {\dot{r}<0} & \text{当}r=r_{k-1}, r_{k-3}, \cdots
  \end{aligned}
\end{equation*}
由Poincare-Bendixson环域定理,
扰动系统~(\ref{eq:2.3.15})至少存在j个极限环.
\par
这里$r_k$的选取使产生的极限环都在$U^{*}$内,
而$\mu$的选取满足$|\mu|<\sigma^{*}$.
我们断言,
能对任意的$\sigma>0$和$r=0$的领域U,
(\ref{eq:2.3.15})在$U$内多于j个极限环,
则我们可以仿照上面的方法选取
$\mu_{k-j-1},r_{k-j}-1, \cdots, \mu_{1}, r_{1}$
以及
$a,r_{0}$,
从而在$U$内的极限环总数大于$k$,这与结论~\ref{item:6}矛盾.
至此,定理~\ref{thm:2.3.6}证毕.
\end{proof}

\begin{proof}[定理~\ref{thm:2.3.1}的证明]
  当$k=1$时,可以从定理~\ref{thm:2.3.6}得到~\ref{thm:2.3.1}.
  事实上,从定理~\ref{thm:2.3.1}的条件~\ref{item:3}和~\ref{item:5}得知
  $x=0$是方程~(\ref{eq:2.3.2})的一阶细焦点,
  因此定理~\ref{thm:2.3.1}的前一部分结论成立.
  再设条件~\ref{item:4}也成立,
  由~(\ref{eq:2.3.8})-~(\ref{eq:2.3.11})可知,
  后继函数
\begin{equation}
    \label{eq:2.3.16}
V\left(x_{1}, \mu\right)=x_{1} \tilde{V}\left(x_{1}, \mu\right),
\end{equation}
其中
\begin{equation}
  \label{eq:2.3.17}
\tilde{V}\left(x_{1}, \mu\right)=\left[\exp \left(2 \pi \frac{\alpha(\mu)}{\beta(\mu)}\right)-1\right]+u_{2}(2 \pi, \mu) x_{1}+O\left(x_{1}^{2}\right),
\end{equation}
再由~(\ref{eq:2.3.14})可知
\begin{equation}
  \label{eq:2.3.18}
\tilde{V}\left(x_{1}, 0\right)=2 \pi \frac{\operatorname{Re}_{1}}{\beta_{0}} x_{1}^{2}+O\left(x_{1}^{3}\right).
\end{equation}
由~(\ref{eq:2.3.17})和条件~\ref{item:4}可知
\begin{equation}
  \label{eq:2.3.19}
\frac{\partial \tilde{V}}{\partial \mu}(0,0)=\frac{2 \pi}{\beta(0)} \alpha^{\prime}(0) \neq 0.
\end{equation}
利用隐函数定理,
存在$\sigma>0$和$0 \leqslant x_{1} \leqslant \sigma$定义的光滑函数
$\mu=\mu(x_1)$,满足$\mu(0)=0$和
\begin{equation}
\label{eq:2.3.20}  
\tilde{V}\left(x_{1}, \mu\left(x_{1}\right)\right) \equiv 0
\end{equation}
至此,定理~\ref{thm:2.3.1}的结论~\ref{item:4}得证.
为了证明结论~\ref{item:5},从~(\ref{eq:2.3.20})求导得到
\begin{equation}
  \label{eq:2.3.21}
  \left\{
    \frac{\partial \tilde{V}}{\partial x_{1}}+\frac{\tilde{\partial} \tilde{V}}{\partial \mu} \mu^{\prime}\left(x_{1}\right)=0\\
    \frac{\partial^{2} \tilde{V}}{\partial x_{1}^{2}}+2 \frac{\partial^{2} \tilde{V}}{\partial \mu \partial x_{1}} \mu^{\prime}\left(x_{1}\right)+\frac{\partial^{2} \tilde{V}}{\partial \mu^{2}}\left(\mu^{\prime}\left(x_{1}\right)\right)^{2}+\frac{\partial \tilde{V}}{\partial \mu} \mu^{\prime \prime}\left(x_{1}\right)=0
    \right.
  \end{equation}
  利用~(\ref{eq:2.3.18})及条件~\ref{item:3}可得
  $$
\mu^{\prime}(0)=0 \quad \mu^{"}(0)=-2 \frac{\operatorname{Re} c_{1}}{\alpha^{\prime}(0)}
$$
由此得定理的结论~\ref{item:5}.
\end{proof}
从以上证明立得以下结论,
\begin{corollary}
  \label{corollary:2.3.7}
  设条件~\ref{item:1},~\ref{item:2}和~\ref{item:3}成立,
  则存在$\sigma > 0$和$x=0$的领域$U$,使得
  
\begin{enumerate}
\item\label{item:8}
  当$|\mu|<\sigma, \operatorname{Rec}_{1} \alpha^{\prime}(0) \mu<0$时,
  系统~(\ref{eq:2.3.1})在$U$内恰有一个极限环,
  当$\operatorname{Rec_1}\alpha^{\prime}(0)\mu < 0(>0)$时,
  它是稳定(不稳定)的;
  并且当$\mu \to 0$时,它缩向奇点$x=0$;
\item\label{item:9}
当$|\mu|<\sigma, \operatorname{Rec}_{1} \alpha^{\prime}(0) \mu\geqslant 0$时,
  系统~(\ref{eq:2.3.1})在$U$内没有极限环.
\end{enumerate}
\end{corollary}

\begin{remark}
  \label{rem:2.3.8}
  在应用定理~\ref{thm:2.3.6}时,需要首先判断未扰动系统$X_{0}$以$O$为细焦点的阶数,
  也就是确定满足条件~(\ref{eq:2.3.5})的$k$.
  在实际计算时,经常应用下面介绍的 \textbf{Liapunov系数法},
  细节请见 ?????????????????????????????????[ZDHD].
  设$X_{0}$具有如下的形式
  
\begin{ode}
\label{eq:2.3.22}
{\dot{x}=-\beta_{0} y+p(x, y)} ,\\
{\dot{y}=\beta_{0} x+q(x, y)},
\end{ode}
其中 $x,y \in \RR,p, q=O\left(|x, y|^{2}\right), \beta_{0} \neq 0$.
我们利用待定系数法,
寻找$V_j \in \RR ,j =1,2,\dots$和函数
$$ F(x, y)=\frac{\beta_{0}}{2}\left(x^{2}+y^{2}\right)+O\left(|x, y|^{3}\right),$$
使得
\begin{equation}
\label{eq:2.3.23}
\left.\frac{\mathrm{d} F}{\mathrm{d} t}\right|
(\ref{eq:2.3.22})
=\sum_{j=1}^{m} V_{j}\left(x^{2}+y^{2}\right)^{j+1}
\end{equation}
满足上式的$\{V_j\}$称为 (\ref{eq:2.3.16})的Liapunov系数.
在下面定理的意义下,它与Hopf分岔系数$\{\operatorname{Re}(C_{j})\}$是等价的.
\end{remark}

\begin{theorem}
  $V_{1}=\dots-V_{k-1}=0, V_{k}>0(\text{或}<0)$,
  当且仅当
  $\operatorname{Re} c_{1}=\dots=\operatorname{Re} c_{k-1}=0, \operatorname{Re} c_{k}>0(\text{或}<0)$.
\end{theorem}

定理的证明见?????????????????????????????????[BL].
下文中,我们把$\{V_j\}$或$\{\operatorname{Re}(c_j)\}$
称为系统的 \textbf{焦点量}.
\subsection{应用}
\label{sec:2.3.3}

\begin{example}
  \label{exam:2.3.10}
  考虑二维系统
  
\begin{ode}
  \label{eq:2.3.24}
  {\dot{x}&=y} ,\\
  {\dot{y}&=-1+x^{2}+\mu_{1} y+\mu_{2} x y+\mu_{3} x^{3} y+\mu_{4} x^{4} y}.
\end{ode}
此系统有两个奇点$(\pm 1,0)$,而$(1,0)$是鞍点.
% 错别字?
所以只需(?????????????????????????????????)
考虑奇点$(-1,0)$附近发生Hopf分岔的可能性.
令$\xi=x+1$,
系统~(\ref{eq:2.3.24})变为
\begin{ode}
  \label{eq:2.3.25}
  \dot{\xi}= &y,\\
  \dot{y}= &-2 \xi+y\left(\mu_{1}-\mu_{2}-\mu_{3}+\mu_{4}\right)+\xi^{2}+\left(\mu_{2}+3 \mu_{3}-4 \mu_{4}\right) \xi_{y}\\
  & +\left(-3 \mu_{3}+6 \mu_{4}\right) \xi^{2} y+\left(\mu_{3}-4 \mu_{4}\right) \xi^{3} y+\mu_{4} \xi^{4} y.
\end{ode}

这个系统在$(0,0)$的线性部分矩阵有一对纯虚特征根的条件为
\begin{equation}
\label{eq:2.3.26}
\mu_{1}-\mu_{2}-\mu_{3}+\mu_{4}=0.
\end{equation}
令$y=-\sqrt{2} \eta$,则在条件~(\ref{eq:2.3.26})下,方程~(\ref{eq:2.3.25})变成
\begin{ode*}
  \dot{\xi}=-\sqrt{2} \eta, \\
  \dot{\eta}=\sqrt{2} \xi-\frac{1}{\sqrt{2}} \xi^{2}+\left(\mu_{2}+3 \mu_{3}-4 \mu_{4}\right) \xi \eta+\\
  \left(-3 \mu_{3}+6 \mu_{4}\right) \xi^{2} \eta+\left(\mu_{3}-4 \mu_{4}\right) \xi^{3} \eta+\mu_{4} \xi^{4} \eta.
  \end{ode*}

  应用Liapunov系数法,可以得到
  $$
  \begin{array}{l}
    
    {V_{1}=\frac{1}{16}\left(\mu_{2}-3 \mu_{3}+8 \mu_{4}\right)} \\
    {V_{2}=\frac{1}{96 \sqrt{2}}\left(5 \mu_{3}-14 \mu_{4}\right)}\end{array},\text{当}V_1=0,\\
  V_{3}=\frac{14}{5} \mu_{4},\text{当}V_{1}=V_{2}=0.
$$
由此应用定理~\ref{thm:2.3.6},可得下列结论:
\begin{enumerate}
\item\label{item:10}
  若$\mu_{3}=\mu_{4}=0, \mu_{2} \neq 0$,则当$\mu_1=\mu_2$时发生一阶Hopf分岔,
  并且系统~(\ref{eq:2.3.24})在原点附近存在唯一极限环的参数区域是
  $$
\mu_{2}\left(\mu_{1}-\mu_{2}\right)<0,0<\left|\mu_{1}-\mu_{2}\right| \ll\left|\mu_{2}\right| \ll 1
$$
\item\label{item:11}
  若$\mu_4=0,\mu_3\neq 0$,则当$\mu_{1}=4 \mu_{3}, \mu_{2}=3 \mu_{3}$时发生二阶Hopf分岔,~(\ref{eq:2.3.24})在原点附近存在两个极限环的参数区域是
  $$
\begin{array}{c}{\mu_{3}\left(\mu_{2}-3 \mu_{3}\right)<0, \mu_{3}\left(\mu_{1}-\mu_{2}-\mu_{3}\right)>0} \\ {0<\left|\mu_{1}-\mu_{2}-\mu_{0}\right| \ll\left|\mu_{2}-3 \mu_{9}\right| \ll\left|\mu_{3}\right| \ll 1}\end{array}
$$
\item\label{item:12}
  若$\mu_4 \neq 0$,则三阶Hopf分岔发生的条件是
  $$
\mu_{1}=\frac{11}{5} \mu_{4}, \mu_{2}=\frac{2}{5} \mu_{4}, \mu_{3}=\frac{14}{5} \mu_{4}
$$
而系统~(\ref{eq:2.3.24})在原点附近存在三个极限环的参数区域是
$$
\begin{array}{c}{\mu_{4}\left(5 \mu_{3}-14 \mu_{4}\right)<0, \mu_{4}\left(\mu_{2}-3 \mu_{3}+8 \mu_{4}\right)>0} \\ {\mu_{4}\left(\mu_{1}-\mu_{2}-\mu_{3}+\mu_{4}\right)<0}\\
  0<\left|\mu_{1}-\mu_{2}-\mu_{3}+\mu_{4}\right| \ll\left|\mu_{2}-3 \mu_{3}+8 \mu_{4}\right|\\
  \ll\left|5 \mu_{3}-14 \mu_{4}\right| \ll\left|\mu_{4}\right| \ll 1
\end{array}
$$
\end{enumerate}
Bautin对右端是二次多项式的平面微分系统(简称二次系统)的一种标准形式导出了著名的焦点量公式(见[Ba]),
并证明了二次系统细焦点的阶数至多为3
(他的第三个焦点量公式在符号及数值上都有误,在[QL],及[FLLL]中得到纠正).
下面介绍的结果把他的公式推广到一般形式的二次系统上,应用较方便.
\end{example}

\begin{example}[[Lc]]
  \label{example:2.3.11}
  设
\begin{ode}
  \label{eq:2.3.27}
  {\dot{x}=-y+a_{20} x^{2}+a_{11} x y+a_{02} y^{2}} \\
  {\dot{y}=x+b_{20} x^{2}+b_{11} x y+b_{02} y^{2}}
\end{ode}
记
\begin{align*}
  A=a_{20}+a_{02}, B=b_{20}+b_{02}, \alpha=a_{11}+2 b_{02}, \beta=b_{11}+2 a_{20},\\
  \gamma=b_{20} A^{3}-\left(a_{20}-b_{11}\right) A^{2} B+\left(b_{02}-a_{11}\right) A B^{2}-a_{01} B^{3}  ,\\
  \delta=a_{02}^{3}+b_{20}^{2}+a_{02} A+b_{20} B,
\end{align*}
则(不计正数因子)

\end{example}

\begin{align}
  \label{eq:2.3.28}
  {V_{1}=A \alpha-B \beta} \\
  {V_{2}=[\beta(5 A-\beta)+\alpha(5 B-\alpha)] \gamma}\text{如果}V_1=0,\\
  V_{3}=(A \alpha+B \beta) r \delta,\text{如果} V_1=V_2=0,
  V_{k}=0,\text{当} k>3,\text{如果}V_{1}=V_{2}=V_{3}=0.
\end{align}
在最后一种情形下,系统可积,$(0,0)$为中心点.

\begin{remark}
  \label{remark:2.3.12}
  从原则上说,
  当$X_0$以$O$为细焦点时,总可以通过有限步运算确定细焦点的阶数$k$.
  但是当$k$较大时,
  对一般系统想用~(\ref{eq:2.3.28})的方式来表达焦点量公式,
  计算量非常大.
  例如,当把方程~(\ref{eq:2.3.27})右端关于$(x,y)$的齐二次项替换成齐三次项时,
  细焦点的最高阶数为5(见[Si]);
  但如果在~(\ref{eq:2.3.27})右端补充上三次项时,
  则它的焦点量公式非常复杂;
  即使利用计算机,目前也仅推导出前几个焦点量公式.
  确定焦点阶数和求焦点量公式,与区分中心与焦点这个困难问题紧密相关.
  在这里我们仅列举我国学者这方面的一些新工作:
  蔡燧林、马晖[CM]给出了判别广义Lienard方程中心和焦点的较一般的方法;
  黄启昌等[HWW]研究了泛函微分方程的Hopf分岔问题;
  沈家齐、井竹君[SI]给出了判别存在Hopf分岔的一种新方法;
  黄文灶[Hw]证明,当非线性方程零点的拓扑度变号时,
  会产生连通的分岔曲线,等等.
\end{remark}
\begin{remark}
  \label{rem:2.3.13}
  若将条件~\ref{item:3}换成
  
\begin{enumerate}
\item\label{item:13}(($H_{3}$)')
  $  F(x, \mu) \in C^{\omega}\left(U \times(-\sigma, \sigma),
    \RR^{2}\right)  $,
\end{enumerate}
其中$U$是$\RR^{2}$中原点的一个开集,
则当条件~\ref{item:1},~\ref{item:2}和~\ref{item:13}成立时,
或者系统~(\ref{eq:2.3.1})当$\mu=0$时以原点为中心,
或者当$\mu \in (-\sigma,0)$或$(0,\sigma)$时~(\ref{eq:2.3.1})在U内的奇点外围有唯一闭轨,
并且当$\mu \to 0$时,
此闭轨缩向奇点.
这是Hopf分岔定理的另一种形式,
证明可以参考[Zj].
注意,此处$F(x,\mu)\in C^{\omega}$的条件不能减弱为
$F(x,\mu) \in C^{\infty}$,
请看下列.
\end{remark}

\begin{example}
  \label{exam:2.3.14}
  \begin{ode}
    \label{eq:2.3.29}
  \dxdt=x \sin \mu+y \cos \mu+(-x \cos \mu+y \sin \mu) \tan A,
 \dydt=-x \cos \mu+y \sin \mu+(-x \sin \mu-y \cos \mu) \tan A,
\end{ode}
其中$A=\mathrm{e}^{-\frac{1}{r}}\left(2+\sin r^{-\frac{3}{2}}\right), r=\sqrt{x^{2}+y^{2}}, 0<\mu<\frac{\pi}{2}$.
容易验证,对此系统而言,条件
\ref{item:1},~\ref{item:2}满足,
且右端函数是$C^{\infty}$的,
但不解析.
所以~\ref{item:13}不满足.
由于
\begin{equation}
\label{eq:2.3.30}
$\left.\frac{\mathrm{d} r}{\mathrm{d} t}\right|_{~\ref{eq:2.3.29}_{\mu}}=
-r \cos \mu\left[\tan \left(\mathrm{e}^{-\frac{1}{r}}\left(2+\sin r^{-\frac{3}{2}}\right)-\tan \mu\right]\right.$,
\end{equation}
其中$0<r<1$.故原点是$~(\ref{eq:2.3.29})_{0}$的渐近稳定焦点.
\par
考虑$(r,\mu)$平面上由下式定义的曲线
\begin{equation}
\label{eq:2.3.31}
\gamma : \quad \mu=\mathrm{e}^{-\frac{1}{r}}\left(2+\sin r^{-\frac{3}{2}}\right)
\end{equation}
它显然界于曲线$\gamma_{1}: \mu=\mathrm{e}^{-\frac{1}{r}}$
与曲线$\gamma_{2}: \mu=3 \mathrm{e}^{-\frac{1}{r}}$之间(见图2-2).
由于
$$
\frac{\mathrm{d} \mu}{\mathrm{d} r}=r^{-\frac{5}{2}} \mathrm{e}^{-\frac{1}{r}}\left(r^{\frac{1}{2}}\left(2+\sin r^{-\frac{3}{2}}\right)-\frac{3}{2} \cos r^{-\frac{3}{2}}\right)
$$
在$r=0$的任意小领域内都改变其符号,
所以对任意小的$\mu>0,$
存在$r_{1}(\mu) \neq r_{2}(\mu), r_{i}(\mu) \rightarrow 0$
当$\mu \to 0$,
使得$r=r_{i}(\mu)$均满足方程~(\ref{eq:2.3.31})($i=1,2$).
从而由~(\ref{eq:2.3.30})和~(\ref{eq:2.3.31})得到$\frac{\mathrm{d} r_{i}(\mu)}{\mathrm{d} t} \equiv 0$.
这就说明对任意小的$\mu$,系统$~(\ref{eq:2.3.29})_{\mu}$在原点附近都至少有两条闭轨.
因此,上述结论的条件
$F(x, \mu) \in C^{\omega}$
不能减弱为$F(x, \mu) \in C^{\infty}$.
\par
请读者验算,此例中对一切正整数$k$,
都有$\operatorname{Re} c_k=0$时,
奇点不见得是中心.
\end{example}

\subsection{对参数一致的Hopf分岔定理}
\label{sec:2.3.4}

考虑$C^{\infty}$平面系统
\begin{ode}
  \label{eq:2.3.32}
  \dxdt = -\frac{\partial H}{\partial y}+\delta f(x, y, \mu, \delta)  ,\\
  \dydt = \frac{\partial H}{\partial x}+\delta g(x, y, \mu, \delta) ,
\end{ode}
其中函数$H=H(x,y)$,
参数$\delta,\mu \in \RR^1$,
且$\delta$为小参数.
设系统有奇点$x=0$,
而且在该点的线性部分矩阵有特征根$\alpha(\mu,\delta)\pm i \beta(\mu,\delta)$.
若存在$\delta>0$及在$0<\delta<\sigma$定义的函数$\mu=\mu(\delta)$,
满足条件
\begin{enumerate}
\item[(H1*)]\label{item:14}
  $$\alpha(\mu(\delta),\delta)=0,\beta(\mu(\delta),\delta) \neq 0$$,
\end{enumerate}
则在一定的附加条件下,当
$\mu=\mu(\delta),\delta \in (0,\sigma)$时,
系统~(\ref{eq:2.3.32})
可在$x=0$点发生Hopf分岔.
对每一个固定的$\delta \in (0,\sigma)$,
利用推论~\ref{corollary:2.3.7}可知,
存在$\epsilon(\delta)>0$,
使得当$|\mu-\mu(\delta)|<\epsilon(\delta)$
且$\mu>\mu(\delta)$(或$\mu<\mu(\delta)$)时,系统无极限环.
问题是:当$\delta \to 0$时,可能有$\epsilon(\delta)>0$.
我们希望找到系统满足的条件,以保证存在正数$\delta_0$和$\epsilon_0 \to 0$.
我们希望找到系统满足的条件,
以保证存在正数$\delta_0$和$\epsilon_0$,
使得对所有的$\delta\in (0,\delta_{0})$,都有$\epsilon(\delta)\geqslant \epsilon_0$.
这就是所谓的对参数一致的Hopf分岔问题,见图2-3???.

代替定理~\ref{thm:2.3.1}中条件~\ref{item:2}和~\ref{item:3},下文需要的条件是
\begin{itemize}
\item ($H_2^*$) $
\alpha^{*}=\lim _{\delta \rightarrow 0} \frac{1}{\delta} \frac{\partial \alpha(\mu(\delta), \delta)}{\partial \mu} \neq 0,
$
\item ($H_3^*$)
  $c_{1}^{*}=\lim _{\delta \rightarrow 0} \frac{1}{\delta} \operatorname{Re}\left[c_{1}(\mu(\delta), \delta)\right] \neq 0.$
\end{itemize}

\begin{theorem}
  
\end{theorem}
\message{ !name(dierzhang.tex) !offset(-1425) }
