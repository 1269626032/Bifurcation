\section{奇点分岔}
考虑一个光滑的依赖于参数并且有奇点的向量场。
当参数变动时,我们关心奇点个数及其附近的轨道结构如何变化。
这种分岔现象称为\textbf{奇点分岔}.
\subsection{一般理论}
\begin{defination}\label{def2.1.1}
  向量场\(X\) 的奇点 \( p \in M \) 称为\textbf{非退化}的,
  如果他在\(p\) 点的线性部分算子是非奇异的,
  即它的所有特征根均非零.
  否则称为退化的。
  \label{def2.1.1}
\end{defination}

\begin{theorem}\label{thm2.1.2}
  光滑的依赖于变量和参数的向量场,
  如果它的奇点是非退化的,
  则奇点本身也光滑地依赖于参数。
  \label{thm2.1.2}
\end{theorem}

\begin{proof}
  设向量场由微分方程
  \begin{equation}
    \frac{\mathrm{d} x}{\mathrm{d} t}=v(x, \mu)
    \label{eq1.1}
  \end{equation}
  给出
  其中
  \(v \in C^{r}\left(\mathbf{R}^{n} \times \mathbf{R}^{k}, \mathbf{R}^{n}\right), r \geqslant 1, k \geqslant 1  \).
   设当\(\mu = \mu_{0}\)时,
  \(x=x_{0}\)为 \ref{eq1.1}
  的非退化奇点,
  既
  \(
  v\left(x_{0}, \mu_{0}\right)=0,\left.\frac{\partial v(x, \mu)}{\partial x}\right|_{\left(x_{0}, \mu_{0}\right)}
  \)
  非奇异。
  由隐函数定理,
  在\(x_{0},\mu_{0}\) 附近存在光滑函数
  \(x= \Gamma (\mu)\),
  使得 \(\Gamma ( \mu_{0}) = x_{0}\),
  且
  \(
v(\gamma(\mu), \mu) \equiv 0
  \).
  定理得证。
\end{proof}

\begin{remark}
  \label{rmk2.1.3}
  定理 \ref{thm2.1.2}说明,
  当奇点非退化时,
  奇点的个数在微小变化下不变,
  它的位置也光滑的依赖于参数的变化。
  需要注意,奇点的非退化性与双曲性是不同的概念。
  例如,一个向量场在奇点处的线性部分是一对纯虚根时,
  按照定义\ref{def2.1.1},它是非退化的,但它是双曲的。
  此时在扰动下,虽然奇点个数(在小领域内)不发生变化,
  但其附近的轨道结构可能变化,
  出现Hopf分岔,或Poincare分岔,
  我们将在\ref{sec2.3}与\ref{sec2.3}中分别予以讨论。
\end{remark}

\begin{theorem}\label{thm2.1.4}
  设M是n维紧致流形,
  \(r \geqslant 1\),
  则
  \(\mathscr{X}^{r}(M)\)中仅有非退化奇点(它们必是孤立奇点)或无奇点的向量场集合形成一个开稠子集.
\end{theorem}
\begin{proof}
  设$X \in \mathscr{X}^{r}(M)$相应于微分方程
  \begin{equation}
\frac{\mathrm{d} x}{\mathrm{d} t}=f(x), \quad x \in U(\xi),
\end{equation}
其中\(f \in C^{r}(M, M), U(\xi)\)为\(M\)中\(\xi\) 点的领域.
如同第一章\ref{1.5}的讨论,考虑投影
\begin{equation*}
j^{\circ} f_{2} \quad \mathscr{X}^{r}(M) \rightarrow J^{\circ}(M, M) : \quad X \mapsto(\xi, \tilde{f}),
\end{equation*}
其中
\(\tilde{f}=f(\xi)\).
具有奇点的向量场集合在空间\(J^{0}(M, M)\) 中有表示式
\begin{equation*}
S=\{(\xi, \tilde{f}) | \tilde{f}=0\},
\end{equation*}
它是\(J^{0}(M, M)\) 中的光滑闭子流形(因为M是紧空间).
设向量场\(f(x)\) 在\(\xi\) 点非退化,
既\(\left.\frac{\partial f}{\partial x}\right|_{\xi}\) 非奇异,
从而由附录\ref{fuluC}中定理\ref{C.13}知,
\(j {\circ} f\) 与子流形S横截相交。
注意,不想交也是横截.
再利用定理\ref{C.15},
得知仅有非退化奇点或无奇点的向量场在\(\mathscr{X}^{r}(M)\) 中形成开稠子集。
\end{proof}
\par
这个定理说明,向量场的一个退化奇点可以经过任意小的扰动转化为(多个)非退化奇点,或经扰动使奇点消失.
但如果我们考虑向量场族\(v(x,\mu)\),
则奇点的退化性往往是不可避免的.
事实上,
虽然小扰动可以把对应于\(\mu=\mu_{1}\) 的退化奇点\(x=x_{1}\) 扰动为非退化的,
但在\(x_{1}\) 附近的\(x_{2}\) 点,
相应于\(\mu_{1}\) 附近的\(\mu_{2}\) 却可能是新的退化奇点.
对一个具体的奇点分岔问题,
通常有两种处理方法:
一种是利用中心流形定理,
把问题归结在中心流形上,
见第一章例\ref{exam1.3.12};
另一种称为Liapunov-Schmidt方法,
或称为更替法(alternative method).
为了说明这个方法的基本思想,
我们先看一种特殊情形.
设变量\(x=(y,z)\),
在\(x=0\)附近微分方程具有下列形式
\begin{equation}
  \label{eq2.1.2}
  \frac{\mathrm{d} y}{\mathrm{d} t}=\boldsymbol{A} y+f(y, z, \lambda),
  \quad
  \frac{\mathrm{d} z}{\mathrm{d} t}=\boldsymbol{B} z+g(y, x, \lambda),
  \end{equation}
  其中\(\boldsymbol{A}\)的特征根均为零,
  而\(\boldsymbol{B}\)的特征根均不为零;
  \(f ; g \in C^{r}, r \geqslant 2\);
  \( f(0,0,0)=0, g(0,0,0)=0, f, g=O\left(|y, z|^{2}\right)\).
为了研究奇点的分布,
在\(x=0\)和\(\lambda=0\)附近考虑方程
\begin{equation}
  \label{eq2.1.3}
  \mathbf{A}{y}+f(y, z, \lambda)=0,
  \quad
  \mathbf{B}z+g(y, z, \lambda)=0
\end{equation}
由隐函数定理,
存在\((y, \lambda)=(0,0)\)
的领域U和\(C^{r}\)函数
\(z=\varphi(y,\lambda)\),使得
\begin{equation*}
  B \varphi(y, \lambda)+g(y, \varphi(y, \lambda), \lambda) \equiv 0,
  \quad
  \forall(y, \lambda) \in U .
\end{equation*}
把函数\(z=\varphi(y,\lambda)\)代入
\ref{eq2.1.3}的第一个方程左端,可得\(C^{r}\)函数
\begin{equation}
  \label{eq2.1.4}
  G(y, \lambda) \stackrel{\mathrm{d}}{\longrightarrow} A y+f(y, \varphi(y, \lambda), \lambda)
\end{equation}
记
\begin{equation*}
  S=\{(y, \lambda) \in U | G(y, \lambda)=0\rangle,
  \quad
  S_{\lambda_{0}}=S  \cap\left\{\lambda=\lambda_{0}\right\}
\end{equation*}
则对不同的\(\lambda,|\lambda| \ll 1, S_{\lambda}\)结构的变化反应了奇点个数的变化规律.
这样就把对\ref{eq2.1.3}的讨论转化为对
\( G(y, \lambda)=0  \)的讨论,
使空间维数得到降低,
通常称\eqref{eq2.1.4}为方程\eqref{eq2.1.2}的\textbf{分岔函数}.
为了应用上的便利,下面在更一般的框架下讨论这个问题.
\subsection{Liapunov-Schmidt方法}
设\(X,Z\)和 \(\Lambda\)为实 \(Banach\)空间,
U和W分别为X和\(\Lambda\)中零点的领域.
\(C^{1}\)映射
\(M : U \times W \subset X \times \Lambda \rightarrow Z\),
满足
\(M(0,0)=0\).
我们要研究方程
\begin{equation}
  \label{eq2.1.5}
  M(x ,\lambda)=0
\end{equation}
在\(U \times W\)中 \((0,0)\)点的某领域内解的结构.
为此,
设\( A=D_{x} M(0,0)\),
并记
\(\mathscr{N}(A) \text {和} \mathscr{R}(A)\)
分别为A在X中的零空间和A在Z中的值域空间.
本节的一个基本假设是
\par
(H)\(\mathscr{N}(A)\)在X中存在补空间;
\(\mathscr{R}(A)\)是Z中的闭集,
并且在Z中存在补空间.
(当A为Fredholm算子时,这个假设总是成立的.在下文的应用中,经常是这种情形.)
\par
因此,在X上存在投影P,在P上存在投影Q,使得
\begin{equation}
  \label{eq2.1.6}
  \mathscr{R}(P)=\mathscr{N}(A), \quad \mathscr{R}(Q)=\mathscr{R}(A).
\end{equation}
\(\forall x \in U\),
可写成
\(
x=u+v, \text{其中}  u=P x \in \mathscr{N}(A)=X_{P},
v=(I-P) x \in \mathscr{N}(P)=X_{I-P}.
\)
这里I是恒同映射,
\(X_{p}\)和\(X_{!-P}\)表示投影
P和I-P的值域.
显然,方程\eqref{eq2.1.5}等价于
\begin{align}
  \label{eq2.1.7}
  {Q M(u+v, \lambda)=0}  \\
  {(I-Q) M(u+v, \lambda)=0}
\end{align}
定义映射
\(\psi : X_{P} \times X_{I-P} \times \Lambda \rightarrow \mathscr{R}(A)\),
\begin{equation*}
\psi(u, v, \lambda)=Q M(u+v, \lambda)
\end{equation*}
则\(\phi(0,0,0)=0, \text{且} \mathrm{D}_{v} \psi(0,0,0)=\left.A\right|_{\mathcal{N}(P)}\)是\(\mathscr{N}(P)\)与\(\mathscr{R}(A)\)间的同构.
由隐函数定理,
存在
\(X_{p}\)在原点的领域\( U_{0}\),
\( X_{I-P}\)在原点的领域\(V_{0}\),
\(\Lambda\)在原点的领域\(W_{0}\),
以及\(C^{1}\)映射\(v^{*}: U_{0} \times W_{0} \rightarrow V_{0}\),
使
\(U_{0} \times W_{0} \subset U, W_{0} \subset W\),且
\begin{equation*}
Q M\left(u+v^{*}(u, \lambda), \lambda\right) \equiv 0, \quad \forall(u, \lambda) \in U_{0} \times W_{0},
\end{equation*}
并且
\(  v^{*} .(0,0)=0, D_{u} v^{*}(0,0)=0\).
利用\(v^{*}\),定义
\(C^{1}\)映射\(x^{*}:U_{0}\times W_{0}\rightarrow U\),
和\(C^{1}\)映射\(G : U_{0} \times W_{0} \rightarrow \mathscr{N}(Q)\),
\begin{equation}
x^{*}(u, \lambda)=u+v^{*}(u, \lambda)
\end{equation}
\begin{equation}
G(u, \lambda)=(I-Q) M\left(u+v^{*}(u, \lambda), \lambda\right)
\end{equation}

容易验证,\(x^{*}(0,0)=0, D_{u} x^{*}(0,0)=I_{M(A)},G(0,0)=0,\mathrm{D}_{u} G(0,0)=0\).
总结上面的讨论,我们有下面的结果.
\begin{theorem}
  \label{thm2.1.5}
  如果条件(H)成立,\(U_{0}, V_{0}, W_{0}\)如上.
  则
  \(\forall u \in U_{0}, x \in U_{0} \times V_{0} \subset X, \text{和}
  \lambda \in W_{0} \subset \Lambda\),
  如下两组结论等价
\begin{itemize}
\item \(P x=u, \quad M(x, \lambda)=0\);
\item \(x=x^{*}(u, \lambda), \quad G(u, \lambda)=0\);
\end{itemize}
其中\(x^{*}\)与G分别有\eqref{2.1.8}和eqref{2.1.9}定义.
\end{theorem}

定理\ref{thm2.1.5}说明,
原来的奇点分岔问题
\(M(x, \lambda)=0\)转化为求解
\textbf{分岔方程}\(G(u,\lambda)=0\).
注意\(x \in X, M(x, \lambda) \in Z\),
而
\(u \in X_{P}=\mathscr{N}(A), G(u, \lambda) \in Z_{I-\mathrm{Q}}=\mathscr{N}(Q)\),
因而使问题的定义域及值域都作了显著的约化.
这就是Liapinov-Schmidt方法的核心思想.
\par
现在我们把上面的一般理论用于$\R^{n}$
上的向量场奇点分岔问题.
考虑依赖于参数\(\lambda\)的向量场
\begin{equation}
  \label{eq2.1.10}
  \frac{\mathrm{d} x}{\mathrm{d} t}=f(x, \lambda),
\end{equation}
其中
\(
f \in C^{r}\left(\mathbf{R}^{n} \times \mathbf{R}^{k}, \mathbf{R}^{n}\right), r \geqslant 2 ; f(0,0)=0, \mathrm{D}_{x} f(0,0)=\boldsymbol{A}.
\)
考虑奇点分岔问题,就是要在
\(\mathbf{R}^{n} \times \mathbf{R}^{k}\)的原点附件考察方程
\begin{equation}
  \label{eq2.1.11}
  f(x, \lambda)=\boldsymbol{A} x-N(x, \lambda)=0
\end{equation}
其中
\(
N \in C^{r}\left(\mathbf{R}^{n} \times \mathbf{R}^{k}, \mathbf{R}^{n}\right), N(0,0)=0, \mathrm{D}_{x} N(0,0)=0.
\)
与前面的一般情况对比,
此时有
\(X=Z=\mathbf{R}^{n}, \Lambda=\mathbf{R}^{k}\).
假设又有
\(\operatorname{dim} \mathscr{r}(\boldsymbol{A})=\operatorname{codim} \mathscr{R}(\boldsymbol{A})=1\),
则存在投影
\(P, Q : \mathbf{R}^{n} \rightarrow \mathbf{R}^{n}\),
满足\eqref{eq2.1.6}.
从而存在
\(u_{0} \in \mathscr{N}(A), w_{0} \in \mathscr{N}(Q)\),使
\begin{equation*}
  \mathscr{N}(\boldsymbol{A})=\operatorname{Span}\left\{\boldsymbol{u}_{0}\right\},
  \quad
  \mathscr{N}(\boldsymbol{Q})=\operatorname{Span}\left\{w_{0}\right\rangle.
\end{equation*}
从上面的一般理论知道,
存在\(\delta>0, \sigma>0\)和\(G^{r}\)函数
\(v=v^{*}(a, \lambda) \in X_{I-P}\),
满足\(v^{*}(0,0)=0, D_{a} v^{*}(0,0)=0\),
使当\(|a|<\delta,|\lambda|<\sigma\)时,
\begin{equation*}
\mathbf{Q}f\left(a u_{0}+v^{*}(a, \lambda), \lambda\right) \equiv 0.
\end{equation*}
由定理\ref{thm2.1.5},
\(
x=a u_{0}+v, v \in X_{I-P}
\)
是\eqref{eq2.1.11}的解,
当且仅当
\(v=v^{*}(a, \lambda) \)且
\((a,\lambda)\)满足\textbf{分岔方程}
\begin{equation*}
g(a, \lambda)=0,
\end{equation*}
这里\textbf{分岔函数}g由下式定义:
\begin{equation}
  \label{eq2.1.12}
  g(a, \lambda) w_{0}=(I-Q) f\left(a u_{0}+v^{*}(a, \lambda), \lambda\right)
\end{equation}

\begin{example}
\label{exam2.1.6}  
用Liapunov-Schmidt方法重新考虑第一章例\ref{exam1.3.12}.
我们考虑\(\R ^{2}\)中一类更广泛的微分方程
\begin{equation}
  \label{exam2.1.13}
  \left\{
    \begin{array}{l}
      {\frac{\mathrm{d} x}{\mathrm{d} t}=y}, \\
      {\frac{\mathrm{d} y}{\mathrm{d} t}=\beta y+x^{2}+x y(1+\varphi(x))+y^{2} \Phi(x, y)}
    \end{array}
  \right.
\end{equation}
其中\(\beta \neq 0, \varphi, \Phi \in C^{\infty}, \varphi(0)=0\).
考虑它的奇点\((0,0)\)在扰动下的分岔问题.
此时线性部分矩阵为
\(
A=\left(
  \begin{array}{ll}
    {0} & {1} \\
    {0} & {\beta}
  \end{array}
\right),
\operatorname{dim} \mathscr{N}(\boldsymbol{A})=\operatorname{codim} \mathscr{R}(\boldsymbol{A})=1.
\)
取满足\eqref{eq2.1.6}的投影
\(P, Q \cdot \mathbf{R}^{2} \rightarrow \mathbf{R}^{2}\).
令
\begin{equation*}
u_{0}=\left( \begin{array}{l}{1} \\ {0}\end{array}\right), v_{0}=\left( \begin{array}{l}{0} \\ {1}\end{array}\right), w_{0}=\left( \begin{array}{l}{0} \\ {1}\end{array}\right), s_{0}=\left( \begin{array}{l}{1} \\ {\beta}\end{array}\right)
\end{equation*}
则
\begin{equation*}
\mathscr{N}(\boldsymbol{A})=X_{P}=\operatorname{Span}\left\{u_{0}\right\}, X_{I-P}=\operatorname{Span}\left\{\boldsymbol{v}_{0}\right\}
\end{equation*}
\begin{equation*}
  \mathscr{R}(\boldsymbol{A})=\mathscr{R}(Q)=\operatorname{Span}\left\{s_{0}\right\}.
\end{equation*}
取\(\mathscr{N}(Q)=\operatorname{Span}\left\{w_{0}\right\}\).
函数\(v=v^{*}(a)=\left( \begin{array}{l}{0} \\ {v_{2}}\end{array}\right) \in X_{I-P}\)由方程
\(Q f\left(a u_{0}+v^{*}(a)\right)=0\)确定,既
\begin{equation*}
0=Q\left\{v_{2} s_{0}+\left[a^{2}+a v_{2}(1+\varphi(\alpha))+y^{2} \Phi\left(\alpha, v_{2}\right)\right] w_{0}\right\}=v_{2} s_{0}.
\end{equation*}
因此
\(v=v^{*}(a) \equiv 0\).
把它代入\eqref{eq2.1.12},得到
\begin{equation*}
\begin{array}{c}{g(a) w_{0}=(I-Q) f\left(a u_{0}+v^{*}(a)\right)} \\ {=(I-Q) f \left( \begin{array}{l}{a} \\ {0}\end{array}\right)=(I-Q) \left( \begin{array}{l}{0} \\ {a^{2}}\end{array}\right)=a^{2} w_{0}}\end{array}
\end{equation*}
从而分岔函数\(g(a)=a^{2}\).
\end{example}
\par
如果我们考虑方程\eqref{eq2.1.13}的\(C^{2}\)扰动,
扰动参数为\(\lambda\),
则扰动后方程的分岔函数\(g(a,\lambda)\)
满足\(g(a,0)=a^{2}\).
利用隐函数定理易知,
存在\(\delta>0\)和\(C^{0}\)函数
\(a=a(\lambda)\),
使得\(a(0)=0, D_{a} g(a(\lambda), \lambda) \equiv 0,
\mathrm{D}_{a}^{2} g(a(\lambda), \lambda) \neq 0, \forall|\lambda|<\delta\).
利用Taylor公式可得,
当\(|\lambda| \ll 1,|a-a(\lambda)| \ll 1 \),有
\begin{equation*}
g(a, \lambda)=\mu(\lambda)+D_{a}^{2} g(a(\lambda), \lambda)(a-a(\lambda))^{2}+o\left(|a-a(\lambda)|^{2}\right)
\end{equation*}
其中
\(\mu(\lambda)=g(a(\lambda), \lambda)\).
因此,
在\((a,\lambda)=(0,0)\)附近,
方程\(g(a,\lambda)=0\)当\(\mu(\lambda) \mathrm{D}_{a}^{2}(0,0)<0\)时有两个零点.
应用定理\ref{thm2.1.5}可知,
原系统\eqref{2.1.13}在扰动下发生鞍结点分岔.
注意,当扰动方程为\(C^{\infty}\)时,
最后的讨论可从第一章定理\ref{thm2.2.12}直接得到.
\begin{note}
  本节中讨论的奇点分岔问题,主要着重于奇点个数随参数变动而发生变化的规律.
  实际上,在奇点个数发生变化的同时(甚至在奇点个数不变时,见附注\ref{ruzhu1.3}),
  轨道结构还可能发生其他变化.
  例如闭轨、同宿轨、异宿轨等的产生或消失.
  这些情形在下章中将会看到.
\end{note}