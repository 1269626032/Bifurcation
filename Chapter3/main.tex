\section{几类余维2的平面向量场分岔}
\label{part:03}
在本章中,我们将综合运用第二章所介绍的几种典型的向量场分岔的理论与方法,讨论向量场在非双曲奇点附近所发生的几种余维2分岔.
\par
考虑以$\mu \in \RR^m(m\geq 2)$为参数的向量场族
\begin{equation*}
\left(X_{k}\right): \qquad \dxdt =f(x, \mu),
\end{equation*}
其中$x\in\RR^n,f\inC^{\infty}(\RR^n\times \RR^m,\RR^n)$.
无妨设$x=0$是向量场$X_{0}$的非双曲奇点,
并设$X_{0}$在该点的线性部分矩阵具有二重退化性.
因此,$f(0,0)=0$,
且把$X_0$化到$x=0$附近的中心流形后,其线性部分矩阵可以化为下列形式之一:\\
$$A_{1}=\left(\begin{array}{ll}{0} & {1} \\ {0} & {0}\end{array}\right), \quad \quad A_{2}=\left(\begin{array}{ll}{0} & {0} \\ {0} & {0}\end{array}\right)$$
$$A_{3}=\left(\begin{array}{ccc}{0} & {1} & {0} \\ {- 1} & {0} & {0} \\ {0} & {0} & {0}\end{array}\right), A_{4}=\left(\begin{array}{cccc}{0} & {\omega_{1}} & {0} & {0} \\ {-\omega_{1}} & {0} & {0} & {0} \\ {0} & {0} & {0} & {w_{2}} \\ {0} & {0} & {-\omega_{2}} & {0}\end{array}\right)$$
其中$\omega_{1} \omega_{2} \neq 0, \omega_{i} \neq k \omega_{j}, i, j=1,2, k=1, \cdots, 5$.
若平面向量场在奇点处的线性部分矩阵有二重零特征根,
并且向量场在旋转$\frac{2\pi}{q}$角度时不变,
则称它有$q$阶对称性,或称为$1:q$共振.
$q=1,2$时,线性部分矩阵具有$A_1$的形式;
$q\geq3$时,线性部分矩阵具有$A_2$的形式.
$q\leq4$陈为强共振,
$q\geq 5$称为弱共振.
除了$q=4$之外,其它情况的余维2分岔的问题都已解决.
本章节\ref{sec:0301}---\ref{sec:0303}分别介绍$q=1,2$,
及其$q\geq 5$的情形.
在高阶项的适当非退化条件下,
扰动系统$X_{\mu}$具有双参数的普世开折.
主要参考文献为:$q=1$的余维2分岔见[Bol,2]和[T],
余维3,4的讨论分别见[DRSI,2]和[LR1];
$q=2$和$q=3$的余维2分岔分别见[Ho],
$q=2$的余维3,4的讨论分别见[LR2]和[Rc];
$q\geq 5$的余维2分岔见[T],
对$A_3$,$A_4$情形的讨论见[Zol,2].
在专著[CLW]和[HZ]中有对所有情形的详细介绍.
\subsection{二重零特征根:Bogdanov-Takens系统}
\label{sec:0301}
在第一章\ref{sec:0105}中,
我们讨论过二维$C^{\infty}$向量场
\begin{equation}
  \label{eq:3.1.1}
  \frac{\mathrm{d}}{\mathrm{d} t}\left(\begin{array}{l}{x} \\ {y}\end{array}\right)=\left(\begin{array}{ll}{0} & {1} \\ {0} & {0}\end{array}\right)\left(\begin{array}{l}{x} \\ {y}\end{array}\right)+O\left(|x, y|^{2}\right)
\end{equation}
它具有正规形(见第一章例\ref{exam:1.4.7})
\begin{equation}
  \label{eq:3.1.2}
  \frac{\mathrm{d}}{\mathrm{d} t}\left(\begin{array}{l}{x} \\ {y}\end{array}\right)=\left(\begin{array}{ll}{0} & {1} \\ {0} & {0}\end{array}\right)\left(\begin{array}{l}{x} \\ {y}\end{array}\right)+\left(\begin{array}{c}{0} \\ {a x^{2}+b x y}\end{array}\right)+O\left(|x, y|^{3}\right).
\end{equation}
当$ab\neq 0$时,
由第一章定理\ref{1.5.13},
系统~(\ref{eq:3.1.2})的任意非退化开折可转化为
\begin{ode}
  \label{eq:3.1.3}
&  \dxdt=y, \\
&  \dydt=\mu_{1}+\mu_{2} y+x^{2}+x y Q(x, \mu)+y^{2} \Phi(x, y, \mu)},
\end{ode}
其中
$Q, \Phi \in C^{\infty},
Q(0,0)=\pm 1=\operatorname{sgn}(a b), \mu \in \RR^{m}, m \geqslant 2$.
为确定起见,
取$Q(0,0)=1 . Q(0,0)=-1$的情况可类似讨论.
\subsection{分岔图,轨线的拓扑分类}
下面的定理是本节的第一个主要结果.
\begin{theorem}
  \label{thm:3.1.1}
  存在$\RR^2$中$\left(\mu_{1}, \mu_{2}\right)=(0,0)$的领域$\Delta$,
  使系统~(\ref{eq:3.1.3})在$\Delta$中的分岔图有原点$\left(\mu_{1}, \mu_{2}\right)=(0,0)$以及下列曲线组成:
  \begin{description}
  \item[($a$)] $\mathrm{SN}^{ \pm}=\left\{\mu | \mu_{1}=0, \mu_{2}>0 \text {或} \mu_{2}<0\right\}$;
\item[($b$)] $\mathbf{H}=\left\{\mu | \mu_{1}=-\mu_{2}^{2}+O\left(\mu_{2}^{\frac{5}{2}}\right), \mu_{2}>0\right\}$;
\item[($c$)] $\mathrm{HL}=\left\{\left.\mu\right|_{\mu_1}=-\frac{49}{25} \mu_{2}^{2}+O\left(\mu_{2}^{\frac{5}{2}}\right), \mu_{2}>0\right\}$;
\end{description}
其中$\mathrm{SN}^{ \pm}, \mathrm{H}, \mathrm{HL}$分别为鞍结点分岔曲线,
Hopf分岔曲线和同宿分岔曲线.
当$\left(\mu_{1}, \mu_{2}\right) \in \Delta$时,
系统~(\ref{eq:3.1.3})在相空间原点$(x,y)=(0,0)$附近的轨道拓扑结构见图\ref{pic:3.3.1}
\end{theorem}
\par
为了证明定理~\ref{thm:3.1.1},注意$\mu_{1}>0$时,
(\ref{eq:3.1.3})在原点附近无奇点;
而当$\mu_{1}=0, \mu_{2} \neq 0$时,
由第二章例~\ref{exam:2.1.6}知发生鞍结点分岔.
因此,下面要考虑的只是$\mu_{1}<0$的情形.
考虑参数及变量的替换:
\begin{equation}
  \label{eq:3.1.4}
  \mu_{1}=-\sigma^{4}, \mu_{2}=\zeta \delta^{2}, x=\delta^{2} \bar{x}, y=\delta^{3}\bar{y}, t=\frac{\overline{t}}{\delta},
\end{equation}
其中$\delta>0$.
再把$(\bar{x},\bar{y},\bar{t})$写回$(x,y,z)$,则~(\ref{eq:3.1.3})化为

\begin{ode}
  \label{eq:3.1.5}
&  \dxdt=y, \\
&  \dydt=-1+x^{2}+\delta[(x+\zeta)+\delta \Psi(x, y, \delta, \zeta)] y.
\end{ode}
我们可以把~(\ref{eq:3.1.5})$_{\delta}$看成是~(\ref{eq:3.1.5})$_{\delta}$的扰动系统,后者为Hamilton系统,
它有鞍点$A(1,0)$的同宿轨道,
以及该同宿轨所围的以点$B(-1,0)$为中心的周期环域(见第二章例~(\ref{exam:2.5.6})及图\ref{pic:3.2.6}),
周期环域中的闭轨族可表示为
\begin{equation}
  \label{eq:3.1.6}
  \Gamma_h: \quad\left\{(x, y) | H(x, y)=h,-\frac{2}{3}<h<\frac{2}{3}\right\},
\end{equation}
其中

\begin{equation}
  \label{eq:3.1.7}
  H(x, y)=\frac{y^{2}}{2}+x-\frac{x^{3}}{3}.
\end{equation}
当$h \rightarrow-\frac{2}{3}+0$时,
$\Gamma_h$缩向奇点B;
当$h \rightarrow \frac{2}{3}-0$时,
$\Gamma_h$趋于同宿轨与鞍点$A$形成的同宿环$\Gamma_{\frac{2}{3}}$.
\par
注意对任意的$\delta$,~(\ref{eq:3.1.5})$_{\delta}$都以$A,B$为奇点,
且$A$为鞍点,$B$为指标$+1$的奇点.
因此,若~(\ref{eq:3.1.5})$_{\delta}$存在闭轨,
它必定与线段$L=\{(x, y) | y=0,-1<x<1\}$相交.
另一方面,
由于$\Gamma_h$与$L$的交点$p_h$在$L$上关于$h$单调排列.
因此,可用$h$把$L$参数化:
$L: \quad\left\{p_{h} | p_{h}=L \cap \Gamma_{h},-\frac{2}{3}<h<\frac{2}{3}\right\}$.
\par
现在我们任取$p_h\in L$,
考虑系统~(\ref{eq:3.1.5})$_{\delta}$过$p_{h}$的轨线.
设它的正向及负向延续分别与$x$轴(第一次)交于点$Q_3$与$Q_1$.
记$\gamma_{(h, \delta, \zeta)}$为~(\ref{eq:3.1.5})$_{\delta}$从$Q_1$到$Q_2$的轨线段(见图~\ref{pic:3.2}).
\begin{collory}
  \label{col:3.1.2}
  当$\delta>0$时,$\gamma(h, \delta, \zeta)$是系统~(\ref{eq:3.1.5})的闭轨,当且仅当
  \begin{equation}
    \label{eq:3.1.8}
    F(h, \delta, \zeta) \equiv \int_{\gamma(h, \delta, \zeta)}[(\zeta+x)+\delta \Psi(x, y, \delta, \zeta)] y \diff x=0.
  \end{equation}
\end{collory}
\begin{proof}
  注意当$|x|<1$时,
  $\frac{\partial H(x,y)}{\partial x}=1-x^{2} \neq 0$.
  因此,
  $\gamma(h,\delta,\zeta)$为闭轨$\Leftrightarrow Q_{1}=Q_{2} \Leftrightarrow H\left(Q_{1}\right)=H\left(Q_{2}\right)$.
  另一方面,由方程~(\ref{eq:3.1.5})$_{\delta}$可得
\end{proof}