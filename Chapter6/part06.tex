\chapter{实二次单峰映射族的吸引子}
\label{part06}
% 第六章
从60年代兴起的动力系统的现代研究,其中心课题之一是双曲理论.
一个系统(流,微分同胚,映射)是双曲的,如果它的极限
彙是双曲的,即极限集中所有轨道的Liapunov指数一致非0(在讨
论流的双曲性时,不考虑沿流方向的Liapunov指数,它为0),从那
时起,以Peixoto,Smale和我国的宴山涛教授等人为代表对双曲系
统做了大量深人的研究,人们对它已逐斯有了比较完整的理解.
另外,当时人们曾相信动力系统基本上是由双曲系统构成的.这一
看法的根本转变是由于在70年代受到物理、天文学等领域的一些
重要动力系统模型的影响.Feigenbaum, Ruelle, H\&on和
L(gnz(他们的工作在70年代开始才受到数学家的重视)等人对
这些模型的大量计算工作表明,这些模型具有极其复杂的动力学
行为,它们似乎不具有双曲结构,相反它们应当属于非双曲翦畴.

对非双曲系统的研究,理抡方面的第F次实质勇破是
JakobsonE在80年代初取得的,他证明了对实二次映射族的一个
正Lebesgue测度的参数集合,相应系统具有关于Lebesgue测度绝
对连续的不变概率测度.这表明,在测度意义下,具有复杂的动力
学行为的非双曲系统并不太少.Benedicks和Carleson\^{}\textsuperscript{1}-\textsuperscript{1}\^{}完善了
Jakobson的结果和方法,并在此基础上,讨论了 H\&ion映射在具有
非退化同宿相切的参数值附近的动力学性质.继而Mora和
VianaWl将此结果推广到对某参数值具有非退化同宿相切的曲
而微分同胚族.与已取得的结果相呼应,PalisE]猜测具有同宿相
切(在高维情况,同宿分支)的系统应该在全体非双曲系统中稠
密.Yoecoz\^{}认为当前讨论此类问题的有效方法应该是,首先要
对某个系统有很好的了解(例如,Logistic映射族在a = 2时,Hfeon

映射在« = 2,6 = 0时),然后考虑对此(全局)分支值的开折.我
国学者在此领域也有一些值得提到的工作,例如{[}Cyl,2{]},{[}Wl{]},
{[}Z{]},{[}EZ{]}.在本章中,我们只介绍{[}BC1{]}关于二次映射族的结果
和证明方法.应当指出,这里介绍的证明思想在当前这一方向的研
究中是十分重要的.