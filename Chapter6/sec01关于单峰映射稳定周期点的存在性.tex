\section{关于单峰映射稳定周期点的存在性}

本节我们要讨论单峰映射Z.Z-I,其中7 = {[}--- 1,1 {]},的稳定
周期点的存在性问题.

定义Li称一个映射rjf,是单峰的,如果它满足下列三 个条件'

(UDf是连续的;

(6)/(0) = 1\textsubscript{(}

\textless{}u\textsubscript{3})/在{[}0,口严格下降,而在{[}一
1,0{]}严格上升.

\begin{quote}
称f是C1单峰映射,如果\emph{f}满足上面三个条件外,并有
(u\textsubscript{4})/是C\textsuperscript{1}的,且尸(z)尹0,当C尹0时.
\end{quote}

令f是C\textsuperscript{1}单峰的,并令F是f的周期为会的轨道.我们称F是
稳定周期轨,如果z £ P,\textbar{}DP (x) \textbar{} 称F是超稳定的,如果0

\emph{€ P.}由稳定性定义,如果F是f的一条稳定周期轨,那么对x €
■P,有一个邻域U,使得对一切\emph{yeu,}有lim \emph{f醇}除了可
«\textasciitilde{}* + 8

能的=1外).

现在我们要讨论的问题:一个单峰映射可以有多少条稳定的 周期轨道?
1918年Julia证明了某类单峰映射至多有一条稳定的
周期轨道.1978年Singe网做了实质性推进,下面我们介绍他的一 些工作.

\begin{quote}
设f是映射J的Schwartz导数Sf(x)定义为 5窗-龄「・
\end{quote}

定义1.2 称f是S单峰映射,如果

(\&)/是仃单峰映射!

(\&)f是U映射;

(S\textsubscript{3}) 5/(x) \textless{} 0,x G 7,并允许 S/Xz) =---o0(

(s)f映j(y)= 3⑴,口到自身?

\textless{}s\textsubscript{5})尸(0) VO.

定理1.3如果f满足(\&),(\&)和(\&),那么每条稳定周期
轨至少吸引工=一1,0和1中的一点.

证明由简单计算和归纳法,下面前两个性质是容易得到 的,此处不再賛述.

\begin{enumerate}
\def\labelenumi{\arabic{enumi}.}
\item
  如果/,g € C\textsuperscript{3},那么• g)S) = (Sn(g3))g,Gr)z +
  \emph{Sg(.x).}
\item
  如M/e C\textsuperscript{5} 且S/(x) \textless{}O,VxG L那么 s(产)(z)
  \textless{} 0, m
\item
  \emph{\textbackslash{}f}丨在(一 1,1)中没有正的局部最小.
\end{enumerate}

如若不然,设1尸\textbar{}在,£ (-1,1)中有正极小.不妨设尸⑶)
\textgreater{} 0,那么舟C)=
0.注意到f是C\textsuperscript{3}的,以及Sf(3-)\textless{}0,r(3-\textgreater{}必
与户(少反号,这与尸3)为极小值相矛盾,特别,上述事实也说
明,如果\textbar{}尸。)\textbar{} =
1,那么至少在y的一边,\textbar{}尸\textbar{} \textless{} 1.如果 贝尹士
D是f的不动点并且I户。)I = 1,那么这个不动点至少 在一边是稳定的.

\begin{enumerate}
\def\labelenumi{\arabic{enumi}.}
\setcounter{enumi}{3}
\item
  如果f有有限多个临界点3称为\emph{f}的临界点,如果产愆)
  =0),那么对每个整数\emph{n\^{}l,f}的周期为n的点是有限个.更明确
  地讲,对每个"\textgreater{} 1,/■的周期为«的点是可分隔的.
\end{enumerate}

如若不然,令g =产并设有无穷多x G /,满足g(x)=工.由 中值公式,便有无穷多x
€,,使营愆)=1.由性质2和\emph{3,\textbackslash{}g'} \textbar{}没
有正的局部最小.因此有无穷多\emph{x,g'} (x) = 0.这与f (因此与\emph{g)}
有有限多个临界点相矛盾.

\begin{enumerate}
\def\labelenumi{\arabic{enumi}.}
\setcounter{enumi}{4}
\item
  如果\emph{a\textless{}k\textless{}c}是\&
  =产的相邻不动点,并且在区间{[}a,c{]}
\end{enumerate}

中不含g的临界点,那么寸。)\textgreater{}1.

事实上,由中值公式,存在\emph{u,v,a\textless{}u\textless{}:b\textless{}v\textless{}c,}使得g,GO
=g'(v) = 1.因为 g 在[a,c]没有临界点,g,Gr) \textgreater{} 0,x G [a,cl
于是由性质3,g'(i)\textgreater{}l.

6.如果xe是g的稳定不动点且\textbar{}g's)i vi,那么 定理结论成立.

因为H是g的稳定不动点,所以它的吸引域中包含勿的连通分
支具有形式[-侦)或者O,口([--- 1,口是平凡情况).首
先我们考虑5,s)情况.\emph{g}将此连通分支映到自身,但小不在工的
稳定流形内,于是3不会映入工的连通分支内.因此,只有以下三 种情况之一出现.

\begin{enumerate}
\def\labelenumi{(\roman{enumi})}
\item
  g(r) = g(s)(= \emph{r} 或 s);
\item
  g(r)=r 并且 g(s) = s;
\item
  g(r) =s 并且 g(s) =r.
\end{enumerate}

如果情况(D出现,那么由Ro\%定理,g在中有一个临界 点声,它被吸引到h.但是g
=产,那么存在使得f将/■映到 f的临界点.从而定理得证.

对情况(ii)(或(由)),类似于⑴,我们不妨设在(r,s)中无临界
点.那么由性质5,这两种情况都可排除(在情况(iii)时,考虑g\textsuperscript{2}).

现在我们考虑连通分支为[-1,5)的情形.此时一 1被吸引到
石类似的结论对□也成立.于是,对\textbar{}/U)\textbar{}\textless{}l,xe(-l,l)
时,定理得证.如果x=±l,定理的结论显然.

.7.如果gS) = \emph{x,} I = 1,那么定理的结论成立.

不失一般性,我伯设g,(x) = 1.如果工=士 1,•那么无需证明. 如果女£ (-1,1),
Ek性质4,有一个z的邻域(r,s)不包含g的其 它不动点.于是 g(y) \textgreater{}
\emph{y,y £ (r,x)}或 g(3F)\textless{} \emph{y,y} G (x,s).否
则,在H两边有点丿,使得营。)\textgreater{}1,因此营有一个正的极小点.为
了确定,假设gQ) \emph{\textgreater{}y\^{}y} G
(r,x).令4是\emph{y\textless{}x}中包含5,女)的 使得£(少\textgreater{}
3-的最大连通分支的最小值.那么g(d) = /(或者\emph{d}
=---1并由此c吸引一1).显然扌以)21,由此有一个点se (d,
工),使得营(矶=1.如果有;y e (d,矿,使得b。)= 0,则利用性
质6,否则由情况(iii),我们完成了性质7的证明,并因此证明了定 理.I

定理L 3有以下几个推论.

推论L4如果/■是S单峰映射,那么它至多有一个稳定的周 期点,加上在区间[一 1
,/(!)]中的一个可能的稳定不动点.

证明 因为/(0)=1,点和1被吸引到同一条稳定周
期轨.由于JCf)是关于■/■不变的,如果一条周期執道有一个点属 于
3,那么这条轨道便落入/\textless{}/).如果这条同期轨是一个稳定
不动点H并且0,由情形3,
\emph{\textbackslash{}f'\textbackslash{}\textsubscript{M}} W
1或者\textbar{}尸\textbar{}母,汀\textless{} \emph{1.}
如果是第一种情况,工吸引Lo.xj并因此吸引0点,如果是第二种
情况,r吸引\textbar{}\textgreater{},口并因此吸引1和0=广】(1).对mvo可类似地
论证.

如果有一个周期\emph{P} \textgreater{} 2的稳定周期轨,那么用完全类似的讨
论可以说明,户的最右边的不动点或者吸引尹的临界点,由此吸
引0点或者它吸引L于是,我们证明了 f在JCf)中至多有一条稳
定周期凱另外,从上面的证明我们也可以看到,在JO)中没有稳
定周期轨,它仅仅吸引一1.

现在我们考虑\emph{f}的稳定周期轨,它不吸引0或1-由定理1. 3, 它一定吸引一
1,并由此至多有一条这样的稳定周期轨・

下面我们证明,这样的稳定周期轨是JCA)丄=(一 1,/■⑴)中
的一个稳定不动点.如同我们已经看到的,如果这条轨道有一个点
在JCf)中,那么整条轨道在JV)中并吸引。和L因此,这样的轨
道必在JCfj丄中.由性质3J在JU)丄中至多有两个不动点,因为 ■/在[一
1,0]中至多有两个不动点.如果在中没有不动
点,那么\emph{f(.yy\textgreater{}y,y\&J(.f\^{}.}因此J在J(/)丄中没有不动点也
没有周期点?如果/在JUV中只有一个不动点,当/U)\textgreater{}1时,
我们有r=---1,因为/W Vy,当了\textless{}z.由此J在JCO丄中没
有其它的稳定周期轨.当0 Vr`(z) W 1时,由定理1.3/吸引
一1,因此丄中没有其它的稳定周期轨.最后,假设\_/■在J(£\textgreater{}丄
中有两个不动点,那么有一个不动点工使得0\textless{}/\^{})\textless{}1.由定理
1-3,/吸引一1点且在丿侦)丄中,/■无其它稳定周期点.I

摧论L 5设f是S单峰映射.如果fD \textgreater{} 1,则在J CO丄
中没有稳定周期点.

证明 如果那么由 3,/(x)\textgreater{}\^{},x€ (-1,0).
因此,在丿(f)丄中没有稳定周期轨.I

推论L 5可以看成是推论1. 4的部分证明过程,用它可以断
定\emph{JUA}中不含稳定周期点.

推论L6存在没有稳定周期轨的S单峰映射.

证明 一个经典的例子J3) = 1 - 2x\textsuperscript{2}.它是Ulam和V.
Neumann在1947年给出的.容易验证,质是S单峰映射.经0点的 轨道为0,1, ---
1,-由定理1. 3它至多有一条稳定周期轨,并且吸 引三点一 1,0,1之一.但是/(一
1) =一 1是不动点并且吸引这三
个点.因此,它们不能被其它周期轨道吸引.但是,由于
4\textgreater{}1它不是稳定不动点,所以須没有稳定周期轨.\textbar{}

附注1.7在本节我们讨论了 S单峰映射的稳定周期轨的存 在性问题.把定理1.
3和它的推论应用到映射族FGr,a) = /\textsubscript{0}(x) =
上时我们看到,如果a靠近2,那么判定\emph{f\textsubscript{a}}是否有稳定

周期轨的问题转化为讨论临界点轨道的性质(在§ 3定理3,1的
证明中,要用到这一性质).我们将看到,映射族\emph{侦如a} e {[}处,
2{]},向靠近2,在参数区间中有一个开稠的参数集,相应的映射有
稳定周期執.这意味着其动力学行为是简单的,不仅如此,更有意
义的是,在该参数区间中也存在一个正Lebesgue测度的参数集,
相应的映射没有稳定周期轨,并且其动力学行为是复杂的.由此结
果,我们讨论清楚了分支值a = 2附近的部分情况.