\section{$ F(x,a) = 1 - ax^2$的基本桂质}


我们将讨论A(a)=硏(0』)返回到一个固定的小区间尸= (一
\emph{S,s)}的方式,其中\emph{a} = exp(--- C).特别要讨论它们是怎样
靠近原点的,以及靠近原点的速度.而F3,4)的迭代的一些基本
性质对于理解这些问题起着重要作用.应当指出,尽管我们只对这
一特殊映射展开讨论,但是这种以Jakobson开始,Benedicks和
Carleson发展的理论是解决一系列此类问题的关键所在.

首先我们将广分成一列不相交的小区间的并= U \emph{I"}

1\^{}\textbar{}\textgreater{}° 其中七==exp(--- 0,而当戸 V 0 时,

---(---- C-J.-1L并记为 \emph{Ip =---}

引理2.1对充分小\emph{8\textgreater{}0,}存在国\textless{}2,使得对ae
[何,2丄 如果% € [- 1.1]和\&满足

函(勿,a)\textbar{} 2* 丿=0,1,2,``麟一 1,

并且

I砂皿\textless{}-?,

那么

2(1.9)匕 (2.1)

证明 令\emph{* = 眄=}sin\emph{淡}在此变换下\textsc{,F(h,4)}可以表示 为

--- ---arcsin(l --- a sin\textsuperscript{2} 扌。).

Tt Z

\begin{quote}
\emph{9}

cos \emph{---n}
\end{quote}

诜(饥\textless{}0 =--- -/2a sgn(5) \sout{,三 一 .}

Jcos或+国京"―刍 令 % = Fu(\%,a),Qu = au(Q"),其中。。=广(\%).假设
I初 \textgreater{}

1一2乩,= S1,\ldots{},j-l,但是区\textbar{} W1 --- 2战 记砰=會萨一如t •
F,・那么我们有

\textsuperscript{fr}-i \textsc{j} \textgreater{}-1

\emph{aF} = p'(n)]Ja『(句,a)反代'(为)!!(- \emph{2am.} (2. 2)

\emph{v---j} w=0

假设%充分靠近2,用归纳法容易证明\textbar{}狷W1 ---揷,,= £••,,
疋因此cos\emph{湛}M 0.于是1\emph{8趴3} I
\textgreater{}1.9.另外,我们有\emph{平'}(向) = 二说,血 IWS 以及
£广(为)=I -\textsubscript{r}■- - ■当 a£

J1 一务 S,2]时,立得

宜(E)IKL9 光 1

引02.2存在常数,\textgreater{}0和\emph{k(S)\textgreater{}0,}当角充分靠近2时,
对所有a£ [\%,2]和所有\textbar{}x\textbar{}法%\_1,存在使得
下面的不等式成立.

\begin{enumerate}
\def\labelenumi{(\arabic{enumi})}
\item
  \textbar{}歹Cr,a) \textbar{} 贏 =、・.. J 一,
\item
  log\textbar{}a\textsubscript{T}F(x,a)\textbar{}\textgreater{}rZ.
\end{enumerate}

证明记% =烈").注意到如果而且 f =-1 + e,那么 f \textless{} F(\&a)
\textless{}- 1 + 4\& 设 瓦[\textgreater{}
c\textsubscript{a}\_\textsubscript{v} 如果

I 如 ■,那么 \textbar{}。顼(%,(0\textbar{} =
\textbar{}2a7\textsubscript{0}\textbar{} 如果 2-'+'N 知 22-',

ZN2,那么

IM(\%,a)\textbar{} \textgreater{}
\textbar{}2晩•前丨\ldots{}\textbar{}2a\%\_J

N \%2-` ♦ 2a(l - 2 • 2」"+2) ••• 2\textless{}z(l - 4'`` . 2 •
2\textsuperscript{\_w+z})

\begin{quote}
=(2a)'2-'(l --- 2 • 2\textasciitilde{}\textsuperscript{w+2})(l -4*2-
2\textasciitilde{}\textsuperscript{2,+2}) ••• (1 -
2\textasciitilde{}\textsuperscript{!}) .=泌(1 \_号)(1 \_\$) ... (/烏).
\end{quote}

因为%充分靠近2,取,=*ln2,那么蜘庖即如从上
面的证明过程可知,对:=1,\ldots{},Z - 1,函(知a)丨2 1 --- \$
\textgreater{} 5- 最后,取*3) = (log2)T"log -1,得I 下面的引理2.
3要证明,当\textless{}5非常小并且吗充分大时,在参数
区间(缶,2)中存在一个小区间d,使得映射Sm/Af厂是一一对
成的,并且保持指数扩张"

引理2.3对任意充分小\emph{S\textgreater{}0,}任意正整数N和任意向V 2,
存在\emph{m法N}和一个参数小区间A\textgreater{} U(00,2),使得

\begin{enumerate}
\def\labelenumi{(\arabic{enumi})}
\item
  对任意 d,\&(a)W--- - 1;
\item
  是\&到广的一一映射且是映上的;
\item
  \begin{quote}
  \textbar{}\&FJ(l,a)\textbar{} \textgreater{}
  (1.9)'一1万=1,2,\ldots{},叫-1. 证明因为
  \end{quote}
\end{enumerate}

(1 + 勾+1) --- (1 + \&)=勾+1 --- \&

=2 --- Q + \textless{}2(1 + \&) (1 如)\emph{t}

当---身时,我们有1 + M+1 \textgreater{} \textbar{}(1 +专).

即1+6是指数增长的.另外,如果那么从 不等式

警=\_号\_2妙爲〈-争

可归纳地证明af釦心)是单调下降的.由这些事实,结论(1)稲
(2)得证.这是因为对上述的\emph{B}和N,我们选出包含2的小参数区
间21\textsuperscript{(}\textsubscript{0}C= {[}00,21由于1 +
\&是按指数增长,只要充分小,总有 \emph{m、}\textgreater{} N,使得 4(a) M
---扌,2 W v W 屿 一 1 \textsc{,q €} 但是氧(赁) \textgreater{}
\textless{}5,其中爲是的左端点.注意到\&,(a) ,a €占。是单调下降
的,所以存在A二得当时,\&,A,f广是一一映射.最 后我们证明指数扩张性.因为'

J

\textbar{}为应(1,\textless{}2)\textbar{} = U(- 2a\&),

y=l

其中\& = 1并且一1 M \& M \textasciitilde{}\^{}*\textsuperscript{v} =
2,\ldots{}W吗---1.从而 结论(3)得证.{]}

下面的引理2.4说明互卧和\&F"是可以比较的.更确切地 讲,M
和為F,的增长速度是相同的,我们已轻知道,

为卧+1 =-\/- 2『为殆,\emph{a\textsubscript{x}F\textsuperscript{a}} = 1

和

平+1 =-\/- 2『"-(尸尸,\emph{3\textsubscript{a}F\textsuperscript{a}} =
0.

因此

\begin{quote}
V---1

M = U(-25), v = 1,2,\ldots{} (2. 3)

:=0
\end{quote}

并归纳地得到

3瑚,=为时多专(1 + 涂)/ = 2,3,\ldots{} (2.4) 显然有

\emph{d\textsubscript{a}F} 一 x\textsuperscript{a}.

\begin{quote}
引理2.4 存在充分小的\^{}\textgreater{}0和㈤V2,\%靠近2,如果 ⑴ 1 一 2N*
\end{quote}

\begin{enumerate}
\def\labelenumi{(\arabic{enumi})}
\setcounter{enumi}{1}
\item
  \% W a V 2;
\item
  \textbar{}为玲-'(\%,a) \textbar{} 2exp\_f'3,j = 8,9,,``", 那么
\end{enumerate}

\begin{quote}
1 {\textbar{}\%Fl(\%,a)\textbar{}}

16、\textbar{}3顼1(知\textless{}2)\textbar{}
\end{quote}

证站 由连续性,如果d充分小,角靠近2,那么对任意%和
a满足条件(1)和(2),容易证明

五弁11 )\textgreater{}丄

\begin{quote}
2。吕「 \textbar{}2aVSa)"" 8
\end{quote}

\includegraphics[width=0.45347in,height=0.43333in]{media/image79.png}\includegraphics[width=0.47361in,height=0.33333in]{media/image80.png}

(2.8)

8

我们仅证第一个不等号,并只就\emph{声}\textgreater{}
°,茶〉\textsuperscript{0}情况给以证明, 其它情况可类似证明.

\begin{quote}
屏2 0,裟 \textgreater{} 0.如果\emph{3\textsubscript{a}Fi} \textgreater{}
0,那么法序2 0.所以1十 \emph{pi o}
\end{quote}

旎戸21---云瀚5・如果%戸\textless{}0,那么勺声\textless{}0.而0/'12

expWJ 于是因为 因

此

\textsubscript{1 +} \_\^{}\_\textgreater{}!\_\_8\_\\
十海尸身\textsuperscript{1} exp",

最后,我们来归纳地证明引理结论.\emph{3} = 7时无需证明,设

j时,有

因为

"许队 I 尸 I

頭吋的\_有丛I「+2辺同

\begin{quote}
fi + 一\emph{沙履\_\_} {]} \textbar{}许\textbar{} h +{屏E}

I十2崩小(E)丿\textbar{}丛\textbar{}丨十料旳")
\end{quote}

因此由(2. 5) --- (2. 9)可证

1 ' {"+W,a)丨} /`` .

行W "珂和7 \$ \textsuperscript{1}

附注2. 5 我们简要地说明一下引理2.1 ---引理2. 4的意
义.在§3主要结果的证明中我们可以看到,執道媚(。)条1将会
反复地进入临界点的小邻域广中.粗略地说,引理2.1和引理2. 2
根据兀的迭代的这种特性,分别讨论了当参数靠近2时,兀进入
\emph{r}前和走出厂后,在广外的扩张性质.引理z.3将作为我们归纳
地得到正Lebesgue参数集丄的基础.我们将F3,a)看成二元函
数,引理2.4说明\emph{F(.x,a)}的迭代对工和a是等度增长的.这一结果
的好处在于对参数或对变量的估算可以相互转化.在§ 3定理3.1
的证明中,我们将看到这些性质所起的重要作用.这些看似简单的
引理的意义还不仅仅如此•对于一般的映射族FGc,a),我们可以
提出与上面基本性质类似的假设.如果FCz,a)满足这些假设,那
么\textsc{F(h,\textless{}0}也具有类似于1 -損的重要结果,细节将在本章的小
结中阐述.