\section{前言}
  动力系统的理论,起源于对常微分方程的研究,近半个多世纪以来得到了蓬勃的发展。
  随着在结构稳定系统的研究中所取得的突破性进展,对结构不稳定系统的研究(既分岔理论)便受到越来越多的关注。
  分岔理论具有深厚的实际背景,又需借助于现代数学的深刻工具。
  在实际应用和数学发展的双重推动下,这一理论的前景是广阔的。

  
  所谓分岔现象,是指依赖于参数的某一研究对象当参数在一个特定值附近作微小变化时,它的某些性质所发生的本质变换。

  
  在自然界中,分岔现象是普遍存在的。
  例如,导管中的流体流动,当流速超过某个特定值时,就由层流变为湍流;
  在生态系统中,当一些自然条件超过某些特定状态时,便可引起生态平衡被破坏或种群灭绝。

  
  既然分岔现象普遍存在于自然界中,因而在描述自然现象的数学模型中,分岔现象也大量存在。

  
  例如,描写磁腔管中磁振荡的模型
  \[\epsilon \frac{d^2x}{dt}+(x^2-1)\frac{dx}{dt}+\epsilon x=b\sin t ,\]
  其中\(0<\epsilon \ll 1\),b是一物理量。
  自本世纪40年代起,这个方程就引起了人们的关注。
  随后发现当b取某些特定值时,系统有非通常意义下的吸引子,从而引出了奇异吸引子的概念。
  事实上,正是在N.Levinson对这个方程研究结果的启迪下,S.Smale给出了著名的马蹄映射的例子。

  
  又如,60年代从气象学中提出的Lorenz方程
  \begin{equation*}
    \left\{
      \begin{aligned}
        \frac{dx}{dt}=-\delta x+\delta y,\\
        \frac{dy}{dt}=-xz+\gamma x-y,\\
        \frac{dz}{dt}=xy-bz,
      \end{aligned}
            \right.
    \end{equation*}
变量\((x,y,z)\in R^3\),参数\(\delta,\gamma,b>0\),其中\(\gamma\)是刻画气体流速的雷诺数。
%这儿书上有个错误
利用计算机研究发现,
若取\(\delta =10,b=\frac{8}{3}\),
则当\(\gamma\) 在三个(分岔)值 \(\gamma_1 \approx 24.06\)
和 \(\gamma_2 \approx 24.74\) 附近时,相应系统的轨道结构呈现某种"混乱"现象。
进一步的研究表明,这种看起来“杂乱无章”的现象却有内在的规律性,这不仅给湍流的形成以新的解释,而且引出了一系列有关混沌现象的研究工作,至今还是物理和数学界关注的热点问题之一。


再如,从生态学中提出的虫口差分模型
\[x_{n+1}=x_n(a-bx_n),a,b>-0,a-bx_n>0,\]
经过适当变换可化为单参数一维单峰映射族
\[f_\mu(x)=1-\mu x^2,0<\mu<2,x\in [-1,1].\]
70年代M.J.Feigenbaum对它进行了细致的研究后发现,当\(\mu\)从0连续增加时,\(f_\mu(x)\)不断出现倍周期分岔点,而且对应于出现稳定周期点的哪些分岔值具有很强的规律性,从而发现了一个新的普适常数,由此引出的相关的研究工作,也受到物理和数学界的关注。


数学上作为研究分岔现象的理论——分岔理论主要研究三类问题:
由常微分方程(或向量场)所定义的连续动力系统的分岔;
由映射所定义的离散动力系统的分岔;
函数方程的零解随参数变化而产生的分岔。
前两类分岔称为动态分岔,而第三类分岔称为静态分岔。
它们既有区别,又相互联系。
本书主要讨论动态分岔,特别是第一类(既向量场的)分岔.


动态分岔理论主要研究动力系统的轨道族的拓扑结构随参数变化所发生的变化及其规律。
例如,奇点(或不动点)的汇聚与分离及该点附近轨道的变化;
周期轨的产生与消失,同宿轨、异宿轨(或环)的形成与破裂;
以及一些更复杂的动力学行为(例如混沌态)的出现与消失等。


虽然分岔理论的某些方法可以追溯到$Poinicar\acute{e}$时代,但在这一研究方向上取得长足的进展,只是近30-40年的事。
迄今为止,大部分工作集中于平面上退化程度不高(既余维\(\leq 2\))的分岔,
也包括同宿分岔和异宿分岔问题等。
分岔理论的发展很大程度上依赖于结构稳定性理论的进展,而目前只有对二维流形上的动力系统的结构稳定性有较完整的结果。
因而,当相空间维数增大或系统的退化程度增大时,问题的复杂性大大增加,完整的工作尚属少见。
此外,最初人们希望在分岔值附近能进行开折,既在分岔值附近存在几张超曲面,他们把参数空间分成若干开区域,每个开区域对应结构稳定的系统。
但人们逐渐认识到,在不少情况下分岔值附近不存在这样简单而理想的拓扑结构,往往只能从测度上进行描述。
本书的第五章\ref{5.3}和第六章将涉及这一问题。


本书的撰写由张芷芬主持。
第一、二、三章由张芷芬和李承治执笔,第四、五章和第一章\ref{1.4}中的光滑线性化部分由李伟固执笔,第六章由郑志明执笔,附录由李承治执笔,最后经集体讨论定稿。


下面简要介绍本书的内容安排。


第一章介绍基本概念和准备知识。
我们假定读者具有常微分方程和常微分方程定性理论的基础知识。
因此,对动力系统的概念只作了简略的介绍。
然后通过实例引进分岔的概念及分岔问题的提法。
本章还介绍了简化分岔问题的两个重要手段:中心流形定理和正规型理论。
最后介绍了普适开折与分岔的余维这两个概念。
第二章介绍几类平面向量场的典型的分岔现象,如奇点分岔、闭轨分岔、Hopf分岔、同宿分岔等,以及研究这些问题的典型方法;还介绍了弱Hilbert第16问题。
在第三章中,我们综合运用第二章介绍的理论和方法,研究了几类平面向量场的余维二分岔现象。
第四章主要介绍二维映射的双曲不动点,并给出一类复杂的不变集(Smale马蹄)存在性的简洁而严格的判别方法。
这些结构在研究三维向量场的分岔问题中有很多应用。
在第五章中,我们研究三维向量场中双曲奇点的同宿轨分岔,以及与前述Lorenz方程相关的一个由双曲鞍点和一个双曲闭轨形成的环的分岔。
第六章介绍实二次单峰映射族在某个分岔值附近的动力性态:在参数空间中可以存在正Lebesgue测度集,使相应的映射族具有非双曲的奇异吸引子。
这说明从测度角度上看,非双曲的系统并不少,并且其动力学行为非常复杂。
本章不属于教材的基本内容,只是向读者介绍近年来动力系统研究的这个新热点。
本书最后的附录涉及到深一些的教学内容,它是为那些想对书中的某些内容(特别是第一章\ref{1.1}和\ref{1.5})进行深究的读者准备的,使他们减少了查找参考书的麻烦。


作为分岔理论的入门教材,本书主要介绍动力系统分岔理论中一些基本概念、主要结果和常用方法,并力图通过最简单的例子涉及到这个理论的一些本质方面。
我们把重点放在向量场的分岔上,但不可避免地会涉及到一些离散动力系统的情形。
我们力求在选材上体现少而精的原则,因而不得不舍弃一些十分精彩但陈述冗长的结果或证明;
在着力于可读性的同时,尽量兼顾一定的理论深度;
并在注重解析推理的同时,兼顾几何直观。
本书的大部分材料选自有关的论文或专著,我们在书中都做了具体的说明。
为了使读者易于接受,我们对这些材料做了整理和加工。
例如,第三章\ref{1.1}中大部分定理的证明和第四、五两章全部定理的证明,是作者重新给出的;
在第六章大部分定理的证明中,作者对原始材料做了必要的补充。
我们也在书中介绍了作者们的一些近期工作。
例如,第二章对参数一致的Hopf分岔定理;
对Abel积分零点个数的估计和有关高阶Melnikov函数的结果等。
限于作者们的水平和能力,书中难免有不妥或错误之处,我们热诚欢迎读者们的批评与指正。


本书可作为大学数学系高年级本科生的选修或者相关专业研究生的基础课教材;
也可供相关学科学生或科技人员当做参考书。
书中的大部分内容,我们曾在北京大学讲授过,部分内容也曾在1990年“南开动力系统年”期间被用作教材。
根据我们的经验,对于每周3-4学时的、一学期的课程,可以讲授第一章\ref{1.2}-\ref{1.4},第二章,第三章\ref{3.1},第四章和第五章\ref{5.1}-\ref{5.2};
对于每周3学时、两个学期的课程,则可以讲授第一章到第五章的全部内容,第六章可选讲或选读。
对于初学者,我们建议在学完第一章\ref{1.2}后立即转入第二、三章,而第一章的其他各节可在适当时候再学,这样可能更容易接受些。


我们愿借此机会对参加过北京大学分岔理论讨论班的曾宪武、井竹君、王铎、高素志、唐云、张伟年、李宝毅、李翠萍、肖冬梅、齐东文、曹永罗、王兰宇、赵丽琴、彭临平等同志和我们的研究生们表示感谢,他们的报告和讨论使我们受益匪浅;
其中有的同志还帮助我们仔细审阅了部分书稿,提出了不少好的建议,避免了一些错误。
我们要特别感谢由伍卓群、黄启昌、曹策问三位教授领导的第一届国家教委理科数学与力学教学指导委员会微分方程教材建设组的各位专家,他们从审定本书的撰写计划到审议书稿都提出了很多宝贵、中肯的意见;
感谢本书的主审人王铎教授和韩茂安教授,他们不仅对本书提出了很多建设性的意见,而且还提供了部分习题;
感谢张恭庆教授,他在百忙中审阅了本书的附录,并提出了宝贵的意见;
感谢高等教育出版社的杨芝馨同志,没有他们的辛勤工作,本书也不可能这么快与读者见面。


在撰写本书期间,作者得到国家自然科学基金和高等学校博士学科点专项科研基金的支持,我们也借此机会向有关方面表示感谢。


作者


1995年8月于北京大学

