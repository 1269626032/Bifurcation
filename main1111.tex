\documentclass{article}


\usepackage{ctex}


\begin{document}
\includegraphics[width=2.33333in,height=3.19306in]{media/image1.jpeg}


in。

\begin{quote}
高等学校教材
\end{quote}

向量场的分岔理论基础

张芷芬 李承治

郑志明 李伟固

高等教育出版社

动力系统的理论,起源于对常微分方程的研究,近半个多世纪
以来得到了薑勃的发展.随着在结构稳定系统的研究中所取得的
突破性进展,对结构不稳定系统的研究(即分岔理论)便受到越来
越多的关注.分岔理论具有深厚的实际背景,又需借助于现代数
学的深刻工具.在实际应用和数学发展的双重推动下,这一理论 的前景是广阔的,
.

所谓分岔现象,是指依兢于參数的菓一研究对象当參数在一
个椅定值附近作微小变化时,它的菓些性质所发生的本质变化.

在自然界中,分岔现象是普遍存在的.例如,导管中的液体流
动,当流速超过某个椅定值时,就由层流变为湍流,在生态系统中,
当\textasciitilde{}些自然条件超越菓些椅定状态时,便可引起生态平衡被破坏
成利■群灭绝等.

既然分岔现象普遍地存在于自然界中,因而在描述自然现象
的数学模型中,分岔现象也大量存在.

例如,描写磁腔管中磁振荡的模型

e 务 +3-1)栄 + 田=6 sin/,

其中0是一物理量.自本世纪40年代起,这个方程就
引起了人们的关注.随后发现当\&取菓些椅定值时,系统有非通
常意义下的吸引子,从而引出了奇异吸引子的概念.事实上,正是 在N.
Levinson对这个方程研究绪果的启迪下,S. Smale给出了著 名的马蹄映射的例子.

又如,60年代从气象学研究中提出的Lorenz方程

d?=一皿+吵

-*=一即+ 如一

= XJ - fe,

变量x,\textsubscript{y}\^{}eR\textsuperscript{3},参数\emph{a,7,b\textgreater{}o,}其中y是刻画气体流速的雷诺
数.利用计算机研究发现,若取10,6 =奇■,则当7在三个(分 岔)值儿2 13.
926," 24. 06和匕忍24. 74附近时,相应系统的

轨道结构呈现出某种``混乱"现象.进一步的研究表明,这利■看起
来``杂乱无章''的现象却有内在的规律性,技不仅给湍流的形成以
新的解释,而且引出了一系列有关浑沌现象的研究工作,至今还是
物理和数学界关注的热点问题之一.

再如,从生态学中提出的虫口差分模型

\textsc{x\textsubscript{m+1}} 一 x„(a ― \emph{bx„), a,b \textgreater{}
Q,a --- bx„} \textgreater{} 0,

经过适当变换可化为单参数一维单峰映射族

\emph{f\textsubscript{P}(.x)} = 1 一 az?, 0 \textless{} \textless{} 2,
\emph{x} {[}--- 1,1J.

70年代M. J. Feigenbaum对它进行了细致的研究后发现,当兴从
0连续增加时J户愆)不断出现倍周期分岔点,而且对应于出现稳
定周期点的那些分岔值具有很强的规律性,从而发现了一个新的
普适常数.由此引出的相关的研究工作,也受到物理和数学界的 关注.

数学上倍为研究分岔现象的理论------分岔理论主要研究三类
问题:由常微分方程(或向量场)所定义的连续动力系统的分岔;
由映射所定义的离散动力系统的分岔;函数方程的零解随参数变
化而产生的分岔.前两类分岔称为动态分岔,而第三类分畚称为
静态分岔.咆们既有区别,又相互联系.本书主要讨论动态分岔,
特别是第一类(即向量场的)分岔.

动态分岔理论主要研究动力系统的轨道康的拓扑绪构隨參数
变化所发生的变化及其规律.例如,奇点(或不动点)的汇聚与分
离及该点附近轨道的变化;周期轨的产生与消失,同宿轨、异宿轨
(或环)的形成与破裂;以及一些更复杂的动力学行为(例如浑沌
态)的出现与消失等.

虽然分密理论的某些方面可以追溯到Poincart时代,但在这
一研究方向上取得长足的进展,只是近30-40年的事.迄今为
止,大部分工作集中于平面上退化程度不高(即余维W 2)的分岔,
包括同宿分岔和异宿分岔问题等•分岔理论的发展很大程度上依
赖于结构稳定性理论的进展,而目前只对二维流形上的动力系统
的结构稳定性有较完整的结果.因而,当相空间维数增大或系统的
退化程度增大时,问题的复杂性大大增加,完整的工作尚属少见.
此外,最初人们希望在分岔值附近都能进行开折,即在分岔值附近
存在几张超曲面,它们把参数空间分成若干开区域,每个开区域对
应结构稳定的系统.但人们逐渐认识到,在不少情况下分岔值附
近不存在这样简单而理想的拓扑结构,往往只能从测度上进行描
述.本书的第五章§3和第六章将涉及这一问题.

本书的撰写由张芷芬主持.第一、二、三章由张芷芬和李承治,
执笔,第四、五章和第-■章§4中的光滑线性化部分由李伟固执
笔,第六章由郑志明执笔,附录由李承治执笔,最后经集体讨论定 稿.

下面简要介绍本书的内容安排. •

第一章介绍基本概念和准备知识.我们假定读者具有常微分
方程和常微分方程定性理论的基貓知识.因此,对动力系统的概念
只作了简略的介绍.然后通过实例引进分岔的概念及分岔问题的
提法.本章还介绍了简化分岔问题的两个重要手段;中心流形定理
和正规形理论.最后介绍了普适开折和分密的余维这两个概念.
第二章乔绍几类平面向量场爾最典型的分岔现象,如奇点分岔、闭
轨分岔、Hopf分岔、同宿分岔等,以及研究这些问题的典型方法;

还介绍了弱Hilbert笫16何题.在第三章中,我们综合运用第二
章介绍的理论和方法,研究了几类平面向景场的余维二分岔现象.
第四章主要介绍二维映射的双曲不动点,并给出一类复杂的不变
集(Smale马蹄)存在性的简洁而严格的判别方法.这些结果在研
究三维向景场的分岔问题中有许多应用.在第五章中,我们研究
三维向景场中双曲青点的同宿分侖,以及与前述Lorenz方程相关
的由一个双曲鞍点和一个双曲闭轨形成的环的分岔.第六章介绍
实二次单峰映射族在某个分岔值附近的动力性态,在参数空间中
可以存在正Lebesgue测度集,使相应的映射族具有非双曲的奇异
吸引子.这说明从測度角度上看,非双曲的系统并不少,并且其动
力学行为非常复杂,本章不属于教材的基本内容,只向读者介绍近
年来动力系统研究的这个新热点.本书最后的附录涉及到深一些
的数学内窓,它是为那些想对书中的某些内容(特别是第一章§1
和§5)进行深究的读者准备的,使他们减少査找参考书的麻烦.

作为分岔理论的入门教材,本书主要介绍动力系统分岔理论
中一些基本概念、主要结果和常用方法,并方图通过盘简单的例子
涉及到这个理论的一些本质方面.我们把重点放在向景场的分岔
上,但不可避免地涉及到一些离散动力系统的情形.我们力求在选
材上体现少而精的原则,因而不得不舍弃一些+分精彩但陈述冗
长的结果或证明;在着力于可读性的同时,尽量兼顾一定'的理论
深度;并%注重解析推理的同时,兼顾几何直观.本书的大部分
材料选自有关的论文或专著,我们在书中都做了具体的说明.为
了使读者易于接受,我们对这些材料做了整理和加工.例如,第三
章§1中大部分定理的证明和第四、五两章中全部定理的证明:,是
作者重舫给出的;在第六章大部分定理的证明中,作者对原始材料
作了必要的补充.我们也茬书中介绍了作者们一些近期工作.例
如,第二章中对参数一致的Hopf分岔定理;对Abel积分零点个数
的估计和有关髙阶Melnikov函数的结果等.限于作者们的水平
和能力,书中难免有不妥或错误之处,我们热诚欢迎读者们的批评

**«* 牛's I

**

.龄 岡粗霎w・衆博加咬屮 罹两账•窿钟

•同片用丼・H\\
辛佥螂・应柑輿宅霎田褪藕彎『\\
汨・1毎也由•湯*"\\
氽1驱用凹・ M盅發貶〔汨整疽笔•來\\
锹麋覊出Y狷-诚,骥\\
汨擊暴新認两彖右\^{}电\%\\
枳氽擧亦岷漏晔驱卦驚卦屮卦\^{}-\/-\\
零刮中会输卜瓯號柔也豪部\\
赛点喧``W企/•奉\\
珈蠟碗叵碧餘,健報,如川,豔,収林\\
,,,,,,,,,\\
蝦聿本聿醐招参亦宿苦噩曖辛

.収霧同 刼・阵湖烟片拒嘛何l・\textless{}Y 迪 zl妙報-卦畀卜•編緡同K
・蜿至蝌圉lM 启 11",毎 lzl国•她舞1,寸昵4 ・妃.匡終匡"*蒂„。661
用位•帮qF・民気鞭金七 •平Y緜泅卦同瘁-龍 応顎加嬰答傘編那* 抿固

.出

s m.\textsuperscript{B}

\protect\hyperlink{bookmark6}{第一童基本柢念和准备知识 1}

\protect\hyperlink{bookmark10}{§1动力系统及其结构稳定性 1}

§ 2分岔与分岔问题的提法 9

\protect\hyperlink{bookmark23}{§3 中心流形定理' 21}

\protect\hyperlink{bookmark32}{§ 4 正规形 31}

\protect\hyperlink{bookmark40}{§5 普适开折与分岔的余维 44}

习题与思考■题一 56

第二章 常见的局部与菲局部分岔 58

\protect\hyperlink{bookmark51}{\emph{§1} 奇点分岔 \emph{58}}

§2 闭轨分岔 64

.§ 3 Hopf 分岔 70

§4平面上的同宿分岔 85

\protect\hyperlink{bookmark86}{§ 5 Poincare 分岔与弱 Hilbert 第 16 问题
94 •}

\protect\hyperlink{bookmark100}{§6 关于Petrov定理的症明 107}

\protect\hyperlink{bookmark118}{习题与思考题二 127}

\protect\hyperlink{bookmark123}{第三章 几类余维2的平面向■场分密 129}

\protect\hyperlink{bookmark126}{§ 1 二重零特征根:Bogdanov-Takens 系统
130}

\protect\hyperlink{bookmark146}{§ 2二重零特征根:1 \textless{} 2共振问题
145}

§3
二重零特征根\textsubscript{:}1\textgreater{}?共振问题(?\textgreater{}5)
152

\protect\hyperlink{bookmark157}{习题与思考题三 157}

第四章双曲不动点及马蹄存在定理 158

\protect\hyperlink{bookmark162}{§1 双曲不劫点定理 158}

\protect\hyperlink{bookmark184}{§2符号动力学简介 165}

.§3 Smale 马跡 170

\protect\hyperlink{bookmark210}{§4线性映射的复合映射的双曲性 178}

\protect\hyperlink{bookmark225}{§ 5 Birkhoff-Smale 定理. 184}

\protect\hyperlink{bookmark236}{第五章空间中双曲鞍点的同宿分岔 190}

\protect\hyperlink{bookmark239}{§1具有三个实特征值的鞍点的同宿分岔 190}

\protect\hyperlink{bookmark268}{§2,空间中鞍焦点的同宿分岔 200}

\protect\hyperlink{bookmark278}{§3环的分岔 216}

\protect\hyperlink{bookmark348}{第六章实二次单峰映射族的吸引子 242}

\protect\hyperlink{bookmark351}{§1关于单峰映射稳定周期点的存在性 { }
243}

§ 2 FG,a) = 1 --- *2 的基本性质 248

§3 \textsc{F(h,u)}不存在稳定周期轨问题 253

\protect\hyperlink{bookmark395}{§ 4 分布问题 271}

附最 284

附录A Banach流形和流形间的映射 284

附录B切丛与切映射,向量场及其流,浸入与複盖\ldots{}287

附录C Thom横截定理 301

奏考文献 307

索引 316

\protect\hypertarget{bookmark6}{}{}第一章基本概念和准备知识

作为全书的准备,我们在§ 1中简述有关动力系统和结构稳
定的基本概念,不加证明地陈述一些重要结果;在§ 2中引入分岔
的概念,并着重阐明分岔问题的提法,在§ 3和§ 4中分别介绍中
心流形定理和正规形理论,在研究分岔问题时它们是进行简化处
理的有效手段;最后,在§ 5中介绍奇异向量场的普适开折和分岔
的余维这两个重要的概念.

\protect\hypertarget{bookmark10}{}{}\textbf{§1}动力系统及其结榆稳定性

动力系统的概念和理论是从人们对常微分方程的研究中产生
和发展起来的,而且对常微分方程的研究,至今仍是动力系统理论
的重要组成部分.

考虑R"中的自治微分方程

\emph{眷=fM) ,} (1.1)

其中/■:
R\textsuperscript{n}-R"是b向量场,r\textgreater{}l-由常微分方程中熟知的结
果,V € R",方程(1.1)以x(0)=务为初值的解a«g)在包
含(=0的某区间上存在.如杲/(x)满足适当条件(或在某种等
价意义下对/"3)进行改造,见LZDHD,pp23-24j),则解明) 可以对一切\emph{t} t
R存在,并且满足:

⑴ a(0,z) = z, Vx€R"\textsubscript{f}

\begin{enumerate}
\def\labelenumi{(\arabic{enumi})}
\setcounter{enumi}{1}
\item
  Ms + V 5,\^{} € R» x 6 R\textsuperscript{rt}*
\item
  g/)对Qu)连续.
\end{enumerate}

我们把满足上述条件(1)-(3)的映射a: R x R"-R"称为
R''中的动力系统,或者称为方程(1. 1)的流,并把点集

\emph{03 =} \{a(Z,z)\textbar{}££ R\} U R'' 称为流a过H的執St.

不难证明,对丫旳口2 G R", 0"为)和0''旳)或者重合,或
者(对有限的时间。不相交.因此的轨道集合依r的不同 而呈现不间的规律.

常微分方程定性理论(或称为几何理论)的首要目标,就是对 于纶定的\emph{f} €
b(R",R''),r》1,研究方程(L 1)的轨道集合的结
构(轨道集合而拓扑结构图称为相图,通常在相图上用箭头标明对
应于时间\$增大的轨道方向).一般而言,方程(1.1)不可能用初等
函数的有限形式求解,因此研究它的相图是一个困难的任务.例 如,即使当« =
2且yw)为某些二次多项式这种最简单的非线性
情形,人们至今尚不清楚相图的确切结构.参见{[}Yl ,.2{]}, \emph{W.}

现在把动力系统的概念加以推广.设M是紧致的b微分流
形,记Diff「(M)为\emph{M}上所有b微分同胚的集合(见附录A中定
义A.8),\^{}r\textsuperscript{r}(M)为肱上所有b面量场的集合,在b拓
扑下,Diff,(M)和缶「(M)均为完备的度量空间(见附录C中定义 C. 1,定义C.
2,附注C. 4). V X £亥「(M),存在X过p £ \emph{M}的极 大流虹(见附录B中定义B.
13,定义B. 15,和定理B. 16).为了讨 论方便,假设M是无边流形,从而使甲€
Diff\textsuperscript{r}(Af)可以往正、负向 无限延伸;使X
£多「(M)的极大流在(一 8, + 8)上存在(否
则要对微分间胚或向量场作适当处理).

\begin{itemize}
\item
  Diff,考虑从整数集Z到Diff\textsuperscript{r}(Af)的映射
\end{itemize}

Z -* Diff \textsuperscript{r}(Af), \emph{==
\textless{}p\textsuperscript{n}.} (1.2)

对固定的饱甲''是M上的一个单参数变换群;对固定\emph{nEZ,} B■给 出Mf
M的微分同胚.

\begin{itemize}
\item
  X £列\textsuperscript{r}(Af),考虑从实数集R到Diff 的映射
\end{itemize}

ax: RfDiff「(M), \emph{t} (1.3)

其中独是相应于X的流.对固定的X, 是M上的一个单

参数变换群;对固定*ER和所有的是Mf M的 微分同胚.

因此,我们可以把上而两种映射统一写成

?\}; Af -* Af,

当zeR时,它是由乐导出的连续流;当\emph{t} e z时,称它为离散流' 9\}满足:

⑴饱=id\textsubscript{M};

\begin{enumerate}
\def\labelenumi{(\arabic{enumi})}
\setcounter{enumi}{1}
\item
  \%。% = \%+,;
\item
  件怂)对,,工一并连续,
\end{enumerate}

有时也把驾称为M上的b微分动力系统.特别,当\emph{tez,}
称为离散动力系统.本书主要讨论连续流.由于它与离散流有密
切关系(参见下面的定义1- 6),必要时也讨论离散流.

定54.1.1设91: 如上,M中的集合

O\textsubscript{p}(x) =(9JU) I z € R(或 t£Z)\}UM

称为连续流(或离散流)9\}过工的轨il.如果把上式中的R(或Z)
改为R\textsubscript{+} (或Z+)或者R\_
(或Z\_.),则相应地得到过工的正半轨if 或者负半轨il,并分别记为0/
(工)或者\emph{0- (\^{}) .}

定义1.2过*的正(或负)半轨道的极限点称为Z的以或a)
极限点.工的全体火或极限点组成的集合称为x的3(成a)极
限集,记为《*S)(或a(z)),即

fc\textgreater{}(z) = \{y £ M \textbar{} 不+ 8,使% (x) -► y\},

\begin{quote}
r
\end{quote}

a(x) = \{y £ M I m if-*--- 8,使 \textsc{%,(h)}
显然,当M紧致时火H)W0,aO)M:0.

定义1.3设如上\emph{,\textsubscript{P}EM}称为游蔼点,如果存在
/\textgreater{}的邻域UUM,和某个正整数N,使V 1H\textgreater{}N,有%(u)nu =
0.不是游荡点的点称为非壽蔼点,\emph{9i}的所有非游荡点的集合称

为非游荡集,记\emph{为岫}.即

\textsc{«\textless{}p)}= {户ewi对户的任意邻域cz,m(,{]}i\textbar{}
\textgreater{}1,

使 性«;)nu\#。}.

显然,叭z) U ``(?\textgreater{}), «(x) C
因此,当\textsc{m}紧致时q(曲

尹.0.下面定义的临界元是皿3)的重要组成部分.

定义1.4设知MfM如上,M的连通子集 L = " I \%3)\emph{= P}对某一
t尹。成立} 称为一个临界元.

给出临界元的定义是为了陈述上简洁.有时要对临界元进行 如下细致的区分.

在微分同胚的情形,如果存在\&ez+,使得 \emph{g=p,}则*为
临界元.满足这个条件的最小数\&称为力的周期,并称/■为为周期
点.特别地,若周期为1,则称户为不动点.对于\emph{k}周期点\textsubscript{P},如果
\emph{M} 孫)的所有特征值的模均不为1,则称力是一双曲不动点◎
=1)或双曲周期点a\textgreater{}i).

在向量场的情形,临界元\emph{L}有两种类型.一种类型是,\emph{L}由一
个点*组成.在这种情况下,''况(力)=\emph{P} (或等价地X(Q =
0),此时称*为向量场的一个奇点.称*为一个双曲奇点,如果Y*
尹0,力是假的一个双曲不动点(或等价地,DX(p)的所有特征根
都有非零实部).另一种类型是,\emph{L}由向景场的闭轨\emph{7}组成.此时
v/\textgreater{}er,3 f尹0,使%。)=/■.这种最小的正数\textsc{z =
t}称为闭轨, 的周期.,称为双曲闭執,如果对某个/\textgreater{} e \emph{y}
(从而v p G,),
\emph{"蓟}的所有特征值的模除了一个以外都不等于1.注意,此时,
上的每一点都不是向量场的奇点,V \emph{P} € 7,x孫)是D\%B)的以
1为特征值的特征向量.

在研究向量场的轨道结构时,局部的困难在奇点附近(就奇
点本身而言,它无非是动力系统的一个平寳点,但在它附近的轨道
结构却可能千变万化);而整体的困难在于非游荡集的结构,它反\\
映出动力系统的本质特征.

定义1.5 M中的集合人称为甲的不斐集,如果有 OQ)UA.

不难证明,0'(工),火/, «(x), Q3)都是p的不变集.
在研究闭轨的分岔时,建立如下定义的Poincarfe映射是一个 很重要的手段.

\includegraphics[width=1.43333in,height=1.4in]{media/image2.png}JfeSCi.6
设y是 C\textasciitilde{}流知Af f M的一 条闭轨,夕£匕取M内
包含力点的一个如此 ``小''的光滑余维1子流 形U,使得流9}相应的
向景场在U上每一点与 \emph{U}都是无切的.从而 3 \emph{U',peU'CiU,}和在
U'上定义的仃函数T, 使得丁伝)等于7的周 期,且 6 \emph{U',
甲N\textgreater{}5eU.}由此可定义Poincar*映射

\begin{quote}
\emph{P: U---U,}户(夕)=传
\end{quote}

见图1-1.

•显然,P(招=力,即夕是映射F的不动点.利用Poinca琵映
射,可以把对O间量场在闭轨\emph{7}附近轨道结构的研究,转化为对
。微分同胚\emph{P}在不动点\emph{P}附近轨道结构的研究.

我们现在转向结构稳定问题.简言之,在``小扰动"下不改变
其轨道结构的动力系统是结构稳定的.

定义L7称两个向量场X]与拓扑執道等价,如果存在同
胚小MfM,它把X】的每条轨道保向地映到X,的相应轨道.称

两个微分同胚饥Mf M拓扑共鏡,如果存在同胚

便 \emph{\% =頌'}称xe wr「w)(或中EDiffrcW))。结构 稳定
XC,如果存在C*拓扑中的邻域U,X£UU务气肱)(或 中£
UuDifflM)),使y泌eu拓扌卜轨道等价于X〈或V \textless{}?• € \emph{u U}
Diff \textsuperscript{r}(Af)拓扑共辄于\emph{9\textgreater{}}).

当同胚血还保持\emph{Xi}与X,相应轨道的时间对应时,称为拓扑
等价.关于。拓扑的定义,见附录C.通常考虑C\textsuperscript{1}结构稳定性.
此时常省略`` C '',简祢为结构稳定.

在向量场的双曲奇点(或微分同胚的双曲不动点)附近,有下 面的局部结果.

定理1. 8 (Hartman-Grobman定理)设C7 U R''是包含\emph{O}点
的开集;向量场X以。为双曲奇点(或甲,C7-R"以。为双曲不
动点),则存在。的开邻域UUU,使X与其相应的线性场DXCO)
在V上拓扑轨道等价(或甲与其相应的线性映射D?\textless{}0)在V上拓
扑共规). I

定理1.9(局部结构稳定性定理)设xe 以O

为双曲奇点,则X在。点附近局部结构稳定,即存在X在务0")
中的一个邻域在。附近有唯一的双曲奇点p,且
y在P附近的某邻域与x在。的某邻域内拓扑轨道等价.\textbar{}

(对离散的情形,也有平行的结果.)

附注1.10如果把定义1.7中的拓扑轨道等价从C\textsuperscript{0}加强到 芒以}
1,即要求力为C*微分同胚,则X]与\&在相应奇点处的线
性化系统的特征根有相同的比值(见[GH,p42]).这就把等价关系
限制过严,使得两个轨道结构相同的向量场也未必等价.例如,由
定理1.9可知,二维系统

\emph{上=工,y = y}

是(局部)结构稳定的,它与系统

= x\textsubscript{t} \emph{y = y + p y}

(``H0)是C。等价的.但它们不是C\textsuperscript{1}命价的.另一方面,如果把定
义1.7中的扰动从加强到C\%则总可以扰动岀不同的轨
道结构,因而无结构稳定可言.例如,由定理1.9可知,一维系统壬
=纟是局部结构稳定的,但它的C°扰动系统工=去+产7M 与
原系统在原点的任意小邻域内都有不同的轨道结构,只要)Al
《1.因此,如无特别声明,下文中的等价都指C。等价,而扰动都 指U扰动.

在给出进一步的结果之前,我们需要下面的定义.

定叉1.11设U并同上.祢集合

"*0)= \{r£U仲%:)---0,当*---十 8\}

和

昨9)=值£ U\textbar{}广(工)-*0,当+ 8\}
分别为W在双曲不动点。的稳定流形与不龍定流形•

对于向量场的情形,有类似的定义.

定义1.12设7是向量场X的双曲奇点或双曲闭轨,集合

Wi(y)= Ue 当 t-*+8\}

和

叼(7) = \{z£ ---儿当 一 8)

分别称为\emph{X}在/的稳定流形和不穂定流形.

利用Poincare映射,可以把向匱场在双曲闭轨附近的研穽转
化为映射在不动点附近的研究.由Hartman-Grobman定理,
\^{}(0\textgreater{},lV«(0)(或昭XO),昭(0))可在O的小邻域内分别与线
性映射A = DX(O)(或明。))的相应集合它们是R"的
线性子空间)建立同胚.因此,w=(o)与irjf(o)(或 昭与
哪(0少在。点附近都是M的子流形.进而可以证明,它们在。
点分别与玲和E``相切,从整体上看归队。)与昭(。)(或
料(。)与畔(0))是M的b浸入子流形(但未必是\emph{M}的子流

形,见附录B中附注B. 24),证明可參考[ZQ,定理4. 9和4.10].

现在,我们可以陈述M.M.Peixoto在1962年证明的一个有关
全局结构稳定性的结果(参见[ZDHD]).

定理1.13设M是紧致的二维光滑流形,X £此\textsuperscript{r}(W).则
X是U结构稳定的,当且仅当

⑴X的非游弱集仅由临界元组成;

(2) \emph{X}的临界元(奇点或闭轨)个数有限,并且它们都是双曲 的;

(3) 任何双曲临界元的稳定流形与任何双曲临界元的不稳定 流形横截相交.I

这里横截性的定义见附录C中定义C. 10.注意,不相交也算
作横截.另一个有关结构稳定集的重要结果,是下面的

定理1.14二维可定向紧致流形结构稳定向量场的 集合在弟对\&
21)中是\textasciitilde{}个开稠集.I

附注1.15 Peixoto等人的上述结果,从60年代开始吸引了
在国外以Smale为代表、在国内以廖山涛为代表的一批数学家,在
微分动力系统方面他们做了大量工作(参见[L口,[Zzs]等).人们
把满足定理1.13中三个条件的向量场称为Morse-Smale向量场.
(类似可定义M-S微分同胚).自然要问:定理L 13和定理1.14在
高维流形上是否仍然成立? Smale在60年代构造的称为``马蹄"的
著名例子(见第四章),以及Newhouse随后对``马蹄''的改造,说明
上述问题的答案都是否定的,即在高维流形上的结构穗定向量场
(或微分同胚)未必是M-S的,而全体结构的稳定向量场(或微分
同胚)的集合在务「(M)(或Diff「(A/))中不一定是稠集.进一步
的问题是:当dimM \textgreater{} 2时,在亥'(M)中(或当dimM
\textgreater{} 1时,在
Diff\textsuperscript{r}CJW)中)结构稳定的充要条件是什么?对此Smale提出了比
M-S条件更广的公理A条件,并且给出两个结构稳定性猜测,结 构稳定K公理A
+强橫截条件;。稳定性(即限制在非游荡集上 结构稳定)司公理A
+无环条件..这两个猜测的充分性部分巳为

Smale本人和其它人所证明;必要性部分则于1987年分别由
Ma艮M和Palis四对微分同胚的情形给出了证明.对向量场情形
的第一个猜测,直到最近才由廖山涛B、胡森\$,和Hayashi聞、
文兰叫分别对三维和一般情形给出证明.

\textbf{§ 2}分岔与分鶴问题的提法

从上节定义1.7可知,若x e务7JW)不是结构稳定的,则在
\emph{X}的任一个C邻域内,都可以找到与\emph{X}轨道结构不同的向量场.
研究结构不稳定向量场在``扰动''下執道结构的变化规律,岌分岔
理论的核心内容.

分岔的柢念

定义2.1设M是光滑流形,£「(")是贸r(M)中结构稳定
向量场的集合,则集合=匆「(9\textbackslash{}苛 W)称为分岔集.

定义 2.2 设 若 X(e\textsubscript{0}) € ,

则称%为族X(e)的分岔值.当参数e逋过分岔值时,在相空间中
向量场XW)所发生的轨道拓扑分类的变化称为分岔.

对于离散动力系统,也可给出类似的定义.

由定理1.9可知,如果向量场在奇点附近发生分岔现象,则该 奇点必是非双曲的.

当M是紧致二维定向流形时,由定理1.13可知,分岔集
*(M)包含的向量场具有如下结构*它有非双曲的奇点或闭轨;
或者它的奇点和闭轨是双曲的,但它们的稳定流形和不稳定流形
不橫截相交(这时出现下面定义的同宿轨或异宿轨),或者它的非
游荡集含有无限多个临界元.

由定义1. 7易知,结构稳定集是\emph{函}\textsuperscript{r}(M)中的开集.当
M为二维紧致定向流形时,再由定理1.14知\emph{,S\textsuperscript{r}(.M}
)是开稠集,从

而分岔集是这个开稠集的``边界''.但在高维流形的情形,

可以有更复杂的结构(见附注L 15).

定义2.3 向量场的相執线称为奇点(或闭轨)的同宿
(homoclinic)轨,如果这轨线不是奇点(或闭轨)本身,而且它的«
极限集与a极限集都与这奇点(或闭轨)一致.相轨线称为异宿
(heteroclinic)軌,如果它的a极限集和a极限集是不同的奇点或 闭轨.

定义2.4发生在奇点(或闭轨)的小邻域内,并且与它的双
曲性破坏相联系的分岔称为局部分岔.笈生在有限个同宿轨或异
宿轨的小邻域内的分岔称为毋局部分岔.所有其余的分岔称为全 局分岔.

上面的定义2.1-2.4取自{[}AAIS{]}.我们将在后面看到,在研
究局部分岔时,可能伴随出现半局部分岔;而研究半局部分岔时,
也可能伴随出现全局分岔.

附注2.5如上所述,分岔集出= 是

倒''(M`)中的闭子集.如果限制在(M)上,仍可以考虑结构稳定 集 WjW)及其余集
MW) = \&r(M)是

中具有更复杂分岔现象的闭于集.我们可以继续对
\&「(M)进行这种剖分,得到各种层次的分岔集.

例2.6考虑R'上的向量场

否=心一产=---x\textsuperscript{8}). (2.1)

显然,x-0总是方程(2.1)的一个奇点.当产\textless{}0时,它是
(2.1)的唯一奇点,并且是双曲的(定义见§1).当户=0时,\textsc{h =}
0仍是(2.1)的唯一奇点,但它是非双曲的.当\textsc{a\textgreater{}0}时,除了工=
0之外,(2.1)还有两个奇点\textsc{h = 一} /花和三个奇点都
是双曲的.图1-2中给出(2.1)拘奇点分布对参数\#的依赖关系.
由此可以看出,当参数``变化通过值产=0时,(2.1)的奇点个数

\includegraphics[width=1.22639in,height=1.34028in]{media/image3.png}

图1-2 图1-3

发生了突变,从而表示奇点个数的图形在\emph{H} = 0处发生了分岔.习
惯上把这种分岔现象称为叉(pichfork)分岔.图1-3给出`` V 0,产
=0,兴>0三种情形下,(2.1)的相图.由此可以看出,相对于不同
的产,(2.1)的轨道拓扑结构所发生的变化.显然'' =0是唯一的 分岔值.

例2.7考虑IT上的向量场

票=产一孙 (2.2)

与上例同样的分析可知/= 0是(2. 2)的唯一分岔值泌V 0时向 量场(2.
2)无奇点;当《>。时,(2.2)有二个双曲奇点;当产=0时
(2.2)有唯一奇点仝=0,它是非双曲的,称为鞍结点
(saddljnode),并把这种分岔现象称为鞍结点分岔.与上例相似,
可分别作出图1-4与图1-5.从下面的例中,我们可以对这一名称 有更直观的理解.

例2.8二维的鞍结点分岔.考虑R2上的向量场

\includegraphics[width=2.6in,height=0.65972in]{media/image4.png}

•不难得知产=0是唯一的分岔值,并可作出轨道拓扑分类图

\includegraphics[width=3.57361in,height=3.32639in]{media/image5.png}\includegraphics[width=1.12014in,height=1.44028in]{media/image6.png}\includegraphics[width=2.65972in,height=1.21319in]{media/image7.png}

(二维鞍结点分岔),见图1・6.

例2. 9考虑W上(0,0)点附近的单參数系统族

与="一 丁 ---工3 + y\textsuperscript{8}),\\
尊 =z + 削一丿(史 + 3\textgreater{}\textsuperscript{2}).

它的线性部分矩阵以产士 i为特征值.在极坐标变换下,方程(2. 4) 变形为 '

糸=心-产)

\begin{quote}
\emph{de} 1

瓦i.
\end{quote}

当``wo时,原点是稳定的焦点以=0时非双曲);当``\textgreater{}o时,原
点为不稳定焦点,并有唯一闭轨/= 它是稳定的极限环匚,

见图1-7.这里,我们把平面上的孤立闭轨称为极限环;它附近的
轨道以它为。极限集,因此称它为稳定的极限环.注意,当产f 0
时,C趋于奇点r=0.

\includegraphics[width=0.52639in,height=0.77361in]{media/image8.png}\includegraphics[width=0.92014in,height=1.16667in]{media/image9.png}

显然,`` =0是一个分岔值.我们可以这样描述这个分岔现象:
当产的取值从小到大通过0时,奇点(zw) = (0,0)改变其稳定
性,并从此奇点``冒出''一个极限环.这称为Hopf分岔.

,例\emph{2.}10 \textless{}{[}DL{]})首先考虑一个平面系统

\emph{dr\\
}瓦

(2.5)

它有两个奇点皿(如)为鞍点-\textsuperscript{OC0}-°\textgreater{}为中心.容易从\textless{}2.5)消
去得到首次积分

+ X\textsuperscript{2} --- X\textsuperscript{3} = C, (2.6)

其中\emph{C}为任意常数.为了下文中的方便,记

C =吝一``, (2. 7)

\includegraphics[width=1.65347in,height=1.41319in]{media/image10.png}

图1-8

则方程(2.5)的轨道依产的不同选取,有如下分布:

(1) 当产=0时,由(2. 6)得到鞍点分界线殆UH。,其中队是
鞍点A的同宿轨道,而把相平面分成左右两部分(见图1-8).

(2) 当产V 0时,由(2. 6)可确定一条(无界)轨道七,它位于
匚所围有界区域之外和\emph{払}的左侧;当 Lb时,%收缩到队L)
\emph{H\textsubscript{a}.}

(3) 当产> 0时,由(2. 6)可确定二条轨道,其中一条为闭轨
球,它位于殆所围有界区域的内部;另一条(无界)轨道\emph{H*}在\emph{H\textsubscript{a
}}右侧.当产f 0+时,珏f匚月户\^{}

令函数

\begin{quote}
= F(w)\_ 陽一』,
\end{quote}

其中的F(z,少由(2. 6)式定义.利用此函数构造向量场族X``如

\begin{quote}
dx
\end{quote}

糸=\textasciitilde{} \emph{--- \^{}X\textbackslash{} + f(.x,y,\^{}y.}

容易算出

dF \emph{IdFdx . 9F e} 、\textsubscript{a}

d? \textsubscript{(},\textsubscript{8)} \textsuperscript{=} ;\^{}dF
\textsuperscript{+} ¥d7)\textsubscript{u}.\textsubscript{a)} =
2/(-3-,\^{},

由此可知,当I产\textbar{}《1时,系统(2. 8)的轨道分布如图1-9所示.(所

\includegraphics[width=3.45972in,height=0.98681in]{media/image11.png}

(a)Y° \textless{}b\textgreater{} \textsc{a=0} (c\textgreater{}
\emph{/\textless{}\textgreater{}0}

.图 1-9

用的论据,类似于用Liapunov函数判断奇点的稳定性).并且容易
得知,当0\textless{}产《1时,(2.8)的唯一闭轨就是(2. 5)的闭轨当 产f
0+时,它趋于X。的同宿轨

在例2. 6 ---例2. 9中,分岔现象都是由于奇点的.非双曲性而
发生的,属于局部分岔■本例则不同,分岔现象是由于同宿轨(对应
于山=0)在扰动下(``尹0)破裂而发生的,称为同宿分岔,它是一 种半局部分岔.

例2.11考虑映射F,史它的线性部分以一1为特
征根,由下文中§4例4.15可知,无妨设

F(Q=---工+ 技 + 0(匕\textbar{}5), (2.9)

其中在附近的一个邻域内,映射F以去=0为唯一
不动点■注意,由隐函数定理可知,(2. 9)的任一扰动系统在工=0
附近仍有唯一的不动点,所以我们不妨取它的扰动系统保持;c = 0
为不动点,且具有下面的形式

\&(工)------(1 + ")Z +
\textless{}Z\textsubscript{2}(/i)X\textsuperscript{2} + 角(``)工'+
0(\textbar{}z\textbar{}'), 其中``e R1,光滑函数如满足但(0)=0,角(0) =a.考
虑F户的两次叠代映射,得到

碎(z) = (1 + p)\textsuperscript{2}x + - (2 +
O(\^{}))ax\textsuperscript{3} + 0( \textbar{}x\textbar{}*),

从而风(/一工可表示为

工E(2 + ``)士 0(\^{})x一(2 +。(兴))心2 + 0( \textbar{}zI,){]}.

因此,利用4=0及下文中的1\^{}1\textsubscript{8}也峥定理(定理2.12)可知,当
用\textgreater{}0且\textbar{}``\textbar{}《1时,"除了有不动点\emph{x} =
0(它是F户的不动 点)之外,又有两个新的不动点,它们是F''的2周期点(图l-lO(a)

\includegraphics[width=3.50694in,height=1.66667in]{media/image12.png}

图 1-10

相应于a\textgreater{} 0的情形).如果把上面的映射F户看成一向量场的流
的Poinca止映射,则当产的值从负到正的瞬间(设a\textgreater{}0),原有的稳
定闭執变为不稳定闭轨K,而在它的邻域内又产生了一个稳定的
几乎两倍于原周期的闭轨\emph{L,}这种分岔现象称为倍周期分费,它发
生在一个M6bius帯上(见图l-10(b)),7,位于这带的轴线,而為是 这带的边界.

分岔问■的提法

\begin{quote}
从实际中产生的分岔问题,常常是带参数的向量场族.例如 等=y(H,``), (2. io)
\end{quote}

其中f X R*,RX).当产=0时,相应系统

是结构不稳定的.当1产1《1时,常把(2.10)称为'(2. II)的一个
D开折(unfolding).我们要研究的是:

问能否找到兴=0在R*中的一个邻域V,使得当``在
V中变动时,弄清系统(2.10)的轨道结构如何变化?

在较简单的情况下还可以考虑,能否把V分成若干于集,它
们对应(2.10)的轨道拓扑结构的不同等价类?在例2. 6中,\textsubscript{P}=0
把参数空间`` £ R'分成二部分,其中``\textless{} 0和K\textgreater{}
0相应于系统 (2.1)两种不同的且分别结构稳定的轨道拓扑类型.

进一步的问题是*

问IRB能否找到(2.11)的开折,它``包含''了(2.11)的任一
开折所能出现的轨道结构?

问塩C能否找到(2.11)的开折,它满足问题B的要求,并且 含有``最少''的参数?

对上述问题中一些名词的确切含义进行街清,将会引导到``普
适开折''和``分岔的余维''这样一些深刻的概念,我们将在§5中
介绍,本节先对这些概念给出直观的描述.这里需要指出的是,由
于向量场分岔集合可以有任意复杂的结构,一般而言,对上述问题
的回答是非常困难的.在多数情形下,只能对问题A作出部分回
等.但当系统(2.11)的相空间维数较低且它的``退化程度''不高
时,目前对问题B和C已有一些完整的结果.

下面以R上的系统

dx \textsc{a}

广r

为例,回答上面的问题.例2. 6中的系统(2.1)是(2.12)的一个开
折,我们要证明,(2.1)不是满足问题B要求的那种开折.现在考
虑(2.12)的任意一个C™开折

\emph{笑=函,}/(x,0) = --- x\textsuperscript{!}

这里\emph{心,舟}在(0,0)
eRXR\textsuperscript{m}的一个小邻域中定义是一个足
够大的正整觐在进一歩讨论之前,我们需要下面的结果,它可以
看成是鹿函皴定理的推广,

定理2.12 (Malgrange定理)设\emph{UURXR''}是包含原点的 开 C-Cf/.R)并且满足
/(t.o) = r*g«),其中 \& e Z+ 階在

\emph{t} = o附近是光滑函数,且莒(0)尹0.则存在R X R"中(0,0)附近
的邻域VUU和光滑函数g,z),以及R-中。点附近某邻域中的 光滑函数
ao(z),'',,aiGr),满足 g(0,0)尹 0,\&(。)= •`` = \^{}\_,(0) =0,以及

六£口)= + 袞%愆)皿 V (f,x) \emph{\& V\textsubscript{t} I}

r = 0

证明可见{[}CH,pp43 ---
45j.定理中的\emph{C\textasciitilde{}}光滑条件后来减弱
到有限光滑性.由上面的定理可知,对于开折(2.13)存在R\textasciitilde{}\textsuperscript{+1}中
零点的邻格V,以及光滑函数q(z,``),a(``),\&(Q和c(Q,满足

q(0,0)尹 0, a(0) =3(0) =c(0) = 0, '(2.14)

以及

--- g(z,``){[}a(*) + --- x\textsuperscript{5}{]}.

因此,当(S/0在RXRm中(0,0)点附近的一个小邻域中取值时, 系统(2.13)与系统

\emph{晉=}+ --- x\textsuperscript{3} (2.15)

有相同的轨道结构.注意上式右端可改写为为顷)+
为3)徉一弩)一@一曾「,其中松)=心)+\^{}\^{} +姓潛,W) = E +也¥令,=H
\_号,然后把 少改写为曰则系统(2.15)变为

糸=*(夕)+制夕元一H (2.16)

显M.C2.16)所能出现的轨道拓扑类型不超出系统

石=\% +人注一*3, (2.17)

所能出现的轨道拓扑类型,其中4,兀6 R'是独立的参数.至此我
们证明了(2.12)的任一开折(可以含有任意多个参数)所能出现
的轨道拓扑结构,都含于双参数开折(2.17)的轨道拓扑结构类型
中.即(2.17)就是问题B中所要求的开折.这时我们称(2.17)是
(2.12)的一个普适开折.对(2.17)进行定性分析不难得知,在A
=(爲,為)平面上使所对应的系统(2.17)成为结构不稳定的点的
集合为原点0(0,0)和由

孔I 4 =±2 囹七 \&>。}

给出的两条分會曲线「+,「-,它们在O点相切,并形成尖点,见分

\includegraphics[width=1.25972in,height=1.44653in]{media/image13.png}\includegraphics[width=1.87986in,height=1.62014in]{media/image14.png}

图 1-11 图 1*12

岔图1-11.相应于不同的A,开折(2.17)的5种執道拓扑分类中有
2种是结构稳定的(分别对应于r\textsubscript{+}u r\_把平面分成的两个区
域),有3种是结构不摄定的,显然,例2. 6中开折⑵1)所能出现
的轨道拓扑类型都含于其中.图M2给出方程(2.17)的奇点个
数对人的依赖关系,图中的``奇点曲面"方程为A + A\^{}-x\textsuperscript{!}
= 0, 沿z轴方向穿越此曲面的次数给出系统的奇点个数.曲面折叠部
分的``边缘''向以垃平面投影,就得到图1-H中的分岔曲线r+ 和

最后,我们来证明(2.12)的任一单参数开折都不具有(2.17)
的如上性质,从而(2.17)还满足问题C的要求,此时称由(2.12)
引出的分岔为余维2的.事实上,由于(2.12)的任一开折都可等
价地转化成(2.15)的形式,所以我们只需在三缝空间(a,方,c)的
原点附近的某邻域U中考虑即可.V产,当\textbar{}产\textbar{}《1时,向量场(2.
15)对应U中的一点.设△为U的一个子集,其中的点对应的向量
场与(2.12)有``相同的奇异性'',即

△ = ((a,们c) I 3 ,使 a + + ex\textsuperscript{2} ---
x\textsuperscript{3} = --- (x ---
x\textsubscript{0})\textsuperscript{3}\}.

故厶中的点满足

a = x\textsubscript{0}\textsuperscript{3}, \emph{b ------} 3端,\emph{c
=} 3x\textsubscript{0}.

由此不难得知,厶

是U中的一条曲线

(余维2子流形). '

所以在[/中至少二

维的曲面(相应于 \emph{I}一十 \textbf{7}

(2.12)的二参数 / \textsubscript{0}1 {/} \textsubscript{:}

开折)才能与厶横 / \emph{A /}

截相交,见图 \emph{L}一\emph{\textsc{J-j.---/}}

1-13,而(2.1.2)的 \textsubscript{0}/ \textsuperscript{/}

单参数开折所对应

的U中的曲线虽然 图冃3

也可以在。点与厶 •

相交,但在任意小的扰动下,它都可以与A分离.换句话说,至少

在二参数的开折中,像(2.12)这样的分岔现象才是``不可去''的.

注意到(a,盼坐标平面在原点与△橫截,所以(2.17)同时具有问

题B、C所要求的性质.

附注2.13利用同样的推理不难证明8当4死。时JV上的系 统专=中E +。(
\textbar{}z\textbar{}'' 是余维\&的,它的一个普适开折可取为
牙=角+心十\ldots{}+ QT + /■七特别地,例2. 7中的系统 (2.
2)是\emph{驚=---}的一个普适开折,但糸="---产则不是•

我们将在§ 5中把这里的讨论精确化、一般化.

\protect\hypertarget{bookmark23}{}{}\textbf{§ 3} 中心流形定理

在考虑一个向量场的分岔问题之前,一般要进行简化处理.
一方面,希望在不改变动力学性质的前提下把相空间维数尽可能
降低;另一方面,力求在等价意义下把微分方程的形式尽可能化
简.前者要用中心流形定理,将在本节介绍,后者要用正规形理论.
将在下节介绍.这两节中的讨论主要对R"中的向量场迸行,不难
把结果推广到微分同胚的情形(例如,参见[Wil]).

观察上节图1-6可以发现,二维空间上的分岔现象其实主要
是由动力系统在一维不变流形A = 0上的结构变化所决定(参照
图1-5),而在这不变流形之外的轨线,无非是向这不变流形的压
编.岀现这种规律并不是偶然的,系统(2. 3)在(0,。)点的线性部 分矩阵为

0 0 \textbackslash{}

0 - 1J '

它分别以0和一 1为特征值,它的非双曲部分是一维的.由此猜
想,当微分方程右端在某一奇点的线性部分矩阵有羚个零实部特
征根,"2个非零实部特征根时,可以把分岔现象的研究,在奇点附
近限制在某一个吼维的不变流形上,从而使问题的难度得以降 低・

线性情形

先考虑线性方程

(3.1)

其中为*阶实数矩阵.我们知道,方程(3.1)的解 \%(z) = e***x (3. 2)

的性态完全被矩阵A的特征值的性质所决定.

设矩阵A的特征值的集合为。),则

。=c UU \emph{\%,}

其中

== \{A G \textless{}r\textbar{}ReA \textless{} 0\},

\emph{a\textsubscript{u}} \sout{d} \{A G tf \textbar{}ReA \textgreater{}
0\}, (3. 3)

\emph{a\textsubscript{c}} \sout{d} \{A G \emph{o\textbackslash{}}ReA =
0\}.

记峦为R''中相应于人\& \%的那些特征值的广义特征向量所张成

的子空间;并可类似定义砂和群,则有直和分解

和相应的投影

旳:R" -* £•, jr\textsubscript{Bl} R" -* £?*, ?Tc: Iff 群.

•这些投影映射的零空间分别为

kerOrs) = E\textsuperscript{u}\textsubscript{,}

ker(ff„) =E«丄群由 E\textbackslash{}\\
kerSQ = \emph{E\textsuperscript{k} E\textsuperscript{s} ® E*.}

上述投影都与A可交换,故E,,E``,卧都是(3.1)的不变子空间,

当+ 8时,从非奇点出发的轨道在\emph{E*}中是指数型``压缩''

的,而在欧中则是指数型``增长''的Qf --- 8时的情况相反).所 有对f f 士
8有界的轨道(特别地,所有奇点,闭轨)都停留在群
内.由于这些性质,通常称玲为稳定子空间,玲为不稳定孑空间,
歩为中心孑空间,= ①身为双曲子空间,并记投影昭:
\emph{E\textsuperscript{h}.}轨道的动力学行为在双曲子空间内是单纯的,而复杂现象发
生在中心子空间快内.

非线性情形

现在考虑3讖性方程

= Ax + /(x), (3.5)

其中 \& C*(R",R"),点 J(0) = 0 且 D/(0) = 0.问题是:方
程(3.5)的轨道结构是否仍然具有方程(3.1)的上述规律?下面
的结果表明,线性方程(3.1)的中心子空间膏推广为非线性方程
(3.5)的中心流形"七虽然在较强条件下\emph{W\textsuperscript{c}}可以整体存在,但通
常实用的还是在奇点工=0的局部.那些``复杂现象''(特别地,所
有奇点、闭轨、同宿轨、异宿轨等)都发生在"上;在一定条件下,
吁外的解指数型地趋于上的解\$且流形胪上解的性质,可通
过对膏上诱导的方程的研究而得到,本节的内容主要参考了 {[}V{]} 和{[}CLW丄

我们先陈述整体的结果.

\begin{quote}
定义3.1设X,Y为Banach空间G Z\textbackslash{}定义映射空间
cj(x,y)丄(geb(x,y)\textbar{} g的 b模有界}.
\end{quote}

在本节中,记以£口)为(3.5)的满足初值条件敏0,H)=z的 \#•并记 IQgll =
supIDgCr)

定理3.2 (全局中心流形定理) 对于系统(3.5),存在与矩
阵A和数\&有关的正数毎,如果須e3(R'',R"),且12刃\textbar{}<施,则
有下列结论:

⑴集合

= \{x G R" \textbar{}
sup\textbar{}ff\textsubscript{A}x(z,x)\textbar{}〈8\} (3.6)

ten

是(3.
5)的不变集,它是R''的C\textsuperscript{4}子流形,即存在唯一的\emph{寧}e
别),使

叩=\{五+中(气)I払£ £\}, (3.7)

⑵如果有少£团(硏矽),使集合

\emph{M\textsuperscript{c}} = \{x\textsubscript{c} + 步(arc) I 气 \& £\}

是(3. 5)的不变集,则"=俨,且步=甲;

(3)如果\emph{y\^{}W\textsuperscript{c},}令*\$) =\%2(£"),则此Q)满足方程

\emph{= Ax\textsubscript{c} + \^{}cf(.x\textsubscript{c} +
\^{}(x\textsubscript{c})), x\textsubscript{c}} 6 \emph{E\^{}.} (3.8)

定义3.3定理3.2中的不变集硏称为(3.5)的全局中心流 形.

附注3. 4定理3. 2中有关全局中心流形唯一性的结论(2) 指出,若研在(3.
5)的流下不变,则\emph{平}e渺)是唯一■确定
的.如果改为PEUCE\textsuperscript{1},?),则结论一般不对,见{[}Sij{]}.

定理3. 2中的条件\textbar{}\textbar{}"\textbar{}\textbar{}
V\&是很强的,它使得该定理实际上 很难应用.由于/(0) = 0,D/(0) =-
0,故在奇点* = 0附近这个条
件却是自然成立的.因此,利用截断(cut-off)■函数从定理3. 2得出
的局部结果更自然,从而更实用,取截断函数\#3) e
C\textasciitilde{}(R\textbackslash{}R)\textsubscript{r}
满足0\textless{}7UXl.且

\begin{quote}
当 IEIW1,
\end{quote}

\textsuperscript{V X \_} Io,当
\textbar{}\textbar{}x\textbar{}\textbar{}\textgreater{}2.\\
再令

/\textsubscript{p}(x) = 言),Y z £ R气 (3. 9)

此时,为了研究方程(3. 5)在x= 0附近的中心流形,我们可以考 虑方程

峯=曷 + /\^{}(矿. (3.10) 显然,当IE w P时,/(X)=
\emph{f\textsubscript{p}(.x)},•且不难证明

当 0. (3.11)

定理3.5 (局部中心流形定理)设

/(0) =0,D/(0) = 0, W3\emph{咋砂聲、砂})和x = 0在R*中的开 邻域贝使得

\begin{enumerate}
\def\labelenumi{(\arabic{enumi})}
\item
  流形
\end{enumerate}

\%= {zc + pS)) \textbar{} 孔 £ 峦} (3.12)

对(3.5〉的流局部不变,即

xa.x) e V X € W\^{}n \emph{U,Y t£} U), 这里冬。,工)为(3.
5)的满足7(0,工)=z的解,西G)为Z在U内 极大流相应的时间区间;

\begin{enumerate}
\def\labelenumi{(\arabic{enumi})}
\setcounter{enumi}{1}
\item
  PC0) = 0, Dp.(0) = 0,
\item
  如果 x € (7,且 J\textsubscript{w}(x) = R,则 h \& 吟
\end{enumerate}

证明 设与A相关的标已经确定(见定理3. 2的条件),则由 (3- 9)和(3.11),可取
\emph{P \textgreater{}} 0,使 4危)\&
Cj(R"\textsubscript{t}R\textsuperscript{n}),且
\textbar{}{]}Dfp\textbar{}\textbar{} \textless{}
\&.对系统(歸10)应用定理3. 2可知,存在由(3. 7)给出的甘子痈 形吓,其中
3〈步,矽),且 = 0,Dp(0) = 0.

另一方面,由(3. 9)可知,若取C/W-{工\textbar{}
\textbar{}国)\textless{}祖,则 系统(3.10)与系统(3.
5)在U中完全相同,故结论(1), (2)成立. 限制在\emph{U}内,此处的吼■就是(3.
6)或(3. 7)式定义的\emph{W\textsuperscript{c}.}

现设e \emph{UMx) =} R,则 v \emph{t €} =2/(\^{}) \emph{CU,}

从而sup()rj\textgreater{}x(r,x)\textbar{} V8,由(3.
6)式,工C俨.限制在U内,也就 是故结论(3)成立. \textbf{I}

定义3. 6 如果W £ 廿(£ ,£*)M21,0(。)= 0,卬(0) = 0, 使U\textsubscript{c}
+(S(x\textsubscript{c}) I充£膏}在(3. 5)的流下局部不变,则称 旳 为(3.
5)的一个C*局部中心流形.

附注3. 7显然,可以对(3. 5)取不同的截断函数,而得到不
同的局部中心流形(尽管对每一个截断函数而言,(3. 1。)的全局
中心流形是唯一的).例如,在图卜6(产=0)中的原点附近,取右
半平面上与x轴相切的任一執线,再拼接上坐标原点及负工轴,都
构成一个局部中心流形.但从定理3. 5的结论(3)可知,(3.5)保
持在U内的任何有界轨道(包括奇点、周期轨、同宿轨、异宿轨等) 都出现在(3.
5)的任一局部中心流形上.因此,对于研究分岔现象
而言,局部中心流形的不唯一性不是一个重要的问题.还要指出,
虽然/的c*光滑性保证了叼的!光滑性,一■般来说,/■的c-光
滑性(甚至解析性)却不足以保证讨會是L的.事实上,从定理的
证明中可以看出,U是以\textsubscript{P}为半径的球形邻域,而\emph{P}的选取要保证
IIDf』\textless{} 眼.一般来说,当时'理f 0,这可能导致Pf 0.

现在我们把局部中心流形、稳定、不稳定流形的结果合写成下 面的定理.

定理3.8 对于方程(3.5),设/' £。"11气产),六0)= °, D/(0) =
0;相对于有如上所述的子空间玲,疥和 \emph{硏} 则在R"
中工=。附近存在开邻域U,和(7中的C*流形和它们
的维数分别与这三个子空间相同,在n = 0点分别与E,夢和\emph{卧}
相切,并且在(7内是方程(3. 5)的不变流形;和W"有定义1.12
所表示的形式,有表达式(3.12),其中C*(ESE"), \textsubscript{P}(0) =0, Dp
(0) = 0. I

类似于对中心流形的讨论,可以定义并讨论(3. 5)的中心稳
定流形眼涉和中心不稳定流形叶,这里不再详述.上面已经说
到,系统在奇点附近的``复杂现象''发生在它的任一局部中心流形
上.下面的两个定理说明了中心流形的其它重要作用.

定理3.9 (断近性质定理)设須£ U(R",R")J(0) = 0, D/(0) = 0,且对于矩阵='0,令
眇为(3. 5)的一个萨局部
中心流形,则可在R"中找到\emph{O}的一个邻域V和正数匕如果X e V,且傍I
i\textgreater{}0}的闭包含在V内,则m to\textgreater{}O,M\textgreater{}
0和;

v,使得

\emph{--- x(t} 一 知力 \textbar{} V *2 \%. (3.13)

定理3.10 (Pliss约化原理)在定理3. 9的条件下,设第\& \% Cl V,且{珀G
\textbar{}/\textgreater{}0}的闭包含于V中.则虱3)作为3 5)的
解是稳定(渐近稳定,或不稳定)的,当且仅当皿Q)作为(3. 8)的
解是稳定(渐近稳定,或不稳定)的.I

定理3. 9说

明,在一定条件下,

在奇点0的一个小

邻域V内,中心流

形外的解可以指数 V- \emph{、 /}

型地趋于中心流形 \emph{/―{\textasciitilde{} \^{}\^{}}\textasciitilde{}7}

上的某一解(当£ / /

土+8,如果%= / /

0 ;或当 r f --- 8, \sout{\emph{L ,} 一 \emph{J}}\emph{g:}

如果彳=0).而

定理3.10说明,在 图i\_M

类似条件下,为了

得到局部中心流形上。点附近的轨道结构,只需要对它在线性子
空间度上诱导的方程(3. 8)来研究即可.事实上,(3. 8)的轨道是
(3.5)在IT上的真实執道向\emph{氐}的投影,见图1-14. 一般而言,从原 方程(3.
5)得到诱导方程(3. 8)不是容易的,需要先知道啊为此我 们给出下面的定理,

定理 3.11 设C*Or,R":M2 1,/(0) = 0,D/(0) = 0,

CYESE*), p (0) = 0, Dp (0) = 0.则 IV\textsubscript{?}---\emph{3 +
\textless{}p} (x\textsubscript{e})

I x\textless{}6玲)}是(3. 5)的一个局部中心流形,当且仅当存在矽中原
点的开邻域Q使得V女e捉,有

Dp (x\textsubscript{c})ff\textsubscript{c}(Ax\textsubscript{c} +
/(x\textsubscript{e} + \emph{\textless{}p} (zQ)

\begin{quote}
\emph{=甲} Sc) + \emph{f3 + 乎} Sc)). (3,14)
\end{quote}

在很多情形下,我们并不需要知道?\textgreater{} (女)的确切表达式,而是
利用(3.14)式算出它的Taylor展式的前几项.

为了简单,先把(3. 5)化成如下的标准形式:

\begin{quote}
专=\textsc{bh +}
\end{quote}

- (3.15)

\emph{= Cy + g(.x,y),}

其中= 的特征根实部均为零,

而

\textbf{c={[}G «1,}

\emph{\textsc{{[}q}} \textbf{cj}

上面的G与G的特征根实部分别为负数与正数.因此,卧N S, 0)\},\^{}=
\{(0,y)\},W。= ((x,p(\^{}))\textbar{}xeR\textsuperscript{n}).T面尝试寻找
jy =卩3)的展式-X 6 R". (3.14)现在成为

'Dp (z){[}Bz + (z)){]} = \emph{C\textless{}p} (x) + g(z,p Cr)),

即

Df + /(x,p (z)){]} --- Cp (x) --- (or)) = 0,

以及条件

\emph{\textless{}p} (0) = 0, Dp (0) --- 0.

利用待定系数法,可逐项计算甲Or).

例3.12考虑二维方程

\begin{quote}
dx

击=夕,

* =印 + ® + \emph{xy,}
\end{quote}

卷过扰动在奇点(0,0)附近可能发生的分岔现象,其中\emph{9*.}注
意(3.17)在(0,0)点的线性部分矩阵为{[}:》,因此当\emph{財0}时,

它有且只有一个零根,中心流形是一维的.我们首先设法找出方
程(3.17)在呂上诱导出的方程,由此推断在中心流形上執道的结
构.为此,先把(3.17)化为(3.15)的形式.

\includegraphics[width=0.33333in,height=0.37986in]{media/image15.png}令

硏 7)(: 则(3.17)化为

\begin{quote}
晋=---*(" +汾'一*(`` +P)t\textgreater{},
\end{quote}

-. .' 1 (3.18)

\begin{quote}
* =伽 十 彩"+ V)2 + \emph{帯(U +'V)V.}
\end{quote}

注意到(3.16)式下面的条件,我们可以设(3.18)的中心流形 由函数

\begin{quote}
\emph{v = («) =\textless{}»? + \&«? +}
\end{quote}

所表达.对方程(3.18)应用(3.16)式,并把P 8)的如上表达式
代入,由待定系数法不难得到

再把上式代入(3.18)的第一个方程,得出(3.18)在£上诱■ 导的方程为

夺一*必+\ldots{}肠0.

由定理3.10,附注2.13和例2. 7可知,系经扰动在中心流形上
发生鞍结点分岔.当\^{}\textless{}0时,它的拓扑结构与图1-6相同.

最后,我们考虑系统(3. 5)依赖于参数的情形.设

\begin{quote}
击=Az +
\end{quote}

这里对矩阵A的假设同上疥和世的维数分别为''十宀和册 /et/CR" X
R\textsuperscript{A},R"),r\textgreater{} 1, /(x,0)
=。(\textbar{})毗*).为简单起见,还\\
设y(0m)= 0,即z = 0总是(3.19)'的一个奇点.

定理3.13在上面的假设下,(3.19)拓扑.轨道等价于如下系 统

\begin{quote}
苧=虹膈),

* 堂=-为 7GR"-, (3.20)

券 T, reR\textsuperscript{n+}.
\end{quote}

这个定理的证明可参考[Sh丄实际上,它的第一个方程就是 (3.的.

附注3.14为了研究(3.19)的中心流形,我们把产也视作 (空间)变量,考虑

等= 4z + \_f(z,Q, 罪=0. (3.21)

題在奇点3,兴)=(0,0)附近,稳定与不稳定子空间仍为疥与
而中心子空间为\emph{E} X R*.故(3. 21)的局部稳定流形与不稳定流
形IT与的结构与"三0时类似.此时中心流形(Gr``G +
\textsubscript{9}\textgreater{}(及,``)丨3,``)e X R*},其中\emph{乎 6,心
£} b(彦 X R\textbackslash{}
卧).由(3.21)的第二个方程易知,{0,产)\textbar{}产=常数}是(3.21)
的不变集,从而对固定的产,"勺\#=常数是(3. 21)第一个方程的不
变流形.注意愆/) = (0,0)是(3.21)的非双曲奇点.一般而言,
对于不同的产,"°\textbar{}户=常数上的轨道结构可能不同,见下例・

\begin{quote}
例3.15 设考虑光滑系统 等=俱一J\textsuperscript{3} + ,
\end{quote}

\includegraphics[width=3.4in,height=1.50694in]{media/image16.png}

图 \textbf{1-15} 图 \textbf{1-16}

其中 /' =。( g = 且g愆,0,Q = 0.由附注

3.14和例2.6可知,在(工设,``)=(0,0,0)的小邻域内,中心流形
归。如图1-15所示.显然,当`` W 0或产〉。时,胪常数有不同
的结构.利用定理3.13及简单计算可知,系统(3. 22)拓扑轨道等 价于

輩H轉一措2

它在(S,彳m) =(o,o,o)附近的中心流形如图1-16所示,它不过是
把图1-15中的中心流形"摊平''在(扌,产)空间而已.

在下文对局部分岔的讨论中,我们大都假定已经把问题化归
到它的中心流形上,即对所论方程的线性部分矩阵\emph{A}而言 =化(即。3,4 =
0). '

\protect\hypertarget{bookmark32}{}{}\textbf{§4}正规形

正规形(normal form)理论的基本思想,是在奇点(或不动点)
附近经过光潰变换把向量场(或微分同胚)化成(在一定意义下,
尽可能简单的形式,以便于研究.这是源于Poincare时代的一个课

题.由于近年来分岔理论的发展,正规形的应用更加广泛,因而重
新引起人们对它的重视,并得出若干计算正规形的新方法,

\includegraphics[width=0.58681in,height=0.26667in]{media/image17.png}众所周知,经非退化线性变换z
\emph{= Ty,}线性微分方程

变换为

这里£ R'', A和\emph{T为 "n}矩阵,且detT尹0.因此,在讨论
线性系统的轨道结构时,我们无妨假设A为Jordan标准形,除了
Jordan块的排列顺序外,它是唯一确定的,接下去的问题自然是:
对非线性部分是否可以作类似的简化?在一定赢义下,答案是肯定
的,但一般不再有唯一性.无论如何,这种简化对进一歩研究分岔 问题是很有用的.

微分方程在奇点附近的正规形

考虑以x = 0为奇点的C\textsuperscript{9}■微分方程(r N 3),它在x=0附近
可表示为

专=膈 +产(工)+ \ldots{}+ 户-3 +0(\textbar{}z\textbar{}「),(4.1)
其中x€R"(或C"),4是线性部分矩阵元m
维\emph{k}次齐次向量多项式所成的空间\emph{,k = 2,}\ldots{},r - 1.

先进行变换

x = \textgreater{} + (4.2)

其中*\textsuperscript{2}(\textgreater{})\textsc{g}
\emph{m}待定,以使变换后的方程具有简单的形式.把变 换(4.
2)代入(4.1),并注意

口 +以气少尸=/ ―以气少+成\textbar{}计),

其中,是n X n单位矩阵,Jacobi矩阵D必(少的每一元素都是
。(旧1),而衣風')表示«Xn矩阵,它的每一元素都是。(\textbar{}卄),
由此把方程(4.1)化为'

尊=2 + \{卢(少一飾(少2 --- 础 g)j\} + ?(>)+

\begin{quote}
••• + 乒7(少+。(》1・, .(4.3)
\end{quote}

其中\emph{F}与(4.1)中的相同,而产是经过运算得到的新的\&次齐次 多项式.

引入算于ad\%:压f

= DA'GMy --- A 於丫少, (4、4)

则(4.3)变成

尊=A/ + 5⑶)一敬(静3)] +夕(少+

\begin{quote}
••• + >一'3)+。(\textbar{}从「), 〈4.5)
\end{quote}

记绥2为算子a眠在而中的值域,而时是"在玷中的一个补 •空间,即

\begin{quote}
\emph{H\%} =绥
\end{quote}

如果产3)£统%则存在\emph{h\textsuperscript{2}(y)em,}使产(少=ad*3(;y)),即
(4.5)中的二次项可以消去;否则,只能找到静《) G部,使 产\&)
\_ad*3(;y)) £ 彩 (4.6)

这样,我们把(4.1)化为

* = 4/ + 屮(少 + 元3)+ - + ?-*(>)+ o(i>io,

(4.7) 其中gp) 6\emph{硏}

其次,再考虑变换

\emph{y --- \^{} + h\textsuperscript{3}}(a), (4. 8)

其中静(言)6 \emph{H如.}重复上面的推理,并注意这个变换不影响■线性
项与二次项,容易得知(4. 7)变为

专=Az + *3) + [户伝)---adW(z)] + \emph{f*(.z) +}

\begin{quote}
...+ 戸-i(z)+0(\textbar{}z\textbar{}「), (4.9)
其中甘七户与以」)中的相同,而元,\ldots{},戸T是经计算得到的,而 且
\end{quote}

adj: \emph{-* Hn,} A\textsuperscript{3}(2) I---*
Dft\textsuperscript{3}(z)Ax --- Aft\textsuperscript{3}(z).

记绥』穿(ad\%),宙a为旋,在中的补空间,即

\emph{Hi,}=外3㊉笆;

则当戸G) e绥s时,存在\emph{h\textsuperscript{s}M,}使得经变换(4.
8)可消去(4. 7)中 的三次顼;否则,只能找到静(幻e叫,使(4. 7)变为

= Az + g,(z) + g,(z) + /'(«) +

\ldots{}+戸(2)+0(\textbar{}吁),

其中廿3) £贵*, \& = 2,3.由此递推下去,就可得到

定理 4.1 设 Xe\&「(R")(或艾「(C")),X(0) = 0,DX(0)
=A,并且\emph{x}有表达式(4.1),则在原点附近的邻域内存在一系列 变换

\emph{x = y + h\textsuperscript{k}(.y)\textsubscript{t} k =} 2,---,r ---
1, (4. 10)

其中爪。)e \textsc{h*,}经过这一系列变换(每次变换后把夕换回H),可
把(4.1)变成如下形式

糸=Az + 妒愆)+ \ldots{}+ 广,(工)+ 0( \textbar{}z\textbar{}"), (4.11)
其中廿(Q 6宙*,參*是\emph{豉=绥}3成)在\emph{Hi}中的补空间,算子
a鸵由下式定义

ad*. ad*(A*(x)) = DA*(x)Ax -

(4.12) \emph{k -= 2\textsubscript{t} --- ,r -} 1. I

\begin{quote}
定义4, 2
微分方程(4.11)的j次截取式(2\textless{}\textgreater{}\textless{}r- 1)
车=\emph{Ax} + g,(z) + ― + 决(z), (4.13)
\end{quote}

其中F(z) e宙';i= 2,\ldots{},j,称为方程(4.1)的J次正规形.

附注4. 3上面进行的只是有限步运算.当(4.1)右端可以展
成收敛的蓦级数时,这种歩骤原则上可以无限地进行下去,问题在
于变换本身和所得到的(形式)正规形是否收敛.在一定条件下,
结论是肯定的,这就是PoincarADulac定理,见[A口,或[CLWJ.

共振与非共振

在定理4.1中,我们希望进一步确定,C4.11)式中哪些寸S) -0.也就是说,(4.
1)中哪些户S)属于烫*?为此,我们假定4
己化成它的Jordan标准形,并引入共振的概念.

定义4.4特征值人=(為,•••,%)称为共振的,如果存在自然 数J 和整数组m 一
(»!!, --- , Win),其中四NO并且\textbar{}m\textbar{}

==N 2,使得

\^{}=1

d \textsuperscript{n}

* = 〈4.14)

\emph{i=l}

正数协I称作共振的阶.

例如4 = 2\&是2阶共振的;24 = 3馬不是共振的仄+為=
。是3阶共振的,因为它可以改写为A = 2A, +如

考察(4.11)中哪些驴S)不出现,就是要考察同伦方程 '

ad§("Gr)) --- 厶责 G)=产3) (4.15)

对给定的户o) e \emph{臨是否}有解住3)e\emph{砒.}

设\textsc{a}是对角矩阵,特征值七互不相同,勺是\textsc{a}相应于4的特征
向量.则构成一组基.设(气,\ldots{},角\textbar{})是相对于(«!,---
\textgreater{} 耐的坐标,则

\textbar{}说\textbar{}=应 (4.16)

就是中元素某一分量中的最简形式.

令方*Gr) = \textbar{}m \textbar{} =虹则D砂(z)Az中只有第s个分量非

夺,并且它等于

\includegraphics[width=1.85347in,height=0.62639in]{media/image18.png}

=(m,A)aT.

另一方面,由于勺是4的相应于4的特征向量,因此

\emph{A\^{}e\textsubscript{s}} = V\%

把上面的结果代入(4.15)的左端,得到

ad5\^{}"e, = ---
A\textsubscript{s}{]}x\textasciitilde{}e\textsubscript{(}, (4.17)

这说明ad々也是对角的,并且它的特征值具有r(«M)-妃\textbar{}的形
式.由此可知,当A的特征值非共振时,adf的所有特征值均非零,
故算子ad*可逆,同伦方程(4.15)可解.

当A有重特征倬时,A的Jordan标准形是上三角形矩阵.此时
a世也有相应的Jordan\^{},并且ad*的特征值仍具有--- 的形式'

定\emph{义4.5}向量值多项式\^{}■勺祢为共撮多项式,如果

4 = \textbar{}m\textbar{} \textgreater{} 2,

其中和\textbar{}m\textbar{}的意义同前.

利用上面的讨论,我们可以得到下面的

定理5 设4是上三角的Jordan标准形,则可适当选取变
换(4.10),使(4.11)右端的诸gtQ仅由共振多项式组成.\textbar{}

正规形的计算

\begin{quote}
例4.7求等=服+尹S十\ldots{}的二次正规形,其中* T ;)•
\end{quote}

解先在中取一组基

\begin{quote}
0
\end{quote}

\includegraphics[width=0.82639in,height=0.34653in]{media/image19.png}\includegraphics[width=0.17986in,height=0.14028in]{media/image20.png}\includegraphics[width=0.10694in,height=0.13333in]{media/image21.png}

现将空间H1与R6等同:\& =命,其中勺,\ldots{},%为R8中的标准
正交基,由Fredholm定理可知

R6 =绽(以)+ " (W ),

其中彳CM')表示£**的零空间.由(4.18)容易看出,

,) = Span(e,, \emph{e\textsubscript{2}} + 2e\textsubscript{t}\},

由此得到

死=亥(8电)\& 了七

其中\&」Span\{q32 + 2£j,由此得出二次正规形为

\begin{quote}
\emph{'2x1}
\end{quote}

\%=叼+ 2\&弱,\\
亨\emph{=a xl + b} 与4

注意,Span国,旦\}同样构成\emph{風}(ad£)的一个补空间.事实上, 取% =跖\&
= \& + 2勺,»1 = \textsc{Ei,w\textsubscript{2}} =电,则内积

\textsc{/ \textbackslash{} \_ a} \_ '当'=九

0,当,*j.

因此,可取Span\{£"与\}为亥(ad\%)的另一个补空间,而相应的二 次正规形为

\begin{quote}
dT \textsuperscript{= x}-
\end{quote}

d\^{}2 g -,\\
瓦• = OX: +如口 2.

用类似方法,可以算出\emph{k}次。> 2)正规形为

\begin{quote}
d£i \_ \emph{At \textasciitilde{}}
\end{quote}

警=心:(1 + PU0) + 如肉(1 + Q(zD),

其中\textsc{P(m)}与Q(©)是氣的\emph{k-2}次多项式,并且P(0) = Q(0)
=0.当沥关0时,可经过尺度变换把(4. 20)化为

亨=X?(l + \emph{P(m)}+ 7 \^{}2\textless{}1 + QGq)),

其中 \emph{q} = sign(oA) (+ 1 或一1).

附注4. 8由于薩(adl)的补空间不唯一,因此正规形也不是
唯一的,从上面的例子中已经看出了这一点.但当取定了一个补
空间后,正规形中的系数就唯一确定了.在例4. 7中,求正规形的 ''
---I\textsuperscript{-} \emph{k ---} 1 \textbackslash{}

方法称为矩阵表示法.由于dimAg = n 「,矩阵表示法

n --- 1 /

的计算量将随兀或火的增大而迅速増大•近年来又发现了计算正规
形的共辄算子法和群表示论法等,见王铎的综述文章[Wd]及其 所引的文献.

例4.9 考虑复方程

£;卜福+。\textbf{5}

其中

/ito 0

Af \_\textsubscript{ia}, ' *。,

即4有一对共辄纯虚特征根,求它的(形式)正规形.

鮮 我们用共振原理求解.记A=ia,\&=\_ia,则共振条件 是

\begin{quote}
爲=@ + 1)A\textsubscript{1} + 虬,\emph{k ------}
\end{quote}

(或\& =想+。+1)为,应=1,2.\ldots{})

由定理4. 6可知,复正规形为

\begin{quote}
= iwz + \emph{Ci} {[}z\textbar{}\textsuperscript{2}z + ― +
\emph{c\textsubscript{k}\textbackslash{}z\textbackslash{}\textsuperscript{ik}z
+ ---} (4. 22)
\end{quote}

(第二个方程与之共辗,故略去不写).

映射在不动点附近的正规形

考虑以工=0为不动点的(Z映射F(r \textgreater{} 3),它在h = 0附近 可表示为

F(z) = Ac + 产(工)+ \ldots{}+ 户\textsc{t(z)} + 0( \textbar{}z
\textbar{},), (4, 23) 其中H e R\textsuperscript{1}*(或C«)
\emph{,A}是线性映射(我们把它在某组基下的矩阵
仍记为A),户3)乍\emph{H*,H*}为''元4维左次齐次向量多项式所成
的空间\emph{,k = 2,-,r-l.}

考虑变换

z = H(y) 丿+ \#(»), (4.24)

其中於GO e \emph{H、}/ + «(.)在原点附近可逆,则(奴24)有如下的 逆变换

\emph{x ---} + O(\textbar{}z\textbar{}*+`), \textbar{}z\textbar{}《1.

令

G(')= 。\emph{F}。\emph{H(y),}

则可把(4. 23)化为

GJ) = * + 户(刃 + \ldots{}+ 尸-'3)+

{[}A少----础*0)){]}+O(b\textbar{}E),
\textbar{}y\textbar{}《L因此,与定理4.1平行,可得到

定理4.10 设F e Diff\textsuperscript{r}CR")(或Dif£「(C")),并且有表达式
(4.23),则在原点附近的邻城内存在一系列变换

\begin{quote}
\emph{\^{} = y + h\textsuperscript{k}(y'), k = 2, --- ,r-} 1, (4.25)
\end{quote}

其中林(少e叫,经过这一系列变换(每次变换后把了换回以,可 把(4.
23)变成如下形式

G(z) \emph{=Ax +} \^{}(x) + - + 厂(£)+ O(\textbar{}x\textbar{}9, (4. 26)
其中『(Q e 补是饱知=庭(%打在中的补空间,算子M 由下式定义

M: \emph{歸} = "(Ar)-\emph{球歎,}

\emph{k --- 2, --- ,r} --- 1. \textbar{}

定义4.11映射(4.26)的j次截取式(2M jMr--- 1) \emph{Ax} + g2(z) + --- +
尸(工), 其中厦(工)6\emph{审七i=2,`` •,厶}称为映射(4. 23)的J次正规形.

定5(.4.12 Jordan形矩阵A的特征值\textsc{A=3lA)UC"}称
为其振的,如果存在s(l MsM'')和整数组用=S1,\ldots{},Win), \emph{m;} d
\textsuperscript{n}

\textgreater{} o, \textbar{}m\textbar{} = \textgreater{} 2,使得

\begin{quote}
\textless{}=1
\end{quote}

A, = A\textsuperscript{M} Aj •••寸. (4.27)

正数g I称作共振的阶.

定义4.13向量值多项式称为共振多项式,如果«和s
满足共振条件(4.27),其中{勺,\ldots{},%}是C"的一组基,S,\ldots{},
办)为相对于这组基的坐标,并且矩阵A在此基下的Jordan标准形
以仇,\ldots{}人}为对角元素.

-类似于定理4. 6,可以得到下面的

定理4.14设A = Diag{爲,\ldots{},则可以选取适当的变换
(4.25),使(4.26)右端的诸事O)仅由共振多项式组成.\textbar{}

例4.15 求以z = 0为不动点的一维映射\textsc{F(z)=-h+}\ldots{}
的六次正规形.

解A=-l是唯一的特征值,所以共振条件为

A* --- A --- 0, \&N2,

也就是

(一 1)1 = 1, 4 \textgreater{}2.

所以共振多项式为工',史,\#,\ldots{}由定理4.14,六次正规形为
G(Z)=---"\textless{}«:3+狩,

其中``,\&为常数.

光滑线性化

定义4.16设志N2是一个自然数.称光滑向畳场(或微分同
胚)的双曲奇点(或双曲不动点)为\emph{k}阶非共振的,如果它的特征
根不满足所有阶的共振关系.如果一个奇点(或不动点)是任
意有限阶非共振的,则称它为无穷阶非共振的,或简称非共振.

从前面的讨论可以看出,一个応阶非共振奇点(或不动点)的月
次正规形是线性的.换句话说,在奇点(或不动点)的邻域里可以
找到一个多项式的坐标变换,使得在新坐标系下系统可以表示为
一个线性部分加上一个\&阶小量.一个自然的问题是,进一步可以
通过什么样的坐标变换能把这个\&阶小量去掉.

定义4.17设龙是一个自然数或A = oo.称R''上的光滑向
量场(或微分同胚)在它的奇点(或不动点)0处可以廿线性化,如
果存在点\emph{0}的邻域\emph{U}和C*微分同胚\emph{HiU} f R",W(O) =
O,使得 经过坐标变换纟f HGr)后,系统在。点邻域内变为线性的.

定理4.18 ([IY])设\&是一个自然数或及=8』是一个孙
阶实方阵,则存在一个依赖于点和4的数,满足

\begin{longtable}[]{@{}lll@{}}
\toprule
\endhead
\begin{minipage}[t]{0.30\columnwidth}\raggedright
\strut
\end{minipage} & \begin{minipage}[t]{0.30\columnwidth}\raggedright
£ Z,当Y 8,

〔8, 当 \emph{k = oa,}\strut
\end{minipage} & \begin{minipage}[t]{0.30\columnwidth}\raggedright
\strut
\end{minipage}\tabularnewline
使得如果原点是b微分方程 & &\tabularnewline
& 糸=〈+..• X \& R", & (4.28)\tabularnewline
或微分同胚 & X p* \emph{Ax} + \ldots{}X G R\textsuperscript{rt} &
(4.29)\tabularnewline
\begin{minipage}[t]{0.30\columnwidth}\raggedright
的片阶非共振双曲奇点(或非共振双曲不动点),则系统(4.

28)(或系统(4.29))可以C*线性化.\textbar{}\strut
\end{minipage} & \begin{minipage}[t]{0.30\columnwidth}\raggedright
\strut
\end{minipage} & \begin{minipage}[t]{0.30\columnwidth}\raggedright
\strut
\end{minipage}\tabularnewline
附注4.19 & 由于£ =\&,@,A)的表达式比较复杂,此处没有 &\tabularnewline
给出.当4 - + & 8时,世线性地依赖于应的增长. &\tabularnewline
\bottomrule
\end{longtable}

\begin{quote}
例4. 20 考虑R\textsuperscript{1}上的C"光滑微分方程

\& 、

瓦• = or +\ldots{} @尹0;
\end{quote}

或微分同胚

x I " 4■\ldots{} \textbar{}产\textbar{} 尹 0,1.

因为特征根A =赤或A =产)不满足任意阶共振关系,故宙定
理4.18,系统在原点处可以C"线性化.即R*上的C\textasciitilde{}向量场(或
微分同胚)在它们的双曲奇点(或双曲不动点丿处可以C"线性化.

例4.21由于辭上的向量场在双曲焦点的特征根为A士油,
A尹0,3尹0.它不满足任意阶共振关系,故軒上的C"向量场在
它的双曲焦点处可以C"线性化.

在考虑分岔问吒时,我们常常只需要C线性化.对此,有下面 较为简单的结果.

定S4. \emph{22}设点。是时中一个C\textasciitilde{}光滑向量场(或微分同
胚)X的双曲奇点(或双曲不动点).如果X在点。的线性部分算
孑的特征根%,\&,•••,▲满足下列不等式

Re". ?S:R 叫十 Re\%,(或乏 MJ , )\&丨),(4.30) V \emph{i,j,W}
\{1,2,•••/\},则系统在点。处可以仃线性化.{[}

上述定理的证明对向量场和映射的情况分别由[Be]和[Go] 给出.

例4.23平面上的C"光滑向量场(或微分同胚)在它的双曲
鞍点(或双曲鞍不动点)处可以C*线性化.

例4.24 如果R'中的向量场的一个双曲奇点有一对复特征
根\textless{}1士云和一个实根产,满足3尹0,孕\textless{}0,则称奇点为鞍篤点.鞍
焦点的特征根显然满足(4. 30).故Y中C\textasciitilde{}向量场在它的鞍焦点
处可以C\textsuperscript{1}线性化.

\begin{quote}
例4.25如果It,中b向量场在它的奇点。处的特征根满足 \textless{} A
\textless{} 0 \textless{} \^{},并且
\end{quote}

兀 +产尹兀, % 尹2爲,

则它在该点可以C*线性化.

在第五章讨论非局部分岔时,遇到的向量场都是依赖于参数
的.因此,下面我们讨论带参数的向量场或映射的线性化问题.

定义4.26设左是一个自然数或A =
oo,Xe(\textsubscript{e}\&R\textsuperscript{m})是R''
上的一个依赖于参数的向量场(或微分同胚)族.设£ = \%时\emph{X\%}
有一个双曲奇点(或双曲不动点)\emph{O.}称族在点附近可
以C*线性化,如果存在参数空间中£ = \%的邻域V,相空间R"中
点。的邻域U,以及一个淨映射H : U X V - R'',满足

(1) 釈林)=0;

(2) 对每一个参数tZ-R"是一个微分同胚,
使得通过依赖于参数e的坐标变换尘后,系统在。点的 邻域变成一个线性系统,

定理4. 27 (DY3)设X*是R"中的C"向量场(或微分同
腫)族,且£=蛤时点。是系统X、的非共振双曲奇点(或双曲不动
点),则对任意自然数加族X。可以在点(。浦)附近C*线性化.

附注4.28由上述定理的结论并不能推出族X\$.可以在点
(O,£\textsubscript{0})附近C"线性化.因为随着\emph{k}的增加,可以实现线性化的点

\emph{0}的邻域17和\&的邻域V可能不断缩小.

本小节给出的定理是我们在第四章和第五章中讨论问题的基 \emph{础.}

\protect\hypertarget{bookmark40}{}{}§5普适开折与分岔的余维

现在,我in把§2后半部分中讨论的一些概念严密化.初次.
接触分岔理论的读者可以略过本节的内容,只需承认定理5.13的
结果,而不影响对随后章节的学习.

番适开折的定义

\emph{定}义5.1设向量场X,y(或映射儿g)都在\emph{PEM}的邻域内
有定义.称X与丫(或/■与吾)在》点有同一芽(germ),如果存在邻
域\emph{u,peuczM,}使 xiu = y\textbar{}u
(或\emph{f\textbackslash{}\textsubscript{v} =
g\textbackslash{}u\textgreater{}.}

附注5.2向量场(或映射)在》点的芽,是向量场(或映射)
的一个等价类.我们把这个等价类中的任一无素称为这个芽的表
示.在考慮局部问题时,利用芽的说法可使陈述简明.附录C中定
义的財式空间或\emph{J\textbackslash{}M,Ny}都可以在映射芽的意义下 给出.

现在考慮向量场族蹟'8(整).在局部情形下,无妨设M =R七兴e
R*.此时,常把X\#与其主部状*,产)等同(参见附录B 中附注B. 17
),而X户的流由微分方程

dx

所决定,其中C"(R" X R*, R").在上述等同意义下,也把"称 为向量场・

定义5.3《开折,局部族)对于向量场族讯並/),当把y视
为从参数空间产£R*在原点的小邻域到向量场空间的映射时,我
们把v(z,``)称为u(z,O)的一个点会数开折(unfolding),或称为彩

3?(deformation),当把视为直积空间R\textsuperscript{m}XR*中在(布>,出)点的
映射芽时,称它为一■个局部族,记为3;皿,险).

定义5. 4 (局部族的等价)称两个向量场局部族(巧互,佝)
与3玖,妇等价,如果存在映射\emph{h,y} =五\textsc{(t,Q,}在(知险)的映
射芽,对于每一个固定的\emph{儿心仍}给出上面两个向量场(在相应
定义域内)觥道间的保向同胚,并且无(五,险)=外.

定义5.5 (导出族)称局部族(旳知,勺)是从局部族(巧册,
及)导出的,如果存在连续映射八产=''戦,在导的映射芽,使得
\textsc{«(x,e) =} (€)),且卩(勺)=佝.

定义5. 6 (普适开折)向量场的局部族3;女,佝)称为向量 场外 =讽♦,妇
的芽在气的普适开折(veraal unfolding),如果任
何一个包含为的局部族都与(PT。,险)的一个导出族等价.

附注5.7注意两个向量场族的等价性要求它们含有相同维
数的参数,而导出族的引进使得同一个退化向量场的普适开折可
以含有不同维数的参数,从而可以进一步考虑含参数最少的普适 开折.

附注5.8定义5. 4 - 5. 6都取自Arnold的书[Al,p267].定 义5.
4中不要求近5成)对产的连续性,称这种等价为員等价,而定 义5.
6中的普适性,一般是在弱等价意义下给出的.若 \emph{岖点} 对
产连续,则称这种等价为强等价,并可得到强等价意义下的普适
性.在[R1]中有例子表明,弱等价意义下的普适开折可以不是强
等价意义下的普适开折.

附注5.9 一般而言,对一个给定的奇异(即结构不稳定)向
景场(芽它的普适开折的存在性并不是明显的,只有在周密的
讨论之后,才能得出结论,参见第三章§1.

读者可以用本节的观点重新考察§2的讨论,在那里利用
Malgrange定理证明了,奇异向量场(2.12)任取的开折(么13)都
与一个形如(2.16)的开折按定义5.4等价.另一方面,(2.16)显
然是(2.17)的一个导出族,按定义5. 6,(2.17)是(2.12)的一个
普适开折.从这里可以看出导出族的作用.

分岔的余维,几何考虑

我们现在对向量场局部族的分岔问题考虑它的余维.在全体
向量场所成的空间\emph{宓}中,奇异(即结构不稳定)向量场%表示一
个``点'',仍记为功,它落入分岔集厶(见定义2,1),A在外附近可能
具有非常复杂的结构.但如果限于考虑A中%邻近的点,它具有与 \%
``完全相同的奇异性",则这样的点集可能具有规则的结枸.例
如,形成务的一个余维,的子流形\&,因此在亥中一个至少\&维
的流形K才能在巩点与川横截相交,而这个K可用外的至少\&参
数开折来实现(图1-17和图1-18中分别对应\emph{k=\textbackslash{}}和\& =
2的情
形).由于横截相交性在小扰动下保持,所以\emph{K}扰为\emph{K'}后仍与川
相交于机或其近旁的(见图1-17).换句话说\emph{,K}所具有的%这

\includegraphics[width=3.38681in,height=1.66667in]{media/image22.png}

图 1-17 以=1) 图 1-18 0 = 2)

样的奇异性是扰不掉的.一个维数小于为的开折所代表的流形,
虽然也能在务中的%点与出相交,但在小扰动下,它就可能与A
分离(这种相支不是横截的).这说明,像%这样的奇异性,至少在
\emph{k}参数的开折中才是``不可去''的.而这个数点,就是外的余维.由
Ttiom橫截定理(见附录C中定理C. 15),满足横截条件的向景场
族的集合在绥的全体向量场族中构成一个稠密子集,称其中的任
一族为一个通有(generic)族,或一般族.因此,可以粗略地说,那
种在至少点参数通有族中``不可去''的分岔现象是余维灸的.当然,
在空间亥中的%点附近,除了 A'之外,还可能有余维低于左的奇
异向量场的集合(即AV).在较简单(较``理想'')的情形,它们形
成在%点附近的余维小于龙的各层次的子流形\&*,如果我们恰当
选择\emph{k}参数族K,使得它在外点与各层次的都橫截,価这个\emph{K}
就是一个普适开折.此时,如果把定义开折\emph{\textsc{k}}的映射记为

多,则\emph{K}与各分岔曲面的截痕在下的象,就形成了参数空间
R*中的分岔图(见图1-18).这种几何的考虑:有时是方便的.

一个余维2分岔的例子(Bogdanov-Takens系统)

考虑一个向量场芽,限制在奇点处的中心流形上,其线性部分
矩阵相似于一个二阶纂零矩阵,它有如下的表示

专=u\textsubscript{0}(x) \emph{\textasciitilde{} Ax + ―} (I 1)

其中\emph{工= ,Vo} G
\textsc{C°°(R\textbackslash{}R\textsuperscript{e}),tj\textsubscript{o}(0)}
= 0, Dvo(O) = 4 =

\textsuperscript{0}牛

0 0/

这是一个有二重零特征根的奇异向量场.我们关心的问题是: 它在H =
0附近是否存在普适开折?分岔的余维是多少?它的分岔
图如何?其开折的拓扑结构有哪些不同的类型?它们怎样随参数的
变动从一种类型变成另一种类型?这些问题的解决不是轻而易举
的.事实上,这是由Bogdanov网㈤和Takens\^{}在70年代中期分别
独立解决的一个难题,并且成为推动分岔理论进一步发展的一个
著名工作.因此,我们将在本节和第三章§ 1中详细讨论这个例
子,借以介绍向量场分岔的一些基本理论与方法.本节主要研究

这个奇异向資场的余维.

由例4. 7的(4.19)式可知,方程(5.1)具有正规形

\begin{quote}
J况=皿+ 0(国3),

1 成=讶 + \emph{bx\^{}Xt} +O(\textbar{}x\textbar{}\textsuperscript{3}).
"
\end{quote}

为了对它的奇异性加以限制,这里假设湖K 0.

设。为R2内包含x = O的一个开邻域,XGr)为x点的全体 C™向量场芽的集合,并记

\begin{quote}
齿={(S,tO\textbar{}a£X(g),
\end{quote}

与(£,初相应的是在£点的向量场芽.记 W)为e的一个小邻城, 则(E,初可以表示为

栄= *), x e v(e. (5.2)

因此(5.1)可以简单表示为(0,\%).我们称(匕初具有与(0,臨〉相
同的奇异性,是指它满足如下两个条件;

(Hl) P(f)=0,且gc)在的线性部分矩阵以也)相似

\begin{quote}
(凤)把(5. 2)在x = f化为正规形(5.1)'后有沥关0.
现在可以把与(5.1)有相同奇异性的向量场表示成:
\end{quote}

\emph{s=} {(\&,〃)e 名 I (\&u)满足(HD 和(瓦)}. (5.3) 记S+=
{(\&t>)£S\textbar{} 条件(H,)中沥>0}和S-= {(£,。>
£S\textbar{} 条 件(瓦)中\emph{ab<Q}.}显然,S+ U S-= S U 亥.

附注5.10 这里函数空间列的取法与上面谈的精有不同.
务中的每个``点'',是R\textsuperscript{2}中的一个点附着一个在该点附近的切向
量场.这符合流形上向量场的一般定义.(见附录B中附注B.14),
由于在欧中的每一点,切空间就是R"自身,所以常把向景场与其
主部等同(见附录B中附注B. 17).我们此处的取法,对描述上面
的集合\emph{3}并研究它在亥中的余维有很大方便;并使得下文中从橫

截性定义非退化族符合定义5. 3的原则(局部族要在相空间与参
数空间的乘积空间中考虑).要注意的是,如果S在亥中构成余
维为\&的子流形,且与\$横截的子流形有参数表示{(S,\%)},则\&的
维数的最小值不再是為而是* 一'',其中\textgreater{}1是扌所在相空间的维数,
此例中H = 2.

下面证明S在房■中的(0,\%)点附近构成一个(局部)余维4 子流形.记

户丄{(\&加)丨応。)£亥},

其中为f在f的农jet,见附录C.从\&到尸有自然投影

爲-\/-\textgreater{} \emph{J\textsuperscript{k},} I---* ,,- ,D\^{}p)(

从尸'到尸S2'')有自然投\emph{影j}尸f尸: ,

\begin{quote}
\emph{\textsc{i (\^{},v,6v, --- ,D"v),}}
\end{quote}

其中坐标\emph{v}是»(x)在x = f点的值(它是2维的),而跃(\#
\textgreater{} 1) 是状工)在m f点为阶导数的一种坐标表示.例如,取加为
在z = S点的Jacobi矩阵(它是4维的),取页为v(x)在x = f点
的Hesse矩阵(它是6维的).

定3 5.11当\& 2 2时,W在\emph{J*}中的每(0,外)点附近形成
余维4光滑子流形,且S在务中的(0,诳)点附近形成余维4光滑 子流形.

证明 由S的定义(5.3)式和条件〈HD易知

ff(5 = ((\$,v\textsubscript{f}Dt») I \% =希:=det Dv = tr \emph{Dv} =
0,Dv * 0} ♦
其中希和布表示\emph{v}的两个分量,det缶和tr崩分别表示矩阵炭的
行列式彌迹.注意括号内右侧4个条件是彼此独立的,故沔S在
旧(0,外)附近是尸中的余维4子流形.

下面设\&N2.注意投影叫是b浸盖,由附录B中定理B.

27,购i(\%S)是户中的光滑子流形;再由定理C.16,痂igS)在尸
中的余维等于电S在尸中的余维,即等于4.另一方面,由S的定 义可知,砍S
=吋(两,\textbar{}咛.注意条件(HQ由不等式泌> 0给
出,其中的a和\&是经过相空间的C"变换得到的.因此,麻S是
响在w*(。,边)点附近的开子集,从而它是户中砾2,边)点
附近的局部余维4光滑子流形.

注意\emph{Sfpg),}重复如上推理(用定理B. 27和定理C. 16)
可知,S在务中(0,\%)附近,构成局部的余维4光滑子流形.\textbar{} 现在设

((f,V\textsubscript{£}) \textbar{} e E 2\} C \emph{3\^{}} (5.4)

是亥中的向量场族(参见附注5. 10),且\%就是原来的奇异向量
场(5.1)在工=0的向量场芽.因此,又把(5. 4)称为(0,\%)的一个
开折;与它相应的微分方程是

等= t/Gr,e),工£卩(\$). (5.5)

\emph{定义5.12}称(5.1)的开折(£,\%)是非退化的,如果R,X R™
到\emph{J\textsuperscript{2}}的映射

(?,£)f 句(\&,\%).

在(\&e) = (0,0)点与空间尸中的子流形舟S(在邑(0,外)点)横 截相交.

下面要证明的主要结果是:

定理5.13

(D \&中的全体非退化开折构成它的一个稠密集.

(2)对于(0,\%)的任一开折(5.4),在Gr,e) = (0,0)的小邻
域内存在C"变换工=玳丁,£),满足x(0,0) = 0,并且把(5. 5)化 为

\emph{yz} =沛(e) + 0(e)叫 \emph{+ yl +} 少队Q(M,矽 + 招(必,e),

(5. 6)

其中

\emph{8、Q,佥冲仲} € C\textasciitilde{},p(O) =。(0) = 0,Q(0,0)=打,0(0,0)
= 1.

(5.7) 这里〃 =±1,取决于(0,\%)£S+或SL在(z,\&)= (0,0)的小邻 域内,(5.
6)轨道等价于如下的标准形

\emph{J} --- \^{}2 •

〔务=中(e)+ \emph{MM*} + z: + 工的Q3i ,£)+ ji®(x,e),

(5. 8) 其中同上.若(5. 4)还是非退化的,则存在参数空间的 6变换e =
e(\^{}),e(0) = 0,把(5. 8)进一步化为

册=心,

.叡 =Qi + 必气 + 犬 + 山以\& 愆1,``)+工莎,

(5.9) 其中©,前满足与Q,◎相同的条件.

定理的结论(1)是定义5.12和定理C.15(Thom定理的jet形
式)的直接推论.结论(2)由下面的几个引理得出.虽然这个定理
尚未回答47页所提出的问题,但我们将以此定理为基础,分以下
两步解决这些问题.为了确定起见,只考虑(0,瓦)e s+的情形.

\begin{enumerate}
\def\labelenumi{(\arabic{enumi})}
\item
  在标准形(5. 9)3= 1)中,取0 (z.Q = 1,5(工,``)= 0,即讨论向量场族
\end{enumerate}

\begin{quote}
\emph{\textbackslash{} x = y,}
\end{quote}

仁 , ,,, (5. 10)

I 丁 =内 + ``2, + 廿 + 的分岔图和轨道拓扑分类图,这里e
R\textsuperscript{1}.

\begin{enumerate}
\def\labelenumi{(\arabic{enumi})}
\setcounter{enumi}{1}
\item
  对任意的\&和c»,Q(o,o)= 1),来证明开折
  (5.9〉都与(5.10)的一个导出族等价,即(5.10)是(0,臨)\& \emph{S+}
  的一个普适开折,并且是余维2的.
\end{enumerate}

在作第(一)步讨论时,除了运用微分方程定性理论的知识和
方法外,还要碰到几种常见的分岔现象.所以,我们首先在第二章
中介绍几种基本类型的分岔,然后在第三章中继续解决本节的问 \emph{题.}

定理5.13中结论(2)的证明

引理5.14设(5. 4)是(0,\%)的一个开折,则存在C\textasciitilde{}变换小
=,火0,0)=。,它把方程(5. 5)化为

(5 11)

其中F,G,雪FC\textasciitilde{},并且

\begin{quote}
9K* a\textsuperscript{2} J?
\end{quote}

F(0,0) = = 0, = 2, G(0,0) = \%

\begin{quote}
Wi(so) \emph{°yi} (0.Q)
\end{quote}

这里彳= sign3),a与\&是把矽60)化为正规形(5.1),后的二次 项系数.

证明 由例4. 7可知,无妨假定(5. 5)具有如下形式

f 上1 =叼 + Wi(X,E),

\textless{} . , (5.12)

I 命=技 + ? + g(Z{[}) + W\textsubscript{2}(x,£),

其中函数 C\textasciitilde{},且g =
O(\textbar{}x\textbar{}\textsuperscript{3}),w\textsubscript{(}(x,0) 0,1
= 1,2.

令

`` =*1, p x\textsubscript{2} 4- w,(x,e),

则(5.12)变成

3 = u,

\textless{} .\textasciitilde{} \textasciitilde{} \textasciitilde{}
(5.13)

• 3 = F (``,e) \emph{+ vG} («,e) + \textsc{(m,v,e),}

其中F.G.V-eC",并且

云(0,0) \emph{=0,¥ =0,碧} =2,

™ (0.0) ™ (0.0)

\emph{G} (0,0) = 0, = 7. (5.14)

™ (O\textsubscript{F}0\textgreater{}

利用上面的条件,可由隐函数方程G(a(e),c) = 0确定C\textasciitilde{}函

数a=a(e),再经过变换

Ji = u --- a(e), \emph{y\textsubscript{2} = v,}

把方程(5.13)变为(5.11)的形式,并满足引理的要求.I

引理 5.15 存在 C\textasciitilde{} 变换 \emph{x} = z\&,e),z(O,O) =
0,在。,e) =(0,0)的小邻域内,它把方程(5,11)变成(5. 6)形式.

证明 对(5.11)中的函数F3,e)应用Malgrange定理(定

理2.12)可得

F(m,e) = 3 (e) +。伝頌 + y?) g3i,e),-

其中\emph{甲£C°°怦}(0) =。(0) = 0,9(0,0) = 1.因此,在(\_y,e)=
(0,0)的小邻域内,(5.11)可以改写成

\$1 = 乂,

A k隹)+扒+贵+ 漑+畿爲诚*(为,£).

(5.15) 令

\emph{Jq(yg}

811(5.15)转化成

\emph{u --- v} -/y,

6 u (兌(E)+ ©(e)化 + / + + 寸)\emph{-Tq}

I V \emph{n 2q 彳 q \textbackslash{}}

\begin{quote}
(5. 16)
\end{quote}

其中 \emph{q} = gCtt,e) \emph{,G =} \textsc{G(«gQ} ,W --- \textsc{W(ig}
Rpq、且

\includegraphics[width=1.31319in,height=0.25972in]{media/image23.png}

因此,(5.16)可以写成

\emph{u --- vS(.u\textsubscript{t}e\},}

£ = 3 (e) +。隹加 + 必 + «vQ(U,e) +

其中0= Q = 为,史=雪\_法愛,而且网0 Q和史满 足条件(5. 7).

下面再迸行一次变换,把(5.17)第二式中的步隹)``变成扒e)v 的形式.令

Xj = a 4- \emph{2\textasciitilde{}\textgreater{}} 处=\emph{V,}

则(5.17)化成

上\textgreater{} =互。(而,\$),

壬z= W (e) +。(e)处 + 此 + \textsc{XjXjQ} (电,©) + z;①(z,e)\}\&
(x\^{}e),

(5.18) 其中

\& (zi,e) = 03 --- ''\^{},号),

\emph{甲 g = @} (e) --- +\#(©),

5(e)=-§©(£)Q+S(Q),

\emph{Q} (z】,Q = QGz) --- ,€)--- 叽

\textasciitilde{} 1

*\&(X,£)= \textless{}P(X! --- 处,e),

而函数He)和``跖,e)由下式确定

Q(Z1 --- -\textbar{}-(i(E),€)= 1 + f(e) +。(而,河・

把\emph{S,\textless{}p,\^{},Q}和\&换回\emph{8,中,W,Q}和①的形式后,方程(5.18)成为
(5. 6),而条件(5. 7)仍成立.引理证完.I

附注5.16 显然,局部族(5. 6)(在3,e) = (0,0)附近的一 个邻域内)等价于(5.
6)右端除以0(队圧)所得的局部族,并且在
定义5.4中的映射中与方都可取为相应空间中的恒同映射,在这个
意义下,我们可以把向量场(5. 6)与(5. 8)视为等同.

引理5.17 (5. 4)成为(0,珏)的非退化开折,当且仅当\emph{H i,j}

S为e的维数),使行列式

3,

e.=e.=o

其中\textless{}\{\textgreater{} (e) "(e)是把(5. 5)化为(5.
6)的形式中所出现的函数,而 e=(勺,

证明 根据定义5.12,(5.4)的非退化性由映射

(S,e) (5.20)

在(扌,e) = (0,0)点与空间\emph{J*}中的子流形心S(在吨(0,如)点)横
截相交来定义.注意的坐标表示分别是£,及%及其对*
的前两阶导数在f的值,从而可代替(5. \emph{20)},考虑映射

(M,e) i Gr,uGr,£),Dt\textgreater{}(z,e),D*v(;r,e))

在S,e) = (0,0)点与空间尸中的子流形旳S的横截相交性.由附 录C可知,向量场(5.
5)经过(把m = £ = 0固定的)微分同胚作用 后,在x = £=
0点与某一子流形的横截相交性不改变.利用引理 5.14,引理5.
15和附注5.16,我们可适当选取坐标系,对(5.5)在 (5.8)的形式下讨论(直接在(5.
6)的形式下讨论,可得同样结果, 但计算稍繁).

另一方面,在尸中心(0,\%)点附近%S可由如下方程确定(见 条件皿)和冋))

fi(x,E)== 0, q(z,e) = 0,

det\^{}\^{} = 0,tr\^{}\^{} = Q.

\emph{dx dx}

因此,由定理c. 13可知,(5.8)在(Z,£)= (0. 0)点的非退化性等 价于

而它显然等价于条件(5.19)对某iWiVjWm成立• I 引理5.18设(5.
4)是非退化的开折,则存在C"变换产=

产◎),产(0) = 0,它在e= 0附近非退化,并把(5. 8)变到(5. 9). 证明
无妨设引理5.17中的i = 1J = 2.令``=``怎)由下 式给出:

``1 =伊(矽,的=。(£),``3 = \& =牝,

则由条件(5. 7)知``(0) = 0.又由于(5. 8)是非退化的,则由引理

5. 17可知,`` ="(e)在e = 0是非退化的,引理得证.方程(5. 9)中 的 Q
(力,产)==QCyMO)), © (丿,``)=①(',£(``)),而 e = e(``) 是产=产隹)在e =
0附近的逆变换.I

利用以上诸引理,可得定理5,13的结论(2).

习题与息考間\_

1.1考虑R'上的动力系统質,设它的轨线分别具有图1-6,图1-8或图1-9
的7种分布.对每一种分布,取平面上不同区域的点工,讨论极限集a(z)和
aGr).并研究系统的非游荡集对哪一种分布,你可以断言系统不是结 构稳定的?

1.2利用对(2.13)的讨论方法,证明R\textsuperscript{1}上的系统(2.
2)是C"系统

音=一/ + 0(\textbar{}邳)

的一个普适开折.

1. 3对7•列杀统,求出与中心流形相应的诱导方程(3. 8).并由此作出原
方程在原点附近的相图(草图).

(1) \emph{X = y \^{}=--- y ---} 3日

(2) ± = 2\# \_,气项='---® .

1.4考虑糸=\& + 0(\textbar{}邛),其中\textsc{h£RM=(:} J.假设向量场
在旋转角度M下保持不变,求其在原点附近的三次正规形.

1. 5设R,上的向村场以(0,0)为奇点,其线性都分在(0,0)的矩阵具有
二重等特征根,而且向量场在旋转角度号下保持不变,其中Q为正整数,并且
满足9\^{}3.证明向量场在(0,0)点附近的正规形为

糸=\textsc{cN + qz} 好 + ... +』《"+片 + A\^{}\textsuperscript{\_1}
4-O(\textbar{}z\textbar{}»), 其中z,c"A为复数,m = {[}号:\textbar{}.

1.6考虑三维系统

\begin{quote}
0 I 0
\end{quote}

糸=-1 0 .0 z + O(0「).

\begin{quote}
0 0 0.
\end{quote}

求其在原点附近的三次正规形.

1. 7考虑L映射\emph{F\textgreater{}} R,f R\textbackslash{}它以一1 +
e为特征«,\textbar{}e\textbar{}《1.证明对
给定的正偶数*,存在\emph{8\textgreater{}0,}使当\textbar{}e{]}
时,F可以经过L变换化为

\textsc{F(h)} =(--- 1 + e)H + \textless{}2j(e)x\textsuperscript{3} 4-
\emph{a\textsubscript{5}(.six\textsuperscript{5} + •••}

+ dW-L + O(\textbar{}\^{}\textbar{}*+i).

利用此结果对例2.11的结论给出简单的证明.

1-8利用上题的结果,讨论以一 1为待征根的一维映射

F(z) =---h + or\textsuperscript{5} +。(女卜),a = 0

在小扰动下的分岔规律.

第二章常见的局部与非局部分岔

本章介绍一些當见的分岔现象,其中包括奇点分岔、闭轨分
岔、Hopf分岔、同宿分岔、Poincare分岔等,其中前三种为局部分岔
问题,后两种分别为半局部分岔和全局分岔问题,除了奇点分岔
外,本章的大部分讨论都限制在相空间为二维的情形.

\protect\hypertarget{bookmark51}{}{}§ 1奇点分岔

考虑一个光滑地依赖于参数并且具有奇点的向量场.当参数
变动时,我们关心奇点个数及其附近的轨道结构如何变化.这种
分岔现象称为奇点分岔.

{ } 般理论

定义1.1向量场X的奇点\emph{PEM}称为非退化的,如果它在
?点的线性部分算子是非奇异的,即它的所有特征根均非零.否则 称为退化的.

定理L2光滑地依赖于变量和参数的向量场,如果它的奇
点是非退化的,则奇点本身也光滑地依飄于参数.

证明设向量场由微分方程

\emph{堂---}Cl. 1) 给出,其中 VG (/(R" X R*,R"), rNl/Nl.设当``=岡时,z
=血为(1.1)的非退化奇点,即叽女,佝)=0,型糾 非

奇异.由隐函数定理,在(血,佝)附近存在光滑函数z =八产),使

/(ft) =x\textsubscript{ot}且吹7以)*)三0.定理得证.\textbar{}

附注L3定理1.2说明,当奇点非退化时,奇点的个数在参
数的微小变化下不变,它的位置也光滑地依赖于参数的变化.需
要注意,奇点的非退化性与双曲性是不同的概念.例如,一个向量
场在奇点处的线性部分矩阵有一对纯虚特征根时,接定义1.1,它
是非退化的,但它是非双曲的.此时在扰动下,虽然奇点个数(在
小邻域内)不发生变化,但其附近的轨道结构可能变化,出现Hopf
分岔,或Poinca龙分岔,我们将在§3与§5申分别予以讨论.

定理L4设M是龙维紧致 则*中仅有非

退化奇点(它们必是孤立奇点)或无奇点的向量场集合形成一个 开稠子集.

证明 设XW"(M)相应于微分方程 -

\emph{笑=心,}非以),.

其中中S点的邻域,如同第一章§5的 讨论,考虑投影.

//: 名,(M) --- \emph{X} I-* (f, \emph{f),}

其中7= /(?),具有奇点的向量场集合在空间中有表 示式

5 - \{\^{},7)1?= 0\}.

它是中的光滑闭子流形(因为\emph{M}是紧空间).设向量场
fS)在f点非退化,即監专非奇异,从而由附录C中定理C. 13知,
与孑流形S橫截相交.注意,不相交也是横截,再利用定理 C.
15,得知仅有非退化奇点或无奇点的向量场在盛"(M)中形成 开稠子集. 1

这个定理说明.向量场的一个退化奇点可以经过任意小的扰
动转化为(多个)非退化奇点,或经扰动使奇点消失.但如枭我们
考虑向量场族gq,则奇点的退化性往往是不可避免的.事实
上,虽然小扰动可以把对应于`` =他的退化奇点\emph{x =
x\textsubscript{t}}扰为非退
化的,但在叼附近的為点,相应于\%附近的ft却可能是新的退化
奇点.对一个具体的奇点分岔问题,通常有两种处理方法:一种是
利用中心流形定理,把问题归结到中心流形上,见第一章例3. 12.
另一种称为Liapunov-Schmidt方法,或称为更替法(alternative
method),为了说明这个方法的基本思想,我们先看一种特殊情
形.设变量.工=梭,霁),在* = 0附近微分方程具有下列形式

\emph{= Ay +} /(y,z,A), 糸=Bz + (1. 2)

其中\emph{A}的特征根均为零,而\emph{B}的特征根均不为零,\emph{f ,g} e
\textgreater{} 2i /(0,0,0) = 0,g(0,0,0) = 0, \emph{f,g}
=O(\textbar{}j\textgreater{},z\textbar{}\textsuperscript{s}).为了研究奇
点的分布,在工=0和A = 0附近考虑方程

\emph{Ay} + y(y,z,A) =0, \emph{Bz} + g(y,z,X) = 0. (1. 3)

由隐函数定理,存在。,A) =(0,。)的邻域U和(7■函数\emph{N =} 人),使得

\emph{B\textless{}p} +gg0,A),gO, V CM) 6 \emph{U.}

把函数\emph{言=甲}G,A)代入(1. 3)的第一个方程左端,可得\emph{頒}函数

G(3\textgreater{},A) \emph{=\textasciitilde{}Ay + f(.y,V}
(\textgreater{},A) ,A). (1. 4)

记

\emph{S} = \{g) G UIGCM) = 0\}, \textsc{Sa} = S n \{Af\},

则对不同的A, \textbar{}A\textbar{}《1,S,结构的变化反映了奇点个数的变化规
律.这样就把对(1.3)的讨论转化为对G(y,A) = 0的讨论,使空
间维数得到降低.通常称(1. 4)为方程(1. 2)的分岔函數.

为了应用上的便利,下面在更一般的框架下讨论这个问题.

\textbf{Liapunov-Schmidt} 方法

设X,Z和A为实Banach空间,U和W分别为X和A中零点 的邻域.。映射M2XWUXX
A---Z,满足M(0,0) = 0.我

们要研究方程

A/愆?)= 0 (1. 5)

在UX"中(0,0)点的某邻域内解的结构.为此,设4 = D■小(0,
0),并记,(A)和饲(A)分别为A在X中的零空间和A在Z中
的值域空间.本节的一个基本假设是

CH) /"(A)在X中存在补空间;绥(A)是Z中的闭集,并且
在Z中存在补空间.(当A为Fydhdm算子时,这个假设总是成立
的.在下文的应用中,经常是这种情形,)1 .

因此,在\emph{X}上存在投影尸,在Z上存在投影Q,使得

竇(P) = "Q4),處(Q) = \^{}(A). (1.6)

V z € C7,可写成 z = `` + v,其中 u \emph{= Px} €,(A) =
\emph{X\textsubscript{P},v =} GT \emph{一 P'' E W} (P) = \emph{Xj}
这里;是恒同映射,Xp和Xf 表示
投影P和\emph{I-P}的值域.显然,方程(1.5)等价于.

\emph{QM(u +} w,A) = 0, (1.7a)

(7 --- Q)M(``+u,A) = 0. (1.7b)

定义映射例XpXXjpX A->處(4),

= QM(u + \emph{v,X).}

则 Mo,O,o)= 0,且Dv机0,0,。〉= A\textbar{}女P)是 w (P〉与使(A)间
的同构.由隐函数定理,存在\emph{X?}在原点的邻域\emph{U\textsubscript{o}
,X\textsubscript{t}\_\textsubscript{P}}在原点的
邻域K,A在原点的邻域叽,以及\emph{C'}映射扩,夂X叭f哓,使
\emph{u\textsubscript{a}} x 吼 uumuw,且

\emph{QM(u +} (w>A),A) = 0, V ('',人)€ S X Wo,

并且<(0,0) =0, D'' (0,0) = 0.利用甘,定义b映射z* : \emph{U°} X
\emph{W.\^{}u,}和 C\textsuperscript{1} 映射 G:U°X 吼f 彳(Q),

\emph{x*} (w,A) = a + v* (a\textsubscript{1}A) , (1. 8)

G(",Q \emph{--- (J ---} Q)M(`` + \emph{v*} (a,A),A). (1. 9)

容易验证,x*(0,0) = 0, D『*(0,0) = \emph{I\^{}\textsubscript{Ar}} G(0,0)
= 0,

D``G(0,0) = 0.

总结上面的讨论,我们有下面的结果.

定S1.5 如果条件(H)成立如上,则V«6
\emph{u\textsubscript{a},\textsubscript{x}eu\textsubscript{a}} x
v°ux,和人e 如下两组结论等价

\begin{enumerate}
\def\labelenumi{(\arabic{enumi})}
\item
  \emph{\textsc{Ph u,}} Af\textsc{(x,A)} 0;
\item
  \emph{x = x*} (w,A), G(w,A) --- 0,
\end{enumerate}

其中z*与G分别由(1.8)和(L9)定义.I

定理1. 5说明,原来的奇点分岔问题M(礼A)=。转化为求解 其分岔方程=
o.注意\emph{x e x,} \textsc{ms,}A)e z,而 '' e Xp = "(A), G('',Q
\textsc{£Z,\_q=} "(Q),因而使问题的定义
域及值域都作了显著的约化.这就是Liapunov-Schmidt方法的核 心思想.

现在我们把上面的一般理论用于R"上的向量场奇点分岔问
题.考虑依赖于参数人的向量场

牙=/〈戒), (1.10)

其中 \emph{f} R\textbackslash{} R''),r\textgreater{}2; /(0,0) =
0,D/(0,0) = A.

考虑奇点分岔问题,就是要在R'' X R*的原点附近考察方程 feg = --- N(z,A) =
0, (1.11)

其中 NEC\^{}ITXR七 RD, \textsc{N(0,0) = 0,DcN(0,0)nO.}与前
面的一般情况对比,此时有X = Z = R", A = R*.假设又有 dim"(A) =codim烫(A)
= 1,则存在投影
P,Q\textgreater{}R\textsuperscript{a}-\textgreater{}R\textsuperscript{n},满足
(1. 6).从面存在 '' (A ),叫,£ '' (Q),使

''(A ) = Span\{``()\}, (Q) --- Span(w\textsubscript{0}\}.

从上面的一般理论知道,存在\emph{8\textgreater{}0,a\textgreater{}}。和(7■函数v
- v* (a,A) 6X,\_\textsubscript{P},满足 \textsubscript{w}*(0,0) = 0,
D\textsubscript{fl}\^{}-(0,0)-0,使当 ia\textbar{}\textless{}E,
\textbar{}A\textbar{} Vb时

\emph{Qf(.au\textsubscript{0} + v*} (a,A), A) = 0.

由定理1.5,工= a«o + v, 是(1.11)的解,当且仅当0 =

\textgreater{}(。,人)且念,A)满足分岔方程

g(\textless{}j,A) = 0,

这里分岔函數g由下式定义:

g(a,A)w\textsubscript{0} \emph{= (.1 --- Q)f(.au\textsubscript{0} + v*
(a,\^{},} A). (1.12)

例L 6 用Liapunov-Schmidt方法重新考虑第一章例3.12.
我们考虑R'中一类更广泛的微分方程

dx

击=夕,

* (1.13)

g =陟 + 廿十 x\textgreater{}(i + 喚工))+

其中夕尹0,啊坦顿0) = 0.考虑它的奇点(0,0)在扰动下的
分岔问题.此时线性部分矩阵为》,din/G4) = codirr应(A) =
1.取满足。.6)的投影
P\textgreater{}Q\textsubscript{t}R\textsuperscript{a}-*R\textsuperscript{2}.令'

\begin{longtable}[]{@{}lllll@{}}
\toprule
\endhead
& \textbf{11} & 0 & & 01\tabularnewline
«0 --- & 」n & & ,Wo = & 」,\&=[」,\tabularnewline
& 0/ & u & & 1/ \textbackslash{}fi!\tabularnewline
\bottomrule
\end{longtable}

彳(A ) = Xp = Span(tt\textsubscript{0}\}, X/\_p ---
Span\{u\textsubscript{0}\},

(A )=我(Q)= Span\{s\textsubscript{Q}\}-

取 彳(Q) = Span\{w\textsubscript{0}\}\_ 函数 v v* (a) = ° 1 g 由方程 \%/

Qf(a«o + * (Q) = 0 确定,HP

0 = Q\{叩 o + {[}茅 + 码(1 + 9\textless{}a)) +
y0(a,\^{}\textsubscript{2}){]}w\textsubscript{o}\} = Wo.
因此1\textgreater{} =尹。)空0.把它代入(L12),得到

g(\textless{}z)玖=(Z --- \emph{Q)f\textless{},au\textsubscript{0}} + v*
(a))

=(J --- \textsc{q)z} :) =(7 --- Q*。J h a\%°・

从面分岔函数g(a)=次.

如果我们考虑方程(1.13)的S扰动,扰动参数为為则扰动后
方程的分岔函数g(a,A)满足g(a,O)=次 利用隐函数定理易 知,存在
3\textgreater{}0和C\textsuperscript{0} 函数a = a(A),使得a(0) =
O.D\^{}CaCA) ,A)= 0, D訪(a(A),A)尹0, V
\emph{\textbackslash{}A\textbackslash{}\textless{}8.}利用Taylor公式可得,当\textbar{}A\textbar{}《
1, \emph{\textbackslash{}a-aW\textbackslash{}}《1,有

g(aj) =\#(A) +I\^{}g(a(A),Q(a --- e(A))2 +。(\textbar{}。一
d(A)\textbar{}z), 其中 ju(A) = g(a(A),A).因此,在(a,Q = (0,0)附近,方程
g(a, A) = 0 当 ju(A)D\^{}(0,0) \textless{} 0 时有两个零点,当
MAJD\^{}CO.O) \textgreater{}0 时 无零点;当产(Q =
0时有一个(二重)零点.应用定理L 5可知,原
系统(1.13)在扰动下发生鞍结点分岔.注意,当扰动方程为C"
时,最后的讨论可从第一章定理2.12直接得到.

附注1.7本节中讨论的奇点分岔问题,主要着重于奇点个
数随参数的变动而发生变化的规律.实际上,在奇点个数发生变
化的同时(甚至在奇点个-数不变时,见附注1. 3),轨道结构还可能
发生其它变化.例如闭轨、同宿轨、异宿轨等的产生或消失.这些
情形在下章中将会看到.

\textbf{§ 2}闭轨分會

考虑微分方程族

(XQ: § = (2.1)

其中`` W \textless{}Z(R\textsuperscript{n} X R* ,R\textsuperscript{n}),
设X。有一条孤立闭執

/.当\textsc{a}尹0,国《1时,我们关心X*在y的邻域内是否还有闭執?
有几条闭轨?这就是爾轨分岔问题.当y为双曲闭執时,问题是平
凡的(见第一章§1).因此,我们要找到一些方法,来判别,的双
曲性,以及当,非双曲时如何研究闭轨的分岔问题.至于,为方程
(2.1)的非孤立闭轨的情形,我们留待§ 5中讨论.

从原则上说,可把闭轨分岔问题转化为它的Poincar色映射的

不动点的分岔问题,从而可利用上节的方法.事实上,任取力\&匕 存在过夕的m
--- 1维``无切截面''U U,按第一章定义L 6所述,
可定义Poincare映射\emph{U'XW\^{}U,}它是U的,其中W是时中
原点的邻域,满足FQ\textgreater{},0) =». X*的闭轨相应于3/U,A) --- FGr,A)
--- h在Cr,A) \emph{EU' XW}内的零点.因此,§ 1中的方法都
是适用的.注意,如果把坐标原点平移到*点,就会满足§1中 M(0,0) =0的条件.

在解决具体问题时,困难在于如何实施上述原则.下面,我们
就平面向量场的情形作进一步的讨论,顺便介绍曲线坐标方法和 某些重要结论.

考虑平面上的微分方程族

(X》 等= 0(z,A), (2.2)

其中w W X R*,R2), 设X。有闭轨匕它有如

下的参数表示

设/以T为周期,并且为负定向,即当*大时,中《)沿,顺时针方
向旋转.取/在夜点沿外法向的单位向量

\textsc{b =} ;:;牛 23)

由 f(O±P\textsuperscript{,}(z)及 \textbar{}並)\textbar{} =1 易知,V
0\textless{}z\textless{}T,

〈。《) \emph{,\textless{}P'}Q)〉m 0, \textless{}f(O,广(D\textgreater{}
w 0, (2. 4)

其中《•,•)表示R'中的内积.取坐标变换

■r = p(s) + 渔)'', (2. 5)

其中工在,附近\textsubscript{(}0\textless{}sCT,\textbar{}n\textbar{}\^{}l,坐标(s,Q可以这样理解,从
\emph{T} (0)沿,轻过时间\emph{s}到达p (j),再从p
(s)点沿\emph{7}的外法向?(s)移
动长度''到达z点(当''V0时,表示向内法向移动),见图2- 1(a). 注意,3
=常数}与{\$ =常数}在平面上形成蛛网形坐标曲线(见

图2・l(b)),称(心矽为曲线坐标.

\includegraphics[width=1.45347in,height=1.42014in]{media/image24.png}\includegraphics[width=1.5in,height=1.64653in]{media/image25.png}

图2-1

我们先把方程(2.2)转换成曲线坐标系下的方程,然后建立
Poincare映射.把(2.5)对,求导,并应用(2.2)得

B(s) + 其0")=牙=(*) +"(s)'')糸 + 照)零.

(2. 6) 分别以推)及时⑴对上式作内积,利用(2. 4)及\textbar{},(s)\textbar{}
= 1得 \emph{华=}g(s), U(甲(s) + r(s)")〉,

\begin{quote}
虫={《V :G) \emph{9} P(W (\$) + 了(s)行,久)》} di - B(s)\textbar{}2 +
''w),尸(s)〉'
\end{quote}

消去*機

dn\_ {(g'(s)\textbar{}'+''3'(s),,`(s)〉)〈,(s), u3(s) +
"s)'',Q\textgreater{}} ds " 〈p'(s), t\textgreater{}(爭(s) + f(s)n,A))

---\^{}=FXn,j,A), (2. 7)

由于= 9\textgreater{}(J)为X。的僻,故

\emph{\^{}p'(s) =} 0)\textsubscript{t} (2.8)

利用(2. 8)和(2. 4),可从(2. 7)算得

F(O,s,O) = 0,

〈r(s),票3 (\$) ,o*(s)〉3 = H(5). (2. 9)

其中尤 J =
O(W).因此,(2.10)满足初值条件«\textbar{}\textsubscript{J=0}=a的
解可表示为

\includegraphics[width=1.34653in,height=0.33333in]{media/image26.png}

(2.11) 现在,取x°的闭轨y上的点工。=\emph{甲}(0),过边以法线«(o)为
方向取一截线匕,建立(2.11)的Poincar卷映射(见第一章定义L 6,
但此时与参数A有关)ROM) iw(T,a,A),这里的函数g",
A)由(2.11)定义.显然,n(T,0,0) = 0.定义后继歯數

GCa.A) = \emph{n(T},a,A) ― \emph{a,} (2.12)

则对每个A,\textbar{}A\textbar{} Cl ,G(a,A)关于a的零点与%在7附近的闭轨
相对应.注意(2. 2)中a 法2,F关于叭人为b的,关于s为

L 的,故(2.11)中的仃,从而(2.12)中的函数G £\emph{矿} 定义2.1
若存在\emph{e\textgreater{}0,}使V4£(0,e)都有\emph{G(")\textless{}。}

(\textgreater{} 0),则称7为外側稳定(外側不稳定)的樋限环.若存在e
\textgreater{} 0, 便VaE
(-£,0),都有G(a,A)\textgreater{}0«0),则称,为内侧糠定(内
侧不稳定)的极限环.双侧均稳定(均不稳定)的极限环称为稳定
(不謹定)极限环;双侧稳定性不同时,称/为半稳定极限环.

从上述定义可知,稳定(不稳定)极限环必为孤立闭轨•下面
定义中的y为非孤立闭轨.

定义 2.2 若 V £\textgreater{}0,3 e (0,e)使得 G(aj,Q = 0,但.
\textsc{g®,a)}尹o,则称,为外fllU型极限环.若存在\textsc{e\textgreater{}o,}使v«e
(0,矽,都有G(a,Q = 0,则称丫为外侧周期环域.

类似可定义内側复型极限环与内侧周期环域.

定理2.3 解析向量场不存在复重极限环.

证明 由于在X。的闭轨y,{* = "\$)iomwT}上无奇点, 利用方程(2.
8)和,的紧致性可知存在\emph{6\textgreater{}0,}使得当\emph{\textbackslash{}n\textbackslash{}\textless{}3,\textbackslash{}M}
\textless{}3 时,〈"(s), u(w(s)+2)")〉,但正.因此,方程(2. 7)的右
端函数F('',A)解析,从而GGM)话析.由于解析函数的非孤立
零点必存在一个邻域,使函数在其中怛为零.因此定理得证.I

定义2.4 若\emph{3 e\textgreater{}}
Q和正整数使当\textbar{}a\textbar{}Ve时, 有

G(a,O) +o(\textbar{}a\textbar{}*), \% = (2.13)

则称y为名的\&童极限环.当龙=1时称为单童极限环,当\&\textgreater{}1
时称为參重椽限环.

显然,当\&为奇数时.\^{}\textless{}o表明,为稳定的极限环,而c*\textgreater{}0
表明\emph{y}为不稳定的极限环;当\emph{k}为偶数时,为半稳定的极限环.
注意,这里说的稳定性为轨道稳定性,而不是结构稳定性.事实
上,与第一章§1的定义相对照可知,单重环是结构稳定(双曲)
的,而多重环都是结构不稳定(非双曲)的.为了判别,是否为单 重的,我们记

。=ftr 敦卩(s),0)ds, (2.14)

定理2. 5若。\#0,则7为X。的单重极限环.当\^{}\textless{}0时,
为稳定的,当。\textgreater{} 0时,为不稳定的.

证明 由(2.115,(2.12)\emph{得到}

G\textless{}a,0) =a(exp匚[HG) + \emph{F,(n(s,a,Q)} ,s ,O)3ds --- 1"
故由 Fi(n,s,0) = 0( \textbar{}n\textbar{})和 n(T,0,0) = 0,可得

G„- (0,0) = Ifan \sout{泡}一\sout{。)芸(0,0)}=吋口7(\$汕 一 1.

a-*o \emph{a} Jo

另一方面,把(2. 9)式的内积按分量展开,并利用(2. 3),(2. 8)可 得

\begin{quote}
H(s)=加),票 =*金何3),0)-祖*汗史
=切票33\textgreater{},0)\textasciitilde{}成£3,敦〉 =tr
書3(s),0)---監的。\textbar{},
\end{quote}

其中 \emph{v =} v(中(\$),0),

①=殺(如)2 + \textbar{}\textgreater{}'姒+鹳时"+参(时)七
注意到,以T为周期,因此

『\emph{H(,s\textasciitilde{})ds} --- j\^{}tr 禀3 (s) ,0)dj = \emph{a,}

代入(2.15)得

GJ(0,0) =e。一 1.

当0时,得到G(0,0) = 0,GL(0,0)\emph{丰}0.故7是单重极限环.
它的稳定性与。的符号之间的关系是显然的.定理得证.I 推论2.
6当X。的团轨7为多重极限环、复型极限环或,附 近为周期环域时,必有«= 0.

\$3 2.7设,为X。的*重极限环@21),则对于丛

\begin{enumerate}
\def\labelenumi{(\arabic{enumi})}
\item
  存在y的(环形)邻域£7和正数8,使只要\textbar{}用\emph{\textless{}S,X\textsubscript{A}}在
  U内至多有为个极限环.
\item
  v 5\textgreater{}o,对任给的y的(环形)邻域vet/,
\end{enumerate}

3 X。的扰动系统\textsc{x\textsubscript{a},\textbar{}A\textbar{}}
\textless{}\$,使\textsc{x}爲在y内恰有,个极限环.当応
为偶数时,上述结论可扩充至\textless{} =0.

⑶ 当\&为奇数时,V VCt/,3
\emph{8\textgreater{}0,}使当\textbar{}A\textbar{} 时,Xa在

V内至少有一个极限环.

证明 当扰动系统X)对应的(2.2)中的D £ b时,可由
Malgrange定理(第一章定理2.12)直接推得以上结论.当v G tT
时,可利用隐函数定理和中值定理证明.详细推导从略.I

例2. 8考虑R,上的系统

专=3S 糸=4 +小+技+\emph{如寸,}

其中参数为尹0.设存在闭轨,,周期为T,则

ff =匚tr di =\emph{+ 2bxy\textasciitilde{})} di =狂\emph{丰} 0,

这里利用系统的第一个方程得到2巧\& = 2点工=d(廿).因此,这 闭轨是双曲的'

附注\emph{2. 9} 上例表明,在一些具体问题中,,利用(2.14)计算。
时,可以不必知道甲3)的表达式,这就给利用定理2.5判别闭轨
的双曲性提供了方便.当。=0时,需要从(2.12)进一步计算(2.
13)式中第一个不为零的会,以便按定义2. 4判断7的重次.这时
一般来说计算量较大.

\textbf{§3 Hopf}分岔

当向量场在奇点的线性部分矩阵有一对复特征根,并且随参
数变化而穿越虚轴时,在奇点附近的一个W维中心流形上,奇点的
稳定性发生翻转,从而在奇点附近产生闭轨的现象,称为Hopf分 岔,第一章例2.
9就是一个典型的实例.既然Hopf分岔发生在二
维中心流形上,为了简单,下面都讨论二维方程.

经典的Hopf分岔定理

考虑向量场

其中 z=(X\textsubscript{1(}X\textsubscript{2}) e R2,`` 6 R1,F(O,O) =
0,D\^{}(0,0) = 0.设 线性部分矩阵\emph{A}
(兴)有特征值\emph{叭心}±道(閃,满足条件

(HQ «\textless{}0) = 0,\^{}(0) = \& 尹 0;

(H\textsubscript{2}) a,(0)尹 0;

(H\textsubscript{a}) Recj 尹 0,

其中\emph{h}为向量场X。的如下复正规形中的系数〔见第一章例 4 9), -

\emph{柴=}+ c; \textbar{}w\textbar{}\textsuperscript{!}w + ― +
\emph{c\textsubscript{k}\textbackslash{}w\textbackslash{}\textsuperscript{2i}u\textgreater{}
+} O(\textbar{}w\textbar{}\textsuperscript{M+3}).

\begin{quote}
(3.2) 定理3.1 (Hopf分抡定理)设条件(HD和(HQ成立,则日。
\end{quote}

\textgreater{}0和x = 0的邻域U,使得当\textbar{}司Vb时,方程(3.1)在U内至
多有一个闭轨(从而是极限环).如果条件(HQ也成立,则3 ff\textgreater{} 0
和在0\textless{}电V。上定义的函数兴=``GD,满足``(0) = 0,而且

\begin{enumerate}
\def\labelenumi{(\arabic{enumi})}
\item
  当产=产(志),OVziW。时,系统(3.1)过点(心,0)的轨 道是它的唯一闭轨.当Req
  \textless{} 0时,它是稳定的,当Req \textgreater{} 0时, 它是不稳定的,
\item
  当 \emph{fia ' WRec, \textless{} Q} 时,0愆1)\textgreater{}必
  当所,(0)Reci\textgreater{}0 时0 (功V 0.
\end{enumerate}

\begin{quote}
在下文中,我们先证明一个更广泛的定理,再证明定理3.1. 附注3. 2
Hopf分岔定理有多种形式和多种证法.例如可參
\end{quote}

考{[}FLLL{]}.实际应用定理3.1时,董要的是计算Rec''对某些常
见的方程,导出计算公式将为应用带来方便.下面给出一个例子.

例3.3 (CGH{]})如果二缝系统X,具有如下形式

件叫

其中 \textsc{/(OtO)} = g(0,0) = 0,D/(0,0) = Dg(0,0) \emph{f,伉砖}。,则
有如下计算公式\textless{}

Re。】\textsuperscript{=}島{(Zzzx + /?.\%• + \emph{g.y + 思*)}

\emph{+ .xx + f} p --- \emph{gH\$,gHH} + g'')

\emph{ffgf} + /yjgy {]} \} I H = J»=O, (3.3)

遇化\textbf{Hopf}分岔定理

当条件(死)与(H,)至少有一个不成立时,仍有可能出现
Hopf分岔,这就是所谓退化Hopf分岔问题.

考虑二维淅参数向量场

(XQ; 糸=顶0,丿,``),\^{} = g(z,/,``), (3.4)

其中e R\textsuperscript{l},/«e R``,/,g e
c\textasciitilde{},/(o,o,o)=g(o,o,o)= o.

引理3.4设在奇点愆,少=(0,0)系统(3. 4)的线性部分矩
阵有一对复特征根虹产)±推(产),满足条件(HQ,则对任意自然数
如存在\emph{S\textgreater{}0}和光滑依赖于参数产的多项式变换,当\textbackslash{}\^{}\textbackslash{}\textless{}\&时,
可以把(3. 4)化为

票={[}a(``)+ i£("){]}w + +

\textsuperscript{at} (3.5)

\ldots{}+水``)『伊+。(\textbar{}如''+'),

其中a(0) = 0,£(0) \textsc{=£o,g(O)} =c"f = 1,2,・``盘,这里c,是把X。
化为(3. 2)后的系数.

证明 记A(Q = \emph{\textless{}W,W\textgreater{},}其中 \emph{W}
和制产)为复特 征根a(G 土 i£("); \emph{m}
=(四,风),其中皿,屿为自然数,\emph{M\textsubscript{k}\^{}=} \{m\textbar{}2
+ »h W 2i + 2\};并且

(»»,人(``))=wiAf/i) +»»A.

由条件(H】)知丄(0) = 给出M 2艮+ 2阶的共振条件,

当且仅当 »i f Af * ---\emph{(m\textbackslash{}mi =} +
l,m\textsubscript{a} = 1, --- ,4\} UAf*.由

于M产)是光滑函数(j= 1,2),故存在\$\textgreater{}
0,使当\emph{\textbackslash{}\^{}\textbackslash{} \textless{}8}时
為(``)尹 S,A(G), 当\textsc{》i £ .}

因地,用第一章定理4. 6和例4. 9同样的推理可得引理的结论.I 定义3.
5稼X。以3,少= (0,0)为左阶细魚点(応法1),如

果条件(HQ成立,并且把X,化成正规形(3. 2)后,満足条件

Req --- ― = Req\_i = 0,Rec* 尹 0. (3. 6)

定理3.6设向量场X'以(0,0)点为\&阶细焦点,则X。在扰
动下可发生\&阶Hopf分岔,即

⑴对它的任一开折系统X''存在和(工成)=(0,0)点
的邻域U,使得当山O时,厘在U内至多有左个极限环;

(2)对任意整数任意常数。'',0<站〈孔以及 (工,少=(0,0)的任意邻域u*
UU,存在一个开折系统X;,使得 X;在b内恰有j个极限环,其中siv(r.

证明({[}RS{]})由引理3. 4,系统(3. 4)可经光滑依赖于参数
的多项式变换化为(3, 5),对(3. 5)及其共辗方程引入极坐标,注
意产=切矛,。2衍=帰-七可得到

r= a(Gr + Re(C\textbar{}("))W + ••• + Re(Q(G)\#+' + 0(严+,),
\$=£(``)+0(产).

(3.7) 由于版。)尹0,所以当时,可从(3.7)得到 东=\&(产'' +禹何,产)尹+ - +
\emph{g 財 *} + 0(户+3), (3.8)

其中r《l,并且

---+7\textsubscript{i\textgreater{}1}Re(c\textsubscript{1}(/«))
+"()\textless{}»(``).(为 N 2),

这里们=们SQ是we \textsc{{[}o,2k{]}}和户(在0附近)的光滑函数.
当兴=。时,(3.8)成为

爵譬 9+1+0(户+3). \textsubscript{(}3.9)

在x,轴上建立方程(3. 8)的Poincarfi映射,并令

--- (3.10)

显然,

V(zi/) = V(r,A),当 \textgreater{} 0. (3.11)

卩(电,户)在电\textgreater{} o的零点个数对应于方程(3.8)非零周期辭的个 数.

令函数

\emph{R(r,乎,ft) =} + ``2(會,")产 + ― + 产+1 + •••

是(3. 8)满足R(r,0/) =r的解;而函数°(7■瑚)是方程(3. 9)满足 択「,0)
=「的解,则有

『愆1,0) = W(r,2O = fi(r,2«,0).

由此可知,当气法0时

赛(脸=C\textsuperscript{(}°\textsuperscript{,0)}-\textsuperscript{1}
=敎。加)- 1,

\emph{-mp -fx,} (3.12)

护0,0)=宓(0,0)=部,2Q, 他\textgreater{} 1.

由于択r挿)是方程(3. 9)的解,我们有

\begin{quote}
0, 当 1 \textless{}m\textless{}24 + l,
\end{quote}

{[}(2,+ 1)门警,当 \emph{m = 2k + l.}

因此

由(3.12)和(3.13)可得

0, 当 1 岳+ 1,

Rec.

\begin{quote}
2«2* + 1)11 当 m = 2i + 1.
\end{quote}

(3.14)

注意方程(3. 8)与(3. 9)右端的函数当r = 0时恒为零,并可 以光滑地开拓到r V
D.因此,= P(nm) --- \emph{Xi}在Si/)
=(0,0)的邻域内是光滑函数,由条件(3.14),利用Malgrange定
理(第一章定理2.12),在(热,``)=(0,0)附近存在光滑函数
hgQ,以。,。)尹。,以及对芯的M+1阶多项式函数Q6成), 使

另一方面,丫(0,司=。,且丫包\textbar{}/)对心的正根与负根成对出现
(这里要用到,当在(0,0)附近的一个小邻域内时,方程(3.

7)中。\textgreater{}0).因此,在(石/) = (0,0)附近的一个邻域内
对电至多有\&个正根,短理的结论(1)得证.

为了证明结论(2),我们假设X。以工=0为龙阶细焦点,即它 具有如下的正规形

z = i炼名 + c」z\textbar{}电 +。( \textbar{}z\textbar{}"+3)= ,
Re々\emph{丰} 0.

取它的扰动系统

£ = F⑴+产為项斗遷-七+ - +叫t房严I上,

\begin{enumerate}
\def\labelenumi{\arabic{enumi}.}
\setcounter{enumi}{14}
\item
  其中\emph{应一 jMmWk一侦}固定(1 W/MD化为极坐标 方程
\end{enumerate}

\emph{r} = "I•户 I* + + 构 + Rec*r\textsuperscript{2i+l} +
O(r\textsuperscript{M+s})

W-G(急

为了使系统(3.15)存在\emph{j}个闭轨,我m按下述方式依次选取
处T,\ldots{},勺\_广设Rec* \textgreater{}
0(当Re々V。时,讨论是类似的),则可

选取OV``V1使

G(0,.••,();小 \textgreater{}0.

选任---1 V0,以 \_1I《Rec*,及 G (0,r*),使得

G(0,仪\_i;々)\textgreater{} 0, G(0,•••,(),任\_{[},々\_{]}) V 0.

类似地,可选分\_2,"\_2,"・,``i,以\_尸使得 具有交善的符号,并且0V
"*\_】《•••《以\_11《\textbar{}Rec\textsubscript{t}\textbar{},O\textless{}
门\_『\textless{}''・0£\_1\textless{}1,使得

r\textgreater{} 0,当 r =
\emph{r\textsubscript{k},r\textsubscript{k}\_\textsubscript{2},---}

\begin{quote}
rVO, 当广=广1,々一3,\ldots{}
\end{quote}

由Poincaifc-Bendixson环域定理,扰动系统(3.15)至少存在\emph{j}个极 限环.

这里门的选取使产生的极限环都在U''内,而产的选取满足 \textbar{}川
我们断言,能得到这样的系统(3.15),它在内恰有j

个极限环.若不然,则对任意的ff\textgreater{}0和r = 0的邻域U,(3.15)在
\emph{U}内有多于\emph{j}个极限环,则我们可以仿照上面的方法选取\emph{既十\textbackslash{},}

1,\ldots{}必,七以及七閃从而在。内再获得另外的\& --- j个极限
环(Ml〈。),使扰动系统在U内的极限环总数大于如这与结论
〈1)矛盾•至此,定理3*证毕、I

定理3.1的证明 当\& = 1时,可从定理3.6得到定理3.1.
事实上,从定理3.1的条件(HQ和(HQ得知電=0是方程(3. 2)的
一阶细焦点,因此定理3.1的前一部分结论成立.再设条件(H,) 也成立,由(3.
8)•--- (3.11)可知,后继函数

VO1/) --- \emph{X1V} 3,产), (3.16)

其中

\emph{V} 愆1,产)={[}exp( 2It ---】{]}+ ``2(2咒/)工1 +
O(X1\textsuperscript{2}),

(3.17) 再由(3.14)可知

\begin{quote}
\emph{V} (\^{},0) = 2``響工
\end{quote}

由(3.17)和条件(HQ可知

部。)=新⑹尊 \textless{}3.19)

利用隐函数定理,存在b\textgreater{} 0和在定义的光滑函数兴=
产5),满足``(0)=0和

VGr"(Zi)) = 0. (3.20)

至此,定理3.1的结论⑴得证.为了证明结论⑵,从(3. 20)求导 得到

S + 2 務"3)+ 麝 33))2 + \emph{紀 g} = 0.

(3.21)

利用(3.18)及条件(H,)可得

\emph{\^{}\textsuperscript{C0}*\textsuperscript{0) = 0}* 齢0,。)= 4璧尹
°,}

,把上面的结果及(3.19)代入(3. 21)得到

``(0) = 0 你0)一 2 熱.

由此得定理的结论(2). I

从以上证明立得以下推论.

推论3. 7 设条件(HQ, (HJ和(H,)成立,则存在。\textgreater{} 0和電
=0的邻域U,使得

\begin{enumerate}
\def\labelenumi{(\arabic{enumi})}
\item
  当I闵Recd(O)产V。时,系统(3.1)在U内拾有一 个极限环,当Reci
  \textless{}0 (\textgreater{}0)时,它是稳定(不稳定)的;并且当点
  f。时,它缩向奇点z = 0;
\item
  当\textbar{}产\textbar{} \textless{}ff,
  RecG(0)``N。时,系统(3.1)在U内没有极
\end{enumerate}

限环.I

附注3,8在应用定理3. 6时,需要首先判断未扰动系统X,
以\emph{0}为细焦点的阶数,也就是确定满足条件(3.5)的\emph{k.}在实际计
算时,经常应用下面介绍的Liapunov系数法,细节请见{[}ZDHD{]}.
设X。具有如下的形式

\begin{quote}
f 任=一崗夕 + p(\_r,y),

1 ■ \textsubscript{a} . , 、 (3.22)

\emph{{ y = Pox + q(x,y),}
\end{quote}

其中z以ER,力归=。(惊,刃,),禺尹0.我们利用待定系数法,
寻找虬£R,j=I,2,\ldots{}和函数

F(w)=辱3+寸)+0(\textbar{}工,刃``),

使得

\begin{quote}
J p \textsuperscript{m}

翌 + (3.23)

w《3.22)
\end{quote}

满足上式的{匕}称为(3. 16)的Liapunov系数.在下面定理的意义
下,它与Hopf分岔系数{Re(罗}是等价的.

定理3. 9 匕 Ui=0,K\textgreater{}0(或V0),当且仅当

Req =\ldots{}=Req\_i = 0,Rea\textgreater{}0(或 VO). I

定理的证明见{[}BL{]},下文中,我们把"}或{Re(c,)}称为系 统的焦点量.

应用

例3.10 考虑二维系统

x --- Ji,

, (3.24)

夕=---1 + 产 + 内 A + \emph{!h,xy} + 仙史 y + \emph{\^{}x'y.}

此系统有两个奇点(土 1,0),而(1,0)是鞍点.所以只须考虑奇点
(-1,0)附近发生Hopf分岔的可能性.令S = z + 1,系统(3. 24) 变为

i,

\$ = --- 2\$ + ---冊一产3 + 内)+ F + (松 + 3产3 --- 4贝)布

+ (--- 3凶 + 6仙湾,+ 3 \textasciitilde{}\textasciitilde{} 4四)净' +
\emph{四土.} (3. 25)

这个系统在(0,0)的线性部分矩阵有一对纯虚特征根的条件为

'\% ---円---产3 + ``4 = 0・ (3. 26)

令丿=一 /万久则在条件(3. 26)下,方程(3.25)变成

f 一 /T\%

■ 7 = ZTf - -\^{}zf\textsuperscript{2} +(X + 3产3 \emph{--- 4心訶 +}

\begin{quote}
(一 3产s + 6"\$)伽+ (版 4``,)尸? +四的.
\end{quote}

应用Liapunov系数法,可以得到

\begin{quote}
=寿 S --- 3ft + 8內),
\end{quote}

S志标-g当S。,

預=亏仪,当咯=K = 0.

由此应用定理3M,可得下列结论:

\begin{enumerate}
\def\labelenumi{(\arabic{enumi})}
\item
  若必=両=0,冊尹0,则当``1 =出时发生一阶Hopf分
  岔,并且系统〈3.24)在原点附近存在雎一极限环的参数区域是
\end{enumerate}

ft (ft 一 所) V 0, 0 \textless{} \textsc{IjUj} ---
"zl《\textbar{}\%\textbar{}《1.

\begin{enumerate}
\def\labelenumi{(\arabic{enumi})}
\setcounter{enumi}{1}
\item
  若內=0,``3关。,则当``1 = 4阮必=3\%时发生二阶 Hopf分岔,(3.
  24)在原点附近存在二个极限环的参数区域是
\end{enumerate}

产3(``2 --- 3内)V 0, ``3(凶一冊一 \%) \textgreater{} 0,\\
。V Ml ---門---妇《伏2 --- 3観《I妇《1.

\begin{enumerate}
\def\labelenumi{(\arabic{enumi})}
\setcounter{enumi}{2}
\item
  \emph{若}ft 7\^{}0,则三阶Hopf分岔发生的条件是
\end{enumerate}

11 2 14

Mi = yft, \emph{th =} 亏产4,\emph{氏---}亏兴"

乂(5代---14Q \textless{} 0,㈤(出一 3出 + 8仪)\textgreater{} 0,

版(``1 --- 出---出 + 向)V °,

0 \textless{} 1仏一 % -内+両丨《 0 --- 3代+ 8代丨

\begin{quote}
《\textbar{}5内一14四\textbar{}《\textbar{}饱\textbar{}《1.
\end{quote}

Bautin对右端是二次多项式的平面微分系统(简称二次系统)
的一种标准形式导出了著名的焦点量公式(见{[}Ba{]}),并证明了二
次系统细焦点的阶数至多为3 (他的第三个焦点量公式在符号及
数值上都有误,在{[}QL{]},及{[}TLLL{]}中得到纠正).下面介绍的结
果是把他的公式推广到一般形式的二次系统上,应用校方便.

例3.11 ({[}Lc{]})设

\begin{quote}
\emph{/ x y} + «2o\^{}\textsuperscript{a} + \emph{+
a\textsubscript{al}y\textsuperscript{2}, @ 泻}

I 力=z + \emph{b\textsubscript{iri}3? + b\^{}xy + ,}
\end{quote}

记

\emph{A} dgo 十 角 2,3 缶0 + tin + 2如2,8 =缶 1 + Zdjo,

7 ------\emph{b\textsubscript{ltl}A\textsuperscript{3} ---}
(a\textsubscript{ao} --- 6\textsubscript{n})A\textsuperscript{2}B +
\emph{(,b\textsubscript{m} ---} \textless{}z„)AB\textsuperscript{!}
\emph{a\^{}B\textsuperscript{3},}

\$ = a歯 + 槎 o + a\textsubscript{0J}A + \emph{b\textsubscript{ia}B,}
则(不计正数因子)

\emph{Vj\^{}Aa- Bg,}

V\textsubscript{2} = U?(54 --- \#) + a(5B --- a)卩,如果匕=0,

\emph{V\textsubscript{3} = (Aa} + 80)",如果 Vi = V\textsubscript{2} = 0,

\emph{V\textsubscript{t} =} 0,当 \emph{k\textgreater{}3,}如果
V\textsubscript{1} = V\textsubscript{2} = V\textsubscript{3} = 0 .

(3.28) 在最后一种情形下,系统可积,(0,0)为中心点.

附注3.12从原则上说,当X。以0为细焦点时,总可以通过
有限步运算确定细焦点的阶数\emph{k.}但是当\emph{k}较大时,对一般系统想
用(3. 28)的方式来表达焦点量公式,计算量非常大.例如,当把方
程E3.2?〉右端关于工,丿的另二IgA商芳三状適时,*®•篇树h 最高阶数为5
(见{[}Si{]});但如果在(3. 27)右端补充上三次项,则它

的焦点量公式十分复杂,•即使利用计算机,目前也仅推导出前几个
焦点量公式.确定细焦点阶数和求焦点量公式,与区分中心与焦
点这个困难问题紧密相关.在这里我们仅列举我国学者在这方面
的一些新近工作:蔡燧林、马晖©C给出了判别广义Lenard方程
中心和焦点的较一般方法;杜乃林、曾宪武皿]给出了计算焦点量
的递推公式;黄启昌等网w]研究了泛函微分方程的Hopf分岔问
题;沈家齐、井竹君'叫给出了判别存在Hopf分岔的一种新方法;
黄文灶证明,当非线性方程零点的拓扑度变号时,会产生连通 的分岔曲线,等等.

附注3.13若梧条件(H,)换成

(H) FGr,``)£ X (---。,。),叫),

其中V是酔中原点的一个开集,则当条件和(H)成
立时,或者系统(3,1)当产=0时以原点为中心,或者当产e (---七
0)或(0,。)时(3,1)在U内的奇点外围有唯一闭轨,并且当兴f 0
时,此闭轨缩向奇点.这是Hopf分岔定理的另一种形式,证明可
参考⑵].注意,此处F(z,``)的条件不能减弱为F(工,``)6 请看下例.

例 3.14

{dr}

在=zsin" + \^{}08戸 + (--- zcos/z + ysin``)tan \emph{A\textsubscript{t}}

糸=---xcos点 + 河快 + (--- xsin/i --- jpcos/iitan \emph{A,}

(3.29)戸 其中A = el(2 +血T), r=山+尹,0\textless{}/z\textless{}y.容易验
证,对此系统而言,条件(HD,(Hz)满足,且右端函数是b的,但
不解析.所以(H)不满足.由于

拿 a --- rcosju\^{}tan (e(2 + sinr\textsuperscript{-}\^{} ) --- tan/i],

出(3.29)\^{}\\
其中0 \textless{}r\textless{}l,故原点是\textless{}3.
29)。的渐近稳定焦点.

考虑5")平面上由下式定义的曲线

\emph{卩=}e\textsuperscript{\_}r(2 + sinrT). (3. 31)

\includegraphics[width=1.42639in,height=1.29306in]{media/image27.png}

它显然界于曲线%,lei与曲线?点=3eT之间(见图

\emph{2-2).}由于

糸=广*厂可瘁(2 +血广*) cosrF)

在r = 0的任意小邻域内都改变其符号,所以对任意小的产\textgreater{}0,存
在门(产)尹疽产),冗(产)-* 0当\emph{叶0,}使得r =小产)均满足方程

(3.31) Q = l,2).从而由(3. 30)和(3. 31)得到 这说

明对任意小的声,系统(3.29)\#在原点附近都至少有两条闭轨.因
此,上述结论的条件F3,G w b不能减弱为e c\textasciitilde{}.

请读者验算,此例中对一切正整数払都有Rec* = O.因此,对
6系统而言,当对一切正整数虹Rec* = O时,奇点不见得是中心.

对彖数一致的\textbf{Hopf}分會定理

考虑C\textasciitilde{}平面系统

一京+次3,心),

\emph{=瑩 + Sg(.x,y,!i,3),}

其中函数\textsc{h = h(z}戒),参数amWR',且\$为小参数.设系统有 奇点*=
0,而且在该点的线性部分疤阵有特征根心产,。)士\emph{诺3,}
»若存在。\textgreater{} 0及在0 V \$ O定义的函数产=\emph{呻},满足条件
(Hf) aO(5),5)=0,伙兴(5)0)尹 0,

则在一定的附加条件下,当产=产3), dW (0,b)时,系统(3.32) 可在x =
0点发生Hopf分岔.对每一固定的(0,。),利用推论 3.7可知,存在e(\$)
\textgreater{} 0,使得当\textbar{}兴一兴(5)\textbar{} \textless{}
£(S)且兴〉 狀心(或 时,系统有极限环.问题是:当df 0时可能有

*
0.我们希望找到系统满足的条件,以保证存在正数\emph{3\textsubscript{0}}和
略,使得对所有的dW (。,出),都有e(3) \textgreater{}
e\textsubscript{0}.这就是所谓对参数 一致的Hopf分岔问题,见图2-3.

\includegraphics[width=3.35972in,height=1.25347in]{media/image28.png}

图2-3

代替定理3.1中的条件(H,)和(HD,下文需要的条件是

(H; ) «■ = lim \emph{丰} 0,

(H/ ) = lim 4rRe{[}ci ("(3) ,\$){]}\emph{丰} 0.

\begin{quote}
J---0 O
\end{quote}

定S3.15 设系统(3. 32)有奇点(布,北),系统在此奇点的
线性部分矩阵有特征根a璀,d) ±诉璀,\$).又设存在爲\textgreater{}0和在
\emph{0\textless{}8\textless{}8,}定义的函数产=产3),使条件(或),(电)和(H\#
)成 立.则存在\& \textgreater{} 0(晶 W 曷),。\textgreater{} 0
和在x\textsubscript{o}\textless{}x\^{}x\textsubscript{o} +
\textless{};,0\textless{}5 \textless{}
上定义的唯一函数兴=Zi(z,S),满足方(女,0) = 0,而且

(1)当兴=\emph{h5)}
,x\textsubscript{0}\textless{}x\textless{}x\textsubscript{o} +
ff,0\textless{}5\textless{}\^{} 时,(3. 32)
过工丿平面上的点(工,0)的轨道是它的唯一团轨乌.当廿\textless{}0时,
玲是稳定的极限环;当\textgreater{}0时,七是不稳定的极限环;

⑵当ay
V0时,g(x,5)\textgreater{}0\textsubscript{(}当宀;\textgreater{}0时,五GM)
\textless{}0.

证明 不妨取愆。,如)=(0,0).与定理3.1的证明类似,所
不同的是以产-产3)代替那里的产,而以\$为参数,则那里的后继
函数\textsc{V(h'')}变为\emph{V(.x,fi -}兴(8),3)的形式.注意到\emph{3 =
0}时, (3.32)为 Hamilton 系统,因此 VGr/ ---兴(0),0) = 0.代替
(3.16),我们有

\emph{V(.x,ft ---}产(3),S) = \&V* (工,兴---兴(5),3).

从条件(H;),(H;)可得

\emph{気■(0,0,0)砖0,} 冢(0,0,0)尹 0.

对广(礼``一兴(5),3)在=(0,0,0)点用隐函数
定理即可,其它推理与定理3.1的证明相同.I

推论3.16在定理3.15的条件下,存在\&\textgreater{}0(\&M曷
。和(如必)的邻域U,使得

\begin{enumerate}
\def\labelenumi{(\arabic{enumi})}
\item
  当 0 V6V,,伏一兴(\$)丨〈"且a七:(兴一兴(S)) V0 时,系统(3.
  32)在U内恰有一个团轨.\emph{当c: V} 00 0)时,它是稳 定(不稳定)的极限环.
\item
  当 0 \textless{}\$\textless{}晶,1``一兴(\$)1 V。,且
  a'C:以一``0))\textgreater{}0 时,系统(3. 32)在U内无闭轨.\textbar{}
\end{enumerate}

在第三章§ 1中,我们将看到这种对参数一致的Hopf分岔定 理的作用.

\textbf{§4}平面上的同宿分岔

由二维流形上的结构稳定性定理(见第一章定理L 13)知道,
当存夜鞍点的同宿轨(或异宿轨)时,系统是结构不稳定的.事实
上,这种连接鞍点的轨线在扰动下可能破裂,从而改变系统的拓扑
结构.第一章例2.10就是一个典型的实例.在本节里,我们要进
一步讨论,当这种分岔发生时产生闭轨的规律.

我们先从凡何上考虑,以获得一些启示.设平面上的单参数向
量场族X户对应于如下的方程

等= M"), (4.1)

其中@ £ C\textasciitilde{}(R2 x R,R*),设X。的轨线结构如第一章图1 -9(b)所
示;它有一条初等鞍点的同宿轨\emph{r,r}内部是稳定焦点的吸引域.
当产n。时,「可能破裂为两条分界线(鞍点的稳定流形与不稳定
流形),见图1-9的(a)与(c).

显然,在图1-9©的情形,分界线破裂的方向与破裂前F的稳
定性相配合,就构成了一个Poincarfi-Bendixson环域,从而系统X户
存在闭轨.当1川充分小时,可以使这个环域充分靠近原来的同宿
轨线.因此,我们可以认为闭轨是从F经扰动破裂而产生的(或反 过来说,当
时,闭轨趋向于「而成为同宿轨).这种分岔现象

称为同宿轨的分笛,或简称为同宿分岔.

考虑向量场族

(XQ: 糸=/\textbackslash{}工,心 (4.2)

其中 /■ E \^{}(R\textsuperscript{2} X R,R\textsuperscript{!}),/(0,0) =
0, X\textsubscript{o} 以 h = 0 为双曲鞍点 (即det莹(0,0)
V0),并且具有同宿轨匚,如图2-4(a)所示(对情 形(b)可类似讨论).

QQ xo

(a) \textless{}b)

ffl 2-4 图 2-5

从前面的讨论中可以看出,在研究同宿分岔时,下面两个问题 是重要的,

\begin{enumerate}
\def\labelenumi{(\arabic{enumi})}
\item
  如何判断X。的同宿轨匚在其内侧的稳定性?(在第一章
  例2.10中,这是利用r\textsubscript{0}内的焦点的稳定性得出来的.我们希望从
  向量场在鞍点和n的特性来获得这个信息.)
\item
  如何判断x户的稳定流形与不稳定流形的相互位置?
\end{enumerate}

为了解决问题(1),我们先对X。在r\textsubscript{0}内侧引入Poincar®映射
(见第一章定义1.6).取\emph{勿 b"} 为过h点与兀正交的无切线
(向内为正).则在如附近存在一个邻域U,使得\emph{peu(\textbackslash{}L+,}从
\emph{P}出发的轨线甲(£,0)经过£ = T5)将再次与L交于一点『3)= P
33)``)(见图2-5).令Mo为沿着\emph{Lt}的单位向量,则V /■ e U
Cl乙才,它有如下的坐标表示

\emph{/\textgreater{} = fla + o ng,} (4. 3)

其中«\textgreater{}0.相应地,P5)有坐标表示

P3)= 0CT3),P)= 6+ R(a)«o, (4.4)

其中供a) eC*,H要o\textless{}a《l(因为X。是的).定义函数\\
\emph{d(a) =} £(a) --- \emph{a,} a \textgreater{} 0.

定义4.1 X。的同宿轨匚称为是渐近稳定(或不穂定)的,如

果存在7\textgreater{}o,使得d(。)V 0 (或\textgreater{}0)对所有的0
成立」

附注4・2 注意limdM) = 0,因此A的穂定性由极限

= limg'(a) --- 1

\begin{quote}
ff-*0 a-*0
\end{quote}

所决定:当lim\#(a) \textless{}0 (或\textgreater{}0),也就是iim伊(a)
\textless{}1 (或\textgreater{}1) *t-*0 ft-*。

时是渐近稳定(或不稳定)的.

引入记号

\% = tr 監(0,0).

定理4.3设X。具有双曲鞍点O及同宿轨儿,如果%尹0, 则当1 \textless{}
0时,匚是渐近稳定的,而当叫\textgreater{} 0时儿是不稳定的.

证明([CH])如上所述,取\emph{p\textsubscript{o}er\textsubscript{a},}可建立X。在A,邻域的
Poincare 映射

\emph{p:} u n 说-灼,尸3)=

并且?,?(?)分别有表示式(4. 3)和(4. 4).把(4. 4)式对«求导, 得到

伊(心° =[翱(7'3),初\sout{气絆}''0 +宴 »o

\emph{™ vp op} i-T(M

\textsubscript{f} {3T3)} 丄亜 八u、

\emph{=§8} \sout{由} ''。+ 敦 «o, (4.5)

\emph{°P \textsuperscript{3}P} \textless{}=-T\textless{}fl

其中/'« = ■/"(皿 o + \^{}0,0) , « € R,而 \_/"\# = \emph{fy}
注意心是沿' 的方向,当a足够小时,内积〈毋,就尹0,在(4.5)式两端以门
作内积,得到

\begin{longtable}[]{@{}lllll@{}}
\toprule
\endhead
& & \emph{S驀} & \%〉 &\tabularnewline
如果记 & & \emph{〈什,q ■} & &\tabularnewline
\begin{minipage}[t]{0.17\columnwidth}\raggedright
\strut
\end{minipage} & \begin{minipage}[t]{0.17\columnwidth}\raggedright
\emph{些}

\emph{\textsuperscript{3}P}\strut
\end{minipage} & \begin{minipage}[t]{0.17\columnwidth}\raggedright
\emph{\% = wn} + 例 °,\strut
\end{minipage} & \begin{minipage}[t]{0.17\columnwidth}\raggedright
(4.6)\strut
\end{minipage} & \begin{minipage}[t]{0.17\columnwidth}\raggedright
\strut
\end{minipage}\tabularnewline
\bottomrule
\end{longtable}

其中,则

\emph{,3 =}品 (4. 7)

\begin{quote}
下面,我们设法把iim?与1建立联系.首先,由的连续性, a-*0
\end{quote}

可以把互表示为

\begin{quote}
= (1 + §)兀 + 取如 (4. 8)
\end{quote}

其中勻舟-* 0当«-\textgreater{} 0.另一方面,由于

\emph{9} (T(/\textgreater{}),力)* U n 命 f Lo"o + 嘔 ipo + 夕(a)\%,\\
它的导映射宴 把洌+嘔处的切向量见映到/\textgreater{}。+ 83版

\begin{quote}
\emph{dp}
\end{quote}

处的切向量刀,即

\begin{quote}
柔\textless{}4.9) \emph{dp t-n.fi)}
\end{quote}

由(4.8),(4. 9)和(4.6)得出

\begin{quote}
\emph{票} /■\# =(1 + 勺 + \emph{e*)fp} + qg (4.10) \emph{°P
E3\textgreater{}}
\end{quote}

(4.10)和(4. 6)给出卖 在基向量S0 下的矩阵为

\begin{quote}
\emph{\textsuperscript{d}p} ■丁S)
\end{quote}

1 4- E1 + \&\& f

5 』'

从而

det 绑 =(1 + \&)\% (4.11)

\begin{quote}
\emph{°P} I
\end{quote}

其次,我们来计算上式左端的行列式(它与坐标系的选股无关).注

意斜/)是变分方程

* =苦(心,》0)``

det(豪(3)) = exp fti•务/(卩\emph{(4.12)}

''(a) = {]} 匚3氏黑/"3«,力),0)山.C4.13) 条件% = \textsuperscript{tr}
£\^{}°,°)V 0 (或〉0),保证存在0(0,0)的邻
域V,使得当般卩时,有tr\^{}/(x,0)\textless{}\^{}\textless{}0(或\textgreater{}号\textgreater{}0).
当\emph{0\textless{}\textsubscript{a}\textless{}S,B}足够小时,T3)=孔+
\emph{T\textsubscript{it}T,}是流? \emph{\{t,p)}停留
在V中的时间.当a-0时7\textbackslash{}\textasciitilde{}+8,而" 是有

界的.因此,由〈4.13)式容易推得

lim夕,(a) = 0 (或 + 8),当% \textless{} 0 (或 \textgreater{}0),

由附注4.2立得定理的结论.\textbar{}

定理4.4设向量场X户由(4. 2)给定.假设X。以。为双曲戟
点,有同宿轨匸,并且% =比基/(0,0)共0.则存在8 A 0和彳〉
0,使得当\textless{}8时,如果X户在匚的寸邻域内有闭轨「户,那么马
是唯一的闭執3并且当% V 0 (\textgreater{} 0)时,「户是渐近稳定(不稳定)
的.

证明 我们只须考虑这样的闭轨「户,当兴\textasciitilde{}0时,它趋于马.
取据€ 并取乙和为同前,则当I川足够小时,「户必与巩横 截相交,记\emph{\% =
5}
玲对于\emph{也}点附近的\emph{\textsc{P\^{}L\textsubscript{b},}}引入坐标表
示 0 = 00 + \emph{呢,}(a\textbar{} \textless{}S.蚂有 0户=角 + \%no,
{]}a``\textbar{} V。,当 I以
足够小时.对于任意固定的产,可在乙上%点附近建立X户的 Poincar色映射%

0 =做。+ 力0 匕。)=§户(a)''o + 久,\textbar{}a ---\%\textbar{}《1,
并且\&(\%)=财则八的稳定性由{[}£/(\%) -1{]}的符号决定.重
复定理4.3的证明方法,并注意当兴〜。时,球f八,气〜0,因 此

炽g=\{:8,如負

这说明当\% \textless{} 0时,弓,渐近稳定;当% \textgreater{}
0时,「户不稳定.

另一方面,两个具有相同稳定性的闭轨不可能并列共存,因此
「户是X*的唯一闭轨.E

现在,剩下要解决的就是我们在前面所提的问题(2),即X。的
同宿轨孔经扰动破裂后,如何判断X兴的稳定流形与不稳定流
形\emph{w:}的相对位置? Medikov函数就是用以描述和昭之间
``有向缝隙''的判定量,从而解决这个问题.

先把方程(4. 2)改写为

(XQ: 糸= ■/"(•!)+点gGr,Q, (4.14)

其中 \& 仃(史,R2),g ee(R\textsuperscript{J}XR,m,r\textgreater{}2.设X。以
x\textsubscript{0}为双 曲鞍点,有同宿轨设匚有表达式

r\textsubscript{o}i \emph{工=\textless{}p 0) , -*x\textsubscript{a},}当
Z-*±8.

a. \emph{b,}

对于平面上的向址4= @=.,定义

的丿 1皈

。丄= \emph{,a l\textbackslash{} b =} \textless{}a,»〉.

容易验证,对于任意二阶方阵A ,有

(如)A 6 + a A (Ah) = trA (a A ft). (4.15)
过匚上的中(0)点取截线E,使它沿法向»(0) =
(P\textsuperscript{,}(0))\textsuperscript{1} = (9=(0)).当I "
I《1时,X*有双曲鞍点七,及其稳定流形吧与
不稳定流形这里珏\ldots{}互,当产\ldots{}0.利用[Sh]关于鞍点分界
线光滑依赖于参数的定理,当\textbar{}川《1技}0(或tMO)时,X户有唯
一有界解卬;(\$)(或 町W)),它与充分靠近,且当\emph{t〜} +
8(或一8)时,"汐)(或昭(少f邛.为了描述";与";
的相互位置,我们引进下面的定义,它可以看成是吧与 昭间'■缝
隙''沿汉0)的投影(相差一个非零常数倍).

dO) =〈W;(0) --- W\%(C)),尸(p(0))\textgreater{}. (4.16)

D(J) =g(p(e),O) A y(P(z))=

定理4.5设/増£仃彳22,乂。有同宿于双曲鞍点瓦的轨 线几,则对扰动系统X''有

泌(Q = M +。(1``卩),

其中

\begin{quote}
\emph{J ---8}
\end{quote}

而 D(Q 9。)由(4.17)和(4.18)定义.

证明记

辭照)匚=冲5,\\
沖)匚=5 V。,\\
zl\textsuperscript{s}(t) = ?(O A /(\^{}(/)), r\textgreater{}0,

\begin{quote}
孕《)= /(£)人 /■(?(,)),£ W0.
\end{quote}

注意应】《)是(4.14)的解GN0),把它代入(4.14),对``求导后取 `` =0,得到

耳戸=堂整以政 r)+g33),0), (4.22) 因此,把(4. 21)对,求导,再利用(4.
22)和(4.15)得

£厶七)=\^{}2 A +\^{}(O A \emph{£心。))}

\begin{quote}
\sout{=可鷲以}如)A /(P(i)) +\^{}(P(O,0) A

+ 打f) A \sout{''鷲地}(M)) 利用常数变易公式可得
\end{quote}

骨\$)=萨)何(0) +jym(gg),O) A

(4. 23) 另一方面,因为 f f8 时 p«)---互,所以 /(\textless{}p (())
-*/(x\textsubscript{0}) = 0, 且趋于零的衰减率为?七这里4
A为X。在双曲極点钊的线性部 分矩阵茎S。)的特征根島\textless{} 0
\textless{} A\textsubscript{2}.再次利用[Sh]关于鞍点分
界线光滑依赖于参数的定理可知,当时营(Q有界.从而由 (4.
21)知"-\textgreater{}8时\textsc{△sq)}〜eV --- 0.记% = tr茎(女),则易知
AV\%〈為,故当\emph{t-g}时厂艸)〜e-财,因此有 limeF**(t)=
\textsubscript{0},

\begin{quote}
(-*QQ
\end{quote}

从而由(4. 23)式及上式得到

\^{}\textsuperscript{s}(0)
=-\textbar{}\textsuperscript{+}™D(z)e\textsuperscript{\_\textless{},\textless{},)}dt.
同理可得

空(0) =一 J-" ZJ(0e\textsuperscript{\_O\textless{}I,}d/.

从(4.16), (4. 20)和(4. 21)易知:d(0) = 0, d,(0)=泌(0)- \^{}(0)
=4.定理得证.\textbar{}

由定理\emph{4.} 3,定理4. 4和定理4.5立即得到

定34.6设X\#由(4.14)给定,X\textsubscript{0}以\emph{处}为双曲鞍点,且有顺
(或逆)时针定向的同宿執兀,并且%尹0,则存在3\textgreater{}0和7\textgreater{}0,
使得当I产\textbar{}〈呑时

(1)若\emph{。花'\textgreater{}} 0 (或〈0),则X\#在匚的牛邻城内恰有一个
从r\textsubscript{0}分岔出的极限环.当tf„\textless{}0时,它是稳定的;当\textless{}70
\textgreater{}0时,它 是不稳的.

(2)若女心<0(或>0),则X``在匚的化邻域内不存在极 限环"

注意在4的表达式中含有\emph{9} (0,这使应用定理4. 6受到限制.
但在某些情形下无须求P (O便知DQ)是定号的,从而厶与\emph{D(t)}
有相同符号,参见下而的例子.

例4. 7设\emph{x,y} € K,平面系统

I x = P(z,y),

I \emph{y} = Q(3),

以原点为双曲鞍点,有顺时针定向的同宿轨匚,并且翌(。,0) +
\^{}(0,0)尹0,则对充分小的\textbar{}产\textbar{},当啊> 0时,投动系统

/ 土 = ,少一产 Q3,y), (4 24)

\begin{quote}
I \emph{y = Q(x,y)} 4- \emph{ft} P(x,y) '
\end{quote}

在珏的小邻域内恰有一个极限环(其稳定性由%的符号决定);而
当\^{}<0时,(4. 24)在匚附近没有极限环.

事实上,

-M)=de\% 二(尸+ Qz)\textbar{}g門)RO,
并且等号仅在个别点上成立,因此\^{}>0.利用定理4. 6,上而的结 论立即可得.

附注4.8在上面的定理4.4和定理4.6中菴有条件%尹0. \% =
0称为临界情形,此时称鞍点为细®点,这就出现了退化同宿
JJ-a.RoussarieM和化殉貫分别讨论了从(退化的)同宿轨分岔
出多个闭轨的问题.他们的基本思想是在奇点附近利用鞍点性质,
与大范围的微分同胚相结合,得出PoincarE映射的表达式,从而在
退化程度较高时,可以经过逐次适当的扰动,反复改变同宿轨内侧
的稳是性,产生多个闭轨,并在最后一次扰动时,使同宿轨破裂面
产生最后一个闭轨.此时,系统(4.14)右端的扰动项g中除了产
外还含有其它参数.由于介绍退化情形的同宿分岔需要较大篇

幅,此处从略.但对于Hamilton向量场的扰动系统,我们将在§6
定理6.4中介绍一个常用的结果.

罗定军、韩茂安和朱德明在[LHZ]和[HLZ]中,对孤立和非
孤立同宿轨在扰动下产生极限环的唯一性作了详细的讨论;冯贝
叶E得出了在临界情形下判别同宿轨或异宿轨的稳定性的方法;
MourtadaMJ对含两个鞍点的异宿环的分岔问题进行了深入的研 究.

对参數一致的同宿分岔

类似于对参数一致的Hopf分岔问题,现在考虑含双参数凯产
的平面Hamilton扰动系统

其中\&为小参数;H = \emph{H\^{}x,y)}为Hamilton函数;而且\emph{H,f,g}有
足够的光滑性.设当0为时,系统有双曲鞍点(叼,无),而且
存在函数``=产。),使当``=产("时,系统(4. 25)有鞍点
的同宿.轨门.则在适当的条件下,对每一个固定的\emph{6\textgreater{}0,}存在
\textsc{e(5)\textgreater{}0,}使当g 一产0)\textbar{}
Ve("时,在乌的邻域内有定理4. 6
的两条结论.我们关心的是:当0时,如何保证e⑹不趋于零, 参见图2-3.

我们不在此给出一般的定理,只在第三章引理丄6中对一类
特殊的系统介绍这种对参数一致的同宿分岔的结果.

\protect\hypertarget{bookmark86}{}{}\textbf{§ 5
Poincare}分岔与弱\textbf{Hilbert}篥\textbf{16}问题

本节考虑平面向量场族

(XQ: \emph{¥ = m + 探3,产),} (5.1) 其中 \emph{f} 6
(7(R\textbackslash{}R0 \emph{,g} e \textless{}\^{}(R\textsuperscript{2}
X R ,R\textsuperscript{a}) ,r \textgreater{} 2.设 X。具有周期
环域,即存在一系列团轨

r\textsubscript{A}\textgreater{} \textsc{\{'' h(z)}\emph{= h,} hi
\textless{}ft \textless{}ftj\},

其中函数\textsc{h}e
\emph{c\textsuperscript{+}'.}我们关心的问题是,x°的哪些闭轨马,(加\textless{}
\emph{損6} 经就动(階l《i)能成为x``的极限环知(即当``f 0时,
\emph{\textsc{Ll
3?}}并且究竟能从r\textsubscript{Ao}扰动出X户的几个极限环1这就是
Poincare分岔问题.

Poincare 分為

本节的■'个基本假设是,闭轨族\emph{r\textsubscript{h}}关于砍在統附近)单调排
列(当X。为Hamilton系统,且\emph{H}为相应的Hamilton函数时,,这个
假设总是成立的).因此过「标上任意一点的无切线可用为参数 化设

马\textbar{} \emph{x --- \textless{}p , 0 W t WTh,} (5.2)

其中\emph{孔}是\emph{r\textsubscript{h}}的周期.为了考察当``尹o时x户过\textsc{"om)}的解能
否成为闭轨,我们沿马在會co,ft)点的法线方向尸3 (0M))取 无切线乙 \emph{设H
= Mt,hg}是系统(5. 1)的解,满足初值条件式0' \emph{3=甲3}设此解当£ =
\emph{T\{h},产)时再次与乙相交,则由微分 方程初值问题的解的唯一性知

x(i,A,0) = p(t,A), \emph{= T\textsubscript{h}.} (5.3)

定义后继函数

G0*) = --- z(0,h,``),\emph{f\textsuperscript{L}}
(p(0,h))\textgreater{},

(5.4) 则显然当0时,关于我的零点对应于X户的闭轨, 且 Geer.

由于X,在马。附近均为闭轨,故利用(5. 3)和马以与为周 期,可以从(5. 4)搭到

G(A,0) = \textsc{\textless{}x(T\textsubscript{k},A,0)} --- x(0,A,0),
X\textsuperscript{1} (?\textgreater{}(0,A))\textgreater{} s 0,

忻一知\textbar{}《1.因此

GS,``)= ``(@(方)+ *(方,``)). (5. 5)

定理5.1 (1)若A = A,O\textless{}
\textbar{}``\textbar{}《1时,衣撬,产)为X``的 闭轨,则必有①(万)=0,

(2)若存在自然数加1 使

\textless{}P(A) = 小1\textgreater{}(万)=0,
\textless{}5\textsuperscript{a)}(A) = 0,(5. 6)

则存在\textless{}7\textgreater{}0,5\textgreater{}0,使当0V
\textbar{}川O时,X户在的呑邻域内至
多有左个闭轨,它们是X``的极限环.特别,若@(兀)=0,\^{}(1)= 0,则当。V
\textbar{}"\textbar{} O 时,X\#在孑的\$邻域内恰有一个极限环.

证明
⑴若@6)尹0,则存在\$\textgreater{}0,使当\emph{\textbackslash{}h-h\textbackslash{}\textless{}s\^{},}
\textbar{}@6)\textbar{} \textgreater{} \sout{性N} \textgreater{}0.由(5.
5)可知存在 \textless{}T\textgreater{}0,当 0 V \textbar{}产\textbar{} V。
且\textbar{}A-A\textbar{}\textless{}
\$时\emph{,G渾函}无零点,即X户在马的呑邻域无闭轨.

(2)反证.设(5. 6)成立,但V
\textless{}7\textgreater{}0,不存在\$\textgreater{}0,使当\emph{\textbackslash{}u\textbackslash{}\textless{}}
b时,X»在马的\$邻域内至多有\&个闭轨.因此,存在佐,和正数
满足当\^{}f+8时土f0,\$„,-\textgreater{}(),并且X\%在马的么邻
域内至少有\&+1个闭執,即GS,\%)对我在\emph{\textbackslash{}h-E\textbackslash{}}
V、内至 少有* +
1个零点.利用Rolle定理可知,存在妇,満足\emph{\textbackslash{}e\textsubscript{m}-K\textbackslash{}\textless{}}
而且

的)(妇+心多晳(妇,丄)=。・

令mf+8,得磴=0,这与已知条件矛盾.I

为了实际应福的方便,下面的定理给出从4的表达式与原方 程来计算(5.
5)式中的\textless{}P(A)的公式.

定理5. 2 对于方程(5.1)和马的表达式(5.2),有

@(五)=丄气-虹`满七以"德),0) A \emph{f(.\textless{}p} (t,A)))dt, (5.7)
其中

\emph{o(t,h) =} j tr 莹(甲(浦))df.

证明 从(5. 5)和(5. 4)知

臥五)=家?(矿户) =\\
叩 夕・0

階钏+劉\textbar{}〃广豢。,",尸皿5», 其中寄与篆在工=*箫,0)取值.由于x
= x(i,A,O)是X。的 解,利用(5. 3)及马为\emph{琮}的周期可得 鬃
=/(x(T\textsubscript{ft},A,O))
\textsc{=/(?\textless{}T\textsubscript{a},A))} =/(p (O,AJ),

i=x\textless{}T\textsubscript{v}A,O)

故

旳)=\textsc{\textless{}\^{}(T\textsubscript{a},A,O)} 一
寄(强,0),尸(p(0,A))\textgreater{}. 若令

△(5=〈制,五,0),产(g,五))〉

\begin{quote}
=孚(赫,0) A
\end{quote}

则

@5) = \&7\textbackslash{}仇)一厶(0,五). (5.8)

与§ 4中(4. 23)式的推导相类似,可得AQM)的表达式如下 ""△(0,丸)+
j;e-心\%何(璀),0) A /(?(i,A))d().

. (5・9)

由推论2. 6知,当g---研《1时,\textless{}7(T*,A) = 0,因此,由(5. 8)和 (5.
9)得到(5.7). \textbar{}

附注5.3 当©(五)三0时,(5. 5)成为

G6,``)= \textsc{a?(3(}五)+ ``攻(知 Q).

此时称饱(五)为高阶MelMkov函数,从它的零点分布可研究扰动
系统闭轨的个数.另外,在应用定理5.1讨坨分岔问题时,系统
(5.1)中的右端除了依赖于扰动参数兴外,还可能依赖于其它参数 A €
R*,从而@6)转化为®(A,A),由此研究对A的不同取值史(如
Q相对于方的零点个数的变化.

\textbf{Hamilton}系统的扰动与將\textbf{Hilbert}第\textbf{16}向通

在实际应用中,经常出现(5.1)的一种特殊形式,即Hamilton 向量场的扰动系统;

糸=-當+心嵐丸小),

-, \_ (5.10)

这里变量参数产£R,AWR七映射Heb+'F'QEtr, r\textgreater{}2.设当``
=0时,未搅动系统有闭轨族,并有表达式

\{(科少 IHCz,/) = \emph{h,h\textbackslash{} \textless{} h} \textless{}
知\}.

假设r\textsubscript{A}关于\emph{h}单调排列.对于系统(5.10),公式(5. 7)成为
心)=-赤」襄P+费加. (5.11)

注意当产=0时,沿马有

\begin{quote}
{dx} \emph{dH} dy \_ \emph{dH dt 3y \textsuperscript{9} dt 3x *}
\end{quote}

因此,可把(5.11)改写为

QCc、勿O,A)dx --- 冷dy・(5.12)

\begin{quote}
\emph{h}
\end{quote}

JLC. mnwrWS,张芷芬(见{[}N{]})和陈翔炎©最先对系统
(5.10)给出了相应于定理5.1的结果.

附注\textbf{5. 4}当\emph{为jy}的'' + 1次多项式,巳\textbf{Q}为工,
\emph{3}的次数不大于孙的多项式时,\textless{}5.12)是〜个\textbf{Abel}积分,寻求它对
\emph{h}的零点个数问题称为弱Hilbert第16问题.由于这个问题是V. I.
Arnolds\^{}】首先提出来的,有时也称为Hilbert-Arnold问题.1900

年,D. Hilbert™在第二届国际数学家大会上提出了 23个数学问
题,其中第16个问题的后半部分是:右端为*次多项式的平面系统
极限环的最小上界H00是多少?可能出现的极限环相对位置如
何?近一个世纪以来,特别是最近几十年来,出现了大量的工作.
例如,史松龄S和陈兰茹、王明淑也WJ最先分别举例证明H(2) \textgreater{}
4;李继彬、黄其明队如举例证明H\textless{}3) \textgreater{}
11;叶彦谦、陈兰茹和杨信
安(见[CY]和CYY])利用[Zzfl,2]中关于极限环的唯一性定理,
证明二次多项式系统中按叶彦谦分类的\textless{} I )类方程至多有一个
极限环,等等.在叶彦谦等的专著[Yl,23,蔡燧林和蔡燧林、张平
光的综述文章[Cs]和[CZ],马知恩的专著[Mz],梁肇军的专著
[Lz],以及Dumortier等人的系列文章[DRR1,2]和[DER]中,读
者可发现大量有趣的结果.经过nyashenko™和Ecalle顷修补证
明后的Dulac關有限性定理指出'一个给定的''次多项式系统的极
限环个数有限.但是,对全体''次多项式系统而言,其极限环个数
的一致上界如何估计(哪怕是否有限),即使对\&=2这种最简单的
非线性情形,仍是一个未知的问题.S. Smal/w认为,对H('')的
研究可能是Hilb*问题中最困难的一个问题,可见这是对数学工
作者的\_个重大挑战.

由(5.5)式可知,Abel积分(5.12)是系統(5.10)在儿附近后
继函数的一阶近似.因此,积分(5.12)的零点个数与系统(5.10)
的极限环个数密切相关.当H是» + 1次多项式,户和Q为刀次多
项式时,(5.10)是一类特殊的/次系统,即Hamilton向量场的扰动
系统,所以把研究Abel积分(5.12)零点个数的问题称为弱Hilbert
第16问题,还应指出的是,当所考虑的多项式系统接近可积而非
Hamilton系统时,需要先乘上一个积分因子,才能把它化为(5.10)
的形式.此时\emph{H,P\textsubscript{t}Q}可能不再是多项式.习愦上仍把(5.12)祢为
Abel积分,把求其零点个数的问题称为弱Hilbert第16问题.

例5 考•虑van der Pol方程

\begin{quote}
而+产(?一1)击+工=0,
\end{quote}

其中0 V I产I《1.可把它改写为如下的等价形式 dx 否=外

\begin{quote}
V 糸=-H + "(1 --- X\textsuperscript{2})\^{}.
\end{quote}

当J« = O时,(5.14)\emph{为}Hanulton系统,它的闭轨族可表示为

\emph{r\textsubscript{k}\textsuperscript{s}} (/Z(x,y)
-\^{}―x\textsuperscript{2} + y = A\textsuperscript{a}, A \textgreater{}
0), 或写成参数方程

\emph{r\^{}t x --- hcost, y} --- ftsini.

代入(5.11)式,得到

\emph{歌h) =} --- 2 J A\textsuperscript{2}sin\textsuperscript{2}f(l ---
ft\textsuperscript{2}cos\textsuperscript{2}/)dt 2成』]---\emph{牛).}

显然儿=2是它的唯一正零点,样且是简单零点.由建理5丄当
供\textbar{}《1时,系统(5.14)有唯一的闭轨匕当兴f 0时,y趋于半径
为2的圆周.

在这个例中,由于\emph{鱼g} 可以积分为显式,使问题很快获解.
但在多数憎况下问题并不如此轻而易举,请看下例.

例5.6考慮平面系统X''如下:

专=好。(\textbar{}冲),

\begin{quote}
(5.15) =--- 1 +X\textsuperscript{2} + "(a + \emph{x\textasciitilde{})y}
+ 0( I芹卩),
\end{quote}

其中变量h J W R,参数产,aWR成为小参数.显然,X°为 Hamilton系统,它有首次积分

\begin{quote}
\emph{H\{x,y)} = 3 + z ---旨=九. (5.16)
\end{quote}

容易得知的轨线分布如图2-6所示,环绕中心点B(--- 1,0)为
同期环域,它以奇点3和鞍点4(1,0)及其同宿机(称为同有环)

\includegraphics[width=2.17361in,height=1.17361in]{media/image29.png}

图2-6

为边界.闭轨族为

{(工``)回3)=丄一号〈龙V ■卜
当A---\textbar{}时,马缩向奇点旦当A-*-\textbar{}时必扩大为同宿环.

利用定理5.1,定理5. 2和(5.12),为了研究当0 V M\textbar{}《l时
X户是否有闭轨,需要对不同的«研究积分

©(方,a) = § (a + (5.17)

关于作\textbar{}-f 4)的零点分布规律.虽然(5.17)形式上很简
单,但它不能积分为初等函数的有限形式,研究它的零点规律并不
容易.本节主要介绍两种研究Abel积分的方法,文献中常见的
Picard-Fuchs方程法(例如,见[CS]),和我们最近得到的直接方 法.

\textbf{Picard-Fuchs} 方程法

首先把(5.17)改写为

= «I\textsubscript{0}(A) \emph{+h(.h),} (5.18)

这里设

\begin{quote}
孔(九)=6 xSdx = 2「
\end{quote}

Jr\textsubscript{4} JfCA)

其中人= 0,l,2,・・・SS)V代力)是马与工轴的交点的横坐标应=

\begin{quote}
满足(5.16).注意

■7(A) \emph{xydx}
\end{quote}

1W 1

\emph{g) \_ \textsuperscript{Lr }}壮

\begin{quote}
J加
\end{quote}

故可定义

又由于、件所围成的紧区域的面 积.所以当A\textgreater{}-y时,

旳,a)=O 0 \emph{F(fi,a)} a - \emph{P(h)} = 0. (5. 21)

定理5.7 ⑴当一\^{}-\textless{}A\textless{}y时,奇VP(方)VI;且 limF(A) = g.

Z2/3 \emph{t}

⑵当一M V九V M时『(曷\textless{} 0)K iim PS)=-吝, lim?(Q=-8. .

12/3

由定理5.1,定理5.7和(5.21)可知,当5/7 VaV 1且\textbar{}川《
1时,X\#存在唯一闭轨,在证明定理5. 7之前,先证一个引理・ 引理5.8
\emph{P0)满}足如下的Riccati方程

\begin{quote}
(9A\textsuperscript{2} - \emph{4)f --- 7P\textsuperscript{l} + 3hP -} 5.
(5. 22)
\end{quote}

证明 将(5.16)对\emph{h}求导,其中y =火"),得到榮=§
再利用\emph{,h)}=夕3以),五)=0,可从(5.19)得到

竹(力)/

(A) = \emph{2} ---dx.

Jf(A) 了

因此,由(5、16),(5.19)和(5. 23)可得

\emph{W} = 2 亨\& = \emph{2hl\textsubscript{k}'(h) -} 24+1,⑶ + 争砰3«).

(5. 24)

另一方面,利用分部积分可得

\emph{I")}=龙{]}{]}(4+F 0)---九+須,"))。

从(5.24)和(5.25)中消去匕+3`以),得到

(2A + 5)Z/h) = 一 4孔+1,伝)+ 6W/ 0).

特别地,

\begin{quote}
J 5爲=---\emph{Mi + ,}

1 \emph{71\textgreater{} ---} - 4" +6柘
\end{quote}

由(5. 16)\emph{有} dH =园夕 + (1 --- \#)\&,从而(1 --- \emph{= ydH}

\emph{-y\textsuperscript{2}dy.}沿积分此式得=/„(*).在(5. 26)中以L代
,,并反解出A/和看`,最终得到

("---4)时=*爲 + 7A,

\emph{-w} =踞+ \emph{訓\\
}把(5. 27)代入尸="'£/*',即得(5. 22). \textbar{}

定理5.7的证明 把(5. 22)改写为系统

=- 7P\textsuperscript{2} \emph{-3hP + Z.} 果=\_M?+4. (5.28)

由于在\emph{h}轴上向量场(5. 28)的方向指向上方,且上面定义的函数 P(A)
-1,当因此/ = P(*)的图形是从系统(5. 28)的. 鞍点(一
\textbar{},lj到结的分界线,见图2-7.向量场(5. 28).

\includegraphics[width=1.74653in,height=1.5in]{media/image30.png}

图2-7

的水平等斜线由方程7产+ 3AF - 5 = 0给出,它的图形为双曲
线,在上半平面的一支为

\emph{3/i} + 79A\textsuperscript{2} + 1401 (5. 29)

显然\emph{,P=P\textless{}fi)}的图形与\emph{P =
q(h)}的图形除了两奇点外不可能相 交;再由(5. \emph{22)}和(5.
29)可以分别算出

lim ?(方)=-丄

\emph{H3 O}

勻=一丄

3丿 4- 因此,当\textasciitilde{}y\textless{}A\textless{}\textbar{}时,F =
F(A)的图形整个落在卩=9(A) 的图形上方,从而沿P = P6)有P (五)VO. lim
?(方)=-8可 U/3

从(5. 22)得到.I

直接方法

为了判断Abel积分的零点个数,目前使用的方法大都要经过
曲折的推导.下面介绍一个可在一定条件下从原方程判别的宜接 方法.

设(5.10)中的Hamilton函數有如下形式

= 6(z) + 世(少, (5. 30)

其中 @ £ \emph{C\^{}a,A3,}啾 e \emph{设}

\begin{quote}
(日)(1) 3 «e \emph{(a,A),}使得W(z)Gr\textasciitilde{}a)\textgreater{}0
(或者\textless{}0), 怎,A)\textbackslash{}{a};
\end{quote}

\emph{(2)3}
使得\emph{H\textsuperscript{!H}(y)(y-\^{}\textgreater{}0(}或者\textless{}0),

\begin{quote}
\emph{Vye} 0,B)財}.
\end{quote}

此条件说明V儿£ (岛也),x°的闭轨马={愆,少旧愆,少=如}
是凸的,其上点的横坐标的最小值\emph{a(h)}与最大值A傘)满足a w
a说)\emph{WaW A(.h)} \textless{} A;其上点的纵坐标的最小值63)与最大值
\emph{8(h)}满足 6 W \emph{Xh) \textless{} 8W} B(A) W
\emph{B;}且(a6)危)与(A(/O ,£)分
别为马i的最左点与最右点,(«,*(*))与伝,86))分别为以的最
低点与最高点.在条件(HQ下,Q(幻,a),存在唯一的7 = m(z) e
\emph{(a,AW),}使臥z)=叡泌;V \emph{y} e。(/0/),存在嗜一的 歹=93 6
(们B3)),使理愆)=晳\&).由条件(同)易知,y * 6
\emph{(a(h)\textsubscript{9}a)\textsubscript{t}y y} 6 (6(方),R)有

dx \textless{}P(x) \textsuperscript{\textless{}0,} \emph{dy
\textasciitilde{}} \^{}(5)V \textsuperscript{0}-

现在考虑两个Abel积分之比

\emph{PfL\textbackslash{}} \_ {4")}

- 脾)\textasciitilde{}顽' 其中

孔(五)=f \emph{f\textsubscript{k}(.X)g(.y)dx,}

「A

这里兀e \emph{k} = i,2, \emph{g} e \emph{河,玲,}并满足

(Hz) (l)/\textsubscript{1}(x)/\textsubscript{1}(x)\textgreater{}o, V r e
(a(Q,a);

(2)(刃 \textgreater{}0, v \textgreater{} e

现在定义两个我别函数

\_ {/'2 0)0 9)一尤(王)a (z)}

一 £ (z)W 愆)一£ (幻a G)'

\begin{quote}
\emph{w 、} {(g。)一g3))世(切 0(力}
\end{quote}

3 \textasciitilde{} g。)'')-g3营⑴'\\
其中克=X(x),x 6 (\textless{}2(方),a),和 \emph{y} =歹(》),丿 £
(6(A),\^{}9).

定理\emph{5. 9} (ELZJ)设H(x,y)具有(5. 30)的形式,并且条件
(HQ与(瓦)成立;当(a,a),y£。洛)时,有 \emph{E)g)\textgreater{}}
0(或v0),则尸以)\textsc{ao} (或vo)对兀e 5* 成立.I

有时所考虑的Hamilton函数具有如下形式

HCr\textless{}y) =S=U)y+ \textless{}5(x), (5.35)

其中的@ e eEa.A{]},且¥ Gc)定号(为确定起见,设s=(X)\textgreater{}
0). 考虑Abel积分之比

Q以)=-p f

\textless{}P /i(x)jdx

", 其中力6 \emph{C'{[}a,A{]},k} = 1,2.类似地,当史愆)满足条件(HQ的
(1)时,V x \& (a以),a)可定义玄=xU) 6 (七\&方)),使@(z)==
⑦G).定义判别函数

,3) = {E)质誕⑴ 一 £(於 石可g)}

* 一£(工)丿反(羚\_九(立)/F旬争G?

(5. 37)

定S 5.10 ({[}LZ{]})设\emph{H(x,y)}具有(5.35)的形式,条件
(H\textbar{})⑴
与(乩)(1)成立,又若当工€怎,a)时L(Z)\textgreater{}。(或V 0),则
\emph{\&W \textless{}} 0(或 \textgreater{}0)对 知)成立.\textbar{}

例\textbf{5. 6}的第二种解法

(5.16)具有(5. 35)的形式,且甲(Q 品■,収z)
=x-yx\textsuperscript{3}. 取a =--- 1,则条件(HQ⑴成立.另一方面,(5.
20)定义的PCA) 具有(5.36)的形式,且为= 1,故条件(H,)(l)
成立.代入(5.37)得

方、\_ {--- H(1 ---另)+ 之(1 一 J?)} \_ {--- (1 + 屹)} b \_ (1一另)一(1
一廿) \_ \textasciitilde{}\textasciitilde{}. 由(5.
31)知若V0,且由图2-6知``V- IV立VI,从而 `3 = 昂亦1 ---另)+(1一釦對
由定理5.10立刻可得尸(方)\textless{}0.

附注5.11上面介绍的方法主要用于Abel积分具有如下的 形式'

7(A) = \emph{alffCh) + blikhy} = 7\textsubscript{0}(ft)(a + ftP(A)),

其中``和\&为常数,A\textsubscript{1}\textless{}A\textless{}A\textsubscript{2},P(A)
而且%(五)\textgreater{}0,当

\emph{h \textgreater{}}奶.如果能证明P (A)尹0,仏 \textless{}
儿〈如则\emph{1(h)}在\emph{0*} 的 零点个数至多为L当Abel积分具有形式

7(A) =aZ\textsubscript{0}(A) +y\textsubscript{L}(ft)
+rZ\textsubscript{2}(ft),

其中\emph{a,b,c为}常数\emph{,h.\textless{}h\textless{}h\textsubscript{z},}而且当\emph{h\textgreater{}h\textsubscript{x}}时厶6)\textgreater{}0,则可
把,(如)改写为

\emph{1(h) = L(h)(a + bP(ft)} +cQ0)), 其中FS)= 鰐,QS)=
端.若能证明PS)与Q6)之一 单调,例如P (;)尹0,炳
\textless{}五则7(A)的零点个数间题转化为 a +妒+
\&CP)对P的零点个数问题,其中Q(P)=QSCP)),而 \emph{h} = A(P)是P =
\emph{P(h)}的反函数.由于\emph{a + bP}是P的线性函数, 所以当曲线Q =
a(P)无变曲点(或有\&个变曲点)时,/以)在(农,
\emph{M}的零点个数至多为2(或\emph{k} + 2).对于研究某些向量场的分岔
问题,这是一个有效的方法,可参见[DLZ].此外,与研究Abel积
分相关,一些作者研究平面上与闭轨族相应的周期函数的单调性
问题,读者可参考曾宪武和井竹君最近的工作[ZJ]及其引文.

\protect\hypertarget{bookmark100}{}{}\emph{§ 6}
关于\textbf{Petrov}定理的证明

当Abel积分有多个零点时,用§5中介绍的方法讨论有较大
困难.因此,我们在本节中介绍另一神方法,却化到复域中用嶺角
原理估算零点的个数.另一方面,为了研究Hamilton向量场的扰
动系统所具有的极限环个数的上界,我们需要把相应的Abel积分
零点个数的估计,与Hopf分岔、同宿分岔、异宿分岔等所能出现的
极限环个数相联系,这是本节将要介绍的另一个问题.

考虑Hamilton向量场的扰动系统

夺察 ++。3),

\begin{quote}
' .(\&1)`` 常=票+心(3)+心),
\end{quote}

其中\emph{n}是小参数,H(z,少,FS,少和QG,少都是*和】的实多 项式,degH = « +
l,degP,degQCn.

设Hamilton向量场(6.1)。有闭轨族

r(ft) --- (3,少)\emph{\textbackslash{}H(.x,y') = h,\^{}}
\textless{} fe\textsubscript{E}),

\emph{其中H = h,\^{}H = h\textsubscript{l}}分别对应系统(6.
D''的中心型奇点和奇闭 轨.设。是奇闭轨\emph{H =
h,}所围成的区域的紧致邻域,则当产充分
小时,系统(6.1\textbackslash{}在\emph{V}中的极限环个数的最小上界与下
面Abe!积分(也叫一阶MeFhikov函数,參宥(5.12)式)

\begin{quote}
= {[} QGr,/)/--- P(工,\$)d»而 V/iV 五z,

J F3〉
\end{quote}

的孤立零点个数(计重次)的最小上界\emph{ZS,册}有如下关系,

(1)
如果伊(矽=0,时(兀)尹0,其中\emph{h\textsubscript{t}\textless{}7i\textless{}h\textsubscript{2},}则系统
(6.1)户有双曲闭\emph{凱ys,}当产f。;反之,如果系统有闭執
乙矶卢-*以兀),当则有。(兀)= 0.

(2) \emph{Z(m,n}}不小于系统(6.1)\^{}的闭轨0,户的个数的总和,此
处\emph{心=叽}当产-*0.

1984 年 A. G. Khovansky〔幻和 A. N. Varchenko\^{} 独立地证 明了 Z〈m,n)
V+8,即对一切工和'的实多项式

y)和 QS,,),其中 degH = m + l,d军W n.Abel 积分

¥= (A)的零点个数有一致的上界. Z(m/)应是m和''的函数.但除
个别情形外,迄今未得到其表达式,对一些特殊的二次或三次
Hamilton向量场在;i = 2,3或4的情况下来估算Z(m,
''),巳有一些有趣的结果,见LDGZ{]}, {[}DRS2{]}, {[}GaHo{]}等.但即
使对于辫='' =2的一般情形,问题也远没有彻底解决.

当 =写 + 3 ---与时,即对 Bogdanov-Takens 系统, G. S. Petrov\^{}m 证明
\textsc{Z(2m)} = n --- 1;当 g)芸 0 时,QBol ,2{]}中 证明BC2.2) =
1,{[}DRS1J 中证明5(2,3) = 2,{[}LR1J 中证明3(2, 4) = 3,
P.Mardesic\^{}证明= »-!>李宝毅、张芷芬``心
证明,当fOO三0而二阶Mel侦kov函数不恒为零时,B(2,n) = 2» 一
2(或\emph{tn}-3),当''为偶(或奇)数.这是在弱Hilbert第16问题
的研究中少有的完整结果.本节主要介绍Petrov和Mardesic的思 氤我们将9
3)扩充到复域,然后对Petrov的方法作一点改迸,用
福角原理来直接估算见{[}LbZ{]}).这种手法可用来研究其 它类似问题.

\textbf{.Abd}积分的构造

当Abel积分卯6)在积分号下的项数较寥时,为了估算\emph{V} (A)
的零点个数,必须先研究它的构造.由Green公式,有

P(A) = {[} Qdx --- \emph{Pdy =} f Qi\&,

\emph{Jrw Jrw}

其中Qy也是工和y的务项式.

令= Qjdr.cu; = Qda:,两个多项式微分1-形式约和《*8称
为有关系``〜",即纯\textasciitilde{}此,如果

, JfW Jr(A)

I此式成立的一个充分条件是

--- tdj --- Ad方 + dg ,

其中A和B都是\textsc{k}和"的多项式.显然``〜''是等价关系.令

\{心〜=/J = \{3〕),

其中3〕表示包含o>的等价类.。对运算

GW。+〔线〕N〔的 + 线〕

而言是一交换静;另外还有

/(A)M = CX(H)办

即

/(A) {[} \^{} = {[} /(77)w,

J p(b) J rw

其中rs)是儿的多项式,即皿可在多项式环浪上进行乘法运算,且
对\textsc{£s),/2(q} e e下列分配律和结合律成立:

(1) £(方)(〔纳〕+〔纱〕)=九3)〔纳〕+ 人以)32、

(2) (儿(五)+£以))〔糾〕=九(/03\textbar{}〕+/20)SD;

(3) £(方)/:(五)〔为〕= £(QC/2(H)3Q.

故。是衣上的模(module).当Hamilton函数\emph{H\{x,y)}=尸+
于危),其中/\textsc{Xh)}是左的多项式时,较易证明口有有限生成元,即口
可表为有限个生成元的线性组合,其系数是頁中的元素,即是五的
多项式•请读者注意,生成元的构造取决于Hamilton函数,而不受
扰动项P愆反)和QGc,y)的影响,后者只决定在线性组合中系数
多项式的次数和形式.

定理 6. 1 ({[}P1{]}) 考虑 Bogdanov-Takens 系统

\begin{quote}
牙=y + \emph{/iP(.x,y)} +。(舟,
\end{quote}

- (6.2)``

\begin{quote}
=--- x + + eoa + o(/*),
\end{quote}

其中户愆点)和Q(z,y)是m和了的多项式,deg巳degQ V 则有 Q=
\{")丄(幻\},且

P(A) =/\textsubscript{o}(A)7\textsubscript{0}(A) (6.3)

0

其中 \emph{L(h) = {[} ydjr\textsubscript{r}i} = 0,1,而 6 R,deg
兀(五)

M {[}与\^{}腿人⑶(也―\textbf{i.}

我们将证明作为习题留给读者.只要对«做归纳法即可证得.

\textbf{Picard-Fuchs} 方程

为了估算?'G)在(0,§)上的零点个数,除了知道甲6)的构 造(6.
3)外,还必须知道厶3)和\&6)的性质,下面我们要证明

定理6. 2考虑系统(6.2)〃此时定理6.1中的LG)和7,6).
满足以下的Picard-Fuchs方程

\begin{quote}
5
\end{quote}

石'

(6.4) 证明 方程(6. 4)中蕴含着I\textsubscript{0}(A)和\emph{IS}
的许多重要性质. 在§5中我们已经对Bogdanov-Takens系统的另一种等价形式
(5.15)推导出了 /\textsubscript{0}(A)和\emph{L0)}所满足的Picardgichs方程
(5. 27).由于Picard-Fuchs方程在研究弱Hilbert笋16何题中的重
要性,此处我们将介绍另一种递推公式匚归幻.由于

= 3 + 芸---守=\emph{h,Q} \textless{} A \textless{} y .

故沿\textsc{H(d)}="有

垫=丄,\\
\emph{dh y \textsuperscript{1}}

并且

(娉+ z---) =0.

\begin{quote}
dr
\end{quote}

利用分部积分得

- {[} \^{}4廿一3戸+2心 + f x"Ty»dz -

说十Z \textsc{Jf(a)}

\begin{quote}
xVdz = 0.

Jr(A)
\end{quote}

特别地,

\textsc{E.t :} --- 2 {[} \emph{xydx} + f \^{}dx --- f 三dz =
0\textsubscript{f}

Jr\textless{}ft) Jr(A\textgreater{} A \textsc{Jf(a)A}

\textsc{\&li: \_ {[}} \emph{yd\^{}+} f 7;dx--- {[} \^{}-dx = 0;

Jr(A\textgreater{} Jr(ft\textgreater{} \emph{y} J®)\emph{y}

\emph{E\textsubscript{2}} 一`` {[} \emph{---dx ---} f \emph{---dx} = 0・

' Jrw y Jr(A)y

另一方面,

\%燃L产『底〉等血

=(强+c 3-{[}孕边\\
Jrcft) 2 Jr(A\textgreater{} \emph{2y} Jr(A) 3y

因此,

\begin{quote}
=j fdx+{[} Jr(A)Z J.

7

---

6

7

6
\end{quote}

由此得到

其中0VhV*.类似地•

\emph{Fg} 五 £ I* •ydz

\textsuperscript{dA} Jr(A)

=f fdx.+ f \emph{fdx} \_ f gdx

J TH) Z Jr(A)2, Jr(A)3\textgreater{}

\begin{quote}
=¥L 必+\#1 V\textsuperscript{dx} (由 \textsc{Elt,E"-i)} b Jr⑶ 6
\textsc{Jt(a)}\emph{y}
\end{quote}

=g {[} gr + 夺 £ I* \emph{xydx,}

6 Jr(A) 6 \emph{ah} J\textsubscript{r(ft)}

由此得到

\emph{心}(A) = -\textasciitilde{}/\textsubscript{0}(A) + \emph{(h),} (\&
6)

其中 0〈五 V*.由(6.5),(6.6 )得

方("一费L«)=(訓---+ 寿Q), (6. 7) 其中 0 \emph{Chv*.}由(6.5),(6.
7)便得(6.4). \textbar{}

几个预备定理

当产充分小时,为了估算方程(6. 2)户在U中的极限环个数,除
了知道\emph{平}⑶在开区间(0,§)上的零点个数外,还要知道当产f 0
时怦6)趋于奇点「(0)和趋于同宿轨「(§)的极限环个数.为此
我们需要引入下面的定理.为了避免行文冗长,我们不引证明,有
兴趣的读者可参看引文.

\begin{quote}
定理 6.3 ({[}LbZ{]},{[}X{]},{[}ALGM{]})考虑方程(6.2)``.如果 机五)=a•頒+1
+。(杖+1), 0, 0〈兀《1,
\end{quote}

则(6. 2)户至多存在\emph{m}个极限环,当产f 0时它们趋于奇点(0,0).

请读者注意,此定理对一般形式的方程(6.1)户也成立,只要 下列条件满足:

\emph{Hcx,\textsubscript{y})}= 7 + 7 + 心 + y)\emph{ecr,}

pg)=。(伝\textbar{} + 仞)\emph{ec\textasciitilde{},}

• QU,v) = og\textbar{} + I刃)\& \emph{c\textasciitilde{}.}

韩茂安和朱德明在专著[HZ]中对非Hamilton系统的线性中 心也证得与定理6.
3完全类似的结论.

定理 6.4 ([R2])考虑方程(6.1)户.设
r\textsubscript{A}i\{(x,\textgreater{})\textbar{}
\emph{H(\textsubscript{x}, 少=h,
h\textsubscript{I}\textless{}h\textless{}h\textsubscript{l}\}}是紧致的,且「(也)对应于(6.1)。的双曲鞍
点薜,刃的同宿轨,则(6. 1)户在同宿執广鶴2〉邻域的一阶Melnikov
函数\emph{V} (A)有以下展式

p (A) = \&0 + 句(方2 --- A)ln(A\textsubscript{s} ---方)+ 缶(検一五)+

\emph{a\textsubscript{z}(.h\textsubscript{2} ---}
A)\textsuperscript{a}ln(A\textsubscript{2} --- A) +
\emph{b\textsubscript{2}}(A\textsubscript{2} --- A)\textsuperscript{a} +
••• (6. 8)

其中0\textless{}ft\textsubscript{2}-A\textless{}l,而且由[HI]和LHm]可知

\emph{b\textsubscript{a} = plh\^{},}

''修+斜 E頒是非零常数'

\begin{quote}
\& =齢3) L = 士 j修+誹当向=0, z a
\end{quote}

其中积分式前符号的选择,取决于闭轨「以)的定向和当\emph{h}增加时
r(A)是扩大或缩小.

如果*+1(或如)是展式(6.8)中第一个不为0的系数,则至 多有r = 2\& + 1(或r
\emph{= W)}个极限环,当产f 0时,E们趋于 「(处).

定理6. 4是研究同宿分岔的主要依据.对于异宿轨而言,韩
茂安和江其保推得了类似的渐近展式,只是展式中的系数与极限
环的个数之间的联系尚待进一步探索.朱德明\^{}\textsuperscript{1}'\textsuperscript{2}'\textsuperscript{33}最近发展了
指数二分性和三分性理论,深化了不变流形的表示理论,用统一的
方法研究了各类系统(高维或低维,可积或不可积)的同宿轨、异
宿轨的存在性和横截性等问题.

定理6.5 考虑方程(6. 2)仃如果屐式(6. 8)中第一个不为

0的系数是印+1或%,则(6. 3)式可改写成如下形式:

P(ft)= (§ 刊(ZTSS +77S)LS)), o\textless{}ft\textless{}-\textbar{}-, 其中
deW (A) \textless{} {[}号加 deg/? (A) C 传{]}---\emph{k -} 1.

这个定理的直观含义是,如果当0时,有+ 1或2么个极
限环趋于同宿轨「(§),则甲(A)在开区间(0,§)上的零点个数将
减少応个'下面我们先证一个引理,然后推证此定理.

作变换\emph{h =} 得

H(w)=夸 十蔘一夸=§ T, 0 v y *

r(o = r(-i-\/-A) (0\textless{}Z\textless{}号)对应系统(6.2)户的闭執族,而
和尸(0)分别对应(6.2)。的中心型奇点和同宿轨.在上述变 换下,展式(6. 8)变为

\emph{甲}(y 一 /)=爲 + fli/lnZ + \emph{b,l +
a\textsubscript{t}P\textbackslash{}rd + b\textsubscript{z}P} + ••- (6.
9) 其中0 VZ《1.

引理6.6 ({[}Ma{]})对方程组(\&2)户,有

L(/)= I- \emph{yd\^{}c} =爲 + 爲Zln2 + o(Z InZ),

\begin{quote}
Jrco

Zi(Z) = {[}- \emph{xydx} = a + 砲InZ + \emph{o(J} InZ), Jr(/)
\end{quote}

其中0 V /《1,并且爲瓦\# 0,衍扁一\emph{a.b.}乏0 .

证明r(o)的方程为

,=土停工一1* + §「' 一\$\textless{}*】.

因此

\includegraphics[width=0.77986in,height=0.42639in]{media/image31.png}

\begin{quote}
甬=亳(。)=2 J音『\textsubscript{T}x(x --- l)Cr + -\textbar{}-)\^{}dx
=--- • 驾丨=亟 =1, \emph{\textsuperscript{3}y!} I(1.0) \emph{3y}
\textless{}i.o)

驾 =奨\textbar{} \textsubscript{=1}

\emph{3yl \textsc{(im} 3y} \textbar{}a,o\textgreater{}
\end{quote}

直接运算便可证明引理.

定理6.5的证明 令

\begin{quote}
\emph{了0} =九(£ \emph{--- D =}勺5(1 +O(Z)),勺尹 0, \emph{7\^{}1)= £(§
--- /)= d,r\^{}i} + o(z)),\emph{心} o, 故
\end{quote}

\emph{布)=}顿 § --- £) = 7oA +

\begin{quote}
=御屿(1 +。(1)) + 祐『(1 +。(1)) =0凿),O\textless{}Z\textless{}1.
\end{quote}

往下分两种情形来证\emph{\textgreater{} k.}

第一种情形I \emph{m】}W斜.

由等式(6.10)立刻得min(m\textsubscript{lt}m\textsubscript{2})日\emph{虹}

第二种情形:码=皿.

假设\emph{m\^{} = m\textsubscript{z}=m\textless{}}知则由等式(6.10),有

祐+ Mi = 0・

由引理6.6,有

L ■盈 + 歸成 + \emph{o(l} ln2),

L = 4 + \emph{b\textsubscript{±}l} InZ + \emph{o(l} Ini).

把它们代入等式(6.10),并利用等式(6.11),便得

祀)=0(3高I - \&而)肿+1成(1 + 0(1))

\begin{quote}
=。0), 0 V2《1.
\end{quote}

从而》»貝知得到矛盾,故nHn(m\textsubscript{lt}m2)

若有minSi ,m\textsubscript{a})
\textgreater{}如可把等式(6.10)中的为换成点+ 1,同 理可证\textgreater{} A
+ 1,从而最终可证minGn】 ,m\textsubscript{s})=应.回 到原变量\&一,定理6.
5得证.I

扩充到复域

定理6. 7 '定理6.1中的爲以)和/\textsubscript{1} (ft)都可单值解析地扩充
到复域

G = C\textbackslash{}(A€R\textbar{}A\textgreater{}-\textbar{}-\},

并且具有以下性质:

\begin{enumerate}
\def\labelenumi{(\arabic{enumi})}
\item
  \emph{I\textsubscript{a}(.h)〜}V, ?!(A) \textasciitilde{} 五、当九一►
  8 .
\item
  L(o)= 0, \emph{U} (0)尹 OMCO) = \&' (0) = 0, L"(0)尹 0;
\end{enumerate}

尹 0,当 * £ G,无尹 0.

\begin{enumerate}
\def\labelenumi{(\arabic{enumi})}
\setcounter{enumi}{2}
\item
  Irn 端尹0,当\emph{heR,h\textgreater{}} ■(沿上下边界).
\end{enumerate}

定理6.8考虑方程

\begin{quote}
r(A) + \emph{+ q(hN(h)} = 0. (6.14)
\end{quote}

\emph{h}
=知是正则奇点的充要条件是待=\emph{h\textsubscript{B}}分别是?以)和\emph{q(h)}的次
数不大于】和2的极点,即(6' 14)可写成

\emph{1心+令吳3十犬\%心=确}(6.15)
其中如6)和\textsubscript{91}(A)在\emph{h = h»}解析' '

对无穷远奇点作变换+ =、则有

\begin{quote}
d . d cP 如"-3 d

而=一『五,诲=『很+ 2*瓦\emph{,}
\end{quote}

而方程(6.14)变为

『器+ (\#\_\%+)愣+ 9忏卩=0 ,

其中Z(0=/(7)-如果彳+卜心日日〜母七则由定理6-8- r =
0是上面方程的正则奇点,即无穷远奇点是舌程(6.14)的正则. 奇点.

定义\emph{6-} \textbf{9} 方程(6.14)称为Fuchs型方程,如果它只有正则
奇点《有穷,无穷).

定理6.10 ({[}Gw{]}) 设九=加是方程(6.14)的正则奇点,则
对充分小正数在0\textless{}\textbar{}A-A\textsubscript{o}l\textless{}/±.(6.14)的基本解组有以
下形式:

8

紿(力)=\emph{(h} 一 加泪(龙 \textasciitilde{}\textasciitilde{} 2、
(6.16)

\emph{1-0}

\emph{心h)=} (A ---人c)P*、如(五---A\textsubscript{o})* + «a»iln(A ---
\emph{h»)}, (6.17)
其中\emph{a,a\textsubscript{k}}和\emph{b\textsubscript{k}}是常数册和ft是下列判定方程的两个根,

\emph{P(.P} --- 1) + + 们(臨)=0,

ReR = 1.令\emph{m = » ---如、}如果0尹。,则方c = L令

z\textsubscript{+}表示非负整数集,如果m e Z+,则知=0,如果m \& Z+,则q
=0.其它\emph{即}和\emph{b\textsubscript{k}}的确定可参看高维新的解析理论讲义「攻】.

由Picard-Fuchs方程(6. 4)可推得

\begin{quote}
L(方)=\emph{島Mg - ,}
\end{quote}

化)=影顷)+亲(5\_財林 及

L' 3)= 一 \emph{6W(K) + 如'0),} L'0) =- 6g(7i) + 6W(Q . 最后推得

")+ \textbar{}占-制杯)=。,\\
眦)+ (狷当-詞时) = ».

由定理6. 8,/i = 0和九=£都是(6. 20)和(6. 21)的正则奇点.作 0

变换五=+,得

『零\^{} + 2?当F +4备+\ldots{}卩0 = 0, (6.22)

『玲集+ 驾普+ 4票十``,如)=0, (6. 23) 其中
椅)=厶囹,如)='』})•由定理6-8,( = 0是方程 (6. 22)和(6.
23)的正Sil奇点,故无穷远奇点是方程(6. 20)和方程 (6.
21)鮒1E则奇点,由定义6.9,(6. 20)和(6. 21)都是Fuchs型方 程.

下面我们来证明定理6.7.为此先利用定理6.10来讨论 \emph{W} 和\emph{I,W}
在奇点\emph{h} = 0及无穷远处的渐近展式.

方程(6. 20)---(6. 23)在0 V '' 一么\textbar{} V r上的通解可写成

/(A) = Cjiu; + (6. 24)

其中G和\emph{C*}是任意常数,5和线由(6.16)和(6.17)给出.

对方程(6.22),奇点(=0的判定方程是

心一1) + 20 + 余=衫 + 0 + 爲=0,\\
解得。1=一§''2=一亨.因師=0J佝=号■冬Z+,故展式\\
(6.17)中的\textsc{q} = 0. 〜,或厂,,当Z f 0\$即\&以)〜侑或

A*,当札一> 8.

同理,对方程(6.23),奇点\emph{t} = 0的判定方程是

p(p --- 1) + 3p + II = J + 即 + 兼=0,

郷得 \emph{P\textbackslash{} =} »\textbar{}°2 .因 m = /Jj --- A = y \&
\emph{Z"故展式}

(6-17)中的c = 0.匕(£)〜厂甘或厂彳,当即/i(A) \textasciitilde{}侑 或

力当儿f + 8.

'由方程(6.18)便知,应取套以)〜糖和匕5)〜吊,当Z\textgreater{} f +
8.定理6. 7中的性质(1)得证.

为求在龙=0附近的展式,先对方程(6. 20)求奇点儿= 0的判定方程PS- 1) =
0的根,得4 = 1和0 = 0.因m = R --- ft = 1 G Z+,由定理6.10,约(0) =
0,此(0) = 6。= 1.如果通解 (6. 24)中的 C\textsubscript{2}\^{}0,便有髦(0)
= (7[纳(0)+。2從(0)=。她尹0, 但此处L(0) = f \emph{ydx} = 0,故必有G =
0.即得L6)= \emph{C"h) = C、h} +\ldots{}〜(0)尹0丄0)在\emph{h =
0}解析.由于 (6. 20)除\emph{h = 0,k =
\^{}}外无其它有穷奇点,故厶(方)可单值解析 地扩充到G.

为求亳(幻在兀=0附近的展式,先对方程(6.21)求奇点\emph{h = 。的判定方程p(p
--- 1) 一 p =寸一舞=}0的根,\emph{得\&} = 2,松= 0.因 m R = 2 £
Z十,由定理 6. 10,\^{}(0\textgreater{} = 0, \^{}2(0\textgreater{}
---\emph{指}

=1.如果通解(6.24)中的G关0,便有L(0) = G糾(0) +
G的(0)\emph{=以逐0,}但此处】1(0) = i = 0,故必有C2 = 0.即得 A(A)---
G纳(力)=\emph{c*} + \ldots{}〜岸 Ji(o)---厶/(0)= 0, 査''(0) 乂
0,LS)在九=0解析.由于(6. 21)除A = 0,ft = -i-外无
其它有穷奇点,故L3)可单值解析地扩充到G.

下面证明性质(2)中关于A6G,A\^{}0时%(方)查6)尹0.

令*= \{A6R \textbar{}方\textgreater{}§\}分别表示区域G的从上下两侧逼
近的水平边界.我们断言下列不等式成立*

ImZo(A)弄 0, Im/j(A) 0, A G 乙*・ (6. 25)

由于当\emph{h} e乙士时,方程(6.4)是一个实解析系统,故CWoJmA)
是方程(6.4)当\emph{h5±}时的一个解.令?(方)=齡务,与§5\\
中讨论的情况类似,PS)满足下列Riccati方程

\begin{quote}
小---訓+仔1专户一払(6.26) 考虑(6. 26)的等价系统
\end{quote}

\includegraphics[width=3.06667in,height=2.82014in]{media/image32.png}

\includegraphics[width=1.08681in,height=0.30694in]{media/image33.png}

系统(\&27)的相图如图2-8所示.其中点\& (A,P) = (§,1}是 蔬点,由方程

\begin{quote}
6方 + 3 + 3 J 4冷---\emph{\%h} + 1
\end{quote}

\emph{P =} g(A) \emph{\% 。 }

麻决定的双曲线是过点S的水平等倾线.利用L6)和 \emph{W} 在A

=*附近的渐近展式,不难算出当\emph{he}乙士时

\begin{quote}
r {Im/】(五)}\_
\end{quote}

\textsubscript{A}\^{}\textsubscript{6}Im7\textsubscript{0}(A)
\emph{一瓦一'}

故P(A)是鞍点S在一侧的不稳定流形.由于P@)介于双 曲线9統)与直线P =
1之间,故有

1 VFS) \textless{}8, A e L*,

因此断言(6.25)得证.

由于金3)是复共辄的,即 \emph{W =W,i-l,2,}故若它们
在G中还有其它零点,必成对出现.这样再加上断宣(6. 25)和性
质(1),不难用辐角原理来证明 当\& W \emph{G,腭。.}

我们将此作为习题.

往下证性质(3) \textsubscript{:}Im什釆W 0,当方£ \emph{R,h \textgreater{}}

用反证法•设

\begin{quote}
\textsubscript{Tm} AW-)

\textsuperscript{Im}w)
\end{quote}

卽

Re(L0;)) 】m0(V)) -Re0(/T))】m(L。*)) = 0.\\
即向\#(Ini7\textsubscript{0}(A) Jm/i(A))和向\textgreater{}
(Re/\textsubscript{0}(A),Re/,(A))在 \emph{h --- h*}

成比例.由于这两个向量函数当\emph{h} e R,A \textgreater{}
I时都是实线性方程 (6.18)的解,如果它们在一点\emph{h =
h-}时成比例,则它们将在一切孔
GR.A\textgreater{}\textbar{}上成比例,即端在五\textgreater{}§上取实值.令\emph{1
= 宀} 代 入等式(6.12),经计算得

lm( ) \textasciitilde{} c(5o6\textsubscript{:}---疝I)紀,0 \textless{};
r *\$; 1,

其中c为非零常数,即虚部不为零,回到原变量\& 此结论

仍成立,即Im[端)走0,当0 得矛盾.故反证法
的前题不成立,性质(3)得证.定理6. 7证毕.\textbar{}

\textbf{Petrov}定理的证明

定理 6.11 ({[}Pl{]},{[}P2L{[}Ma{]},{[}LbZ{]})考虑系统(6.2)\^{}.
如果degP(w), degQ(")Q,则对同宿轨「(普)所围成区
域的任意紧致邻域U,存在£\textgreater{}0,当\textbar{}产\textbar{}\textless{}£时,\textless{}7中极限环个数
的最小上界8(2,n) = n --- 1.

证明 我们将U中的极限环也/分成三类:

\begin{enumerate}
\def\labelenumi{(\arabic{enumi})}
\item
  ---。(0,0),当 \#-*0;
\item
  瓦,户*「6),。〈五 \textless{}§,当
\end{enumerate}

⑶\emph{乙%} 当 ``f Q.

由定理6.1,

p(A) \emph{= KhWh) + G\textless{}h\textless{}\textasciitilde{},}

0 «

其中\emph{代W)}和\emph{fs} 是\emph{h}的多项式,且deg
/\textsubscript{0}(A) M {[}专旦L deg/;6)M {[}芸{]}一 1.

\begin{quote}
由定理6.7,L@)和L6)可单值解析地扩充到复域G,故 也可单值解析地扩充到G
\end{quote}

由定理6.5,有

\begin{quote}
P(A) = (\^{}■一 \emph{h)\textbackslash{}f; (h\}I\textsubscript{0}(h) +
f\^{}hyi\^{}h)),} AEG, D
\end{quote}

其中 \&g J7(h) \textless{} {[}号\^{}一 \emph{h,} deg \emph{f;}
W\textless{} {[}3{]} --- 1 - \emph{k.}此处 不妨假设当\emph{heG}时,yj⑶
和\emph{拭}統)无公共零点.

令

由定理6. 7的性质(2) ,F(五)在\emph{G}上解析.取区域\emph{D} U G,其边界
那满足下列条件:

\emph{3D --- C卞} U \emph{L\textasciitilde{}p} IJ \emph{C;} U
\emph{L如'}

其中

Cg= \{Aec\textbar{} \emph{\textbackslash{}h\textbackslash{} =}

\emph{c\textsubscript{r}= \{he} \textbf{Cl m - §\textbar{}=r, OVYl\}, .}

\emph{L\^{}= \{heR\textbackslash{} § +} 血 WF\}.

而今和C;分别表示沿C\#逆时针和沿%顺时针方向盘旋,乙糸和
\emph{临}分别表示沿\emph{L\textsubscript{ap}}关于\emph{h}从小到大和从大到示移动,

令P(F6))aD表示当\emph{h}绕3D逆时针方向旋转一周时,由复数
F3)所定义的向量盘旋圈数的代数和.我们先分别估算当九沿\emph{3D}
的每一部分移动时,相应的KF(\&))的取值,则

p(F(A))\textsubscript{c}- W ---点 + *, (6. 28)

由定理6. 7中性质(1),

以⑴丄;Mmax\{[十],[罚一\{(+4-

由定理6. 7中性质(3),当h沿匕亲U互> 移动时,侦F(A)至多有 deg/?个零点,故

队卩⑴九扣垢deg /; + 1 =[芸]一 \textsuperscript{k}-

综合以上各式,得

移(F(五也彫W '' 一 2\&---.

由于盘旋圈数为整数,所以有

移(F(幻)彫V '' 一 2爲一1.

由辐角原理,F(Q在D中至多有\&一2左一1个零点,从而F0)在 0 M血V
*上至多有\emph{2k} --- 1个零点.由定理6. 7的性质(2),
pCA)比F0)在0£\&\textless{}§上的零点个数多1,故會(\&)在0M\& \textless{}
§上至多有''一2\&个零点.由定理6. 3,甲(\&)在0 V h \textless{}号上
至多有\emph{n-2k-m-l\^{}} 点.由定理5.1,相应系统至多有n --- \emph{2k-m-}
1 个(2)类环.

由定理6. 3,系统至多有m个(D类环.

.由定理6. 4,如在展式(6. 8)中第一个不为。的系数是丄则系
统至多有狄个(3)类环.综上所述,得

如展式(6.8)中第一个不为0的系数是%+i,则系统至多有 \emph{2k} +
1个(3)类环.此时由\emph{祁}的展式(6.13),可知不等式(6. 28) 变成

故一 1仍成立.

往下来证等号成立.

令3表示次数不大于''的多项式向量场空间,即丫 =户£ +
Q务£3,其中戶(3)和Q(W)蕭是次数不大于''的m和/的
实多项式.令\emph{\^{}\textgreater{}n=} (Z\textsubscript{r}(Z)\},其中

/\textgreater{}•(/)= ( Qdx-Pd\^{}, y e p\textsubscript{n},
o\textless{}z\textless{}4-\/-

Jr«) 6

,首先用[LbZ]中同样的论证,在用归纳法推证等式(6. 3)时,
可一步步地推出,对一切可能的\&次多项式P(x,y)和QU.j),满 足等式(\&
3)中的一切多项式兀6)和/,(A)都是可以实现的.

(1)设 n = 24 + 1.取 7(/) = 2产L(Z) € St,如尹 0.于是
I(Z)〜\emph{b\textsubscript{k}l\textsuperscript{k}.}令

r(Z) = +R/-】La)+7(/) e 啊.(6.29)

取\emph{th \textasciitilde{} 21,}使得

\begin{quote}
5-叽(Q + 向井=--- 70)\emph{,}

(6.30) 旳1切)+卽尸如2)=-以)■
\end{quote}

由引理\& 6,其系数行列式

''财) ''择)

心-\textless{})= \textbackslash{} ,

\begin{quote}
(2/)*-呪(2,)('')*-飞(2。

=2*一1(爲么-\&損)尸-1应 + o(/\textsuperscript{2i\_1}lnZ)
\end{quote}

壬 0,当 0 \textless{}Z\textless{}T\^{}1. (6.31)

故方程(6. 30)有唯一解(%电),使得=「0)= 0 .可证Z,
和4是的简单零点,我们将此作为习题留给读者・

由(6. 30)可解得

=?-\textsuperscript{1}7\textsubscript{l}U)7(2Z) --- (2/)*-吼(22)7(2)

\textsc{=2*t} 反血淬-1 +。(沔一1).

将它和(6.31)右侧相比较,由引理6. 6便知当时,%-().同 理可证,当Zf
0时,\&f 0.这样便在(0,§)中有了两个简单零点
4和上且它们可任意小,只要产充分小.

取定了 4和上后,(务国)也就唯一地确定了 ,对于这样的W, \&),有

「0)〜虹\_/T,虹\_1 尹 0,0 V 2《1 .

请读者验证,此处虹\_1 =- 26仏In 2/lnA.重复以上歩骤,如此继
续做下去,在,步以后,可得产Q)e多\&,它在(0,i)中至少有2A
个互异的简单零点,回到原变量\&=\textbar{} - 5咐)=y 一 A)
在(0,\textbar{})中便至少有2A - n - 1个简单零点.

(2)设门=2Q+ 1).由引理6. 6,可取為(Z) + \emph{B『La)}G 夕'',使得 4-
\emph{= 0, a\textsubscript{t} = a\textsubscript{t}b\textsubscript{a} +
瓯} W 0,于 是殉)〜漆*+1血,0\textless{}ZCl.取在象"中的充分小扰动
尸(/)=虹/打。(/)十/\textless{}7)

=外产爲 + io + \^{}j/lnZ +o(ZlnZ).

\begin{quote}
显然,存在\emph{k \textgreater{}} 0,有= 0,并且当虹f 0时山f 0.这样就
将情形(2)化归情形(1),即 \emph{g 〜御.}由(1)中的讨论,可以•
构造\emph{\&,}它在(0,\$)中至少有跃个简单零点,即JQ)在(0,§) 中至少有2命+
1 =從一 1个简单零点.回到原变量兀=- 一 %,则
\end{quote}

Q

函数5P (A)=产空一A)在(0,§)中至少有« 一 1个简单零点.

由定理5. 的每个简单零点对应系统(6. 2)\#的一条闭

轨,故有\emph{B(2,n)\^{}n-} 1,即等式8(2,'')=從一 1成立.

\protect\hypertarget{bookmark118}{}{}习题与思耆题二

\begin{quote}
2.1向量场奇点的双曲性与非退化性有什么联系,有什么区别?光滑向
暈场的双曲奇点或非退化奇点在向量场的C7r\textgreater{}l)扰动下有何变化规律?

2.2证明当g\textbar{}《1时,例5. 5中的van der Pol方程(5.14)的唯一闭轨
是双曲的.
\end{quote}

2.3若(2. 2)式中的向量场X。是\emph{n}次多项式系统,试问定理2.7的结论
i(2)能否在"次多项式系统X*中实现?

\begin{quote}
2. 4考怎方程组
\end{quote}

\emph{x = y,} \^{} =■--- x + oar\textsuperscript{a}y + ``(1 --- f,

\begin{quote}
其中aN0,0\textless{}
I川《1.请验证定理3.1中的条件(H,),(H\textsubscript{2})fn(H\textsubscript{3}),并以
a = 0时的情形来说明条件(HD不能缺少.否则(0,0)可既不是方程组的中
心,其唯一闭轨也不收缩到(0,0),当点\textasciitilde{}0.
\end{quote}

2. 5确定下列系统中原点作为细焦点的阶数:

\begin{enumerate}
\def\labelenumi{(\arabic{enumi})}
\item
  £ =一,一 3® + \emph{2xy} --- y, j •= z + z,一 2ry.
\item
  扌=一 \textgreater{} + h' + xj,, \emph{y = x+}
  -j-x\textsuperscript{2} + 3zy --- y.
\end{enumerate}

2. 6试证明:当妃--- %缶\textgreater{}0时,蔡飽

\begin{quote}
\emph{I} a,x --- \emph{a\textsubscript{t}y + y\textsuperscript{!},}

I \emph{y- `` + b\textsubscript{2}y} \_ 玲 +
\emph{cy\textsuperscript{1}}
\end{quote}

有两个一阶细焦点的充分必要条件是

\emph{by} = €0(, at = \emph{ca\textsubscript{lr}} --- 2 \textless{} c
\textless{} 0.

2. 7证明定理3. 15.

2.8若系统(6.1),存在双曲闭轨々俨■*「('),当其中/\textgreater{},\textless{}A\textless{}
丄,则有P(X) = 0,试问是否必有尹0?

2. 9试用定理6.3来求方程组

Z = \emph{y,y} =--- Z --- \emph{+ P1X + i\^{}xy + Pi\^{}y,}

当?(A)\^{}0时,细焦点(0,。)的阶数,其中1,2,3)是小参数.

2.10请读者讨论§ 5中后继度数G\%,``)和一阶Melhikov函\emph{财(h)}对
人和对/T的光滑性问题.清读者参考引文{[}HZ{]}.

2.11试证(6. 30)式中的4和4都是公式(6. 29)中/'(/)的简单零点.

2.12试证定理6. \emph{7}中的/„«)和匕除了零点A = 0外,在G上无其它零 点.

2.13对系统(5・15)・试用§6中的姓推公式,导出L(X)和丄侬)所满足 的
Picard-Fuchs 方程.

2.14\emph{若}Hamilton向量场为一蝮 +工务,丫£ R•,试求其Abel积分
零点个敬的最小上界,即求Z(2.«).

2.15若Hamilton向量场为丿备+ (-工一釦务,Y 6 丄试证''=
\{L』%\},其中\emph{I.W} == L⑴物,心,\emph{i =} 0,1,2.并用§
6中的递推公式求 出 4(»)« = 0,1,2)所满足的 Picard-Fuchs方程.

2.16考虑二维Hamilton向蠹场的扰动系统

\emph{x = 2xy} ---如 y\textsuperscript{1}\textsubscript{9}

j = --- 1 + 2工一--- )2 + 8(爲4- \%)- J\textsuperscript{3}),

其中\&为小参数.假设由Abel积分(5.12)定义的一阶Melnikov函数不恒为
零,且其同宿分岔的最高阶数为2,利用定理6. 4求其一阶和二阶同宿分畲
曲线在(%,\%)平面上的方程(即对\emph{3}的一阶近似方程).

\protect\hypertarget{bookmark123}{}{}第三章几类余维2的平面向量场分岔

在本章中,我们将综合运用第二章所介绍的几种典型的向量
场分岔的理论与方法,讨论向量场在非双曲奇点附近所发生的几 类余维2分岔.

考虑以产£ R"(«\textgreater{}2)为参数的向景场族

(X*)' ¥ =

其中工£ R\textsuperscript{n}./6C™(R\textsuperscript{n}X
R-,R\textsuperscript{n}).无妨设工=0是向量场X。的
非双曲奇点,并设X。在该点的线性部分矩阵具有二重退化性.因 此,/〈0,0) =
0,且把X。化到工=0附近的中心流形后,其线性部 分矩阵可化为下列形式之一:

\begin{longtable}[]{@{}llllllll@{}}
\toprule
\endhead
° S & \textbf{.} & & \textbf{\& =} & 0 0{]} & \emph{r} &
&\tabularnewline
0 0/ & & & & 0 o\textbar{} & & &\tabularnewline
& & & & 0 & 糾 & 0 & 0\tabularnewline
0 & 1 & 0 & & -糾 & 0 & 0 & 0\tabularnewline
---{[} & 0 & 0 & ,\& = & & & &\tabularnewline
& & & & 0 & 0 & 0 &\tabularnewline
0 & 0 & 0 & & & & &\tabularnewline
& & & & .0 & 0 & \_糾 & 0\tabularnewline
\bottomrule
\end{longtable}

\textbf{A,=}

其中 sg H 0,的=她,=】,23 = 1,\ldots{},5.

若平面向量场在奇点处的线性部分矩阵有二重零特征根,并
且向量场在旋转響角度时不变,则称它有\emph{q}阶对称性,或称为"g
共臉厂=1,2时,线性部分矩阵具有\&的形式項\emph{法}3时,线性部
分矩阵具有去的形式.称为强共振.«\textgreater{}5称为弱共振.除
\emph{\textsc{Tl}} 4之外,其它情况的余维2分岔问臨都已解决.本章 §1一
§3分别介绍冬=1,2,及冬法5的情形.在高阶项的适当非退
化条件下,扰动系统X``具有双参数的普适开折.主要参考文献

为:q= 1的余维2分岔见[Bol,2]和[T],余维3、4的讨论分别见
[DRS1,2]和ELR1]河=2和g = 3的余维2分岔见[Ho], q = 2的
余维3、4的讨论分别见[LR2]和[Re]; \textsubscript{9} \textgreater{}
5的余维2分岔见[T丄 对\emph{A\textsubscript{3}}情形的讨论见[Zol,
2].在专著[CLW]和[HZ]中有对所 有情形的详细介绍.

\protect\hypertarget{bookmark126}{}{}\textbf{§ 1}二重琴特征根3
\textbf{Bogdanov-Takens}系统

在第一章§5中,我们讨论过二维b向量场

嚣H:摧I + 8W), Q.D 它具有正规形(见第一章例4. 7)

当湖尹0时,由第一章定理5.13,系统(L 2)的任一非退化开折可 转化为

=夕,

=出 + ``2丁 十十 \emph{xyQ\textless{}.x,ti)} + 寸顽,``),

(1.3) 其中Q仲£6,。(0,0)=±1 =幫11(沥),六£呻,後22.为确定 起见,取0(0,0) =
1. Q(0,0) =- 1的情况可类似讨论.

分岔图,轨线的拓扑分类

下面的定理是本节的第一个主要结果.

定理1.1 存在R,中(凶心)=(0,0)的邻域△,使系统 Q.
3)在△中的分岔图由原点(岡/2)= (0,0)以及下列曲线组 .成: -

(a) SN士= \{"\textbar{}内一0必\textgreater{} 0 或 \textless{} 0 \})

(b) H = 3\% =-妃 +。(出¥)必\textgreater{}0\};

(c) HL = \{川产1 = 一 舞妃 +0(版3),冊 \textgreater{}oj,

其中SN士,H,HL分别为鞍结点分岔曲线,Hopf分岔曲线和同宿 分岔曲线.当6
△时,系统(1. 3)在相空间原点愆,少=

\includegraphics[width=3.87361in,height=2.81319in]{media/image34.png}

图3-1

(0,0)附近的轨道拓扑结构见图3-1.

为了证明定理1.1,注意当\^{},\textgreater{}0时,(1. 3)在原点附近无奇
点;而当M = 0,出尹0时,由第二章例1.6知发生鞍结点分岔.因
此,下面要考虑的只是内\textless{} 0的情形.考慮参数及变量的替换:

\emph{樵=---§、出\textsc{h =} y =}罗 =号,(1. 4)
其中\emph{S\textgreater{}o.}再把(5,5』)写回则(1.3)化为

石=弘

■

, * =---1 + 工'+ +,)+ 州(\^{}'',凯 QB.

(1.5 為 我们可以把(L
5為看成(1.5)\textsubscript{0}的扰动系统,后者为Hamilton系
统,它有鞍点A(l,0)的同宿轨,以及该同宿轨所围的以点B(-l,
0)为中心的周期环域(见第二章例5. 6及图2-6),周期环域中的闭 轨族可表示为

* = A, ---(1.6)

其中

孤3:,少=专+工---号. Q.7)

当方〜一等+。时心缩向奇点岛当\ldots{}号一6时可趋于同
宿轨与鞍点A形成的同宿环「勇

注意对任意的8,(1. 5為都以\emph{A,B}为奇点,且\emph{A}为鞍点为 指标+
1的奇点.因此,若(1. 5爲存在闭轨,它必定与线段\emph{L =} {SoO W =。,一
1 \textless{} * \textless{} 1}相交.另一方面,由于与X•的交
点处在乙上关于\emph{h}单调排列.国此,可用h把Z,参数化:

\emph{L'} {用也=5死,一号\textless{}兀\textless{}号}.

现在我们任取\emph{P\textsubscript{h}} 6 \%,考虑系统(1.
5"过\emph{p\textsubscript{h}}的轨綫.设它的
正向及负向延续分别与工轴(第一次)交于点Q与Q.记,(如8,了) 为(L
5為从\emph{Q、}到Q的轨綫段(见图3-2).

引理L2当3\textgreater{}0时是系统(1.5爲的闭轨,当且 仗当

\emph{F(足8Q m} f [(? + ⑦)+ 冲= 0.

(1.8)

\includegraphics[width=3.10694in,height=1.04028in]{media/image35.png}

图3・2

证明 注意当\textbar{}以\textless{}1时\sout{严£")}=1 ---充尹0、因此,\\
\emph{"h,a,G} 为闭轨 G Q\textbar{} = Q Q H(Q\textbar{}) = \emph{H0.}

另一方面,由方程(1.5''可得

\textsc{{AH(ht)}}

\emph{At}

因此

(1. 9) 由此引理得证.I

我们可以用取极限的方法把F(如8,了)的定义域扩大到集合

\emph{U =}件,标)

引理 1. 3 ({[}Bo2{]})
Vf\textsubscript{1},\^{}(f\textsubscript{1}\textless{}r\textsubscript{2}),3\^{}\textgreater{}0,使得函数
F(方,凯。)在集合U上连续.此外,F关于3与。在U上是C\textasciitilde{}的,关
于儿在集合

丫=件,标)

上是L的.

证明利用微分方程的解对参数与初值的连续性、可微性定
理,F在U上为C。,在卩上为利用[ALGM]关于在中心或焦
点附近解对参数的光滑依赖性定理,F在儿=一 ■!■关于草为C-;
利用【Sh]关于鞍点分界綫对参数的光滑依赖性定理,F在 关于凯了为b的.I

从上面的两个引理可得如下推论.

引理1. 4系统(1. 5"(8> 0)存在鞍点\emph{A}的同宿轨,当且仅 当 F 停,8,q =
0. I

下面,我们把F6,凯,)看作F6,0,G的扰动,注意

\emph{\textsuperscript{Jr}h} (1.10)

=矽)(---旳)),

其中 \emph{g,P(h)}如第二章(5.19)和(5. 20)所定义.

引理1.5 3 *>0和在円定义的函数。=岩(\$),使
当。=\%(\$)时,系统(1.5)\$存在鞍点\emph{A}的同宿轨.

证明 由引理L4可得,系统(1.5爲存在同宿轨的充要条件
为F(号,明=0.再利用第二章定理5.7可知』号,0,弔=
L囹传-?囹)="并且款方号=4囹>0.由隐 函数定理洱叫> 0,和在0 W8 W
\%上定义的函数了 = *(\$),使 fi(0) = \textbar{}并且F停,杯(叫三0・I

引31.6存在曷>0 (\& W务),折>0,并存在(1.5)。的同 宿环 E
的邻域山,使得当看(S) V 了 <X⑶+ \%, 0 V 8 M备时, (1.
5為在r\textsubscript{2/3}的\emph{U,}邻域内含有一个闭轨,且它是不稳定的极限
环;而当灯(力一丸咼时,系统(1.5)\textsubscript{a}在「必 的邻域内无闭轨.

证明 由引理1.5,当* =看3)时,(1.5姦有同宿轨,(\$).我
们需要证明同宿分岔对参数3的一致性(见第二章§4),即证明 存在\&
\textgreater{} 0和% \textgreater{}
0,使得对任意\#和\textgreater{}?,只要0 V 8 V也,\textbar{}4 一
宗8)1\textless{}\%,就有

\begin{enumerate}
\def\labelenumi{(\arabic{enumi})}
\item
  与灯相应的同宿轨汽8)的稳定性与d的具钵取值无关,
  且它在「涵的某个\emph{U,}邻域内至多分岔出一个闭轨;
\item
  y(3)在扰动下其稳定流形与不稳定流形具有固定的相对
  位置(与\$的具体取值无关).
\end{enumerate}

我们把系统(1.5)\textsubscript{3}改写为第二章(4.14)的形式
牙=\emph{f\textless{}z,8) +} - g))g(\textgreater{},以),

其中

1+\^{} + 註 + 牌)"

"m'e=L(i+o(i5i + 1了一洒)1丄

由于对任意的凯系统的鞍点均在愆,少=(1,0),而且系统在该点 的发散量为

7(3) = \$(1 + \^{}(3)+0( 0\textbar{} + 化一匕(``丨).
注意匕(0)=辛,故存在\&\textgreater{}0和宛\textgreater{}0,使只要。V3\textless{}\&,\textbar{}'
--- fiWI \textless{}弛,就有T(5) \textgreater{} 0.结合第二章定理4.
3和定理4. 4可 知,结论(1)成立.

重复第二章定理4.F乾证明,并注意以了 一 \&⑦)代替那里的 ,产,则与第二章(4.
19)责癸似的紅 \emph{为}

I 妃,8)= T- 0({[}5\textbar{} + I,T\&W),

其中

z=广火1+。(\textbar{}8\textbar{} + ir-g)\textbar{})eT 腿+印》\%.

\emph{J ---8}

由于沿「旳与,(冷都有帅=\&.因此,存在為e(0,舄)和执e
(0,如),使只要0\textless{},\textless{}\&,1了一,3)丨〈払,就有 25=
{[} 3-(1+0(\textbar{}5\textbar{} +
\textbar{}了一畠(3)\textbar{}))迫法! I" ydz\textgreater{}0.

J\% \emph{L *3}

所以结论〈2)也成立.引理1.6得证.I

引理L7存在\^{}\textgreater{}0,\%\textgreater{} 0和至定义的函数了
=匕(为,\&(。)= 1,以及奇点方(一 1.0)的邻域S,使得当%(8)
-兔畐时,(1.5為在B点的S邻域内恰有
一个闭轨,它是不稳定的极限环;而当匕(8)WU\textless{}\&(8)+\%,
0\textless{} 8 W \&时,(1. 5)\$在S邻域内无闭轨.

证明(1.5)\textsubscript{a}在奇点8( --- 1,0)的线性部分矩阵为

\begin{quote}
\emph{I \textsuperscript{0}} L \textbar{}

丨一2 3(,一 1) + (- 1,0,\#£)丿'
\end{quote}

它有特征根a(S,G 土谄(凯§),其中 \^{}

心,,)=1) + 冲(一 1,0,

1 1

阳,c =权8七。(序)界

因此,存在禺\textgreater{}0和在0W3W\&定义的函数:=孔⑹,使当了 =
匕3)时点(3,最3))=。伊3,匕。))尹0.即第二章§ 3中的条件
(Hj)成立.计算表明,

\begin{quote}
1
\end{quote}

歹'

\begin{quote}
\emph{ci} M lim4Re{[}ci(\textless{}J,。){]}
\end{quote}

f---V O

因此,第二章§3的条件(H:),(HJ)也成立,而引理1.7可由第二
章推论3.16得到.I

引理L 8对任意给定的旨\&,号V \& V导\textless{} 1,存在晶〉

0,使当看晶时,系统(1.5為恰有一个闭轨,它是

不稳定的极限环.

证明 vreid令''=pT(r),其中函数卩= \emph{P\{h)}由第二章(5.
20)所定义.由第二章定理5.7, T\^{}CAXO.因 此,当?!<r〈匕时,一号<妃 gy
財 <音,其中蚌= 厂(成),妨(盘).由(1,10)及P(h,)= \#可知

F6*,0,O) =7。6*)(尸-PCA")) = 0,

\emph{aF}

剥矽,0,广)=Z„\textsuperscript{,}(A*)(r -P(AO) -4(*)8(v)>0.
这里要用到玲(\&)的有界性.事实上,由第二章(5. 23)和本节
(1.5*的第一个方程可知

\begin{quote}
L'(妒)={[} 丄 dH=f dt-T(A-),
\end{quote}

其中T(ft-)是r\textsubscript{(}-的周期.当-1 < Ar C A' < A; < I-时,
\emph{W)}显然是有界的.

\begin{quote}
因此,由隐函数定理可知,存在广>0,7' >0,0* >0,及在
\textbar{}<7*定义的唯一函数\emph{h =} 使得
\end{quote}

F(狱 = 当 0 V3W\^{},K 一广丨 w矿

以及

\begin{quote}
aF

瓷 SM)>o.当。\emph{vaw\&,}

ir 一广I g 矿,M ---妒 \textbar{}M<r.
\end{quote}

再利用区间传,X{]}的紧致性,可找到舄>0 ,及在

\begin{quote}
壹定义的唯一函数hfg,满足\emph{F(丄眾)眾)} OL'
\end{quote}

云0以及鈴SM〉>0,当0V3W,,如 W久 这里岛=\emph{p-wwy).}

由引理L 2知,上述结论说明,丫(禺,G,0<畐〈爲,旨 <匕,系统(1.
5)a有唯一闭轨,它相应于五。=即F(岛, y) =
o.为了判断此闭轨的稳定性,我们考虑Ao附近的\emph{h,}则

由(1.9)式可知・

H(Q) - H(Q) = \emph{\&F(h,y)="}务-五。),\\
其中\emph{h}介于如与\emph{h}之间.由于\emph{霊僞,\&,微>}
0,推知相应于知的\\
闭轨是不稳定的权限环.I -

由引理1. 6 - 1. 8可得下面的定理.

定理L9存在g,心平面•上原点的邻域冬,使当\emph{3,饱)}
€4且介于曲线H与曲线HL之间时,系统(1. 3)有唯一闭轨,它
是不稳定的极限环.当(尚,色)趋于H时,此闭轨缩向奇点鬲当
S必〉趋于HL时,此闭執变为鞍点\emph{A}的同宿环.相应的轨线结 枸见图3-1.

证明 由引理L 6和引理1.7,可找到满足要求的曷办和頻 阳现在取€ {[}I,y +
71{]} (1-\%,1),对\&,食{]}应用

理1. 8,可找到相应的晶.取正数爲< ,為,禺)且足够小,则

V3e(o,\^{}),?e(§(心,靠3)),系统Q.5為恰有一个闭轨,它为
不稳定的极限环,参见图3-3.

\includegraphics[width=2.80694in,height=1.68681in]{media/image36.png}

为了从系统(1. 5爲返回到系统(1. 3),由变换(1. 4)知

納=---尸,兴2 =

因此,区域0 豈(必变到S 必)平面的尖形

\includegraphics[width=1.41319in,height=1.17986in]{media/image37.png}\includegraphics[width=1.57986in,height=1.20694in]{media/image38.png}

图3・4

区域0 \textgreater{}内 \emph{A M}且(为必)在曲线H与曲线HL之间,参见图

3-4.事实上,由于炬=涉,而灯(\$) = y(l+。⑹),所以当凶f
时时,8=。(冰〃),而且了 =京\$)转化为

代\_ {{]}} 49丄八,八

曲2 \_ 择(6)(1 +O3))Z 25 + \emph{顷\&}),

即尚=一保尻+os*'类似地,,=龄)\emph{转化为}M =---蘇+ \emph{0(.\^{}).
\textbar{}}

定理Lio存在(巧,出)=(0,0)的邻域4,使当(由,凹)e
\&并且在曲线H上方或HL下方,则系统(1. 3)无闭轨.

证明 由定理1.9可知,当Crt,\textsubscript{ft})eH(或HL)时,相应系统

(1.3)无闭轨,而且从线段

乙=\{(工,\$),\emph{---』---}\textsc{m} \textless{} x \textless{} V---
\emph{Px ,y} = o\}

上任一点岀发的轨线均为向外(或向内)盘旋的螺线.现在把系统 (L3)写成

等=贝3),尊=丫(工十),

则当\emph{yj\^{}o,} \textbar{}x\textbar{},
\textbar{}j\textgreater{}\textbar{}适当小时,

\begin{quote}
\emph{U} V \_

\emph{迎} 亟"Q + z誤+丁務)w. \textsc{(lid} \emph{3f\textgreater{}2} 3ft
\end{quote}

注意向量场的奇点为(一丿二;,。)与(丿二,。),且其任意闭
轨(若存在)必与连接此两奇点的线段\emph{L}相交.因此,在考虑闭轨
的存在性时,总可以适当选取A,使少在原点附近变动,从而
(1.11)成立.此时,系统(L 3)关于的形成广叉旋转向量场(见 [ZDHD]或[Y1J).V
e勺,且位于H上方,取(由回)G

\includegraphics[width=3.06667in,height=1.35972in]{media/image39.png}

图3-5

H,且总\textless{}化,则从乙上同一点出发的相应于(内,%)的轨线,必保
持在相应于的轨线尸的外侧.但尸是外旅的螺线,因此7不
可能成为闭轨(见图3-5(a)).同理可证(咼心)e A且在HL下方 的焙形(图3-5(b)). I

由定理1. 9,定理1.10及变换(1. 4)之前对情形的讨 论,得到定理L1的结论'

普适性的讨论

由定理1.1知道,两个具有(1. 3)的形式但相应于不同的Q与
①的向量场族有相似的分岔图和轨线拓扑分类,其差异表现在分
岔曲线H与HL的表达式的高阶项0(\^{}1)中.这种差异为我们证
明它们的等价性,并进而证明普适性带来困难.为此,我们先从几\\
何上考虑一个较一般的问题.

设匕,巴,丫3是平面上在原点相切的三条C,曲线,在适当的坐
标(工十)下,它们有表达式夕=\%(",£ = 1,2,3,并满足条件
匕(0)------\sout{。{七?})}0, \emph{i =} 1,2,3;

以及匕''(0)
,v\textsubscript{2}\textsuperscript{w}(0),為"(0)彼此不等.因此,可以令
\emph{KY Y Y)-}{巴"(°)一 匕''(°)}

y ,3 \_匕〃(0) \_Y*0),

则适当排列匕•的顺序,可使0 \textless{} \emph{KY,}
,y\textsubscript{2},為)\textless{} 1.在下面的引理
中,我们对[Bo□的结果作了必要的修改.

引理1.11设为与乙,£=1,2,3,是两组满足如上要求的曲
线,则⑴如果存在原点邻域内的C\textsuperscript{2}变换,保持原点不动,并把匕.
映到1,2,3,则

\emph{\textsc{Y\textsubscript{z},Y\textsubscript{3})}}
=7(Z\textsubscript{1},Z\textsubscript{2},Z\textsubscript{3}). (1. 13)

(2)如果(1.13)成立,则存在原点邻城内的拓扑变换,把与 映到奏''=1,2,3.

证明先证(1).设存在宓变换

\begin{quote}
z --- \emph{u =}
\end{quote}

它把曲线 \emph{Y\textsubscript{i;}y = y\textsubscript{;}(.x)}变到命 3
=召(``),\emph{i=} 1,2,3.

/"(z,M(z)) = \%(g(z 商・(z)))

是对H的恒等式J= 1,2,3.把上式对;r求导,得到

\emph{fz + f\textsubscript{r}'yi} =矽(居 + \emph{gy'y'),}

再对m求导一次,得到

\emph{U + fj'yr + 3 =}

\emph{X9 + gjy/y + + g\textsubscript{y}'y"\textgreater{}-}

(1.16) 由于/\textless{}0,0) 一 g(0,0) = 0,且 j,(0) = \%(0) =
\textsc{m'(0)} = z/(0)= 0,在(1.15)中取x = 0可得77(0,0) =
0.再由(1.14)的非退化 性得/7(0,0)尹0,g」(0,0)尹0,从而可由(1.16)得到

把上式代入(1.12)可得 7(匕,丫2,丫3)= \emph{KZ\^{}z\^{}z,).}

为了证(2),只须证明在条件(1.13)下曲线族{匕}和{Z,},都
可以经拓扑变换化为{X,}盘=1,2,3,这里X,.的方程为XJr)三
0,X2(Z)=宀乂3愆)=澎9=/(匕,匕,匕)=/(Z\textbar{}ZZ).下
面只须对{匕}证明即可.首先,容易看出Q变换愆,少-\textgreater{}
\emph{(x,y -}

把{匕}变到 \&,},R(z) = 0,五(z)=妃£(z),W(z)=
\emph{坤6,}其中\emph{\%3}为不等的非零常数,扌与彳为C\textsuperscript{0}函数,满足f(0)
=7(0) = 1.由前面的结论(1),可算出£ = \emph{c.}再作一次原点附近

的拓扑变换z-*z, 则Z变到 五这里无3)三0,

丫2愆)=护,丫3愆)\textsc{=chW(z),}其中 \emph{c =}
C\textsuperscript{2},"。)= 1.由于

o\textless{}c\textless{} 1,我们可定义在Z = 0附近的连续函数r(z)如下:

\emph{心=\sout{{尸 \_}}} (1.17)

\emph{■J}1 --- \emph{eg)}

令区域玖={(Z,少 IzNOuNx\textsuperscript{3}} u {Gr,R)HW。},玖={(礼
\^{})\textbar{}x\textgreater{}0,0\textless{}j\textless{}\^{}),A-
{(x,y){[}x\textgreater{}0\^{}\textless{}0).取Gr,» =
(0,0)的充分小邻域U,定义变换

'« =z, \emph{y,} 当(z,丿)£ 以 n S

-"=z + Zr? --- y/O) , u =夕, 当\emph{(.x,y) Q D\textsubscript{2}
C\textbackslash{}Ui}

.u + ''(z) , \emph{v = y,} 当(z,丿)6 D\textsubscript{3} Q U.

(1.18) 此变换把曲线3 = \#'和;y =由孩z)分另{[}\textbar{}变到v = o,u =

``z和。=点.利用/\textsubscript{eC}°和/(0) = 0可知,(1.18)在U fl (玖U
D\textsubscript{2} U D\textsubscript{3})是保持原点不动的拓扑变换, I

引理1.12对于具有(L 3)形式相应于不同的\emph{Q,\^{}}的两个
系统族,存在(内/2)平面上保持原点的U变换,它把其中一个系\\
统族的分岔曲线分别变到另一个系统族的分岔曲线.

证明 对任一CL3)形式的系统族,由定理1.1,它的分岔曲

线为 ``I = 0,``1 =---必 +。(``丿),和 \emph{Hl} =---聂妃
+O(As\textsuperscript{T})\textsubscript{t}(当 ft f
o+),因此「由公式(L 12)得 J(SN,,H"HL,) = i = 1,2,
由引理1.12的结论(2)立即推得本引理成立.I

引理L13 具有(1.3)形式相应于不同Q,◎的两个系统族 彼此等价.

证明 设这样的两个系统族分别为

dr

击=夕'

栄=``1 + 陌 + / + \emph{xyQi} (x,\^{}) + 奶JwQ

\begin{quote}
=4 + \textsc{AjR} +工'+ H列z(Z,Q +夕2电(们),心,
\end{quote}

其中 \textsc{a,A6} R",»i\textgreater{}2\textsubscript{(}Q,\^{}(i=
1,2)W\^{}(1,3) 的条件. 根据引理1.12,存在(佝,役)到(4,过的拓扑变换A
\emph{= 脸,i =}

1,2,满足HO) =0,且把X】的分岔曲窪SN产,H\textbar{},HL\textbar{}分别映到
\emph{X\%}的相应分岔曲线SN?,H\textsubscript{a},HL\textsubscript{2}.因此,代替X''可考虑系统族
(X/)"

---了,

一 Aj(u) +涙点)夕+ z: + Z列2愆,2(产))+夕2孔3,夕,2(产)).

这里人=人(产)表示A =嵐(产),``=为(``),为=产3,``•,丄=四.
这样,当产取遍原点的邻域△的值时,取遍原点的邻域云- 中的值.V \^{}6
△,我们可以利用本节前半部分的结果,先建立X

在原点某邻城U(Q中的极限集(奇点、闭執、同宿轨)到X/在原
点相应邻域\emph{U'} (Q中极限集间的同胚砍Q,然后把此同胚扩展到
中其它轨道上,从而建立\emph{Xi}在中轨道到X/在\emph{V} (兴)
中轨道的同胚,I

定理1.14双参数族

\begin{quote}
ir -

否=义

* =祖 + 如 + \# + 时
\end{quote}

是奇异向量场(1.2)(沥>0)的一个普适开折,

证明 由第一章定理5.13知•,(1.2)的任一开折可经过b

变换转化为与下列系统等价的系统(指在原点的邻域内)

dz

在=贝

• (1.20)

浆=p(e) + a(e)\_y \emph{+ x\textsuperscript{2} +} xjQ(x,e) + N(礼 W),
其中参数 e 6 R*,m N 2,函数\emph{甲仲,Q,6} \& C\textasciitilde{},p(O)
=(6(0) = 0, Q(0,0) = 1.与(1. 20)相联系,考虑含有m + 2个参数的系统

\begin{quote}
\emph{=y,}
\end{quote}

(1.21)

\begin{quote}
=1 \emph{+ / + xyQ(.X,£)}+ y\^{}(x,v,e),
\end{quote}

其中3, A)是独立的参数.如果我们把(1. 19)看成是含成+ 2个
参数的系统(除内,牌外,另外推个参数e不显含),则由引理1.13
及其证明过程可知,存在与UA)之间保持原点的拓扑
,变换巧=出(方,\&),凹使得对每一个固定的3 A), 系统(1.21)与系统

\begin{quote}
dx

d (1.22)

盅=MA A)+ ``2(4,為》+ \# + \emph{xy\\
}拓扑等价,这里把(1.22)看成含参数e的系统.特别地,在(1.21) 与(1.
22)中分别取4 =祁),兀=\emph{炒},则(1. 21)转化为(L 20), 而(1.
22)变为(1.19)的一个导岀族(见第一章定义5. 5,此时把
(1-19)看成一个双参数系统族).这就是说,(1. 2)的任一开折,
都与(1.19)的某一导出族等价,按第一章定义5. 6, (1.19)是系
统(1.2)的一个普适开折.I
\end{quote}

\protect\hypertarget{bookmark146}{}{}§2二重専特征根:1:2共振间题

由第一章习题1. 4,在奇点(0,0)的线性部分矩阵为籍零阵,
并且在旋转角度\emph{K}下不变的平面向量场具有如下的三次正规形:

¥ = = ar\textsuperscript{3} + \emph{bx\textsuperscript{2}y.} (2. 1)

为了陈述简洁,我们在这里只考虑三次正规形而去掉了高阶项.事
实上,在本节的基本假设泌尹。下,保留高阶项对整个讨论没有本
质的影响(可参考§1).

当沥尹0时,可经尺度变换,把(2.1)化为

这里正负号的选取与沥的符号有关.与§ 1中的方程(1. 2), (1.3)
不同,这里取正号与取负号的结果是不同的.

类似于第一章定理5.13,有下面的结果.

引理2.1对于(2. 2)士的任一个旋转角度n不变的C"开折,
都可以经过C"变换,在相空间与参数空间原点的小邻域内,把此
开折化成与下列向景场等价的形式

\emph{dx}

栄=\emph{叭启H} + a(g 士史+工*愆,产)+J*(3,G,

(2. 3)士

其中夕仲,@,攻 £ b,衣0)=扒0)= 0,B(o,o)=--- 1. \textbar{}

设\^{}eR-,m\textgreater{}2,衣岡\emph{冲}是独立的,则可进一歩经过参数
空间的变换,把\textless{}2. 3)士化为

dx

否=义

§ =有工 + \emph{S\textsubscript{2}y} ±x\textsuperscript{3}
+x\textsuperscript{2}j\^{}(x,e) +y\textsuperscript{!}'5'(x,3-,£),

(2. 4)士 其中6 C\textasciitilde{},\textless{}5(0,0) =一 1.与§
1类似的讨论可得,(2. 4)土
的分岔规律及轨线拓扑分类与@,亚的具体取法无关.而最简的取 法为0 = -1,\^{}
= 0,即

牙=丿,糸=砧+3±刀3-4 (2.5)*

本占的主要结果是

定理2.2({[}Ho{]}) Q)在旋转角度\textgreater{}r下不变的开折中,系统
\textless{}2. 5)士是系统(2. 2)士的普适开折.

\begin{enumerate}
\def\labelenumi{(\arabic{enumi})}
\setcounter{enumi}{1}
\item
  (2'5)+的分岔图由(勻,弓)平面上的原点及下列曲线组成: pichfork 分岔曲线
  R±=⑴勻=0,\& \textgreater{} 0 或 e\textsubscript{2} \textless{}
  0)\textsubscript{(} Hopf分岔曲线11=住旧=0国V0\},
\end{enumerate}

\begin{quote}
异宿分岔曲线HL="旧=---+0(旧什)疤\textless{}o\}. (2.
5)+的轨线拓扑分类见图3-6.
\end{quote}

\begin{enumerate}
\def\labelenumi{(\arabic{enumi})}
\setcounter{enumi}{2}
\item
  (2.5)-的分岔图由(勻,\&)平面上的原点及下列曲线组成,
  pichfork分岔曲线R±= \{司取=0,\& \textgreater{} 0或勺V 0\} ?
\end{enumerate}

Hopf分岔曲线巨=\{€\& = 0舟\textless{}0\}¥

\begin{quote}
Hopf 分岔曲线 H\textsubscript{2} = (£(£; = e, + O(W),£i \textgreater{}
0\}\textsubscript{t} 同宿分岔曲线HL = "\textbar{}弓=+ 0(e\#)E
\textgreater{} 0 j \textsubscript{f} ' 二重闭轨分岔曲线B=
(e\textbar{}e\textsubscript{2} =«, +0( J),弓\textgreater{}0 \}, 其中ex
0. 752. (2. 5)一的轨线拓扑分类见图3-7.
\end{quote}

\includegraphics[width=3.58681in,height=1.45347in]{media/image40.png}

图3-6

\includegraphics[width=3.55347in,height=2.54653in]{media/image41.png}

图3-7

对(2. 5)\textsuperscript{+}的讨论与§ 1很相似,所以下面仅对(2.5广的情形
作一简要介绍,特别是要指出与§1的不同之处.

首先,(2.5) 一的奇点满足y=0及---\^{} = 0.与第一•章例 2.
6对照易知,当勻=0(些尹0)时发生pichfork分岔.

其次考察\textless{} 0的情形.这时(z,y) = (0,0)是(2.5)-的唯
一奇点,并且在该点的线性部分矩阵为

(°牛

住1色/

当嶋=0,与 \textless{}。时,作变换工=则对变换后的方 程应用第二章公式(3.
3)可得,Re(\%) =-\textbar{}\textless{}0.因此,由第二章
定理3.1知,在曲线H,上发生Hopf分岔.

最后考察勻\textgreater{} 0的情形.此时进行变换

\emph{=淨,电=}邳 \emph{x} =应,\emph{y} =巧,Z =号.(2. 6)
然后记G,§,£)为\emph{(.x,y,t),}则(2.5广转化为

等=丿,尊=\^{} ---书+席―f. (2. 7) 它可以看作是Hamilton系统

{d,r} dv ,

五=队-£ = \textsuperscript{x}\textasciitilde{}\textsuperscript{x} (幻8)

当\emph{0\textless{}3«} 1时的扰动系统.系统(2. 8)有首次积分

H(w)=扌一苏+ 十=九 (2.9)

记「x = \{(日力旧(吋)=幻.当一-i-\textless{}A\textless{}0时必是围绕两
奇点(-1,0),(1,0)的两条对称闭轨;当五f---j时,它们分别缩
向这两个奇点;当A-*0\textsuperscript{+}时,它们扩大而形成鞍点(0,0)的对称双
同宿轨匚;当兀\textgreater{}0时,马是围绕三个奇点及匚的闭轨.以上分布
见图3-8.若令£ =丄UA,其中

乙 1 = \textsc{Ir=0,0WhW1\},}

\includegraphics[width=2.50694in,height=1.14028in]{media/image42.png}

囹3-8

2/2 = \{\textbar{}x = 0,夕 \textgreater{} 0),

则当一时,匕与Z相交于唯一的点处=(«(A),0)\textsubscript{f}当
A\textgreater{}0时,马与乙相交于唯一的点么=(。丿(邳)).因此上可被五参
数化.

现在考虑扰动系统(2.7).y 0 Vd《1,点(一1,0), (0,0),
(1,0)仍是奇点.如果对以.7)的对称团轨只考虑落在右半平面的 部分,则(2.
\emph{7)}的闭轨必然与乙相支,取為6乙,记系统(2. 7)过么
的轨线沿时间正,负向延伸分别与正z轴的(第一次)交点为Qz与
Q,再记,(方04)为从Q到Qz的轨线段,见图3-9.则与§1引理 L2---1.4类似可得

引理2.3 V 0\textless{}5\textless{}l,7(A,5,f)是系统(2. 7)的闭轨,当且
仅当

\emph{F(h,S,\^{})} = f (了一工,奸=0; (2.10)

而且,成为(2. 7)的同宿(双同宿)轨,当且仅当F(0-,8,C = 0 (或 F(0+,M)= 0).
\textbf{1}

为了研究分岔函数\emph{F(h0©}对\emph{h}的零点个数,先考察函数

F(A,0,r)- f (,一 (曷一匕3),(2.11)

\begin{quote}
J气
\end{quote}

其中 \emph{I\textsubscript{k}W} =丄 \emph{x*ydx, k} =
0,2.与第二章中(5.18) - (5. 20)的

\includegraphics[width=2.45347in,height=1.54653in]{media/image43.png}

\begin{longtable}[]{@{}ll@{}}
\toprule
\endhead
& 图3、9\tabularnewline
\bottomrule
\end{longtable}

结果类似,容易得到

1

引理2.4 ⑴当\emph{h\textgreater{}\_ 土}时,/心)\textgreater{}0,

\begin{longtable}[]{@{}ll@{}}
\toprule
\endhead
\begin{minipage}[t]{0.47\columnwidth}\raggedright
\textless{}2)/\textsubscript{0}(-\textbar{}\textbar{} = A(-

因此可定义\strut
\end{minipage} & \begin{minipage}[t]{0.47\columnwidth}\raggedright
11 c ..【3 ,

刃=。「咀顽=L\strut
\end{minipage}\tabularnewline
\begin{minipage}[t]{0.47\columnwidth}\raggedright
\begin{quote}
P(ft)=-
\end{quote}\strut
\end{minipage} & \begin{minipage}[t]{0.47\columnwidth}\raggedright
{'2(五)}少厶、 L

\textsuperscript{11} 当 \textsc{\textsuperscript{a=\_}t}\strut
\end{minipage}\tabularnewline
\bottomrule
\end{longtable}

从而(2.11)成为

= Z\textsubscript{0}(A)\textless{}f - \emph{P〈h». (2. 12)}

引理2.5函数FS)有如下性质

\begin{enumerate}
\def\labelenumi{(\arabic{enumi})}
\item
  limF(A)=纟,lim F(A) =+ oo;
\end{enumerate}

\emph{hf。} U \emph{h-\textasciitilde{} + aa}

\begin{enumerate}
\def\labelenumi{(\arabic{enumi})}
\setcounter{enumi}{1}
\item
  a A* \textgreater{}0,当一十 Q vr 时尸 3) VO,而当
  \emph{k\textgreater{}h-} 时 F(A)\textgreater{}0;
\item
  P(A* ) \textgreater{} \emph{j,P'} (ft- ) = 0, W ) \textgreater{} 0. I
\end{enumerate}

与§1中函数P(Q不同的是,这里的PS)不是单调的,上面

的性质(3)将导致二重闭轨(半稳定极限环)•的存在.我们可以从

\includegraphics[width=3.06667in,height=2.21319in]{media/image44.png}

图 3-10

P(盼的图形粗略推得(2. 7)轨线的拓扑分类,见图3-10.事实上,
从(2.12)可知,研究F6,O,Q的零点可用直线,=常数么去截\&
=PS)的图形•当n> 1时,截得的交点相应于如 >它对应于
系统的环绕三个奇点的``大极限环";当\textbar{}<?0< 1时,截得两个交
点的橫坐标分别为和瓦,>。,前者相应于一对对称
的``小柢限环",后者相应于一个``大极限环'':当\& = §时,系统出
3双同宿轨,且外侧仍有一个"大极限环七而当c <爲V *(c =

))时,两个交点的橫坐标\&也> 0,系统出现两个``大极限 形;当土 f
c时,这两个``大极限环''合为一个半稳定环;当協V C

时,极限环消失.

以上仅从F(奴3占)的一阶近似F\%,0,;)来讨论的.严格的
数学论证要利用(对参数一致的)\textsc{HqM}分岔定理,同宿分岔定理,
隐函数定理等等.

对于具有%字型,,双同宿轨的平面系统(见图3-8),在[Lw]
中证明:它在一般三參数扰动下至多可出现5个极限环.

\textbf{§3}二童零特征根:\textbf{1W}共振问题\textbf{3 2 5)}

由第一章习题1.5,以(0,0)为奇点并具有二重零特征根,旋 转号(g
\textgreater{} 3)不变的向量场具有如下的复正规形

糸=5+ 以打 + ••■ + C„z"+'z" +
\emph{A\^{}\textasciitilde{}\textsuperscript{l}} +。(
\textbar{}z\textbar{}q), (3.1)
其中\emph{z\textsubscript{f}c,;A}是复数,整数\emph{m} =[写\^{}丄

若Re \$尹0,A尹0,则(3. 1)是余维2的,它的十个普适开折 为

糸=\& + 勺土 + c\# + ... + W+它 + \emph{A\^{}-\textsuperscript{1},}
(3.2) 其中e =勻+ i\&,G = q +耕.由于%尹0M尹0,总可对作 变换化为\emph{A}
= l.a, =- 1的情形,再利用极坐标变换,可把(3. 2) 化为

\begin{quote}
£ = M ---尸 + \emph{a\textsubscript{2}r\textsuperscript{5} + •••} +
0„户+1 + 卢 Tcos(g。), £ =勺 + 缶r\textsuperscript{2} +
\emph{b\textsubscript{z}r* + ― +} -
\^{}\textasciitilde{}\textsuperscript{z}sin(g5).
\end{quote}

(3.3) 显然,当勺=0為尹。时,发生Hopf分岔.若£,\textless{}0,则由3 3)的
第一个方程可知,在r = 0的小邻域内的所有轨线当+8均趋 于唯一的奇点\emph{r
= Q.}当% \textgreater{}0时,作变换

弓=萨,成 =州,尸=Sp, £ =普, (3.4)

其中\$\textgreater{}0.将f仍写成£.下面假设?\textgreater{}5,则(3.3)化为
等=成1 一尸)+况趴"=R(p,章),

\emph{■,0} (3.5)

茅=,+ M + \emph{Sgts,P,8}} = \&頒,脆),

其中

\emph{f\^{},p,ey} = a』裆 + \ldots{}+ \^{}-\^{}-+1 +
\emph{3\textsuperscript{9}\textasciitilde{}\textsuperscript{5}\^{}\textasciitilde{}\textsuperscript{1}cos(.\textsubscript{q}e'),\\
}g(凯p,8) = \emph{b\textsubscript{t}Sf?} + \ldots{}+
60-7产-\emph{字-5广Sg"\\
}当3=0时,(3.5)成为

£ = p(l --- Q,黑=亍 + \emph{b\#.} \textless{}3. 6)

此系统有唯一的不变环£= {頒,\&)侬=1},它是吸引的.当女+
虹R0时,它是一稳定的极限环t当§ + \& = 0时,其上充满了奇 点.记勤为(3.
5)3 A 0)的流的时间1映射,{[}RT{]}证明了,存在
\$\textgreater{}0(相应地,由(3.
4),存在\&\textgreater{}0)使得对每一个(0,孑),(相 应地\& €
(0,百)),流形列玛以)当''-»+ 8时趋于(3. 5)的一个 不变流形d,它是(3.
5)的一个吸引的不变环,而系统(3. 5)的动
力学行为完全由此系统在这个不变环上的性质所决定.记L =
{(司,取)晶\textgreater{}0}(即\& =。,见(3. 4)),则下面要
证明:存在曲线SM和S1\%,它们从两侧与L在勻=蓟=。相切,
并在珏=叼=0附近形成一尖狀区域XX见图3-11),当(牝勻)e
验\textbar{}时,(3.3)在耳上有Q个鞍结点;当(勻,\&)进入。后,它们分
解成\emph{q}个鞍点和\emph{q}个结点;当(与,M £ SNz,这匆个奇点又组合成
g个新的鞍结点,但\&上的流向与色,9£SN,时相反;而当3,
弓)在。之外时,耳上无奇点,即它是(3. 3)的极限环.

既然奇点的分岔现象发生在不变环上,由第二章定义1.1,定
理1.2和附注1.3可知,对每一固定的\^{}\textgreater{}0,发生这种分岔的奇点

\includegraphics[width=1.57986in,height=1.31319in]{media/image45.png}

图 3-11

U/)必满足

R頒,8,6) = 0。,们3) = det( = 0, (3. \emph{7)}

其中R,曲是(3.5)右端的函数.由于

\begin{quote}
d《醫剧=
\end{quote}

其中

\emph{D〈p,8,a)}= (3尸---l)cos(g0) + 2 缶衫sin(g0) + 0(3).

(3.8) 因此,(3.7)可改写成

R(p/,3)=包(°,们<J) = \emph{Dg。)} = 0. (3. 9)

下面,我们在3,们``\$)空间的集合

\emph{W} = \{(p,们。,3) Ip 1,0 = 0\}

附近考察(3. 9)的解.

由(3. 5)的第一个方程及隐函数定理易知,在IV附近,集合
\{(F,8,``6)\textbar{}R= 0\}有表达式印(0,§,8),歹4濟\},其中戸工1 + 如.

由(3. 8)知D Ixi = 0在归附近对<?有匆个根.因此,集合
\{(p,0,``8)\emph{\textbackslash{}R} = 0,D = 0\}在IV附近具有以下形式:
勺'"酒0(洒,。,心,

\begin{quote}
J-0

其中\emph{歸再(j} = 0,l,・・・,2g --- 1)满足以下条件}
\end{quote}

\begin{enumerate}
\def\labelenumi{(\arabic{enumi})}
\item
  9•以0)= A + \emph{+ Og}其中馬満足'
\end{enumerate}

\begin{quote}
cos(ggo)+ fiisin (\^{}0)= 0; (3.10)
\end{quote}

\begin{enumerate}
\def\labelenumi{(\arabic{enumi})}
\setcounter{enumi}{1}
\item
  ft\textsubscript{+2} =再且为2 \emph{=们*}籍(方程的对称性);
\end{enumerate}

\emph{q}

\begin{enumerate}
\def\labelenumi{(\arabic{enumi})}
\setcounter{enumi}{2}
\item
  两=1 + 0(6).
\end{enumerate}

我们先估计A - \emph{K}由于糸髙,瓦,以)=务:再,\&,E = 0,得到

爲一祝 +
\textsc{\^{}\^{}\textsuperscript{4}cos(\textsubscript{9}\^{}\textsubscript{!i})=}

内一裨+ /\_\%OS(用g+卩)+ gf , '

整理得

(A 一 时口 一(府 + \emph{两禹} + 离]=2跑TcosWV + 0(跟一3), 从而

\begin{quote}
崗一禹=一户一 AssCgg。)+0(觀一 3). (3.11)
\end{quote}

由(3. 10〉知 cos(gg°)尹 0.因此两-ft =O(/T).

最后,考虑集合{3,。占,》\textbar{}R = \emph{0,D} = 0,8 = 0),注意

创R=0,£\textgreater{}\textasciitilde{}0 \emph{= 了 +} K衫 +。(6),

因此,存在函数= 0,1,-,2? - 1,使

©(两(Mj3),Q0(Mg),a),机⑹ 渺)=0. (3.12)

由对称性知,M,+2(8) - M/5).下面估计MJS) - 在等

式(3.12)中取j= 0,1得

M0)+ - f-Gin包\& + nJ

= M°(S) + 加禅Wj⑹ 0) - 觀一站认(点)+。(浇-3). 利用(3.12)和為=1 +
03)整理上式可得

Mi(8) -
M。⑴=2fi\textsuperscript{5}\textasciitilde{}\textsuperscript{J1}Cfe,cos(\textsubscript{9}\^{})
一 sinC"。)]十 O(S一3). 再由(3.10)知sin(g\&,)\emph{尹}0,并且

f = 20cos(g烏)---sin®。。] = --- 20 + l)sin(''(,)尹 0.

总结上面的讨论可得

\begin{quote}
引理3.1 \textless{}m)在空间中w的邻域内,集合 \{3,0,,a)\textbar{}R = 8
= D = O\}
\end{quote}

由2?条曲线组成,它们在応,\$)平面上的投影由二条曲线 = 1,2)组成,满足 M(S)
- *0)=勢T +

O(8\textasciitilde{}3),其中常数\$尹o.

定理3.2系统族(3.3 )的分岔图由空间中的原点及 下列曲线组成:

Hopf 分岔曲线 H±= \{e\textbar{}勻=\textsc{0,e\textsubscript{2}}
\textgreater{} 0 或 ej\textless{}O\}'\textsubscript{f}

鞍结点分岔曲线SN,=怔低=Cq = Q(s),s为参数\}, j=
1,2,其中\emph{心)滝}足为(0) = 0,而且

(3.13)

\includegraphics[width=3.78681in,height=2.85347in]{media/image46.png}图
3-12

证明 由上面的分析,(3.5)的分岔现象发生在席上.由引
理3.1,分岔曲线在平面上为点(一知0)及曲线⑴,的, j = l,2组成,且

\begin{quote}
一 =澎一4 +。(序一3)\emph{况圭} 0.
\end{quote}

利用变换(3. 4)从系统(3. 5)返回到系统(3. 3),从而把分岔曲线 的方程从(O
空间返回到魚,\&)空间,得到习=\& = 0以及两条 曲线

〈(J弓)丨%.=翳,勺=孕虹⑹},

{(勻,勻)1勺=序,\& =孕肱⑹},

其中 8为参数.令 A/\^{} =
\textless{}J\textsuperscript{2}\^{}\textsubscript{1}(\textless{}J),;=1.2,则

如(3) --- \emph{he} =序(M3) - Af\textsubscript{0}(5))=\emph{財\_2}
0(\^{}-1),

其中£尹。.定理3.2得证.\textbar{}

\protect\hypertarget{bookmark157}{}{}习题与思考■三

3.1在引理1. 6和引理1. 7中,为什么要分别利用对参数一致的同宿分
岔定理和对参数二致的Hopf分岔定理?这对最终得到定理1.1有什么作用?

3. 2证明引理2.1.

3. 3证明引理2. 5.

\begin{quote}
3. 4对方程(2. 5)+讨论它的分岔现象,并证明分岔图3-6的正确性.
3.5考虑方程\textless{}2.7).试证明:
\end{quote}

\begin{enumerate}
\def\labelenumi{(\arabic{enumi})}
\item
  \begin{quote}
  当 ⑷充分小时,(2.7)在双同宿环(记为兀)附i£有极限环的
  必要条件是\textbar{}£一号丨《1.
  \end{quote}
\item
  \begin{quote}
  当均充分小时,(2.7)在匚附近至多有一个包 围三个奇点的大彼限环.
  \end{quote}
\end{enumerate}

(提示,当 *0山一新《1时,(2. 7)在原点的发散量为aeo.)

第四章双曲不动点及马蹄存在定理

本章中我们将给出关于R'中映射的几个定理.这些定理在第
五章中是我们研究空间R'中鞍点同宿分岔的基础,同时这些定理
也有其自身的重要价值.在§ 1中我们证明一个双曲不动点定理.
在§沖引进符号动力学的基本概念.在§ 3中给出马蹄存在定理. 在§
4中给出一个关于两个线性映射的复合映射的双曲性的判别 引理.在§ 5中作为§2-
§ 4中诸结果的一个应用,我们将给出R\textsuperscript{3}
中Birkhoff-Smale定理的证明.

本章和下一章的主要定理在Sizikov的文章[Sill-3]和
Wiggins的书[Wi2]中都能找到,但此处所有定理的证明都是独立
给出的.我们力图把几何直观与数学的严密性统\textasciitilde{}起来,并给予
读者一套易于掌握的方法,用以解决高维空间中其它类似的问题.

\protect\hypertarget{bookmark162}{}{}\textbf{§1}双曲不动点定理

定理的舔述

首先我们引进一些基本概念.令

\textsc{r\textsuperscript{1}} x \textsc{r\textsuperscript{1}} = \{(3)k
e R\textsuperscript{1}, e R\textsuperscript{1}\}.

定义LI设是一个常数.平面RXR1上的一条C水平
曲线指的是一个李氏常数为C的K的函数伝0)的 图象.当一
8VaV\^{}V+8时,称点为该曲线
的端点.类似地,R,XRi上一条C垂直曲线指的是一个李氏常数
为C的了的函数\^{}=g(y),ye(a,b)的图象.特别地,当a=-8,3

=+8时,则称其为无限\emph{C}水平曲线或无限C耋直曲线.

\emph{定义1.2}称平面RiXK上两条不交的无限C水平曲线所界
区域为\emph{C}水平带域;两条不交的无限C垂直曲线所界区域为\emph{C}毒
宜带域.

以下我们总是假设内是两个正常数,满足\emph{说}

定义1.3平面RIXRL上的一个区域D称为(埃,E)矩形,若
它是一个内水平带域与一个阳垂直带域的交,那组\emph{林}水平对边
称为边界的水平部分,记作也那组冉,垂直对边称为边界的整 直部分,记作為D

对\emph{pE\^{}XR\textsuperscript{1},}我们记\^{}(R'XR\textsuperscript{1})为平面R'XR'过点巾的切
空间.令

\emph{K,} = {孩=(?;居)e T/R\textsuperscript{1} X RD I 班丨 a
\textsc{kH}頒丨},

\emph{K\textasciitilde{}} = \&=(頒,或)\& T\^{}CR' XRD\textbar{} 吟丨
£妇頒\textbar{}}, 见图(4-1)-

\includegraphics[width=1.47986in,height=1.87986in]{media/image47.png}定义
1.4 令 DCR'XR' 是一个区域./;!\textgreater{}---R'XN是
一个微分同胚.称/在D上满 足(如%)锥形条件,如果有

⑴ \emph{DfK;} U K必), vyen,

(2) \emph{Df-\^{}K-}

\emph{y pefD\textsubscript{t}}

且存在常数41,瘻得

(3) l(D驾\textgreater{})+ \textbar{}3A\textbar{}\#\textbar{},

\emph{Y Z DA/ K;;}

(4) \textbar{}(n/-'Q- I

直观地说,映射/■满足(払,知)锥形条件指的是r在水平方
向压缩而在垂直方向拉伸,并且它将任意\emph{匕}垂直曲线奏为\emph{孔}垂

直曲线,而其逆須T将任何内水平曲线变为内水平曲线.

定理1.S (双曲不动点定理)令D是平面RiXRi上一个
(\&,\%)定形,儿DfIVXRi是一个微分同胚,满足(物,为)锥形条 件,并且有

\begin{enumerate}
\def\labelenumi{(\arabic{enumi})}
\item
  相交条件成立
\item
  边界条件成立,為Z)fl\_D=0,
\end{enumerate}

则\_/■在D中有唯一双曲不动点

\emph{O} =但8如

\includegraphics[width=1.44653in,height=1.76667in]{media/image48.png}\includegraphics[width=1.54028in,height=1.67986in]{media/image49.png}

图4-2 上述定理的几何直观见图4-2(a).

证明的思路

令 '

\begin{quote}
\emph{D\textsubscript{0} = D,} D,. = /(/);\_!)CID,

D\_f = /-1(Z)T+1 n 知), \emph{i} e N, (1.1)
第一步证明D土,是(用,片)矩形,并满足条件
\end{quote}

3g・U沪 \emph{為J U}沪

以 UD-i, DtUQt+i, ie N. (1.2)

\begin{quote}
8 OO
\end{quote}

第二步证明V =Q以和\emph{H} 分别是端点属于\emph{a\textsubscript{k}D}和

初的冉垂直曲线和\emph{B}水平曲线,

第三步证明\emph{H(\textbackslash{}V}是\emph{f}的一个双曲不动点.

见图 4-2(b).

几个引理

引理1.6平面WXIV上一条内水平曲线与一条㈤垂直曲 线最多有一个交点•

证明 令
\emph{\textbackslash{}-\textgreater{}\textasciitilde{}y---h(.x), v;y}
\textbar{}-*x---z)(j)是两个函数,它
们的图象分别为内水平曲线H及片垂直曲线V.再令(血,少),
(X2,\^{})ewnv,则

\_ 务丨 一 \textbar{}» ° A(X,) --- \emph{v -} I

W 從g(zi)---儿6)\textbar{} W0 • \%\textbar{}屯一处\textbar{}.

因为知%\textless{}1,故思一志I =0.进一步I乂一队I = o. \textbf{I}

引31.7令DCR\textsuperscript{l}XR'是一个(3,\%)矩形,H,VUD分别
是端点集属于\emph{3\textsubscript{V}D}和\emph{d\textsubscript{k}D}的%水平曲线及%垂直曲线,则\emph{H}
与卩相交于唯一一点・

证明 交点的存在性可由连续性得到,而唯一性由引理】• 6保 证.\textbf{I}

引理1. 8令,Z)2URi Xie是两个(内,%)矩珍 设\emph{f,D\textsubscript{t
}}fRiXW是一个微分同胚,并满足(内,队)锥形条件及

(1) 相交条件,/z), n o\textsubscript{2} y: 0 ;

(2) 边界条件:\emph{fD,} n \emph{3\^{}2} = 0, 则

\emph{i-fDi} n风是一个(何,%)矩形,满足

毎(/■以 n
\textsc{d\textsubscript{2})}cza\textsubscript{A}D\textsubscript{2},

2.广(玖n \textsc{rd}是一个\textsc{Sb)}矩形,满足

礼(广YQn/A))u 礼\textsc{a}

上述引理可參见图4-3.

\includegraphics[width=1.37361in,height=0.62014in]{media/image50.png}\includegraphics[width=1.52014in,height=1.74028in]{media/image51.png}

图 4-3 ,

证明 因为f满足(㈣\_,\%)锥形条件,故尹将%垂直曲线变
为%垂直曲线,而fT将用水平曲线变为内水平曲线.因此f将
区域以变成一个左右两边为\%垂直曲线的曲边矩形\emph{fD\textsubscript{t}.}再由相
交条件及边界条件,月\textbackslash{}是上下穿过玖的,也就是说\emph{fD\^{}D.}形成
一个十字形,因而\emph{fD\^{}D,}也是一个曲边矩形,其上下两边属于
0的边界的水平部分,而其左右两边属于的左右两边,因而
加\textbar{}\textbar{}"\textbar{}玖是一个(内,氏,)矩形,满足无(/A
0玖)匸布玖.类似地可 以证明另一个结论.I

令D土,.,沱N,是由(L1)定义的区域,用归纳法及引理1.8我 们有

引理1.9 Q土渥(用,Q矩形,满足(1.2).

国P/Q矩形的高和宽

令

\begin{quote}
K+= \{(石少€R' XR屮刃质妇牛\textbar{}\}, \emph{K-=} \{愆,刃 € R' X
Ri\textbar{} 牧I W"z\textbar{} \}.

给定R*XR'±两点户点,我们记讶€K± ,若连结\emph{p,q}的向量属于 \emph{K士.}
\end{quote}

令\emph{D}是一个3*,\%)矩形.下而我们定义\emph{D}的高及\emph{D}
的宽3(D)如下;

\emph{h(.D) ---} sup dist(/\textgreater{},g),

\begin{quote}
\emph{\textgreater{}q} GD

泰K*
\end{quote}

\emph{w(D) ---} sup dist(p.g),

\begin{quote}
洪K-
\end{quote}

这里dist(・,•)表示R'XR\textsuperscript{1}上两点之间的距离.

引理1. 10令DUfTXRt是一个闭区域,/iD\^{}R'XR\textsuperscript{1}是一
个饑分同胚,满足(用,气)锥形条件.设是一条g光滑的\emph{也}
垂直(或险水平)曲线.若对某自然数i\&N,r(或 广')在/上有 定义,则

1/\^{}1 质\&/\textbar{},1,(或 I厂叫
\emph{\textgreater{}A\textsubscript{k}X\textbackslash{}7\textbackslash{}\}.}

这里\textbar{},1代表曲线的长度( , A = ----\textsuperscript{1}------,而A

\begin{quote}
\textsc{a/1} + 成 -V1 + \emph{队}
\end{quote}

\textgreater{}1是定义1.4中的常数.

证明 我们只对,是%垂直曲线的情况证明引理,另一种情
况的证明可用类似方法得到•

首先设» = 1.令曲线/是一个映射項)丄0, 口---D的象,令。

=\textless{}r.r\textsuperscript{+})=Ko,则 rwK+.令

?(«)=(广,时)=D/?,

im(以121?\textsuperscript{+} a\textgreater{}i \^{}A)r si

求积分,得

\emph{m} = j:ng)忡 \textgreater{} = 4/阳.
对£\textgreater{}1,令\emph{g=f,}则g也满足(内,气)锥形条件,只是其中的扩张
及压缩常数A变为兒再将前面对£=1的讨论应用到映射\&即可 证明引理. I

引理1.11令Q土,.是由(1.1)定义的(用,%)矩形,则存在 常数00,使得

\begin{quote}
\textless{} CA-*, \emph{i} GN,
\end{quote}

这里是(厄,%)锥形条件中的常数.

证明 令少,亦\textsc{Z\textgreater{}t}满足苟£K+ .设是一条连结
的。光滑的公垂直曲线.由于/■满足(母,㈤)锥形条件,戸7是Q
中的\%垂直曲线.令£是Q中所有光滑的%垂直曲线长度的 上界,则由弓F理1.10,有

E 濠 \textbar{}/\textgreater{}\textbar{} 质4旳"£\&Xdist3,g)

或

distg) \textsc{w\&ieL.}

因此

方酚 t.

故存在常数G,使得

同理可证存在常数C2,使得

w(D,) C\textsubscript{S}A-. I

定理1.5证明的完成

由引理1.9和引理1.11立即得到

V = ".

i = 0

是一条角垂直曲线,且满足\emph{avu\^{}hD.}

是一条内水平曲线,且满足由引理1.7 WAV是一个 点,记为0,且有

\begin{quote}
\emph{fo= fw} nv)= A.\_n\textsubscript{m}/z\textgreater{}\textgreater{}
\end{quote}

=n \emph{戸 \textsc{d = h} n v = o.}

\begin{quote}
i" ---8
\end{quote}

故。是\textgreater{}■在D内的唯一不动点.因为尸在水平方向压缩而在垂直
方向拉伸,故不动点\emph{O}是双曲的.

\protect\hypertarget{bookmark184}{}{}\textbf{§2}符号动力学简介

本节将扼要地介绍符号动力学理论,它是研究动力系统的复 杂动力行为的基础.

符号序列空间及其结构

令

\begin{quote}
S = \{1,2,\ldots{},N\}, NN2.
\end{quote}

在S上引进如下度量,

如果"=如 (21)

如果\emph{ag \textsuperscript{( 5}}

则集合S在度量(2.1)下构成一个紧致的,完全不连通的度量空 间•令

这里§=SX⑴,换一句话说,*中的每一个点是\emph{S}的元素构成 的双边无穷序列

3 3,

则

3= \textless{}\textsuperscript{a}ihez = a\_i,Oo,-

这里\emph{a\^{}S}是s在分量S'上的值.在上我们如下引进度量.

设

\textsc{3=\{w,€z,} 函=\{\%\}此2,

则

\begin{quote}
f, d(a;,布 d(s,5J) = 2 --- . (2.2)
\end{quote}

---oo 2

上述定义表明\emph{梦}中两个点接近指的是它们在一个很长的中间段

上一样.下面的引理将要精确地说明这一点.

弓[理2.1 令
«\textsuperscript{(}=(«i\}feZ'\textsuperscript{Si}\^{}\{\textsuperscript{a}ihez\^{}-\textsuperscript{sJV}-'

\begin{enumerate}
\def\labelenumi{(\arabic{enumi})}
\item
  如果 d(sa\textgreater{})\textless{}2T,则傳•=\%,对
\item
  如果 «,=«,,对 HI W点,则 d3,a\textgreater{})W2T+i.
\end{enumerate}

证明结论(1)用反证法证明.设存在花使得勺尹 夺则

这便导出矛盾.

现在证结论(2).如果煲=瓦对\emph{\textbackslash{}i\textbackslash{}\^{}k,}则我们有

,,一、m d(a:,q)\\
火 3,S)=〉 ---

---«o \emph{b}

\includegraphics[width=0.76667in,height=0.34028in]{media/image52.png}

在研究\emph{玲}的结构之前让我们回忆一下有关概念.一个点彙
称为完全的,如果它是闭的且没有孤立点.一个点集称为完全不连
通的,如果它的每一个连通分支只包含一个点.同时具有上面两条
性质的经典例子是Ri上的Cantor三分集.

引理2.2空间玲 在度量(2. 2)下是

(1)紧致的N2)完全不连通的;(3)完全的.

引理2. 2的三个结论恰是一般Cantor集合的定义.

证明(1)令K是中的一个无限集,我们证明K在*
中有聚点.因为\emph{K}包含无穷多个点,故存在\emph{K}的一个无穷子集
K),使得对其中任意序列《>= {\%}強z,\textsuperscript{55} =伉}旭2,有%
=\&。•下 面我们归纳地构造一个无穷子集列\emph{Kf}

令弟是K的一个无穷子集,满足以下性质:对任意0 = {妇旭z,函k
{弓尾z€K,.都有%=初仔\textbar{}@.因为Y包含无穷个
元素,故存在岛的一个无穷子集Kf+i,使得对Kf+i的任意元素 3= (fljJyez-有

■勺=弓,\textbar{}刃+ 1.

这样我们得到\emph{K}的一个无穷子集死K,.,满足

1-对任意 3=
{a\textsubscript{>}(\^{}\textsubscript{€Z},a5=(ay}j\textsubscript{€Z}6K,-有勺=弓,\textbar{}丿\textbar{}
V.

2A+iUK,.,。。.

任意取一点緬6 K,,现在归纳地选取一点列% E 如下:假
设此,纳,\ldots{},総已经取定,取<w\textsubscript{n+1}eK\textsubscript{n+1}使得%+济叫.』=。,1,
•••,\&.令\%=(a"}\textsubscript{;ez}.令 «,=«;,及 3= 0.},wz,则显然
% 收敛到

(2)因为\emph{S}是完全不连通的,而fN是S的乘积空间,故酣 也是完全不连通的.

⑶因为\emph{S\textsuperscript{N}}是紧的,故它是闭的.对任意《•= {\%.}心£
V, 下证a是梦中某点列的极限点.对V 3,取。一盤,令囱=
(砧ez£梦是这样一个点,使得

4=a;,VI W 兀 + 1,弓+2 尹 %+2・

由引理2. 1,奇属于3的£邻域且防d.因此,我们证明了\emph{梦}的极
限点集等于它自身,即梦是完全的.I -

移位映射

定义*到自身的映射。如下:

3=伉}心£潛,

=也雇z,6 = \%+i.

a称为事位映射.

引理2.3移位映射b是甘\^{}到自身的同胚.

证明 显然。是一一的和映上的,故为了证明引理,只需证明 。与矿】是连续的.

令《»= \{\%槌z,a= \{\^{}hez*由定义

。3)= 饱\},wz,缶=\%+i,b(方)=\textsc{\{4"gz,}4 =
\textsuperscript{s}\textsubscript{i+}i*

。-1(3)= \{q'ez,\emph{\textsuperscript{c}i =} i,\textless{}ri(a)=
(c\textsubscript{f}\}\textsubscript{iez}, = q\_i. 因此

,,,、一、、m d(4,矽 再+i)

d(久 3)mO))= \textgreater{} S 771

\begin{quote}
\textsc{e=} ---cc \emph{£i} f \emph{--- oa Li}

Q觉牛* =収3,粉,
\end{quote}

E» ---CC g

dSW),广㈤)=力警=力竺号徉

Q玄竺*° = 2如林I

---a\textgreater{} \emph{L}

令蛇梦,集合

0(«t) = G \emph{Z)}

称为映射。过点\emph{s mat.}当\emph{。(砂是有限集时,仞称为周期点,而}
。(《0称为周期執,此时满足『3=3的最小正数*称为财的周
期.一个非周期点在。的正向及负向迭代下,如果趋于同一个周期
轨,则称它为该周期轨的同宿点;如果分别趋于不同的周期轨,则
称它为该两周期執的异宿点.

定理2.4对■\^{}中的移位映射。,下列结论成立.

(1)存在任意周期的周期点,周期点在*中柄密;.

'(2)任意同期轨的同宿点集在K中稠密,任意两个周期執 的异宿点集在中稠密;

(3)映射。在fN中有.稠密轨道.

证明 先证明(1).首先我们引进一些记号.我们对由S中
的元素枸成的一个周期性重复的双边无穷序列用在其重复段上加
一个横线表示.例如{•••,1,2,1,2,\ldots{}}用{飪}表示.对左(右)向无
穷的周期重复的序列用一个在其重复段上面向左(右)的箭头表
示,例如(-.1.2,1,2}用(如}表示,{1,2,1,2,\ldots{}}用( 12)\^{}.对 w=
{佝}此*孕我们用表示也的长度为2«+1的中间段,即 虹小={四}
EW''显然,中的一个周期性重复的序列是\emph{a}的周
期点.对于任意正整数知点3= (共\emph{k}个数字)是\textless{}7

的周期为\emph{k}的周期点.现在我们来证明周期点在中是处处稠
密的,对V\emph{关梦,}任意绐定e\textgreater{}0,令死\textgreater{}0是这样一个整数,使得
2\textasciitilde{}\textsuperscript{n+1}\textless{}E.由引理2.1周期点{E}属于s的\textsc{e}邻域,故周期点稠
密.

现在征明(2).令«\textless{}= {苛},S= {矿}是两个周期点,这里a*与
U分别是©和而的周期重复部分.对任意\emph{咋奕}任意给定eAO,
正整数``满足2一''+1 Ve,则由引理2.1

\emph{a} = {a\textsuperscript{-}»;(«)\&■}

属于寸的e邻域,而a在。的正向及负向迭代分别趋向于O(s)和 。㈣.

最后我们来证明⑶,脚证明存在一个点岐*,使得对任意 吒寸
及e\textgreater{}0,都有一个整数''满足我们将直接
构造这样的点皿对任意整数如首先构造所有由\emph{S}中元素构成的
长度为\emph{k}的序列.因\emph{S有N}个元素,故这样的序列有\emph{州}个:

{/},{M},\ldots{},{如},

则?是一个长度为\emph{kN\textsuperscript{k}}的序列.现在考虑下列序列

3 = \{・・・次,次,\textless{}?.•・・\}・

这样S包含任意给定长度的所有可能的序列.我们断言8的轨道
在式中稠密事实上,对V 5J£\#,w\textgreater{}0,令正整数''满足不等式
2\textsuperscript{\_n+1}\textless{}£.由3的构造可知,长度为2\&+1的序列洌:'')必在3中的
某一段出现.因此,由引理2.1,存在整数/,使得

出产(《0,函)V\&, 故a的轨道03)在\_gN中稠密.\textbar{}

\textbf{§ 3 Smale} 马蹄

本节首先将描述美国数学家Smale于60年代初给出的一个分
段线性二维映射的例子.这个例子表明一个看上去简单的映射可
以拥有十分丰富的动力行为.它在现代动力学理论的发展•中起到
了非常重要的作用.然后我们将Smale的例子推广到一般的非线
性映射的情况,给出所谓马蹄存在定理.最后我们将看到马蹄可以
在三维向量场的Poincart映射中出现•

马蹄映射的例子

考虑If上的单位正方形D={[}0,1{]}X{[}0, 口.我们引进D到
\textbf{R'}的微分同胚\emph{f}如下.貫先将\emph{D}沿垂直方向拉伸5倍沿水平方向
压缩5倍,然后将所得到的细高矩形在中部弯曲得到马蹄形区域,
最后将马蹄形区域与\emph{D}按图4-4的方式相交\emph{.fDQ£\textgreater{}}是两个高度
为1宽为§的矩形而是两个高为*宽为1的
矩形\emph{死=厂'Vi,t=}1,2.映射y在上是线性映射,并具有 {[}±1 0

Jacobi矩阵 5 ,它在Hi上取正值,在上取负值.

\begin{quote}
.0 ±5.
\end{quote}

\includegraphics[width=1.55347in,height=1.95972in]{media/image53.png}

4-4

现在我们考虑\emph{D}中的所有在\emph{f}的任意次迭代下都不离开\emph{D}
的点集,即

A= \{次 R\textsuperscript{2} l/x 6 i € Z\}.

不难看出

\emph{A = hfD}

---8

是\emph{f}的一个不变集,即\emph{fA=A.}首先我们说明A非空.

U珞是两个宽为1高为"I■的矩形.而广叨=广\textbar{}刷U广'払是四
个宽为1高为*的矩形.用归纳法,一般地可得到

厂也(=/■-"+'', «e \textsc{n,}

且\emph{r\textsuperscript{n}D}是2•个宽为1高为5一''的矩形.因此,集合5厂"0是一
0

个工轴上[0,1]区间与力轴上[0,1]区间内的一个Cantor集K,的 积,即

P/-\/-D = {[}0,1{]} X K

类似地可证明存在z袖上[0.1]区间内的一个Cantor集K''使得

\emph{\^{}\textbackslash{}f\textsuperscript{n}D = K\textsubscript{x}} X
[0,11

因而

\emph{A = K\textsubscript{x}XK\textsubscript{y}}

是一个非空Cantor集.

下面我们描述7■在厶上的动力行为.因为4是7■的不变集, 故对丫 \emph{z\&A}

\emph{fx} € A, V « e Z.

注意SfMUHi UH''因此存在一个由1,2组成的无限双边序列\emph{项} EZ,使得

\emph{\textsc{Hq.}}

\emph{i}

这样我们定义了一个映射

\begin{quote}
z I-* @(z) = («;}\textsubscript{iez}.
\end{quote}

令\emph{。}代表上节中我们定义的习上的右移位映射,则根据定义有

\emph{鱼。f ■= G。}@辰

下面我们将说明企是厶到习的间胚,即企是连续的,一一的和映
上的.首先我们说明企是连续的.设Hi与之间的距离为E,则
对任意两点\emph{a,b€:A,}若d(a』)VE,有\emph{a,b}同时属于反i或\emph{H\textsubscript{2}.
f} 在及Hz上是一个水平方向压缩5倍,垂直方向拉伸5倍的线性
映射,故对\emph{a,b\^{}A,}若

5妇心)VE/\&N,

则有\emph{fa,fb\^{}\textbackslash{}i\textbackslash{}\^{}k}同时属于Hi或故序列在
长度为24+1的中间段是一样的

{臥办}卩心=仲0),w

只要取\emph{d5)}小/便可以充分的大,由引理2.1,企食)与企⑶在
呂中的距离可以充分的小,这便说明了企是连续的.

对一点uGA,我们用呛分别代表``的c与;y坐标.

\begin{quote}
现在我但来证明@是一一的,即对不同的点\emph{a,b\&A,} ©(a)尹 @(6).
\end{quote}

用反证法,设

\begin{quote}
@(a) = @(\&) = {勺}话Z,
\end{quote}

即

\begin{quote}
\emph{.fa\textsubscript{t}fb € H\textsubscript{Ci}, i} € Z,
\end{quote}

故对芷Z ■

1 \textgreater{} I\emph{-顔吼} I = 5''% -妇,

1 \textgreater{} I---(厂'糾I = 5'厄一妃,

因此

勾=勺,=如,

这与\emph{a\#b}矛盾,

最后为了证明企是同胚,还需证明企是映上的,即对任意

都有一点\emph{a€:A,}使得@伝)=纵

我们将它作为练习留给读者.

通过上面的讨论我们得到\emph{f\textbackslash{}\textsubscript{A}}与。序是拓扑共辄的.因而根
据定理2. 4我们有

定理3.1马蹄映射/■在D中有一个不变的Cantor集4,满 足 ,

\begin{enumerate}
\def\labelenumi{(\arabic{enumi})}
\item
  A中有尸的任意周期的周期点;周期点在厶中稠密;
\item
  任意两个周期轨道的异宿点及任意周期轨道的同宿点在 人中稠密,
\item
  /在4中有稠密轨道.
\end{enumerate}

马蹄存在定理

下而我们将定理3.1推广到非线性情形.

定理3. 2令所,\%是两个正常数满足所•光VL令以UR】

XR项=1,2,\ldots{},N是N个两两不交的(物,佐)矩形,映射/:D=
UD.-\/-R\textsuperscript{{]}}XR\textsuperscript{{]}}是一个微分同胚,并且

i=i

\begin{enumerate}
\def\labelenumi{(\arabic{enumi})}
\item
  /在D上满足(的,%)锥形条件,
\item
  相交条件成立,7)/0巾;老0,,\ldots{},N;
\item
  边界条件成31 \emph{zfDj}Q3.\^{}=0,/O\textsubscript{A}Dp
  Q\emph{D\textasciitilde{}}0, \emph{i,j=}
\end{enumerate}

1,2,\ldots{},M 则集合

\emph{OQ}

s n尸\textsc{d}

\emph{X= ---8}

是一个不变的Cantor*./限制在A上的映射f \textbar{}丄拓扑共辄于
\emph{沙}上的移位映射°,即存在厶到寸的一个同胚金,满足

。。。刃4. (3-1)

由定理2.4我们有

推论3.3定理3. 2中的映射\emph{f}有一个不变的Cantor巢A,满 足

\begin{enumerate}
\def\labelenumi{(\arabic{enumi})}
\item
  厶包含任意周期的周期点,周期点在厶中稠密,
\item
  周期轨道的同宿点及异宿点在A中稠密?
\item
  /在A中有稠密轨道.
\end{enumerate}

定理3. 2证明的思路

给定3= {\%}心我们定义有限或单边无限序列如下:\\
=仏}0W" = {aJ\_*W。,

\begin{quote}
s+= 区}爲0, 虹={\%}《0・
\end{quote}

如果\emph{xED}满足 Aen\textsubscript{fl}, 则我们记作

\begin{quote}
I
\end{quote}

遂(z)= \textless{}.

如果xenw则我们记作

\begin{quote}
---I

也愆)=练・

D(\textless{}) = U e = \textless{}\}, (3.2)

\emph{DCa\textgreater{}\textasciitilde{})} = \{工 e \emph{=
to\textasciitilde{}\}.} (3. 3)
\end{quote}

第一步证明IX诺)是〈隔,爲)矩形,且满足

DO:) UD(蜡一1),

以及

財海)u aR\&g) u\emph{衲``}

第二步证明

p(it»\textsuperscript{+})-I--- nD(s:)

是一条端点属于泣么的内水平ai \^{}

是一条端点届于初\%的%垂直备线,

第三步记

。(《\textgreater{}) = D(a»\textsuperscript{+}) (1
\emph{D(a\textgreater{}\textasciitilde{}).}

证明映射

\emph{\textless{}p\textsubscript{:}S\textsuperscript{N} -*■}
A,a\textgreater{} H* (\&(O\textgreater{})

是一个同胚,满足\emph{时=A.}

第四步令◎=\textless{}\}\textgreater{}"证明等式(3. D成立.

定理3. 2的证明 令

笑=\{h £ = \{\%T+1,・・,%\} \},

其中f="2L・M+l.我们断言\emph{左s} 是一个(处,气)矩形,

\begin{quote}
---i+l
\end{quote}

满足

U---\textless{}+1

事实上,由为的定义,我们有xe\^{}\textsubscript{+1}当且仅当处£驾及
£\textgreater{}\textsubscript{a} ,因此

乌+1=广1(乌0/\^{}1.). (3-4)

下面用归纳法证明我们的断言.对\emph{i=l\textsubscript{t}A\^{}=D\textsubscript{a}},断言自然成立.

\begin{quote}
H
\end{quote}

现假设对\emph{i=j}断言成立,即A:U以1+1是一个(用,%)矩形,满足
材;匸初皿心.由(3.4),再利用引理L 8,我们的断言对A;+I也 成立.

注意到尤+1=1)(磅),于是我们有

\begin{quote}
引理3. 4。(虹匸气是一个\^{}砂心矩形,满足 村(虹)次%.
\end{quote}

类似地我们有

引S3. 5 D3QUD\%是一个(珞,珞)矩形,满足

\emph{apgua}

\begin{quote}
关于(件,阳)矩形o\textless{}\textless{})的高和宽有
引理3.6存在一个正常数C,使得
\end{quote}

A(D(a»+)) \textless{}CA-"\textsubscript{(}
w(D(a\textgreater{};)) VGT", '' £ N,

这里艇•),\&(•)分别表示(险,阳)矩形的高和宽,\emph{而心\textgreater{}1是(位,}
從)锥形条件中的當数•

引理3. 6的证明与引理L II的证明几乎完全一样,我们把它留 给读者•

根据D(妒)的定义,有

D(咁)UD(屹I).

因而

£)(好)亠初(坤)

U=O

非空•由引理3. 4和引理3. 6,D(s+)是一条端点属于神%的内水
平曲线.而引理3. 5和引理\$ 6表明以『)是一条端点属于\emph{和气}
的件垂直曲线,由引理1.
7,D(«\textgreater{}+)与D(«T)有唯一的交点,记作

\begin{quote}
D D(«r)・
\end{quote}

这样我们得到一个映射pW-Aa i扒oO.

下面我们证明©是一个同胚,为此只需验证。是一一的、映上 的和连续的.

\begin{enumerate}
\def\labelenumi{\arabic{enumi}.}
\item
  0是一一的.这指的是对任意给定如果s尹瓦 则\emph{中3)我C}
  用反证法证明.若不然,假设\emph{瞄=坤)=屿} 4.设
  Aeo\textsubscript{fl},iez,w
\end{enumerate}

\begin{quote}
I
\end{quote}

\emph{3 = \&=} \{㈤\}fez・

这与4U尹而矛盾.

\begin{enumerate}
\def\labelenumi{\arabic{enumi}.}
\setcounter{enumi}{1}
\item
  \begin{quote}
  。是映上的.这指的是对任意给定xEA,都存在一个点@ 使得步3)=\#.因为xEA,故
  \end{quote}
\end{enumerate}

A e \emph{\textsc{d,}} v \emph{i e} z.

假设 \emph{fx\&D\textsubscript{ai}},»ez.令 3= \{q辰由(3. 2)和(3. 3),

\emph{x} e n(\textless{})n \textsc{d(\%),} VYN.

因此

8

xen\textsubscript{o}(\textsuperscript{D}\textless{}\textless{}\textgreater{}
n \textsc{d3:))= d(s+)}n \emph{\textsc{d〈c} = \textless{}pg}

\begin{enumerate}
\def\labelenumi{\arabic{enumi}.}
\setcounter{enumi}{2}
\item
  步是连续的.这指的是任\emph{给蛇梦,40,}都可找到正数d, 使得对任意
\end{enumerate}

商=\{q.ez e £``,

若\emph{d(.a\textgreater{},\^{}\textless{}3,}则 d(Q(s)仲(a))O
由引理2.1,如果 d(w,a»)\textless{}5,
则4=为,对\textbar{}f\textbar{}WN(S).这里N(S)---+8,当6---0.故由引理3. 6
有

d") *)) \textless{} 2CA* V e,当 S《1.

令 ,现在我们来证明(3.1)成立.令xEA是一

个点,设

\emph{fx} e \emph{D\textsubscript{a} , i} e z, (3.5)

则由©T'得 ,

B(Z)=3= {\%}沱 *

故

°。叡'' =b(<u)=㈤此z,q = a;+i・

另一方面,由(3. 5)得

\begin{quote}
\textbf{MZ,}
\end{quote}

这意味着

①。/(x) = {爲}心,

因而

° ® = (P ° /1 A- 定理证毕, I

\protect\hypertarget{bookmark210}{}{}\textbf{§4}线性映射的箋合映射的双曲性

本节将给岀关于两个线性映射的复合映射的双曲性的一个重
要引理,这一引理将在第五章中多次应用.

问题的提出

第五章将斯究奇点同宿轨的分岔,这一问题的解决是通过研
究同宿軌的Poincare映射得到的.这时,Poincart映射一般可表示
为两个映射的复合,一个映射由奇点的邻域中的向量场决定,另一
个映射由奇点邻域外的向量场决定,第L个映射根据正规形理论
可以得到它的精确表达式.它具有很强的双曲性浩定义域取得充
分小时,它沿着稳定流形方向的压缩常数和沿着不稳定流形方向
的扩张常数分别充分小和充分大,我们对第二个映射除了知道它
是一个微分同胚外,凡乎没有其它任何信息.我们的目标是,证明
在一定条件下,第一个映射的强双曲性可以保证Poincare映射的
双曲性.由锥形条件可看出,一个映射的双曲性是由其映射的导算

子体现出来的,而由链锁法则,两个映射的复合映射的导算子是由
两个映射的导算子的积给出的.因而本节我们将对两个线性映射
的复合映射的双曲性给出一个判别引理.

锥形条件和引理的陈述

令处,%是两个正常数,满足内缶VI.令

K+=((广,C+)ER\textbar{}\textgreater{}\textless{}R】\textbar{}欧丨法必\textbar{}厂丨\},\\
\emph{K} = ((广,广)6 R' XR屮尸丨M见广I\},\\
\emph{靜=}((广4+) e R' X R屮广丨法用广\textbar{} \}.

我们称一个可逆线性映射A'ItXRifRiXIt满足(由,知)锥形条 件,如果

(1) \emph{A\^{}+UK+;}

⑵ l(4C+\textbar{}NA\textbar{}u+\textbar{},y。=(广4+)\&K+,

\begin{quote}
(3). i(A-'?)-\textbar{}\textgreater{}Air i,v 了 =(,-,r•沱
K\textbackslash{} 这里尤\textgreater{}1是一个常数.
\end{quote}

因为*是\emph{K\textasciitilde{}\textbackslash{}W}的余集,故条件(1)可推出4-*-UK-.

\includegraphics[width=1.46667in,height=2.00694in]{media/image54.png}

引理4.1考虑R,XRL到自身的两个可逆线性算子

\textbf{H={[}: \}} \emph{\textsc{\textsuperscript{J}={[}d m{]}}}

満足\emph{d・M也}令乙\textgreater{}0是一个常数,使得

\textbar{}\textbar{}H\textbar{}\textbar{}V 乙,\textbar{}\textbar{}矿却
V 乙, (4.1)

\emph{\textbackslash{}d\textasciitilde{}'\textbackslash{}\textless{}L,}
(4.2)

(4.3) 则对任意存在一个依赖于£,内,%的正常数凯 如果下列不等式

\begin{quote}
IM-\textsuperscript{1}! \emph{\textless{}S,} (4.4)

\emph{\textbackslash{}A-BM\^{}D\textbackslash{} \textless{}8,} (4.5)

\emph{\textbackslash{}AM\textasciitilde{}' \textbackslash{}
\textless{}3,} (4.6)

IDA/-\textsuperscript{1}1 \textless{}5, (4.7)

IcBM-\textsuperscript{1}! \emph{\textless{}3} (4.8)
\end{quote}

成立,则线性映射A=町满足(払,%)锥形条件.

我们将在不等式

応'\textgreater{} \% \textgreater{} max\{2L\textsuperscript{2} + 22? +
1,2\textless{}L + £\textsuperscript{2})(L + £\textsuperscript{3})\}

(4. 9)

成立的前题下,分三步证明引理.

不稳定備形区域内的扩张性

令

C=(L4+)€K+, 〃=(阳,时)

对满足(4. 9)的常数旳,%我们将证明

\textbar{}?\textsuperscript{+} I
\textgreater{}A\textbar{}f\textsuperscript{+} I, 对 (4-10)

因为

?\textsuperscript{+} = (M + \emph{dD)Z\textasciitilde{} + (.cB + ,}

由三角不等式

I俨 I n \emph{\textbackslash{}dMz\textsuperscript{+}}1 一
\textbar{}\textsubscript{c}sr \textbar{} - w \textbar{} ---
\emph{\textbackslash{}diK\textasciitilde{}} I.

(4.11)

由(4. 2) '■

\emph{\textbackslash{}dM\^{}\textsuperscript{+}} \textless{}4.12)

由(4.8)

\begin{quote}
丨 \textless{} \textbar{}c\textbar{}\textbar{}\_BAfT\textbar{} .
\emph{\textbackslash{}M\^{}} I 广 \textbar{}. (4.13)
\end{quote}

由R+的定义

I 依-I £ 百1 ⑷• ir I w外1\textbar{} 仙 T\textbar{} • IMO I,

I m应\textsc{''di} • ir • iA/r i.

这些不等式与(4. 6)和(4. 7) 一起推出

\begin{longtable}[]{@{}lll@{}}
\toprule
\endhead
\begin{minipage}[t]{0.30\columnwidth}\raggedright
\strut
\end{minipage} & \begin{minipage}[t]{0.30\columnwidth}\raggedright
\textbar{}頌丨5听网+ \textbar{},\strut
\end{minipage} & \begin{minipage}[t]{0.30\columnwidth}\raggedright
.(4.14)

(4.15)\strut
\end{minipage}\tabularnewline
由(4.1) & SI + 賀俱+ & (4. 16)\tabularnewline
由(4.111,(4.12),(4.13)和(4.16)我们得 & &\tabularnewline
1旷◎ & 『-队1 +制.g )\textgreater{}*"\textbar{}姓+ 1 &\tabularnewline
& & (4-17)\tabularnewline
\bottomrule
\end{longtable}

对充分小的\emph{3}成立.再由(4. 4)

\textbar{}Mr 丨2矿%+.{[},

这与(4.17)一起推出(4.10).

锥形区城的不变性

现在证明

▲X+UK+. (4.18)

设£=(广,尸沱民+,咛3「,时)=鸟,由定义欧IN如广I.我 们将证明

1?\textsuperscript{+} I I- (4-19)

不等式(4.10)给出了
\textbar{}旷\textbar{}的下界估计,下面我们给出\textbar{}厂\textbar{}的上界

估计.由定义

\begin{quote}
i?-= (M + \&D)厂 + \emph{(aB + bM)\^{}\textsuperscript{+}.}
\end{quote}

由三角不等式

\textbar{}\textsubscript{7}- \textbar{} CI\^{}IIAT I +
\emph{\textbackslash{}bm\textasciitilde{}} I + \textbar{}a\textbar{}
\textbar{}B§+ I +
\emph{\textbackslash{}b\textbackslash{}\textbackslash{}M\^{}\textsuperscript{+}}

(4.20) 由(4.1), \textbar{}a\textbar{},\textbar{}3\textbar{} VZ
,故由(4.14),(4.15)有

\textbar{}\^{}\textbar{}\textbar{}\textless{}- I +
\textbar{}6\textbar{}\textbar{}Df* \emph{\textbackslash{}
\^{}2L\^{}'6\textbackslash{}M\^{}\textsuperscript{+} \textbackslash{}.}
(4.21)

进一步由(4. 3)

\textbar{}战+ 丨 M IBM-\textsuperscript{1}! {[}A/r I
\emph{\textless{}L\textbackslash{}M\^{} \textbackslash{}.}

因此由(4.1)

lall-Sr I +
\emph{\textbackslash{}b\textbackslash{}\textbackslash{}M\^{}} I
\textless{} \emph{(L\textsuperscript{l}+L)IM\^{}+} I, (4.22) 将(4.
21)和(4. 22)代入(4. 20)得

\begin{quote}
IT \textbar{}M〈2``a + Z?+\_L)g \textbar{}.
\end{quote}

这与(4.17)-起得

\begin{quote}
I厂I \^{}2乙(2乙成蜕+ 〃 +心\textbar{}旷\textbar{}.
\end{quote}

由(4.
9),上式中右边的\textbar{}俨\textbar{}前面的因子当d充分小时小于心,这

便证明 了 (4.19),即(4.18).

稳定锥形区域内的压缩性

令

\emph{?=(旷,旷)£ K-,} ? =(r\textless{}r)= \textsc{a-'?.}

我们将证明不等式

ir u 对■?\textless{}!.

令 \_ \_

则.

\begin{quote}
\emph{A-\textsuperscript{1} = A -}

\emph{c = --- d\textasciitilde{}'c a,} A-'B =-
\emph{BM\textasciitilde{}',} 矿'\emph{=a ---
bd\textasciitilde{}\textsuperscript{1}c.}

\emph{\textbackslash{}a\textasciitilde{}' \textbackslash{}} \textless{}
\textbar{}a\textbar{} + \emph{\textbackslash{}bd\textasciitilde{}'c\{}
\textless{} L + L\textsuperscript{1}. (4. 26)
\end{quote}

由(4.3) '

= \textbar{}BM-* I \emph{\textless{}LL.} (4.27)

现在我们证明〈4. 23).注意到我们有

,-=(及 + \emph{Sc)r +} \textsc{\textless{}a5} + S3)时.

因此

ir I \textgreater{} \textbar{}才\textbar{}{[}\textbar{}矿I \_ ik网-1 \_
\textsc{im} I \_ \textbar{}玲孩时门.

(4. 28)

由(4.26)

\emph{\textbackslash{}ar} \textbar{}\textgreater{}(L +
L\textsuperscript{s})-')\textsubscript{?}- {]}. (4. 29)

由(4. 27),(4. 25)和(4.8),有

\emph{{[}A-'Bcr} I才t剧团团

\begin{quote}
\textless{}L\textsuperscript{2}\textbar{}BM-\textsuperscript{1}{]}\textbar{}c\textbar{}\textbar{}7-
I \emph{a.} 30) £乙%I厂j.
\end{quote}

由(4.1)

{]}\^{}\textsuperscript{+} I \textless{}L{[}?\textsuperscript{+}
\textbar{}. (4.31)

由(4.27)和(4.1)

\emph{{[}A-\^{}Bd\^{}} 丨 M \textbar{}3-'B\textbar{}
\emph{\textbackslash{}3\textbackslash{}}\textbar{}7+ 丨 \textbar{}.
(4.32)

注意到 \emph{代K\textbackslash{}}由(4. 31)和(4. 32)得

\textbar{}切+ {]} + 0753时 \textbar{} VkCL + *)/ \textbar{}. (4. 33)

由(4. 24),(4. 29),(4. 30)和(4. 33)得到(4. 28)右端的下界估计.
因此

\textbar{}厂 \textbar{}法尸+ + 叮 1广 I.

(4. 34) 由(4.9)

\emph{(L +} 乙\textsc{3)t} \emph{-CL +} \textbf{+ ")T.}

因*当\emph{6}充分小时,由(4. 34)可推出(4. 23).引理的证明完成.I

\protect\hypertarget{bookmark225}{}{}\textbf{§ 5 Birkhoff-Smale} 定理

本节我们利用§ 2- § 4所给出的结果证明R,中的Birkhoff- Smale 定理.

定理的陈述

令bUR,是向量场\emph{X}的一条双曲周期轨,设人,1"是它的三
个特征指数,满足\textbar{}A\textbar{}\textless{}Klp\textbar{}.这样的周期轨有二维稳定流形
巧,(。)和二维不稳定流形\emph{W-M.}向量场\emph{X}的一条轨道y称为周
期轨b的同宿轨,如果万\&且7UM3) 1吁5)・进一步,同宿
轨y称为植截的,如果流形与\emph{w«w}沿着轨道/横截相交, 见图4-6.

定理5.1 (Birkhoff-Smale定理)令X是R,上的一个C°°光
滑向量场,。是X的一条双曲周期轨.如果,是周期轨b的一条
横截同宿轨,则X在出,的任意邻域里有无穷多条双曲周期轨.

事实上,我们将证明比定理5.1更强的结论:在的任意邻
域里都可以构造一个存在马蹄的后继映射.我们将证明分几步进 行. • .

周期轨道的\textbf{Poincart}映射

令\emph{S}是一个与。横截相交于0点的平面,则在S上0点的邻

\includegraphics[width=1.94028in,height=1.57986in]{media/image55.png}

图4-6

域可以定义Poincart映射.由于RZ上的微分同胚在它的双画不动
点处可以U线性化(参见第一章定理4. 22),故在S上O点的邻域
可以选取一■个坐标系,使得

(1)0点是坐标原点,

⑵ PoincarS 映射 P 是线性的,\emph{P(.x,y) = (.kx,ny'),}

这M\textbar{}A)\textless{}1\textless{}
I产I.直线去=0和分别对应局部稳定流形和局
部不稳定流形.令两点分别属于局部稳定流形和局部
不稳定流形,作一个尺度变换便可假设/\textgreater{}=(1,0)四=(0,1),令

B = \{9,少卩0■ ---11 V\textbar{}刃 M 禺\},

\begin{quote}
菖=\{(工,丫)\}闵 MM "
\end{quote}

分别表示点夕与q的一个邻域,当畐,毎充分小时,有

P8r)B = 0,P-'AnA=0. (5.2)

令p\textsubscript{n}=p-"snB表示\emph{\textsc{b}}中所有在\emph{p"}作用下映到2的那些点所
构成的集合,当n充分大时以是B中一个高度很小而与\emph{B}等宽 的矩形.由(5.
2)有

0 = 0. \emph{n\^{}m.} (5. 3)

再令是百中一个宽度很小而与S等高的矩形,

\includegraphics[width=1.8in,height=1.7in]{media/image56.png}

图4-7

见图4-7. ■

同宿轨at的后继映射

因为同宿轨y连结P,g两点,故当畐,易充分小时,系统X的
每一条从S出发的正半轨将与S上力的一个邻域\emph{U}相交于一点,
我们用F表示这一对应

\emph{F.} (5.4)

由常微分方程的解对初值的光滑依赖性知,F是b微分同胚.令 F(Z,少=
(Px,Fy).由于稳定流形与不稳定流形沿y横截相交, 故有

\begin{quote}
\emph{3Fv}

---乂 0. (5. 5)
\end{quote}

\emph{3y ,}

令

\emph{B\textsubscript{h} = B {[})} 3 = 0\},瓦=3 n(X = 0\}.
取定为充分小,使得

\begin{quote}
\emph{F珥} D Sft = \emph{\{p\}f} (5.6)

凱攻 \textsuperscript{\textless{}5}-\textsuperscript{7)}

F\^{}riaBL。, (5.8)
\end{quote}

唤昭=0, (5.9)

这里 \emph{aB尸\{\^{}---l±S\textsubscript{l},y=0\}} ,3乱=\{z=0,y =
l士\&\}.由(5. 7)--- (5. 9)可以取定毎充分小,使得

賈L尹。, (5.10)

\begin{quote}
\emph{FE n 為B = 0,} (5.11)
\end{quote}

\emph{, Fd\textsubscript{h}S} n B = 0, (5.12)

这里

\emph{3\textsubscript{V}B = € B\textbackslash{}x} = 1 ± 昂\},

祝骑=\{(工,夕)e舌\textbar{}\_y = 1 ±\&\}.

现在我们定义后继映射厶:£)「*{[}/如下:

\emph{A'' = F。已}

在集合卩玖,上定义后继映射/■为

月% = 1

下面将证明对所有充分大的整数在区域玖,UQ+i上是一个
马蹄映射.为此,我们将逐条验证/满足定理3. 2的所有条件.\emph{D„}
是矩形.它的边界的水平及垂直部分分别是

電以={愆,丿)丨B --- 1丨V - 1\textbar{} =晶} 和

\emph{3\textsubscript{v}D\textsubscript{a}} = {(x,3-) \textbar{}
\textbar{}x --- 11 = \& , I心一 11 M "

引理5. 2对任意正数岡都存在正整数N,使得当
\&\textgreater{}N时J-满足(払,%)锥形条件,并且对\emph{i,j\textgreater{}N}有

\begin{quote}
\emph{n}
\end{quote}

(1) 边界条件成立:為以nq=。,/D,n\^{}=0\textsubscript{f}

(2)
相交条件成立\textsubscript{!}/D,.no\textsubscript{\textgreater{}}\^{}0.

证明 首先我们利用引理4.1证明满足锥形条件.根据链

\begin{quote}
\emph{Df\textbackslash{}\textsubscript{D} =DF-DP"}
\end{quote}

这里 D=B=O, A-A", \emph{M=f,}令

\includegraphics[width=0.57986in,height=0.32639in]{media/image57.png}\includegraphics[width=0.57361in,height=0.34653in]{media/image58.png}

由(5-10)及\emph{F}是一个b微分同胚,我们有

\emph{L} V+ 8.

令。是引理4.1中所确定的常数,注意到\emph{B=D=0,}

\textbar{}厶\textbar{} = W\textbar{} f 0,成7\textbar{} =
\textbar{}厂''\textbar{} 〜0,当 ''\textasciitilde{}+8, 对充分大的(4.
3)---(4. 8)显然成立.

下面验证边界条件.注意到

\emph{DjUB,} PquE,戸(次 Z\textgreater{}f) uaq,\emph{礼Dj U 礼B}
以及(5.11)和(5.12),便知边界条件成立.

最后验证相交条件.在以•中有两族直线,即所谓水平直线族 和垂直直线族:

«(\_\textgreater{}) = B\textsubscript{A} X ⑶\},r G (R)。,\\
q(z) = \{苗 X (£\textgreater{}A,z G \emph{B\textsubscript{k}.}

这里(玖),=\{ \textbar{}止y---l \textbar{}M\&\}.注意到

G.3)--- --- 及,当Z---+8,

再利用(5. 6)得到,当\emph{i,j}充分大时,曲线
与C£(j\textgreater{})在\emph{p}点

附近相交于唯一一点.因此

\emph{fDj} n Dj 尹 0.

引理证毕.I

最后,由引理5. 2及定理3. 2可推出定理5.1.

与本章§1, §3和§ 5中相应的在高维(`` \textgreater{} 3)情况下的结
果,可.参见[Wi2]和[Sill].

\protect\hypertarget{bookmark236}{}{}第五章空间中双曲鞍点的同宿分岔

本章将考虑空间R'中鞍点的同宿分岔.在§ 1中将讨论特征
根都为实数的鞍点的同宿分岔•在§ 2中将讨论有复特征根的鞍点 的同宿分岔.在§
3中将讨论由一个奇点和一条闭轨以及连结它们
的两条异循轨所组成的环的分岔.通过本章的学习,读者将会对如
何利用奇\&或不动点附近的线性化理论(见第一章§ 4)来研究非
同部分岔问题有初步的了解.

\protect\hypertarget{bookmark239}{}{}\textbf{§
1}具有三个实特征值的鞍点的同宿分岔

本节介绍的结果是平面上相应结果(见第二章§4)在空间中 的推广.

周期软at的产生

假设R3中一光滑向量场有\textasciitilde{}个特征值都为实数的双曲鞍点
及其同宿孰.我们考虑这样一个向量场在一般的单参数扰动下所
能发生的分岔.不失一般性,我们总可以认为鞍点有两个负特征值
和一个正特征值(否则考虑其时间反向系统).这样的鞍点具有二
维的稳定流形和一维的不稳定流形.我们称最大的负特征值与正
特征值之和为粽点量.

定理L1令Xe是R3中一个一般的单参数向量场族.设当参
数e=o时,向量场X。有一个鞍点0,它具有两个负特征值和一个
正特征值,并且鞍点\emph{0}有一条同宿轨则存在7U0的邻域\emph{U}和
参数空间中e=0的邻域V,使得当参数e位于V中零值的某一侧
时,向量场Xe在。中有唯丁f条双曲周期轨,当L0.进
一步,如果鞍点量为正,则乙有二維的稳定流形和二维的不稳定

流形.如果鞍点量为负,则4是稳定的;而当参数£位于\emph{V}中零值
另一侧时,X°在U中没有周期轨.

一般性假设

在证明定理之前,我们首先解释定理陈述中``一般''一词的含
义,即所给出的单参数向量场族Xe应满足下面四条假设,其中前
三条是针对向量场X。的,最后一条是对族\emph{X\textsubscript{e}}本身的.

(1)鞍点。的特征值两两不相同且为非共振.

此假设意味着向量场X。在点\emph{0}邻域内光滑等价于向量场\emph{x«}
在点\emph{o}处的线性部分(参见第一章定理4. 18),故在线性化坐标系
下,稳定流形是一个平面,而不稳定流形是一条直线.在稳定平面
上,除了奇点\emph{0}和一条通过O点的直线外,所有轨道当1+8时
都沿着对应较大负特征值的特征方向趋于\emph{O}点,这一方向称为主 稳定方向.

以)当\$f + 8时,同宿轨沿主稳定方向趋于。点.

下一条假设是基于如下的事实:对应两个较大特征值的特征
向量张成一个不变平面",这一不变平面沿着同宿轨延伸.

(3) 流形归与鞍点。的稳定流形沿着同宿轨横截相交,见图 .5-1.

(4) 当参数通过零值时,同宿轨以横截方式产生和消失(确切
定义将在下面给出).

定理\textbf{L1}证明的愚路

我们将通过研究向量场在同宿轨附近定义的Poincare跳射来
证明定理.为此,在鞍点邻域内可以选择两个与同宿轨横截的曲面
「+与其中以「+上的点为初值的正半轨道进入邻域,而以\emph{r\textasciitilde{}}
上的点为初值的正半轨离开邻域.首先,定义映射厶
这一映射定义在「+上的某个区域,它将尸上的一点映到以该点
为初值的正半轨与「一的第一个交点,假设(D使我们可以把向量
场在。点的邻域里看成是线性的,因而映射厶'舛可以精确地用初

\includegraphics[width=2.98681in,height=2.77361in]{media/image59.png}

,B05-1

等函数表示出来.当鞍点量分别是正和负时,厶:皿分别具有强双
曲性和强压缩性.另一方面,因为同宿轨分别与「+和「「相交,因
而沿轨道我们可以定义从同宿轨与的交点的邻域到同宿轨与
L的交点的邻域的映射,并将它记作厶普,见图5-2,由解对初值及
参数的依赖性定理可知,厶尸是一个光滑依赖于参数e的微分同
胚・于是,同宿轨的Poincart映射可定义为

4 = △罕。兮.

最后,我们分别利用双曲不动点定理和压缩映象原理,讨论当鞍点
量为正和为负时,映射厶的不动点的存在性.

映射厶普的定义和性质

由假设(1),利用第一章定理4. 27,在鞍点\emph{0}的一个邻域里,

\includegraphics[width=2.72639in,height=2.56667in]{media/image60.png}

图5-2

我们可选择一个坐标系,使得族X.具有形式

\emph{x =}

\emph{' y} 为(0) VAJO) VOV\#(0). (1-1)

\emph{z} == ju(£)z,

令

r\textsuperscript{+}= u = 1, \textbar{}yl \textless{} 1, \textless{}
1\},

\emph{r\textasciitilde{}=} = 1,1钊
\textsc{\textless{}i,}\textbar{}\textgreater{}\textbar{} M i\}.

由假设(2),我们可以认为同宿轨与L相交于一点色=0,外,0),
\textbar{}由\textless{}1.不失一般性,我们还可假设,0点邻域中正z轴是同宿轨
的一部分,这样,同宿轨与广相交于点如=S,O,1).下面我们来
计算笙叫系统(1.1)的以为初值的解为

= x exp(A{]} (e)z),

・贝,)=\emph{y} expQ(€)£), (1. 2)

z(£) = \emph{z} exp(jU(£)f),

注意到在r±\^{}=h因此由等式

\emph{1 = z} exp (声(QQ

可解出轨道从r\textsuperscript{+} n
\{z\textgreater{}0\}上一点出发走到「-所需时间

ln«

---而

然后代入(L 2)中前两式,并注意到在「+上z=l,便可得到 攵%O,Z)f
\textless{}y ,z,)= (*,*, (1-3)

这里 a=---A(e)/产(e) = /S.

映射\emph{骨}的定义和性质

因为当e=0时,同宿轨连结上的点条=(0,0,1)与「+上的
点勤=(1,少,0),由解对初值与参数的依赖性可知,存在r-±\textsubscript{9o}
的邻域Q,使得对于所有充分小的参数e,Xc从Q中的点出发的正
半轨与L交于戊附近一点・我们将这一对应关系记作

\begin{quote}
骨:QU
\end{quote}

映射A尸是一个光滑依赖于e的微分同胚.令

w(g。)= (Y(e),Z(e)),(Y(0),Z(0))=(丸,0).

不难看出,(y(e),z(\textsubscript{e}))是鞍点的不稳定流形与「+的交点,面
Z(e)是该交点到稳定流形的距离.假设(4)的精确含义是

£z(e)\textbar{}g。\#:。. (1.4)

作参数变换Z(e)=产,然后把``和y(Z-r頒))仍记为£和y(e),便 有

=(Y(e),§). (1.5)

由(1.3)

gg(\_y,z)l 稣=0(1 湖 f 0,当 \emph{hf、}

故当\emph{h}充分小时,对必然属于 e 的定义
域Q.这样在\emph{n\textsubscript{h}}上,我们可以定义Poincare映射如下:

浦定周期轨的产生

现在我们对鞍点量为负的情况来证明定理L 1.

\begin{quote}
因为鞍点量\textless{}7=馬(0)+"(0)VO,所以 \#=-芸\^{} \textgreater{}1,对
l«d《L
\end{quote}

由(1.3)

浏板 \textsubscript{(1}.\textsubscript{6)}

= 0(1)尸 T ---0,当五 f 0.

注意到 缨 是一个光滑依赖于参数=的微分同眨,故它的导算子
关于参数一致有界.因此,(1.6)意味着

II以当 A-0.

特别地,当h充分小,41\%是一个具有压缩常数+的耳缩映射.
因此,4在\emph{血}上至多有一个不动点.下面我们分别对情况e\textgreater{}0和
6\textless{}0来讨论映射\&的不动点的存在性.

(1) e\textgreater{}0.

对任意点\emph{(\textsubscript{y},\textsubscript{z})en\textsubscript{2tf}}有

hd(yx)--- = E 冬(y/)---冬。,0冲

1 1 (1. 7)

Walley/) --- C\^{},o)\textbar{}\textbar{} = ----z\textless{}£.

注意到 y(o)=y\textsubscript{0}e(-ia),故

M2eU 丑2'' Y1.

由压缩映象原理可知厶在丑2e内有唯一吸引不动点,又由(1.5), 4L=Q =
(Y(e),e),故不动点不也于直线言=0上.因此,系统兀
过不动点的轨道是一个吸引周期轨.而不等式(1. 7)意味着不动点
与点(Y。),矽的距离不超过e.因此,周期轨当L0时趋于X。的同 宿轨的位置.

(2) Q.

\begin{quote}
令△=(\&,\&).对任意点\emph{(.y,\^{}en\textsubscript{h},} △z(y,£)WAzG,z)
--- \& w \textbar{}\textbar{}(4,Az)(y,z) ---
(Y(e),\&)\textbar{}\textbar{}

=114。浦)---冬⑶,o)\textbar{}\textbar{} ---。,。)11

=丄 \textless{} z.

4
\end{quote}

上边不等式中等号成立,当且仅当

\begin{quote}
2 --- € = 0.
\end{quote}

这样,一个点的M坐标在映射\&的作用下减少,因而\&在区域
风'2=0}上没有周期点.

附注1.2在上边的证明中没有用到假设(3).

鞍类周期執的产生

现在对鞍点量为正的情况来证明定理.与鞍点量为负的情况
不同,我们将利用双曲不动点定理来研究映射4不动点的存在性 和唯一性.令

WW 卍)=(骨,頌旳(时,e). 假设(3)意味着

京=碧3 尹°, (1.8)

\emph{如} I(0,0,0)

情况d \textgreater{}
0和\emph{d\textless{}0}分别对应所谓可定向同宿轨和不可定向同宿
轨.在区域丑月上,我们将y方向看成水平方向,而把z方向看成垂\\
直方向.

引H13存在正常数\emph{土\&、盹向0}使得对所有矿\textbar{}时《
1,映射任在瓦上满足(网,化)锥形条件.

证明我们用第四章引理4.1来证明・

令,x\textsuperscript{,})=4T\textsuperscript{8}(\^{}\textsubscript{tZ}).由链式法则,

D``,£)= D骨(寸,'')・ 心p,z)

\includegraphics[width=0.35347in,height=0.74653in]{media/image61.png}\includegraphics[width=0.3in,height=0.70694in]{media/image62.png}\includegraphics[width=0.42014in,height=0.14028in]{media/image63.png}

其中

ff = D\^{}(y ,/), A =『,\emph{B = ay\^{}-l,}

\textsubscript{a} (1.9)

\begin{quote}
D=O, \emph{M =此L}
\end{quote}

由第四章引理4.1可知,引理1. 3成立,如果存在常数£\textgreater{}0,使得

以及一个依毯于£的常数\emph{SCL1,}使得

max\{ \textbar{}AfT
I,照\emph{t\textbackslash{},\textbackslash{}A-BM\textasciitilde{}\textsuperscript{l}D\textbackslash{},\textbackslash{}M\textasciitilde{}'D\textbackslash{},\textbackslash{}AM\textasciitilde{}'
\textbackslash{}} \}\textless{} 5(£).
注意到(1.8)和友\textasciitilde{}吨尸(条),当如LO,我们立即得到上面
第一个不等式.现在我们来验证第二个不等式.因为鞍点量为正, 故有

夕=-端\textless{}1,园《L

因此

IM-'HDI =0,

\textbar{}M\textsuperscript{\_1}1 ---十事―,-* 0,
\textbar{}BAf\textsuperscript{\_1}1 =\emph{专}

\textbar{}A- - z°-* 0\textgreater{}
\emph{\textbackslash{}A\textbackslash{}},\textbar{}M-'\textbar{}
=x«T+l-*O,

第二个不等式必然成立,引理证毕.I

下面我们分情况讨论.

(1)泌\textgreater{}0.

对于。,/£耳,我们有

\begin{quote}
△z(B=頌\textsuperscript{8}。輩f 比)-豹0,0``)+ £ =頌《(泌,那,e)-胥(0,0,e)
+ £ 等宀掌\ldots{} \textsubscript{(1}.\textsubscript{10)}

2 3z + e, OOVYL
\end{quote}

先设eNO.由(1-10),当z\textgreater{}0时\emph{My,\^{}\textgreater{}z-}这意味着\&\emph{在}
页\textbackslash{}仏=0\}上没有周期轨.

再设Y0.下面用双曲不动点定理来证明,\&在内有唯一 鞍类不动点.因为\&3,0) =
e\textless{}0,我们可以找到这样一个依赖于
\&的常数0\textless{}G\textless{}一e,使得

\emph{\^{}(y,z) \textless{}} 0, OWzVCe. (1. 11)

考虑r+上的矩形

\emph{D\textsubscript{t}} = \{(y,g)
\textbar{}C\textsubscript{e}\textless{}zC---e.hl \textless{} 1),

则矩形\emph{D\textsubscript{c}}的边界的水平部分为

狗色=g,M)£\textbar{}z = C\textsubscript{o} --- £, \textbar{}y\textbar{}
\textless{} 1\},

\emph{D\textsubscript{t}}的边界的垂直部分为

\emph{\^{}D\textsubscript{e} = \{(y,z)} € F\textsuperscript{+}
\emph{\textbackslash{}y} =± l,C\textsubscript{e}C«\^{}--- £\}.

对于任一点(y,z)WZ\textbackslash{},我们有

I3,o)丨 M ME -骨(0,0) I

+ {]}-\^{}(0,0)--- 3,o))

=心03) --- 骨(0,0)丨 + \textbar{}(珀),©) --- 3,0)\textbar{}

\includegraphics[width=0.45347in,height=0.40694in]{media/image64.png}\includegraphics[width=0.45972in,height=0.40694in]{media/image65.png}

注意到I必\textbar{}\textless{}1,我们\emph{有}

Mng = 0,)e《L

由(1-10)有

为(R, --- e) \textgreater{}--- 2e\textgreater{}--- e.

(L 14)与(1.11) 一起推出

冬策玖n玖=0,

以及

(1.16) (1.13),(1.15),(1.16)和引理1. 3保证了 Poincare映射冬在区域
\emph{以}上满足第四章定理1. 5的所有条件.因而\&在内有唯一双
曲不动点.进一步从(L 12)可以看出,此不动点当『-0时,趋于系
统X,的同宿轨与E的交点3,0).因而,当L0时,通过该不动
点的周期轨趋于同宿轨.另外,(1-10)和(1.11)还保证了在
中,映射\&没有其它不动点•

(2) d\textless{}0.

对于有

\begin{quote}
=頌sg宀e) - △羿(0,0,e) + £
\end{quote}

3骨八坤疽丄

先设 Y0•由(L17)得

矽 V 0,当。产)€ 瓦\textbackslash{}伝=。\},

故映射A在""3=0\}上没有周期点.

再设£\textgreater{}0.因为d(''0)=弓故存在一个依籁于參数£的常
数0\textless{}C\textsubscript{£}\textless{}\textbar{},使得

△zG,z)\textgreater{}\$,对OWzWCe. (1-18)

由\emph{(1-17)}得

鸟⑶,专)\textless{}一爵\textless{}0. (1.19)

现在考虑「+上的矩形\emph{D\textsubscript{e}=\{}(X)EL ICeWY爵,侦I Ml
\}.矩 形D\textsubscript{e}的边界的水平部分和垂直部分分别是

a\textsubscript{A}-D\textsubscript{E} =((\textgreater{}.«)€
r\textsuperscript{+}u = C\textsubscript{e},c/2,
\textbar{}j\textgreater{}\textbar{} \textless{}1\}

电玖=\{(" e「+ IGW Ye/2, \emph{y=± 1\}.}

不驚验证,不等式(1-12)仍成立,因而(1.13)也成立.进一步由
(1.18)和(1.19)可推出(1.15)和(1.16).与情况d\textgreater{}0时完全类
似,我们可以证明映射在\emph{D\textsubscript{t}}内有唯一双曲不动点.定理证毕.\textbar{}

\protect\hypertarget{bookmark268}{}{}\textbf{§2}空间中鞍焦点的同宿分岔

我们将R'中具有一对宴特征值和一实特征值的双曲鞍点称
为鞍焦点•本节将考虑鞍焦点的同宿分岔•不失一般性,我们可以
假设鞍焦点的一对复特征值具有负实部,而它的实特征值为正数,
否则可通过时间反向而转化为上述情况•实特征值与复特征值的
实部的和称作鞍点■•鞍点量为正或为负的鞍焦点所对应的同宿
分岔有着本质的不同,当鞍点量为负时,分岔与本章§ 1中相应的
情况类似;而当鞍点量为正时,在鞍焦点的同宿轨的任意邻域中定

义的Poincarfe映射都存在``马蹄''.这样一类向量场在单參数扰动
下的分岔现象,迄今为至还没有完全研究清楚.

具有负鞍点■的鞍焦点的同宿分岔

本节第一个主要结果如下:

定理2.1令Xe是R3中一个一般的单参数向量场族.假设当
\textsc{e=0}时,X。有一条具有负鞍点量且不稳定流形为一维的鞍焦点0
的同宿轨4则存在 \emph{W0}的邻域U和参数空间中零值的邻域\emph{V}
使得当参数\textsubscript{e}位于V中零值某一侧时,Xe在\emph{U}中有唯一周期轨,
此周期轨是稳定的,当时趋于七而当e位于V中零值另一侧 时,Xe在U内没有周期轨.

在定理2.1的陈述中,``一■般''一词的含义是

CD向量场X。的鞍焦点的特征值非共振;

(2)当参数通过零值时,系统Xe的同宿轨以橫截方式产生和
消失(确切含义见(2.10)).

定理2.1证明的1:现倉义

在给出严格证明以前,为便于读者理解,我们先给出证明的直
观描述,为此我们将尽量把问题简化•首先我们假设,在鞍焦点的
邻域里向量场可由下面线性微分方程组给出

\emph{(tii =} (A + \emph{ia\textgreater{})w,}

. C2-1)

\begin{quote}
岳=\emph{fiZ,}
\end{quote}

其中 i = -/---If \emph{iv=x-\textbackslash{}-iy=r} exp(W), «CR,
XVOV'',3尹0.令 r\textsuperscript{+}= = i\},r-= \emph{\{z =} i\}.
\textless{}2.2)

不失一般性,我们假设同宿轨y与r+和r-的交点分别为

\begin{quote}
/\textgreater{} = yn「+:\{(们z)=(o,o)\}, \textsubscript{(2} 3)
\end{quote}

q =,n r\textsuperscript{-}\emph{i(.x,y,z) =} (0,0,1).

\emph{n\textsubscript{h}=} (W.z) G r\textsuperscript{+} \textless{}1).
(2.4)

注意到(2. 1)是变量分离的.换句话说,它是分别定义在稳定流形
与不稳定流形上的两子系统的积.因此,(2、1)的任何轨道希z轴
方向向(\#,少平面的投影是方程\&= + i«»)w的轨道.这样,所
有以\emph{n\textsubscript{h}}上的点为初值的正半轨位于一个平行于z轴的柱形区域
里,这一柱形区域的底是一个``粗螺线''匸,其中匸是由区域瓦n
蒔=0}上出发的正半轨线的并组成.我们用厶畅表示沿系统
(2.1)的轨道从\emph{n\textsubscript{h}}到的映射,即\emph{n\textsubscript{k}}上每一点映到从该点出发
的正半轨与「\textasciitilde{}的第一个交点、为了描述区域\emph{a\textsubscript{h}}在哲喩作用下的
象,只需将匸沿\%轴方向提升到平面然后再去掉其在圆周c: r = H =砂,V = 一
\$之外的部分.有两种情况需要考虑

(1) 鞍点量为负:a=A+\^{}\textless{}0.此时V\textgreater{}1,\emph{椒HTF}

(2)
鞍点量为正2=人+心\textgreater{}。,此时V\textless{}1,\emph{故H=*》h}
见图5-3..

\includegraphics[width=3.46667in,height=1.12014in]{media/image66.png}

图5-3

在定理2.1中只考虑第一种情况.根据假设同宿轨连结「+的点\emph{p
与}L上的点\emph{Q.}由解对初值及参数的依赖性,对于所有充分小的
参数,r-±\textsubscript{9}点附近出发的正半轨道将与「+交于\emph{p}点附近的一
点,我们用中表示这一对应关系.为了简单起见,我们假设骨 是一个平移,

肖:愆,力 f (0,z) = W«(z,力=(z\$ + e). 记A =
\^{}\textsuperscript{g}(0,0) = (0,e)是鞍焦点的不稳定流形与「+的交.不
难看出在Poincare映射\&=杰弟„杰姑的作用下的象是一个
以点处为极点的``粗螺线'',并且它位于圆周耳:\textbar{}(们言)一如1 =
\emph{m} 的内部.当£\textgreater{}o时,因为集合位于C\^{},H=(2C的
内部,因而\&是丑''到自身的压缩映射,故有唯一吸引不动点.当
Y0时,圆周内部的点的m坐标小于无,这意味着映射
\&象的z坐标比其原象的\%坐标小,故\&没有周期点,

定理2. 1的证明

我们将证明分成几步

\textless{}1)映射夔\textsuperscript{8}的定义及性质

由假设1,应用第一章定理4. 27,在鞍焦点的一个邻域\emph{0}里存
在一个坐标系S,y,z),使得Xe在。内是线性系统

\begin{quote}
\emph{i} = A(e)x --- tu(e»,
\end{quote}

\emph{■ y =} + A(e);y, (2.5)

\begin{quote}
\emph{z} =产(e)z,
\end{quote}

这里H0)V0V产(0),a = A(0)+产(0)V0.如果必要的话,可对 \textless{}2.
5)做尺度变换,故我们总可假设\{(z,加时\textbar{} E\textbar{},
\textbar{}y\textbar{}, \textbar{}z\textbar{} \textless{} 1)
U0.不失一般性,我们还可进一步假设,当『0时屛中正z轴是 同宿轨的一部分.令

\begin{quote}
i\^{}\textsuperscript{2} + y =
1,\emph{\textbackslash{}z\textbackslash{}} i\}\textsubscript{t}

戏=\{愆成,歎e r* \textbar{}0WzV五\}, r\textasciitilde{}= \{Czj,z)I/ + y
w * i\}・
\end{quote}

方程(2. 2)的以点愆①,QEZV为初值的解有下列形式 工(£) =
exp(A(£)i)Ccos(tu(€)f)x ---如(火以) 火 Q = exp(A(e)r) {[}sin(w(E)f)x +
00»3(£)*»丄 z(£)= exp(产(矽

从上面第三个等式解得凱道到达「■所需时间为产(。71眼、

将其代入前两个等式,得到轨道与的交点坐标(\#\emph{,矿},1)满足

('',护)---W Czcos/? --- win\#)
\textsubscript{f}x\textsuperscript{\_a}(xsin\^{} + \emph{ycosR)),\\
}这里a =人(矽加(c) \textless{} 0,夕\emph{=一} gglnz.我们用笙哗代表系统\\
(2.5)的沿轨道从;V到r-的映射.如果我们在"上引进坐标\\
(0,z),0=tanT,则梦\textsuperscript{8}可表示成一个二维映射

«x

\emph{3 ,y'})=茂'理但幻=(2-`008(0 + 戶),厂`血(。+ \textsc{j3».}

(2. 6)

(2.7)

此处

I

------\&(€)(A(。弘。---o)(e)5i'n0)

因为鞍点資a-A(O)+M0)\textless{}0,故

---a --- 1 =--- --- 1 \textgreater{} 0, \textbar{}d《L

(2.8)

(2)映射骨的定义及性质

用》戒分别表示同宿轨,与计和「-的交点.将坐标系做一
个绕\%轴的旋转,我们可以假设\emph{P}点位于H轴上(0=。).令 性3= wqe/l 冏
(2.9)

这里SO是常数.当£=。时,同宿轨y建结「一上的点 \emph{m ,y')}
=(0,0)与点\emph{p.}由解对初值与参数的依赖性定理,存在r-±q点
邻域Q,当I时充分小时,每条从\emph{Q}中一点出发的正半執都与r+交
于\emph{P}点附近一点,这种对应关系我们用 骨 表示.「+,是 一个微分同胚■令

笙%) = g),z(e)), (0(0),z(0)) = (0,0),

馳假设的确切含义为

亲z(e)\textbar{}5 尹 0. (2.10)

做参数变换z(矽Ze,我们可假设

骨 0) =g,£),

即不稳定流形和「+的交点与稳定流形伝=0}的距离为\textbar{}e\textbar{}.

(3) P晚ncart峡射和它的不动点

由(2.6),当兀充分小时,象星\textsuperscript{118}「办包含在出**的定义域Q
里,故在「払上可定义Poincare映射如下

\& =斷。史

由(2.8),当五充分小时,\&是丄个压缩常数小于f的压缩映射.
接下去对d的不动点的存在性的讨论与§ 1中关于戰点量为负情
况的讨论类似,我们在这里从略.

定理2.1证毕,\textbar{}

鞍点■为正的鞍焦点的同宿分岔:马蹄的存在性

定理2. 2设空间中一个光滑向量场有一条具有二维稳定流
形的鞍焦点的同宿轨.若鞍焦点的鞍点量为正,则在同宿轨的任意
邻域里都有无穷多条双曲周期轨.

定理2. 2证明的思路 我们将通过验证同宿轨的Poincare映
射存在``马蹄''来证明定理,下面我们仍然利用公式(2. 2) ,(2. 3)和
(2.4)中的记号.问题的关键是要证明Poincare映射厶在「+的某
个子集上满足第四章的马蹄存在定理的条件.正如我们前面所说,
象\emph{皿}是一个以\emph{p}为极点的粗螺线,粗螺线距点\emph{p}的最大距离约
为0(1)尸,集合皿'3=0)有可数个连通分岔.我们考虑那些位

于(\^{}=0\}±方的连通分岔的原象,这些原象都是曲边矩形,映射△
在每一个曲边矩形上都是一个马蹄映射,见图5-4.

\includegraphics[width=2.31319in,height=1.8in]{media/image67.png}

图5-4

定理2. 2的证明 由第一章定理4. 22,我们可以在鞍焦点的邻
域里取一个U卡,使得向量场在这个卡下是线性系统

s = Az --- ay,

\emph{■ y = cuz + Ay,} (2.11)

\begin{quote}
2 =
\end{quote}

这里人 \textsc{V0V\#,2+m\textgreater{}0.}

r\textsuperscript{+},r-如(2.2)所定义,厶蛔 代表沿(2.U)的轨道

定义的映射.由(2. 6),厶血\textsuperscript{8}有表达式

厶蛔 0/)=(厂%08(。+ 8),旷《血(。+ 0)), (2.12) 这里8= 一
?1心,一10=対产\textless{}0.令

瓦=\{(们功 \emph{er\textsuperscript{+} \textbackslash{}e = e»e}
\textsc{e-53J,o\textless{}\textsubscript{2}\textless{}a\}.}

由(2.12),象濟昭气是「-上的以3,°)为极点的对数螺线•注意
到厶财是一个微分同胚,故象\&£=厶哦。△血8駅是尹上以点\textsubscript{p}
为极点的螺线.令「力兀如(2. 9)定义,则象AT*五是介于两条螺线
应主d之间的区域•它在直线3=0}上方有可数个连通分支,其中
每个连通分支都绕\emph{P}点半圈.我们给这些分支由外向里给定序
号,令瓦,表示第«个连通分支在4作用下的原象,见图(5-4),则
丑*是一个曲边矩形.它的上下边属于直线{言=0}的原象,我们断
言,存在不依赖于\emph{n}的常数G\textgreater{}1,使得下面不等式成立

\textsc{Up"}嗇K *\textsuperscript{1} g*. W G exp(-\emph{粽),}

(2.13)-

Cr\textsuperscript{1} exp(窗)w II 0,£)I匝 W G exp(舒),(2.14)
这里表示点(H,z)到点步=〈0,0)之间的距离•为了证明上
述两不等式,在(们Q平面上对点(8,z)£瓦,我们如此定义其象点
△(。点)的辐角arg ,使得这个辐角连续地依義Sz),并且

arg \emph{A(3,z)} \textbar{}叫£{[}0,0,这样便有

\textbar{}arg 苛理(°,2)\textbar{}渉*)€'' e {[}(2'' 一 2侦,(2''一
1)\textless{}\textbar{}.

(2.15) 由(2.12),有

\begin{quote}
\textbar{}arg 潛W,z)\textbar{} = 誹* +0(1).
\end{quote}

注意到辐角arg厶如在微分同胚厶也作用下只能改变有限 量,故

\textbar{}argA(H,z)\textbar{} = + 0(1), (2.16)

比较(2.15)和(2.16),得

---Ina =2* +。(1)・

\begin{quote}
``冲
\end{quote}

由上式便推出(2.13),注意到

\textbar{}\textbar{} 厶(们z)\textbar{}\textbar{} 厶蛔(们"\textbar{}
=O(l)\textbar{}z\textbar{}f,

再利用(2.13)便可推出(2.14).区域丑*的另一个重要性质由下 面引理给出,

引理2.3对任鳶给定當数席所>0,当''充分大时,区域 \emph{L是}S,所)矩形.

证明瓦,是一个曲边矩形.它的左右两边是直线佰=士8}.
这样,为证明引理,只需证明\emph{a\textsubscript{n}}的上下边是\textsubscript{w}水平曲线,即一个
李氏常数为任关于0的函数的图象.

令\emph{w}表示直线z=o在映射厶呻作用下的原象,则\emph{w}是r-±
的一条通过点g=(o,o)的u曲线.曲线"在卩点的切线或不与》
轴平行或不与x轴平行.下面我们只讨论不与3-轴平行的情况,
不与x轴平行的情况可类似地论证.

令中是C\textsuperscript{1}函数的图象

\emph{w =} {(z,了)£6 (R.o)},
SM/(o)=o\textsubscript{t}则\emph{n„}的上下边中的点的坐标(仇/满足方程
/愆)=夕, (2.17)

这里 cos(8+/3),丿=厂。sin(0+§).等式(2.17)两边对 \emph{9} 求导,得

条一中苓+凱

=1亙而+ 41 一万云,而)

或

\begin{quote}
割\_ 3, (x) +\emph{訶}3) +中+誹)⑵18) \emph{=xz+ yzf'} (x).
\end{quote}

将(2.18)除以工,得

飢-寸5+唯+辭3) +方

=*1 + +户(。). (2.19)

注意到当工f。时,主f尸(0).因此,(2.19)的等号右边括号内的
值趋于非零常数,故

务=。⑴

再由(2.13),当厂*8时有 \textsc{lO.}引理证毕.\textbar{}

令

玖,=\{(。,釘€瓦 \textbar{}8\textbar{} M2C】exp\{標)\}. 由引理2.
3可知,ZV是一个(处,所)矩形,它的水平边界属于原象
厶顼\{*=()\}•因为鞍点量b=A+产>0,故当时

G F(- 符>《G-1 exp(帶')•

\includegraphics[width=3.22014in,height=2.00694in]{media/image68.png}

由不等式(2.13)和(2.14),Poincare映射△限制在区域\emph{D\textsubscript{n}}上类似
一个马歸映射,参见图5-5.下面证明\emph{D„r\textbackslash{}\^{}D\textsubscript{B}}确实包含了一■个马
蹄•为此,我们将验证映射△在区域\emph{D„nA-'D\textsubscript{n}}上满足第四章定
理3.2的所有条件.首先,我们证明映射△在区域an"玖上

满足S,向)锥形条件,

引理2. 4对任意给定常数您出》1,都存在依赖于故,
代的常数N,使得当\emph{n\textgreater{}N,}映射厶在区域上满足
(冲,為)锥形条件・

证明我们将利用第四章引理4.1来证明.

;)的辐角,令 \emph{cos\%} --- sin .sin爲 \emph{cosg。}

因为向量[J是矩阵片的特征向量,故

(2.20)

我们断言,对于\emph{C0,\^{}eD„\^{}-'D„}有

\begin{quote}
0, \emph{P} \^{}Inz f go(mod it), \emph{n -*+
\textless{}x\textgreater{}.} (2. 21)
\end{quote}

由区域2.的定义,(2.21)中第一个极限是显然的•现在我们证明
第二个极限.因为(丄■?)€/)`` ,故厶(们由

(2.13)和(2.14)有

\begin{quote}
argA(0,z) f 0 (mod ''), \emph{n f + 8.}
\end{quote}

因此'当 8 时 arg不-傷god \textsc{k)}.由(2.12)可知,
arg△血们z)=8+/3,这意味着当8时,戶―禺(mod ``).

现在我们来计算雅可比阵D△.对(2.12)求微分得

\begin{longtable}[]{@{}lll@{}}
\toprule
\endhead
& 脳")=驚 & cos/? \emph{I}\tabularnewline
其中矩阵 & &\tabularnewline
4 = & `一厂侦泌-\/- & \textsuperscript{1} (Acos\^{}
---如in。)\tabularnewline
\bottomrule
\end{longtable}

\begin{quote}
Z\textsuperscript{\_}"costf ------``TgfT (Asin。+ \emph{a/cosff)}
因此,由链式法则有
\end{quote}

D/i0,z)=。厶哼(厶血。(们戒)•

\begin{quote}
\emph{=HJ =}
\end{quote}

其中

\emph{I a b \_ i} cosg --- sing\}

ff= 」=D次Kg)).: :,

\emph{c d} \textbackslash{} sin夕 cos0 \emph{I}

\begin{quote}
\emph{A B \textbackslash{}}
\end{quote}

\emph{\textsuperscript{/=} D M{]}'}

\emph{A ---} --- --- --- osin。),

\emph{D ------z\^{}cosfffM
------}------/(\textsuperscript{\_1}2\textsuperscript{\_a\_1}(Asin5 +
\emph{\textsc{okosG').}}

现在我们来验证第四章引理4.1的条件,即存在常数L,以及由引

理4.1所确定的依赖于乙的常数乩使得不等式

\begin{quote}
\emph{L \textgreater{}} max\{ \textbar{}回\textbar{},
\textbar{}\textbar{}H-'\textbar{}\textbar{}
\emph{,\textbackslash{}d\textasciitilde{}' \textbackslash{}
\textbackslash{}\}} (2. 22)
\end{quote}

以及

\begin{quote}
max(\textbar{}MT\textbar{},\textbar{}MT\textbar{} •
\textbar{}D\textbar{},\textbar{}A\textbar{} • (? \textsubscript{23)}
\end{quote}

成立.由(2. 21),对(8,2)£口》,当\&-\textgreater{}8时有H---点或一角.故由

(2. 20)有

c-* 0, \emph{\textbackslash{}d\textbackslash{}} f 位\textbar{} 尹 0.
(2.24)

注意到当''f8时,10,我们有

〈鼎,5, (\textsubscript{2}.25)

因此(2. 22)成立.由(2.13),当 L8时有

\textbar{}AfT\textbar{}=O(lW+J0,

\textbar{}MT\textbar{} \textsc{\textbar{}Q\textbar{}=0(1)lO,}

⑷\textbar{}心\textbar{}=0⑴\textsc{lO,}

\emph{\textbackslash{}A-BM\^{}D} I =0( WJ 0.

另一方面,由(2.24)和(2. 25)推出,\textbar{}c\textbar{} •
IBM'\textsuperscript{1}1 - 0,当

\&f+8,故(2. 23)成立.引理证毕.\textbf{I}

最后来验证,4在上满足边界条件和相交条件• 首先ang
包含两个连通分支,故aru-\%''有两个连通分
支团与H:.因为它们当中的每一个都是一个曲边矩形,它们的
上下两边属于\emph{D\textsubscript{n}}的水平边界\emph{d\textsubscript{h}D„}的原象A-】眾如因而\emph{勇H}
都是(陶,所)矩形,且满足

(活 U 卅)Cl 氟叫=(2.26) 注意到 缽死U氟玖,,由(2.
26)有皿研0(研11死)=0・由 不等式(2.14)与\textsc{d„}的定义推出 g
n毒以.=。,故\&理n \%玛=0盘」=1,2.由连续性,相交条件朝n码尹0显然 航.

上面的论证表明,对所有充分大的自然数*,映射△在区域
\emph{D\textsubscript{H}n\^{}D„}上有一个不变集4,使得与a上的右平移映射\emph{。}
拓扑共辄,因而△在4上有无穷多个周上点.因而向量场在同宿
轨附近有无穷多条双曲周期轨.

定理证毕.I

轨道等价不变■的存在性

令\emph{M}是一个紧致光滑流形,幽代表\emph{M}上的所有C向
量场在C拓扑下所构成的Banach空间.在缶'(M)上定义一种
等价关系``〜''•对X£缶「(M),令

X =(Y e \emph{\textbackslash{}My\textbackslash{}Y - X\}}

表示与x等价的向量场的集合.称xe缶Pw)(关于等价关系
\textsuperscript{1}•〜'')结构稳定,如果\emph{x}属于集合玄的内点.

定X2-5 一个复数c称为向量场X£缶'(M)的模,如果对
\emph{X}的任意邻域U,存在一个包含\emph{X}的连通集合\emph{A(=u}和定义在\emph{A}
上的一个非常值连续函数f满足

\begin{enumerate}
\def\labelenumi{\arabic{enumi}.}
\item
  /(X)=C\textgreater{}
\item
  v Y,zeA,若 y 〜z,则/'(59=六2).
\end{enumerate}

由定义2. 5可以看出,任何有模的系统都不是结构稳定的.例
如,如果等价关系``〜''取为U等价,则孤立奇点处的任意两个特
征根的比值即为模.因而,任意有孤立奇点的向量场都不是结构稳
定的.再如,如果我们在拓扑轨道等价中要求相应同睡保持时间,
则向量场的孤立周期轨的周期便是模.因而,任意有孤立周期轨的
向量场不是结构稳定的.显然,上述两种等价关系过于*严格'',使
``过多''的系统不是结构稳定的.于是,我们通常是在更弱的拓扑轨
道等价(见第一章定义1. 7)下,研究结构稳定问题和分岔问题.寻
找有模的系统是分岔理论中一个很有意义的问题.

定理2. \emph{6} (EAAIS3)设X是IT中一个具有鞍焦点外的同宿
轨的光滑向量场,则鞍焦点%的复特征根的实部与它的实特征根
的比值是系统X的模.

设奇点京的特征根为A士国,产,为了证明定理2. 6,我们只需证
明,〃产是轨道等价不变量.在下而的讨论中,我们仍利用前面的
记号和概念.我们仍然假设,〈产.

若A+"MO,我们定义

店(``)=min'' € \emph{0\}\textsubscript{f}}

若人+40,我们定义

\begin{quote}
''(\&) = max\{n £ N,氏 0 心''\emph{手} 0\}.
\end{quote}

引理2.7序列*(»),««)有下列性质

\textsuperscript{71} 产 4-* + ™ « 人

证明如果

maxfzllo* Wmax\textbar{}z\textbar{}\textbar{}、,

则显然

\emph{以}n皿!\emph{手}01

如果

min\textbar{}z\textbar{} 1\% \textgreater{}max\textbar{}z\textbar{}
\textbar{}业,

则屜然

\begin{quote}
\emph{风}n徵舞=\emph{0・}
\end{quote}

由(2,13)和(2・14)易知,如果

(-當RUP(符), 则(2. 27)成立.换言之,如果也骅而,则

\begin{quote}
mn\&T逐。. (2,29)
\end{quote}

同理,如果

Up(-符)\textgreater{} G exp(窗)` 则(2. 28)成立.换言之,如果一也竺
\textgreater{}网+糸,则

\begin{quote}
耳 n 斗=\emph{0.} (2. 30)
\end{quote}

由(2.29)得

岫亀哨-分顶``宇-令.

由(2.30)得

\textsuperscript{kw} A 企肾-7\textsuperscript{n, n(A)} W-企蜉-f*- 因此

陪柴\textbar{},酵+普同齧!. 引理证毕.I

定理2.6的证萌 令X]和Xz是R,中分别具有鞍焦点。雨。2
的同宿轨的两个光滑向畳场•设4•士讪,,曲是勺的特征值•假设
\%和尤是轨道等价的,即存在一个同旺H,R---R3,使得H将X\textbar{}
的轨道保向地映成X,的轨道.正如前面考虑的那样,我们可以假
设\emph{粕在勺}的某个邻域3中是线性的.

\begin{quote}
x = AjX --- \emph{a\textgreater{}jy,}
\end{quote}

X,= 3产 + 人疗,A,. V 0 V 的,s,•尹 0.

通过一个线性变换,我们可使Ui中的点6=(1,0,0)属于q的同 宿轨.我们取

lYnUHCsQElASz + y = 1,⑵ Ml\}.

在S中做一个线性变换,可使点\textsc{a=h\textless{}a)}的坐标成为(1,0,0). 取

n = \{o,外 z) € + y = i,\textbar{}\textsubscript{z}\textbar{}\textless{}
1\}.

在町 的某个子区域上,我们定\emph{X} Poincare映射乌.我们断
言,映射厶与\&是拓扑共純的,即存在一个从上点队的邻域到
IV上点為的邻域的一个同胚G,使得对\&的定义域上的任意点 有

G - A3)= 4, - G3).

事实上,同胚G可以用公式

\emph{G = .。H}

给■出,这里玲定义为点所在R冲的邻域沿X面轨道向Z7的投影
映射.显然洞胚G将Z7上的曲线住=0\}映到有上的曲线修=0\}
.如前所述,映射4的定义域是一族曲边矩形疋的并•因为昭i
的上下两边属于原象午%=0\},故G 一定将\emph{m,}的上下边映成某
个曲边矩形\emph{吗}的上下边.因此对所有4》1,存在一个不依赖于«
的整数m,使得

\emph{GK = \%.}

因此

碇》)=如(舛\emph{+ m) ---m,}当鞍点量非正;

= »2(4 + \emph{m) --- m,}当戰点量为正.

由引理2.17,这意味着

太=至

\emph{\% th}

定理证毕.I

\protect\hypertarget{bookmark278}{}{}\textbf{§3}环的分岔

有限个奇点和闭轨tfo.Oi,---.\textless{}?4-1\^{}* =
\textless{}7\textsubscript{0}以及连结它们的轨
道乙•,满足和弘为)=鸟+1并=0,1,``財一1,所构成的轨 道集合称作环.我们在§ 1,
§ 2中所考虑的奇点的同宿轨是一类特
殊的环,邮及色是一个双曲奇点.本节我们考虑R*中由一个
双曲奇点和一条双曲闭轨以及两条连结它们的轨道所构成的环.
值得指出的是,在著名的Lorenz方程中,对某些参数值上述环是
存在的,而且环附近的分岔现象非常复杂,迄今为止尚未彻底研究 清楚•

环的分类

设R3中一个光滑向量场\emph{X}有一个包含奇点6和闭轨电的环
A,满足下列四条假设

\begin{enumerate}
\def\labelenumi{(\arabic{enumi})}
\item
  C是一个双曲奇点,它有二维稳定流形"'3。)和一维不 稳定流形
\item
  \textless{}7,是一个具有二维稳定流形``飞凤)和二维不稳定流形
  於``(公)的双曲闭轨,它的特征指数为正.
\item
  令"UW(为)n'''(a,+D并=0,1表示\emph{A}中连结«,和缶+1
  的執道,则流形胪3D与\^{}3。)沿曲线儿横截相交・
\end{enumerate}

⑷如果奇点%的特征值都为实数爲\textless{}0〈产,则以是2阶 非共振,即``尹%
Aj, AjK2人1,且

\begin{enumerate}
\def\labelenumi{(\alph{enumi})}
\item
  设"是向量场\emph{X}的二维不变曲面,它在。。处与特征值
  4,产相对应的特征向量所张成的平面相切,则\emph{W}与稳定流形IV*
  (急沿曲线7。橫截相交.
\item
  轨道么沿着奇点c的主稳定方向趋于\%.
\end{enumerate}

满足上面四条假设的环\emph{A}有三种不同的\emph{类型:}

1-奇点。。是鞍焦点,这时人称为鞍魚环.

\begin{enumerate}
\def\labelenumi{\arabic{enumi}.}
\setcounter{enumi}{1}
\item
  奇点勺特征值都为实数.且右UQ(X Z),这里Q(X \textbar{}u)表
  示向量场\emph{X}限制在A的某个邻域\emph{U}内的非游荡集,这时\emph{A}称为
  非游蕩环.
\item
  奇点凤所有特征值均为实数,且\emph{7\^{}£i(.X\textbackslash{}\textsubscript{v}),}这时厶称为
  游蔼环.
\end{enumerate}

上面三种类型的环可参见图5-6.下面我们首先证明鞍焦环和
非游荡环的任意邻域里都有无穷多条双曲闭執,具有这样环的向
量场在一般的单参数扰动下的分岔,还没有被研究清楚.本节的最
后将讨论游荡环在一般的单參数扰动下的分岔问题.

炒环

定理3.1设光滑向量场,X有一个鞍焦环A,满足假设(1) ---

\begin{enumerate}
\def\labelenumi{(\arabic{enumi})}
\setcounter{enumi}{2}
\item
  ,则双曲周期轨。I有横截同宿轨.
\end{enumerate}

证明定理结论的几何意义可参见图5-7.

令\emph{S}是一个与队横截相交于\emph{O}点的平面,在S上\emph{0}点的邻域
取一坐标系3,V),使得\emph{Q}点是坐标原点,并且

\begin{enumerate}
\def\labelenumi{(\arabic{enumi})}
\item
  \{(\textsubscript{W},\textsubscript{W})\textbar{}
  l«\textbar{},k)\textless{}2)有意义,
\item
  ((``,0)\textbar{} miW2\}UW,(s);
\item
  ((0,0){]} \textbar{}0\textbar{}W2\}UTV(s);
\item
  \textsubscript{9o}=(i,o)e/\textsubscript{0}ns,A=(o,i)e/\textsubscript{1}ns.
\end{enumerate}

因为bo是一个鞍焦点,故向量场在它的邻域里可以U线性
化,即在%的邻域里存在一个C\textsuperscript{1}卡使得X在该邻域里可由下面线
性微分方程组表示

\begin{quote}
X = Ax J \emph{a)y\textsubscript{t}}

\emph{y = atx + Xy\textsubscript{t}} A\textless{} 0 \textless{}
\emph{住''}弄 0・ (3・ 1)

\emph{z --- \^{}z\textsubscript{9}}
\end{quote}

不妨假设正m轴对应轨道像在§ 2中那样,我们引进无切截面 「士,rj如下:

「+== ((x,.y,z) \textbar{}x\textsuperscript{2} + y = 1,
\textbar{}z\textbar{} \textless{}1\},

(b)罪*善拜

图5・6

\includegraphics[width=2.69306in,height=1.56667in]{media/image69.png}\includegraphics[width=2.76667in,height=1.20694in]{media/image70.png}\includegraphics[width=2.72639in,height=1.34028in]{media/image71.png}

\includegraphics[width=2.90694in,height=1.64653in]{media/image72.png}

图5・7

广-=\{愆,丸 Qlx\textsuperscript{2} + '* W 1, \textbar{}划=1\},

\emph{- \{(x,y\textsubscript{t}z)} £尸+ \textbar{}OWnW 五\}.

(3.1)所导曲的从n到的沿執道所定义的映射仍用厶财表 示.由(2. 6),濟《有形式

厶林(们•?)= \emph{(z-cos(.S + B)} ,z-smC\textless{}? + R))

---=(X(0,z),y(8,z)),

这里a=A/\^{}\textless{}0, £=---芸屜.令点 g 为(x,3\textgreater{},z) =
(0\textsubscript{t}0,l)er- n
儿,点/■为(们/=(0,0)£「+0九因为轨道儿连结点g与点知,
轨道万连结点公与点由方程的解对初值的依赖性定理,每一条
从「-上点\emph{q}邻域及\emph{S}上点戊邻域出发的正半轨必然分别与\emph{S}及
「+交于如及\emph{p}的邻域内一点,我们分别用F和G代表这一对应,
则\emph{F,G}都是微分同胚.因为流形胪(色)与IP(c)沿着轨道为横
截相交,故曲线 可以表示成

。=03), (R,0),

这里。⑴ec\textsuperscript{1}.因此象厶林(w"s)nF)是上的以\textsubscript{q}点为极
点的螺线,这样象\emph{F}。厶畅(WS) C1F+)是S上以知点为扱点的 螺线

(",吋=「。政(。\&)/)''£皿,0). (3.2)

上述曲线属于闭執乾的不稳定流形1¥"(此)与平面S的交.下面将 通过验证曲线(3.
2)与曲线貯0) flS, 3=0}有横截相交点来证 明定理3.1.令\emph{F=}
(Fu,Fv),则我们只需证明方程

\begin{quote}
Fv(X(0O),z),y(0O),z)) = 0 (3.3)
\end{quote}

有简单零点.令

。=票\textbar{}<待\& =割``,提

那么(3. 3)可写成

\emph{V(z) = aX(心},z) + 成(@(z) \emph{,z)} + O(X\textsuperscript{!} +
Y\textsuperscript{2}) = Q, 或

V(z) \emph{= a} cos(i? \emph{+ \^{}+b} sin(ff + j9) + O(z-) = 0.

\begin{enumerate}
\def\labelenumi{\arabic{enumi}.}
\setcounter{enumi}{3}
\item
  注意,当0时,函数V(g)是一个在士 ``T夢之间无穷振荡
  的函数加上一个小量.因此(3. 4)有无穷多个根处f 0.下面证明 砲是(3.
  4)的单根.由V(z\textsubscript{t}) =0,
\end{enumerate}

\emph{\textbackslash{}a} sin(0 + g) --- 8 cos(0 + £) \textbar{} 勺=\&
十罗(1 + o(l)),

\begin{enumerate}
\def\labelenumi{\arabic{enumi}.}
\setcounter{enumi}{4}
\item
  这里当4---8时,o(l)-*0.另一方面
\end{enumerate}

{[}V 6)I = j 亟 {京X(g(z) ,z)} + \emph{dFv} {史w)} =I (a + 0(广°)){[}---
\textsc{ok\textsuperscript{\_}"\textsuperscript{\_1}cos(\^{}(k)} \emph{+
g) --- z\textasciitilde{}\textsuperscript{a}\^{}n(0(z')} + j3) ♦
(伊\textsuperscript{(x)} + 浄{]}十。+ O(D){[}-
«---\textsuperscript{\_1}sin((?(2)+ /?) + zTcos(8(z) + g)(伊(z) +
?){]}\textbar{}

=号瑚7(,"" +。⑴)X0.

上面最后一个等式是由(3.4)和(3.5)推出的.定理证毕. I

非游蕩环

定理3. 2设R'中光滑向量场X有一个非游荡环A,満足假
设(1)-(4),则在\emph{A}的任意邻城里\emph{X}有无穷多条双曲周期轨.

证明实际上,我们将证明一个更强的结论:在厶的任意邻
域里都可以定义场X的一个具有马蹄的后继映射•我们将分成几 步来证明定理.

\textbf{A}周期執的\textbf{Poincare}映射

令S是一个与周期轨气横截相交于Q点的平面.在s上Q点
邻域可以定义闭轨%的Poinca推映射R因为Q点是二维映射F的
双曲不动点,故在\emph{Q}点的某个邻城\emph{U(Q)}内,可以选取一个C\textsuperscript{1}卡
(",u)使得

(DQ点是坐标原点(0,0),.

\begin{enumerate}
\def\labelenumi{(\arabic{enumi})}
\setcounter{enumi}{1}
\item
  \textbar{}``\textbar{},W\textbar{} V2\} Ut7(Q),
\item
  \emph{qj=} (1,0) £ 為 C\textbar{} S,例=(0,1) \& 7)PI S,
\item
  Poincare映射F在U(Q)内是线性的
\end{enumerate}

(3- 6) 这里OV\^{}Vl V元.令

\emph{D,, ---} G \emph{U} (Q), \textbar{}iz ---

■玖=((u,u) £ U(Q), \textbar{}``\textbar{} 1\textbar{} Wd\}.

取S充分小,则有

m n a = 0, n Di = 0.:

令

则由(3.6),有

H'' = \{(«,») 6 Do,

m-11 -»)員寸員穴*(1 + 3)\}.

令叫=尸払,则有

H: = \{(``0 6 \emph{D\textsubscript{lr}}

\begin{quote}
x(i ---。)w * w 尸(1 + 3), m -11 \emph{\^{}s\}.}
\end{quote}

\textbf{B}鞍点附近的映射.

由假设(4),我们可以在鞍点%的一个邻域V中选取一个C
卡Gr,y,z),使得下列条件成立

\begin{enumerate}
\def\labelenumi{(\arabic{enumi})}
\item
  \{(z,\textgreater{}i,z),\textbar{}z\textbar{}Wi,I\^{}IWi,\textbar{}z\textbar{}Wi\}uV.
\item
  向量场X由下列线性系统给岀
\end{enumerate}

\begin{quote}
\emph{'x ---}
\end{quote}

\emph{-y} =危了,\emph{\textless{},x,y,z\textasciitilde{})} 6 V, (3. 7)

\begin{quote}
.£ =四,
\end{quote}

这里兀v爲vov兴是。。的特征值.令

\begin{quote}
r\textsuperscript{+}= \{(x,\textgreater{},z): x = i,{[}j\textbar{}
\textless{}i,{[}«! \^{}i\}, 「-= \emph{\{(x,y,z)t}
\textsc{\textbar{}h\textbar{}M1,\textbar{}3i\textbar{}W1,\textbar{}k\textbar{}=1\},}
\emph{Uh --- \{(x,y,z\textasciitilde{}) £}: 0 WzW/i\}. '
\end{quote}

由假设(4)的(b),我们不妨设轨道%与「+相交于\emph{p}点:

P = 4 n「+= \{O,丿,Z)=
\textsc{(1,j\textsubscript{o(}O),j\textsubscript{o}} 6 (--- 1,1)\}.
沿系统(3.7)的轨道所定义的从\emph{n\textsubscript{h}}到「二的映射厶血\textsuperscript{8}根据(L
30) 有形式

\textless{}y,/)=曲\textsuperscript{1}g\&,z)=(沱,事), 这里0 V伊=一
4/产V一危/兴=\emph{a.}

\textbf{C 4}到尸的映射

因为轨道儿连结点仑和点》=y】D「+,故当\emph{a}取得充分小时,
系统\emph{x}的每一条从s上的任一点岀发的正半轨将与「+交于\emph{p}

附近一点・我们用\emph{G}表示这一映射:

\emph{G*D\textsubscript{I} f} , (w,w) H* \emph{(.y,z)} ---
G(u,v\textgreater{} - (Gy(a,®) ,Gz(",v)).
由假设⑶流形)与W,(g沿人横截相交,故有

\emph{dGz\\
9u}

因此,可选取3充分小,使得

\textbf{D}广到巩的映射•

因为轨道连结「-上的点g: \textsc{Gt) =} (0,0)与\emph{S}上的如点,
故从r-± \emph{q}附近的点岀发的正半轨必与s交于争附近一点.我们
用F表示这一映射:

\emph{F'} G,Z)f(K,K)= \emph{F(.y,X)}=(玲(》,工),\&(>,2)).
假设(4)的3)意味着

,晒(0,0),``

\emph{曲} 釘-乂。・

因此有

\emph{d =}\sout{押,獸力}\emph{丰}。,国,\textbar{}丁\textbar{}《1.
必>0和必V0分别对应非游荡环和游荡环.

\textbf{E} 环的\textbf{PoincaM}映射和它的双曲性

令

府=\emph{G\textasciitilde{}'n\textsubscript{h}} 0 也,\emph{K„ =} P-»K'
U \emph{H\textsubscript{n}.}
在区域\emph{K„}上定义环\emph{A}的Poincart映射为 厶 \textsc{Ik} =
F。厶血\textsc{「GoL.}

\begin{quote}
ft
\end{quote}

Q, = \{3汐)6 以伊一''(1 + s)一 \$)\},

则有

Q Z) \emph{Hi} U 払\_2・

令

叫=21-U+2 A \emph{K„,D\^{} =} \textsc{△tq``+2} \textsubscript{n} K''+1.
则映射厶在区域玖■ U理上是马歸映射,见图5-8.

\includegraphics[width=3.75972in,height=2.13333in]{media/image73.png}

图A8

引理3. 3存在正常数\emph{叭心踞向0}使得当\&充分大时,映
射A在畠U理上满足(出,S)锥形条件・

证明对令

\begin{quote}
fn. 若(``,!\textgreater{}) 6 Gw) = \emph{G -} \textsc{P"(r,u)},tn =
\textless{}

In + 1,若\&伏,
\end{quote}

(歹,z')=厶蛔(了,2, Q,5)=厶(@,0).

因为(W,V)\&Qn+2,故

\begin{quote}
石一 *-2(i+\$)w板 w 万f+i(i --- 3). (3. \textsc{id}
\end{quote}

由定义

矛.心,心=齢,+井*\\
专宀财

因此有

亨/VBV2O,对九《1.

这与(3.11) ---起推出

(碧阡甲〈Y(导''* \textsubscript{(3},\textsubscript{2)}

现在我们计算雅可比阵DA.由链式法则有

DU DF(y ) • Dg(y,z) • DG • DP»

\includegraphics[width=0.77361in,height=0.65972in]{media/image74.png}\includegraphics[width=0.20694in,height=0.34653in]{media/image75.png}

\emph{a b\textbackslash{}IA} Bl\\
\emph{c d{]}\textbackslash{}D M{]}}

其中

:卜\textsc{dfw}旧),\\
QE-P 萼+ a,登〉,

\includegraphics[width=0.63333in,height=0.27986in]{media/image76.png}

\emph{D= 8E\_1} 芸,

\emph{M=陀"孩\_1}

OT?

由第四章引理4.1如果存在一个正常数乙,使得

\begin{quote}
max\{\textbar{}\textbar{}DF\textbar{}\textbar{},
\textbar{}\textbar{}DFT\textbar{}\textbar{}
\emph{,\textbackslash{}d\textasciitilde{}\textsuperscript{J}
\textbar{},\textbackslash{}BM\textasciitilde{}'\textbackslash{}\}\textless{}L,}
(3.13)
\end{quote}

则对任意给定正常数网曲V巧【《I,都存在一个依赖于\emph{L,}
尚,为的常数勇使得如果不等式

\begin{quote}
max(\textbar{}A\textbar{} • lAf-'JAf-'l •
\emph{\textbackslash{}D\textbackslash{},\textbackslash{}A\textasciitilde{}}
网T\textbar{},jMT\textbar{}\}v/ (3.14)
\end{quote}

成立,则映射厶满足(闻,勤)锥形条件.

由(3.12),当'' ---8有 « -*0,因此⑶,,/ )---(0,0),故\\
\textsc{\textbar{}®/t\textbar{}} =*-"晉广屮登*中登\}* °,当一+冲

(3.15) .注意到 DF(y' * )---DF(0,0),由(3. 9)和(3.15),R要取

乙=2max\{\textbar{}\textbar{}DF(0,0)\textbar{}\textbar{},\textbar{}\textbar{}DF(0,0)T\textbar{}\textbar{},\textbar{}ak\textbar{}),

则对充分大的n,(3.13)成立.现在证明(3.14).当 L0时,有 ⑷•
=身妝方*"驴一書+呵警)f 0,

\begin{quote}
成T\textbar{} .四=貯-I聋)T聲)\textbar{}f o.
\end{quote}

因此,为了证明(3.14)只需证明

\textbar{}Af\textsuperscript{\_1}1 -* 0,当如-*■+ oo.

而这等价于

石-''事''---。,当 n---+8. (3.16)

\begin{quote}
对于 聞1,(3.16)是显然的;对于,8\textgreater{}1,由(3.12) 广宀\textless{}
啰•泞s

=(墙d `` 一\^{}\^{}。,当\ldots{}+8・
\end{quote}

引理3. 3证毕.\textbar{}

\textbf{F}马蹄的存在性

下面我们将利用第四章的马蹄存在定理来证明,映射厶在区
域昌U珏上有马蹄构造.引理3. 3保证了映射厶满足锥形条件.
因此,为了证明定理,我们只需验证下列条件:

⑴昌,功是(用必)矩形;

(2)相交条件:△身n功乂

C3)边界条件:M席n巧=0,朝1為玲=0.3=1,2.

首先,区域叫是曲边矩形,它的左右两边是垂直线段电=1士
\emph{3}的一部分,而它的上卞两边分别是Q+2上下边在映射△下的原
象,因而是网水平曲线,故条件(1)成立.其次,相交条件不难由连
续性推出,参见图5-8.最后,我们来验证边界条件.因为象\emph{国淄}
属于矩形Q+2上下边,而由Q+2的定义它的上下边不与\emph{巧}相 \emph{交,故}
勞mn勇=0.另一方面,注意到

(禹3) = \emph{A(u,v)} f (1,0),对\emph{(u,v)} G -D? U D?,n-►+ 8,
并且曲线\emph{椚}属于垂直直线»=1±3,因此

\emph{叫 n \&巧=0,}对n»l.

定理3. 2证毕.\textbar{}

游荡环

下面我们考虑具有游荡环的向量场,在一个单参数族扰动下
所发生的分岔.设光滑向量场\emph{X}有一个游荡环A,满足假设(1)一

\begin{enumerate}
\def\labelenumi{(\arabic{enumi})}
\setcounter{enumi}{3}
\item
  .令Xe是一个光滑单参数向量场族,满足\emph{XT.}因为奇点%
  和闭轨为是双曲的,故向量场\emph{X\textsubscript{t}}在队,%及轨道\emph{A}附近有双曲奇
  点\&(。,双曲闭\emph{轨g}和轨道4(e)UW*S\textgreater{}(e)),满足%(0)=
  \% --- \textless{}7, =/\textsubscript{0}.
\end{enumerate}

定义3. 4称参数值%为向量场旗Xe的同宿分岔值(环分岔

值),如果 WGUTWoCQ) (TT(Meo)).进一步,分岔值称为
正规的,如果当參数值e通过勺时,轨道\^{}怎)以非零速度接近及
离开流形叩(角①))(胪5心)).

下面我们用分别表示正规同宿分岔價集和环分岔 值集.

定理3.5令Xe是R3中一个典型的单参数光滑向量场族,设
当£=0时,场X。有一个游荡环A,满足假设(1)一(4),则存在\emph{A}的
一个邻域\emph{U}和参数空间中零值的邻域V,使得对于所有位于\emph{V}中
零值某一侧的参数值(不妨设£\textless{}0),系统Xe限制在\emph{U}中的非游
荡集c(xm)只包含两条轨道以(矽和扣矽.对于\emph{V}中零值另一
側的参数值3\textgreater{}0),下述结论成立.

(1)双曲周期轨%(e)有一条横截同宿執,

⑵集合氏的闭包庇是一个包含g=o的Cantor集,且满足
BcC=B«,这里\emph{\textsc{B'h}}代表集合\emph{\textsc{Bh}}的导集»

(3)令V\textsubscript{w}=(e\textgreater{}Olfl(X\textsubscript{t}k)是奇点。°(s)加上一个有稠密轨
道的双曲不变集\},

则存在一个常数使得

lim \^{}mGO\textless{} e\textless{} t\textbar{}e \$ \textsc{Vh))} = 0,

这里m(・)表示R\textsuperscript{1}上的集合(♦)的Lebesgue测度.
附注3.6结论⑴表明,当参数通过游荡环分岔值8=0)时,

X,发生\emph{Q}爆炸,即非游荡集的规模突然大量増加.结论(2)表明,
在心\textgreater{}0附近有不可数个分岔值.结论(3)表明,当參数e通过。附近
的绝大部分值时,Xe在\emph{A}附近不会发生分岔现象.

乙是典型族的含义 定理3.5中的向量场旗Xe除了满足第
216页的假设⑴一⑷外还要满足下面两条假设.

\begin{enumerate}
\def\labelenumi{(\arabic{enumi})}
\setcounter{enumi}{4}
\item
  向量场X。的奇点%和闭執气都是非共振的;
\item
  (\textsubscript{£}=0\}eBc.
\end{enumerate}

证明将分成几步进行

\textbf{A}周期轨的\textbf{Poincar£}映射

令S是一个与。I横截相交于某一点Q的平面,因为乾的特征
指数非共振,故在S上Q点的一•个邻域U(Q)内存在一个依赖于
参数£的有限光滑坐标卡使得

\begin{enumerate}
\def\labelenumi{(\arabic{enumi})}
\item
  ffi(E)n s = (('',v)=(o,o)\};
\item
  \{(",〃)11"I,回 M2\} UU(Q),闭轨\textless{}11 (e)的 Poincare映
  射在U(Q)上有形式
\end{enumerate}

(a,u) -\textgreater{} \textsc{F\textless{}«,p)} = (A(e)tt,/j(e)u),

这里 0VK(e)VlV 石 h);

\begin{enumerate}
\def\labelenumi{(\arabic{enumi})}
\setcounter{enumi}{2}
\item
  \emph{\% =(1,0)}e 為 n \emph{s,fi\textsubscript{0}} =(0,1)\emph{er,
  ns.}
\end{enumerate}

令

\emph{\textsc{d„}} = ((``,\#)e u(Q)11`` 一 11M 団 w \#\}

和

\emph{D, = \{(``,◎)} 1丨 M 科

分别表示争与例的邻域.取\$充分小,使得

\emph{PD\textsubscript{a}} D 玖=0,FT" 0 玖=0.

\textbf{B}鞍点附近的映射

因为鞍点吹非共振,故在%的一个邻域卩中,可以取一个依
赖于参数e的有限光滑卡愆切岸),使得

并且向量场族Xe有形式

\textsc{x = A,(e)x,}

\emph{-y} =杯加\\
三=``(e)z,

其中为(e)V\&Q)V0\textless{}\textgreater{}(e).令

「+= \{(工商,刃工=1,切 WL寥Ml),

不妨假设轨道7。在V中与正z轴重合,即

g = 4 0 广=\{(W)= (o,o)\}.

而轨道为与「+相交于0点

n「+= \{(丁,言)=3,0),烫\textbar{} \textless{} 1).

由§ 1中的结果,沿\emph{x\textsubscript{c}}的轨道所定义的对应映射萨
「一有形式

6,吋1 =屛照愆/)=(对,法),

其中

E---潟〈-潟*

c沿執at定义的小到尸的映射

因为轨道匕连结点决和P,故对充分小S利\textbar{}e\textbar{},沿轨道我们
可以定义一个从例邻域玖到\emph{p}的邻域的一个映射,记作

G: (a,w) -\textgreater{} \textless{}y,g) = \emph{G(.u,v) ---}
(Gy(*,u,e),Gz3,se)). 由假设⑶,不稳定流形"°(s)与稳定流形盼0)沿4横截相交,
故有

@ =孕 尹 0. (3.17)

D 沿軌道定义的广到S的对应映射

因为轨道7,连结点\emph{q}和%,故对\textbar{}€\textbar{}\textless{}1,沿轨道可以定义一个
从r-± \emph{q}邻域到S上価的邻域的一个映射,记作

F: 3 ,x\textsuperscript{f}) f 3她)=F(y H ,£)

\begin{quote}
\emph{=(Fu3},工'»e) \emph{\textgreater{}Fv(y'} M (e)).
\end{quote}

假设(4)意味着

游荡环对应情况doVO.

E僵设(6)的瑞切含憲

假设(6)的确切含意是

\emph{实} 尹 0. (3.18)

™ (0,。,0)

通过引透参数变换Fv(O,O,ele,可以假设

F(O,O,e) --- («(€),£)---\^{}=?\textsubscript{£}.

F £\textless{}。时结论的证明

对Y0,我们有

Fr(y \emph{,x'} ,e) = e + 静+ \emph{碧h,}

\textsc{=e} + 警k + 碧孫=e +'按或(1 + °⑴)\textless{} 0.

因此,系统Xe在4附近的轨道的结枸非常简单:除了奇点%(€),
闭轨皿3)以及连结它们的轨道/,(£)以外,当L十8或一8时,
所有的轨道都将离开A的邻域.

G结论⑴的证明

现在证明,当e\textgreater{}0时,闭轨乾3)有横截同宿轨.这一结论从图
5-9容易看出.事实上,由假设(3),流形卬-(\%)与沿为横截
相交,而匕是沿着主稳定方向趋于晚因此,流形W飞如)属于流形
"*(6)的边界,并且有边流形与流形IV(系统X。
的在相处与山,产的特征方向相切的不变流形)在,。上的任意一点
处相切,故由假设(4)的(a),它与流形IVYs)沿儿横截相交.另一
方面,上述所有的流形光滑依赖于参数e,故当e\textgreater{}0时,流形
町(Me))与俨(% (Q)将沿着在附近的一条曲线横截相 交.

\includegraphics[width=3.05972in,height=3.04028in]{media/image77.png}

图5-9

H正规同宿分岔值序列及正规环分岔值序列的存在性

为了证明结论(2),我们需要下面的

引理3. 7存在正规同宿分岔值序列玲和正规环分岔值序列
盾,它们具有浙近表达式(1 +。(1)斤(0)f.

证明

L正规同宿分岔值序列的存在性 由假设(3),奇点席(e)的
稳定流形叩(风(。)与S的交包含〜条曲线弗它与闭轨。心)的
不稳定流形{``=0)横截相交,并且当e=0时通过点贝=(0,1).因

此,曲线0可表示为

Cj: v = Vi(iM),(u,e) £ (Rz,O), (3.19)

这里Vi\textgreater{}-个C'函数,满足Vi(O,0) = l.曲线C:在映射卩一"的作
用下的象U,''为

u =沁)(3.20) 因此,如果

\& = F(0,0,e) = (a(6),e) £ \%,»,\\
则鞍点有同宿轨.g*CL■,当且杈当参数£满足方程\\
閃(e)丄£一")= o,vj,(e) = Z(QfK(m)W),e).

(3.21)

对任意给定小常数3\textgreater{}0,有

甜(0)=- 昨(0) vo, H?(5)\textgreater{}0,n»l, 故方程(3.
21)在区间(0,/)中有根,下证它是单根.

会如沪1-A*)

=1-3)ViQ(e)\%(Q,矽+Z定)f"'

\begin{quote}
(翠以5十隹)蛇)+勢(g,隹)+登日. =1+。(1), , (3.22)
\end{quote}

故V n»l,(3. 21)在区间(0,9)上有唯一单根,记作e.因为

碍=吗(玲)=\emph{以教-飞1 +}。⑴)\textless{} A"

对某个常数A\textless{}1成立,我们有 '

\$=(而))+ ``(例)一如(1 +。⑴) ,\ldots{}、

\begin{quote}
Zo ) d(0L(l +。(1)).
\end{quote}

在证明正规环分岔值序列的存在性之前,让我们先给出下面 的重要附注.

附注3.8令"(Q表示过Q的那条轨道.事实上,我们在上
面已证明了下述结果:设We(尹中,(方(矽)是一个光滑依赖于参数
e的二维U不变流形,如果W。与『(乾)横截相交,则存在一序列 方f
0,当\emph{代f} 时,:M\%)u叽■,并且当e通过\&时,点%以非零
速度接近及离开曲线\emph{w\textsubscript{e}ns.}

\emph{2.}正规环分岔值序列的存在性由假设(3)稳定流形『3)
与不变流形叽h=o}沿轨道横截相交.因此,叩(b】s)mr- 可由下面方程表示

\begin{quote}
\#=W,6), (V ,e) £ (R2,0), (3.24)
\end{quote}

这里方(0,0) = 0. - \textgreater{}0.因此曲线(茂网广(叩0(e))n

\begin{quote}
™ (0,0〉
\end{quote}

r-),即护0(e))nr*上的点的坐标・,z)满足方程

\emph{= h(ynf\textsuperscript{,},E').} (3. 25)

令\emph{法=s,}则(3. 25)变为

\emph{s = Kys\textsuperscript{9} \textsc{,e\textasciitilde{}).}} (3.26)

令(国。)是点/.nr\textsuperscript{+}的坐标,我们对方程(3.26)\^{}点(),財)=
(辨,0,0)应用隐函数定理可解得\$ =\emph{队S},s),或

\emph{z =} (3- 27)

我们断言,函数S有下面的性质

口 = 0(1),孺=。5十),票=O(»). (3.28) 由(3. 25)有

"宀制

或

\emph{法=}臺 e/ (1 ---\emph{备滤-B)} = 0(e).

这意味着

g = O(J).

为了得到(3. 28)中后两个估计,我们对(3-25)分别关于乡和£求 导数,得

"T寿=券3+紗宀叙, go)

\emph{"7隼}+耕1出=霸(宀皇+耕皿+瓮

(3.31)

由(3.30),

由(3. 31),

畝 {条I 一黑血\emph{+ 囁}挥T+ll心}

\begin{quote}
=辱-。)=。(»).
\end{quote}

令価表示曲线G-i。砂(的*)),设点(``仲沱以, 则``,''满足方程

Gz(M,*£)--- A(Gy(W,V,€),€). (3. 32)

由(3.17),对(3. 32)在点(",sQ = (0,l,0)用隐函数定理,可得

\emph{v ---} V\textsubscript{2}(«\textsubscript{f}€)"(0,0) = 1 ,w €
(R.O)» eN 0・

对(3. 32)求微:分,可得

因为曲线5,''属于闭轨"矽的稳定流形胪("Q),故若点

条=(小),e沱匀m则在\emph{A}附近存在包含奇点''(©)及闭轨ffi (e)\\
的环,而g定6,'',当且仅当参数£满足方程

---e-\textsc{V*(e)} = 0,

\begin{quote}
\textless{}
\end{quote}

令\#V1是一个小正数,満足

丽\_/\textgreater{}{]},业豆呉\textless{}1.\\
云(0) - \emph{s}

(3. 36) 因此,在区间上方程(3、35)存在一个根,记作硫.下面我们
证明耳是单根,即

(3. 37)

由(3.35)

響L

{[}疝(e)fT -沁)-"鲁\^{}\^{}u(e)

+碓尸噂)]一殓)-

由(3. 33)可知,如果RW1,则有

沁)

\begin{quote}
\emph{o€}
\end{quote}

如果成\textgreater{}1,则有

■ '\textsuperscript{p} .

風沪嘰。⑴回L吊J:

■

\textless{}0(1)

I结论(2)的证明

由引理3. 7,集合\emph{\textsc{B\textsubscript{h}}}和Be非空,对任意% £
\emph{B\textsubscript{c}\textbackslash{}\{0\}}由假设 (3)和结论(1),流形
护(%。))和W,(乾(%))与流形 昧(们(彘))
横截相交.因此内是集合B"和氏的聚点(参见附注3. 8).故集合
\emph{B\textsubscript{c}}的闭包\emph{Sc}是完全集并且是\emph{\textsc{B\textsubscript{h}}}的导集\emph{\textsc{B'h}}的一个子集.下面证
\emph{明\&是}Cantor集.为此只需证明它不含任何开区间.令\emph{众\textsc{Bh,
}}则系统X\%的奇点心(公)有一条同宿軌.因此,对3的某一侧位附近
的所有参数值e,轨道4(e)当+8离开A的邻域,这意味着\&
氏\textsc{Bh.}因此\textsc{Ek}的闭包瓦,不能包含任何开区间.另一方面EcU
\emph{\textsc{B'hUEh,}}故\textsc{Zc}不包含任意开区间,因而是Cantor集.

\emph{J}结论⑶的证明

由(3.17),

\emph{dGz}

我们设«\textless{}0,对于情况a\textgreater{}0,讨论是类似的.因为\textsubscript{a}\textless{}0,曲线Ct位
于以的上方.显然,只有那些通过U与6之间的点的轨道才能回
到区域6.令K。表示U与C1之间的区域,

\emph{K„ =} ((a,v) \textbar{}V\textsubscript{a}(w,£)It \textless{}
7i(w,e), 0 W `` W舌). 令

Z5= \textsc{\{(",v)\textbar{}0WkW} 1 十 +句,

\emph{K„=} n \emph{D,}

\emph{L\textsubscript{n}=} \{(a\textsubscript{f}v) \textbar{}0
AC®)\textsuperscript{-1-}"
\textsc{V\textsubscript{1}(A(e)\textsuperscript{b+1}w,e),}

\begin{quote}
0W Y1 +3\}.
\end{quote}

令硝,球是引理3. 7所确定的参数值序列,则球+1\emph{〈耳〈喋} 令\\
\emph{K =碧+1} + 产(0)Tl+''n,\emph{禺}=gle*,

则由玲总 的渐近表达式可得力 令S表示区间(旗,《)•对

于我们定义由\emph{b=K„\{jL„}到Z>的映射如下`

\emph{PCu.v')} = (A(£)W,/t(€)T>),

\begin{quote}
尸。梦《七。尸(3),
\end{quote}

\emph{\^{}■n} = 4'乙n---1 n K»,

则4在\emph{以UJ}上看上去像一个马蹄映射.参见图5-10.

\includegraphics[width=2.58681in,height=2.14028in]{media/image78.png}

图 5-10

引理3.9存在正常数旭,使得当n充分大,并
且\textsc{GES---i}时,映射\&在区域以U厶,上满足(任,任)锥形条件・

证明因为\&1七是线性映射F,故对任意常数用,阡,只要
W\^{}<1,它就满足\emph{g心}锥形条件.对于3,3 e巩,有

冬(u,u) = \emph{F}。.中。\emph{G}。尸(``,0).

在利用第四章引理4.1来验证4满足锥形条件时,所有运算 与引理3.
3的证明中完全一样,从那里可以看出我们只需验证

万(QfgiTf 0,对 e £ \emph{U„\_\textsubscript{lt}} ;!---+8, (3.40)
这里z是点G。尸(``,3的m坐标.对于\emph{心},(3. 40)的成立是显
然的.下面假设j8\textgreater{}l-因为,故点器=35) ,e)\emph{与点(3)}
=\&M)WL„---1的距离大于員醯―1一4).这样,存在正常数C,
使得点(0,0)£「-和点F-W,m)的距离大于2以苏\_1一玲).

因此.

/ \textgreater{}\textless{}7(慕\_1 一理),

或

沁)-事-y 沁)-由-i(思t \_球)»\textasciitilde{}1

\begin{quote}
=cH沁l *o)〈T〉Q+志叮t
\end{quote}

这里广隹)=次0)(1-新1+志)/\&(£).注意到

r(0)=万(0)-力 \textless{} 1,

故(3. 40)成立.引理证毕.\textbar{}

由引理3. 9,当射d在集合殄(\_14上存在马蹄・ 国而集合珥(0
=,\&出(珥U丄)是一个马蹄,氏在仏(e)上有
稠密轨道.由区间S\_i的定义可知,当*S\_i,奇点的不稳定流
形7。(启)当+8时离开A的邻域,因而是游荡的.另一方面,不
难看出A附近除了奇点以外的所有非游器轨必定与\emph{S}相交
于码(e)中菓一点.因此,集合"(Xe lu)是%(e)加上一个具有稠
密轨道的双曲不变集.这意味着\emph{U„\_!} U V", ''》1.令

国=U \emph{u'',}

«3\textgreater{}1

则

C7? U \emph{V\textsubscript{H}.} (3. 41)

最后我们来证明极限

\begin{quote}
lirn\^{}y- m( \{0 \textless{} £ \textless{} r \textbar{}e \$ U?\}) =
0. (3. 42)

"+范
\end{quote}

令 A =(彳+1,彳)» 碍 h» 磴=(6球)\emph{,}则

\begin{quote}
m ㈤)\emph{=3\textsubscript{n}-} \textsc{£+i} =万(0)-"抖,
\end{quote}

由(3.39),

m㈤)=O⑴元(O)T1+抄,

故

\begin{quote}
£ m(殆 U 风)=。(1帀(0)-«+''七

\emph{n---k}
\end{quote}

另一方面

\begin{quote}
8

ZmUQ =《=(1+。(1))元(0)-*,

\emph{n=i}
\end{quote}

因而

\begin{quote}
8
\end{quote}

»(足u用)

lim 弓 =0. (3, 43)

\begin{quote}
i8(、mGQ)】+藐

\emph{n=k}
\end{quote}

由《3. 43)不难导出(3.42).定理证毕.\textbar{}

推论3.10设向量场族Xe满足定理3. 5的条件,并且奇点队
的鞍点量为负,则使系统\emph{Xe}具有稳定周期轨的参数值集是一个
边界包含原点的开集.

证明 对每一个由定理L1可知,存在4的一个单侧
邻域U(£o),使得对\textsubscript{e}et7(€)\textbackslash{}\{€\textsubscript{0}\},x\textsubscript{(}有一个稳定周期轨,另一方
面,由引理\emph{\textsc{3.7,B\textsubscript{h}}}中的点是可以逼近e=o,故推论成立.I

在一般高维情况下,本章的京理L 1、定理2.1和定理2. 2的结
论也成立,读者可参见[Sil2,3]和DLJ.

\protect\hypertarget{bookmark348}{}{}第六章实二次单峰映射族的吸引子

从60年代兴起的动力系统的现代研究,其中心课题之一是双
曲理论.一个系统(流,微分同胚,映射)是双曲的,如果它的极限
彙是双曲的,即极限集中所有轨道的Liapunov指数一致非0(在讨
论流的双曲性时,不考虑沿流方向的Liapunov指数,它为0),从那
时起,以Peixoto,Smale和我国的宴山涛教授等人为代表对双曲系
统做了大量深人的研究,人们对它已逐斯有了比较完整的理解.
另外,当时人们曾相信动力系统基本上是由双曲系统构成的.这一
看法的根本转变是由于在70年代受到物理、天文学等领域的一些
重要动力系统模型的影响.Feigenbaum, Ruelle, H\&on和
L(gnz(他们的工作在70年代开始才受到数学家的重视)等人对
这些模型的大量计算工作表明,这些模型具有极其复杂的动力学
行为,它们似乎不具有双曲结构,相反它们应当属于非双曲翦畴.

对非双曲系统的研究,理抡方面的第F次实质勇破是
JakobsonE在80年代初取得的,他证明了对实二次映射族的一个
正Lebesgue测度的参数集合,相应系统具有关于Lebesgue测度绝
对连续的不变概率测度.这表明,在测度意义下,具有复杂的动力
学行为的非双曲系统并不太少.Benedicks和Carleson\^{}\textsuperscript{1}-\textsuperscript{1}\^{}完善了
Jakobson的结果和方法,并在此基础上,讨论了 H\&ion映射在具有
非退化同宿相切的参数值附近的动力学性质.继而Mora和
VianaWl将此结果推广到对某参数值具有非退化同宿相切的曲
而微分同胚族.与已取得的结果相呼应,PalisE]猜测具有同宿相
切(在高维情况,同宿分支)的系统应该在全体非双曲系统中稠
密.Yoecoz\^{}认为当前讨论此类问题的有效方法应该是,首先要
对某个系统有很好的了解(例如,Logistic映射族在a = 2时,Hfeon

映射在« = 2,6 = 0时),然后考虑对此(全局)分支值的开折.我
国学者在此领域也有一些值得提到的工作,例如{[}Cyl,2{]},{[}Wl{]},
{[}Z{]},{[}EZ{]}.在本章中,我们只介绍{[}BC1{]}关于二次映射族的结果
和证明方法.应当指出,这里介绍的证明思想在当前这一方向的研
究中是十分重要的.

\protect\hypertarget{bookmark351}{}{}\textbf{§1}关于单峰映射稳定周期点的存在性

本节我们要讨论单峰映射Z.Z-I,其中7 = {[}--- 1,1 {]},的稳定
周期点的存在性问题.

定义Li称一个映射rjf,是单峰的,如果它满足下列三 个条件'

(UDf是连续的;

(6)/(0) = 1\textsubscript{(}

\textless{}u\textsubscript{3})/在{[}0,口严格下降,而在{[}一
1,0{]}严格上升.

\begin{quote}
称f是C1单峰映射,如果\emph{f}满足上面三个条件外,并有
(u\textsubscript{4})/是C\textsuperscript{1}的,且尸(z)尹0,当C尹0时.
\end{quote}

令f是C\textsuperscript{1}单峰的,并令F是f的周期为会的轨道.我们称F是
稳定周期轨,如果z £ P,\textbar{}DP (x) \textbar{} 称F是超稳定的,如果0

\emph{€ P.}由稳定性定义,如果F是f的一条稳定周期轨,那么对x €
■P,有一个邻域U,使得对一切\emph{yeu,}有lim \emph{f醇}除了可
«\textasciitilde{}* + 8

能的=1外).

现在我们要讨论的问题:一个单峰映射可以有多少条稳定的 周期轨道?
1918年Julia证明了某类单峰映射至多有一条稳定的
周期轨道.1978年Singe网做了实质性推进,下面我们介绍他的一 些工作.

\begin{quote}
设f是映射J的Schwartz导数Sf(x)定义为 5窗-龄「・
\end{quote}

定义1.2 称f是S单峰映射,如果

(\&)/是仃单峰映射!

(\&)f是U映射;

(S\textsubscript{3}) 5/(x) \textless{} 0,x G 7,并允许 S/Xz) =---o0(

(s)f映j(y)= 3⑴,口到自身?

\textless{}s\textsubscript{5})尸(0) VO.

定理1.3如果f满足(\&),(\&)和(\&),那么每条稳定周期
轨至少吸引工=一1,0和1中的一点.

证明由简单计算和归纳法,下面前两个性质是容易得到 的,此处不再賛述.

\begin{enumerate}
\def\labelenumi{\arabic{enumi}.}
\item
  如果/,g € C\textsuperscript{3},那么• g)S) = (Sn(g3))g,Gr)z +
  \emph{Sg(.x).}
\item
  如M/e C\textsuperscript{5} 且S/(x) \textless{}O,VxG L那么 s(产)(z)
  \textless{} 0, m
\item
  \emph{\textbackslash{}f}丨在(一 1,1)中没有正的局部最小.
\end{enumerate}

如若不然,设1尸\textbar{}在,£ (-1,1)中有正极小.不妨设尸⑶)
\textgreater{} 0,那么舟C)=
0.注意到f是C\textsuperscript{3}的,以及Sf(3-)\textless{}0,r(3-\textgreater{}必
与户(少反号,这与尸3)为极小值相矛盾,特别,上述事实也说
明,如果\textbar{}尸。)\textbar{} =
1,那么至少在y的一边,\textbar{}尸\textbar{} \textless{} 1.如果 贝尹士
D是f的不动点并且I户。)I = 1,那么这个不动点至少 在一边是稳定的.

\begin{enumerate}
\def\labelenumi{\arabic{enumi}.}
\setcounter{enumi}{3}
\item
  如果f有有限多个临界点3称为\emph{f}的临界点,如果产愆)
  =0),那么对每个整数\emph{n\^{}l,f}的周期为n的点是有限个.更明确
  地讲,对每个"\textgreater{} 1,/■的周期为«的点是可分隔的.
\end{enumerate}

如若不然,令g =产并设有无穷多x G /,满足g(x)=工.由 中值公式,便有无穷多x
€,,使营愆)=1.由性质2和\emph{3,\textbackslash{}g'} \textbar{}没
有正的局部最小.因此有无穷多\emph{x,g'} (x) = 0.这与f (因此与\emph{g)}
有有限多个临界点相矛盾.

\begin{enumerate}
\def\labelenumi{\arabic{enumi}.}
\setcounter{enumi}{4}
\item
  如果\emph{a\textless{}k\textless{}c}是\&
  =产的相邻不动点,并且在区间{[}a,c{]}
\end{enumerate}

中不含g的临界点,那么寸。)\textgreater{}1.

事实上,由中值公式,存在\emph{u,v,a\textless{}u\textless{}:b\textless{}v\textless{}c,}使得g,GO
=g'(v) = 1.因为 g 在[a,c]没有临界点,g,Gr) \textgreater{} 0,x G [a,cl
于是由性质3,g'(i)\textgreater{}l.

6.如果xe是g的稳定不动点且\textbar{}g's)i vi,那么 定理结论成立.

因为H是g的稳定不动点,所以它的吸引域中包含勿的连通分
支具有形式[-侦)或者O,口([--- 1,口是平凡情况).首
先我们考虑5,s)情况.\emph{g}将此连通分支映到自身,但小不在工的
稳定流形内,于是3不会映入工的连通分支内.因此,只有以下三 种情况之一出现.

\begin{enumerate}
\def\labelenumi{(\roman{enumi})}
\item
  g(r) = g(s)(= \emph{r} 或 s);
\item
  g(r)=r 并且 g(s) = s;
\item
  g(r) =s 并且 g(s) =r.
\end{enumerate}

如果情况(D出现,那么由Ro\%定理,g在中有一个临界 点声,它被吸引到h.但是g
=产,那么存在使得f将/■映到 f的临界点.从而定理得证.

对情况(ii)(或(由)),类似于⑴,我们不妨设在(r,s)中无临界
点.那么由性质5,这两种情况都可排除(在情况(iii)时,考虑g\textsuperscript{2}).

现在我们考虑连通分支为[-1,5)的情形.此时一 1被吸引到
石类似的结论对□也成立.于是,对\textbar{}/U)\textbar{}\textless{}l,xe(-l,l)
时,定理得证.如果x=±l,定理的结论显然.

.7.如果gS) = \emph{x,} I = 1,那么定理的结论成立.

不失一般性,我伯设g,(x) = 1.如果工=士 1,•那么无需证明. 如果女£ (-1,1),
Ek性质4,有一个z的邻域(r,s)不包含g的其 它不动点.于是 g(y) \textgreater{}
\emph{y,y £ (r,x)}或 g(3F)\textless{} \emph{y,y} G (x,s).否
则,在H两边有点丿,使得营。)\textgreater{}1,因此营有一个正的极小点.为
了确定,假设gQ) \emph{\textgreater{}y\^{}y} G
(r,x).令4是\emph{y\textless{}x}中包含5,女)的 使得£(少\textgreater{}
3-的最大连通分支的最小值.那么g(d) = /(或者\emph{d}
=---1并由此c吸引一1).显然扌以)21,由此有一个点se (d,
工),使得营(矶=1.如果有;y e (d,矿,使得b。)= 0,则利用性
质6,否则由情况(iii),我们完成了性质7的证明,并因此证明了定 理.I

定理L 3有以下几个推论.

推论L4如果/■是S单峰映射,那么它至多有一个稳定的周 期点,加上在区间[一 1
,/(!)]中的一个可能的稳定不动点.

证明 因为/(0)=1,点和1被吸引到同一条稳定周
期轨.由于JCf)是关于■/■不变的,如果一条周期執道有一个点属 于
3,那么这条轨道便落入/\textless{}/).如果这条同期轨是一个稳定
不动点H并且0,由情形3,
\emph{\textbackslash{}f'\textbackslash{}\textsubscript{M}} W
1或者\textbar{}尸\textbar{}母,汀\textless{} \emph{1.}
如果是第一种情况,工吸引Lo.xj并因此吸引0点,如果是第二种
情况,r吸引\textbar{}\textgreater{},口并因此吸引1和0=广】(1).对mvo可类似地
论证.

如果有一个周期\emph{P} \textgreater{} 2的稳定周期轨,那么用完全类似的讨
论可以说明,户的最右边的不动点或者吸引尹的临界点,由此吸
引0点或者它吸引L于是,我们证明了 f在JCf)中至多有一条稳
定周期凱另外,从上面的证明我们也可以看到,在JO)中没有稳
定周期轨,它仅仅吸引一1.

现在我们考虑\emph{f}的稳定周期轨,它不吸引0或1-由定理1. 3, 它一定吸引一
1,并由此至多有一条这样的稳定周期轨・

下面我们证明,这样的稳定周期轨是JCA)丄=(一 1,/■⑴)中
的一个稳定不动点.如同我们已经看到的,如果这条轨道有一个点
在JCf)中,那么整条轨道在JV)中并吸引。和L因此,这样的轨
道必在JCfj丄中.由性质3J在JU)丄中至多有两个不动点,因为 ■/在[一
1,0]中至多有两个不动点.如果在中没有不动
点,那么\emph{f(.yy\textgreater{}y,y\&J(.f\^{}.}因此J在J(/)丄中没有不动点也
没有周期点?如果/在JUV中只有一个不动点,当/U)\textgreater{}1时,
我们有r=---1,因为/W Vy,当了\textless{}z.由此J在JCO丄中没
有其它的稳定周期轨.当0 Vr`(z) W 1时,由定理1.3/吸引
一1,因此丄中没有其它的稳定周期轨.最后,假设\_/■在J(£\textgreater{}丄
中有两个不动点,那么有一个不动点工使得0\textless{}/\^{})\textless{}1.由定理
1-3,/吸引一1点且在丿侦)丄中,/■无其它稳定周期点.I

摧论L 5设f是S单峰映射.如果fD \textgreater{} 1,则在J CO丄
中没有稳定周期点.

证明 如果那么由 3,/(x)\textgreater{}\^{},x€ (-1,0).
因此,在丿(f)丄中没有稳定周期轨.I

推论L 5可以看成是推论1. 4的部分证明过程,用它可以断
定\emph{JUA}中不含稳定周期点.

推论L6存在没有稳定周期轨的S单峰映射.

证明 一个经典的例子J3) = 1 - 2x\textsuperscript{2}.它是Ulam和V.
Neumann在1947年给出的.容易验证,质是S单峰映射.经0点的 轨道为0,1, ---
1,-由定理1. 3它至多有一条稳定周期轨,并且吸 引三点一 1,0,1之一.但是/(一
1) =一 1是不动点并且吸引这三
个点.因此,它们不能被其它周期轨道吸引.但是,由于
4\textgreater{}1它不是稳定不动点,所以須没有稳定周期轨.\textbar{}

附注1.7在本节我们讨论了 S单峰映射的稳定周期轨的存 在性问题.把定理1.
3和它的推论应用到映射族FGr,a) = /\textsubscript{0}(x) =
上时我们看到,如果a靠近2,那么判定\emph{f\textsubscript{a}}是否有稳定

周期轨的问题转化为讨论临界点轨道的性质(在§ 3定理3,1的
证明中,要用到这一性质).我们将看到,映射族\emph{侦如a} e {[}处,
2{]},向靠近2,在参数区间中有一个开稠的参数集,相应的映射有
稳定周期執.这意味着其动力学行为是简单的,不仅如此,更有意
义的是,在该参数区间中也存在一个正Lebesgue测度的参数集,
相应的映射没有稳定周期轨,并且其动力学行为是复杂的.由此结
果,我们讨论清楚了分支值a = 2附近的部分情况.

\textbf{§2 F(x,o) = 1 - ax\textsuperscript{2}}的基本桂质

我们将讨论A(a)=硏(0』)返回到一个固定的小区间尸= (一
\emph{S,s)}的方式,其中\emph{a} = exp(--- C).特别要讨论它们是怎样
靠近原点的,以及靠近原点的速度.而F3,4)的迭代的一些基本
性质对于理解这些问题起着重要作用.应当指出,尽管我们只对这
一特殊映射展开讨论,但是这种以Jakobson开始,Benedicks和
Carleson发展的理论是解决一系列此类问题的关键所在.

首先我们将广分成一列不相交的小区间的并= U \emph{I"}

1\^{}\textbar{}\textgreater{}° 其中七==exp(--- 0,而当戸 V 0 时,

---(---- C-J.-1L并记为 \emph{Ip =---}

引理2.1对充分小\emph{8\textgreater{}0,}存在国\textless{}2,使得对ae
[何,2丄 如果% € [- 1.1]和\&满足

函(勿,a)\textbar{} 2* 丿=0,1,2,``麟一 1,

并且

I砂皿\textless{}-?,

那么

2(1.9)匕 (2.1)

证明 令\emph{* = 眄=}sin\emph{淡}在此变换下\textsc{,F(h,4)}可以表示 为

--- ---arcsin(l --- a sin\textsuperscript{2} 扌。).

Tt Z

\begin{quote}
\emph{9}

cos \emph{---n}
\end{quote}

诜(饥\textless{}0 =--- -/2a sgn(5) \sout{,三 一 .}

Jcos或+国京"―刍 令 % = Fu(\%,a),Qu = au(Q"),其中。。=广(\%).假设
I初 \textgreater{}

1一2乩,= S1,\ldots{},j-l,但是区\textbar{} W1 --- 2战 记砰=會萨一如t •
F,・那么我们有

\textsuperscript{fr}-i \textsc{j} \textgreater{}-1

\emph{aF} = p'(n)]Ja『(句,a)反代'(为)!!(- \emph{2am.} (2. 2)

\emph{v---j} w=0

假设%充分靠近2,用归纳法容易证明\textbar{}狷W1 ---揷,,= £••,,
疋因此cos\emph{湛}M 0.于是1\emph{8趴3} I
\textgreater{}1.9.另外,我们有\emph{平'}(向) = 二说,血 IWS 以及
£广(为)=I -\textsubscript{r}■- - ■当 a£

J1 一务 S,2]时,立得

宜(E)IKL9 光 1

引02.2存在常数,\textgreater{}0和\emph{k(S)\textgreater{}0,}当角充分靠近2时,
对所有a£ [\%,2]和所有\textbar{}x\textbar{}法%\_1,存在使得
下面的不等式成立.

\begin{enumerate}
\def\labelenumi{(\arabic{enumi})}
\item
  \textbar{}歹Cr,a) \textbar{} 贏 =、・.. J 一,
\item
  log\textbar{}a\textsubscript{T}F(x,a)\textbar{}\textgreater{}rZ.
\end{enumerate}

证明记% =烈").注意到如果而且 f =-1 + e,那么 f \textless{} F(\&a)
\textless{}- 1 + 4\& 设 瓦[\textgreater{}
c\textsubscript{a}\_\textsubscript{v} 如果

I 如 ■,那么 \textbar{}。顼(%,(0\textbar{} =
\textbar{}2a7\textsubscript{0}\textbar{} 如果 2-'+'N 知 22-',

ZN2,那么

IM(\%,a)\textbar{} \textgreater{}
\textbar{}2晩•前丨\ldots{}\textbar{}2a\%\_J

N \%2-` ♦ 2a(l - 2 • 2」"+2) ••• 2\textless{}z(l - 4'`` . 2 •
2\textsuperscript{\_w+z})

\begin{quote}
=(2a)'2-'(l --- 2 • 2\textasciitilde{}\textsuperscript{w+2})(l -4*2-
2\textasciitilde{}\textsuperscript{2,+2}) ••• (1 -
2\textasciitilde{}\textsuperscript{!}) .=泌(1 \_号)(1 \_\$) ... (/烏).
\end{quote}

因为%充分靠近2,取,=*ln2,那么蜘庖即如从上
面的证明过程可知,对:=1,\ldots{},Z - 1,函(知a)丨2 1 --- \$
\textgreater{} 5- 最后,取*3) = (log2)T"log -1,得I 下面的引理2.
3要证明,当\textless{}5非常小并且吗充分大时,在参数
区间(缶,2)中存在一个小区间d,使得映射Sm/Af厂是一一对
成的,并且保持指数扩张"

引理2.3对任意充分小\emph{S\textgreater{}0,}任意正整数N和任意向V 2,
存在\emph{m法N}和一个参数小区间A\textgreater{} U(00,2),使得

\begin{enumerate}
\def\labelenumi{(\arabic{enumi})}
\item
  对任意 d,\&(a)W--- - 1;
\item
  是\&到广的一一映射且是映上的;
\item
  \begin{quote}
  \textbar{}\&FJ(l,a)\textbar{} \textgreater{}
  (1.9)'一1万=1,2,\ldots{},叫-1. 证明因为
  \end{quote}
\end{enumerate}

(1 + 勾+1) --- (1 + \&)=勾+1 --- \&

=2 --- Q + \textless{}2(1 + \&) (1 如)\emph{t}

当---身时,我们有1 + M+1 \textgreater{} \textbar{}(1 +专).

即1+6是指数增长的.另外,如果那么从 不等式

警=\_号\_2妙爲〈-争

可归纳地证明af釦心)是单调下降的.由这些事实,结论(1)稲
(2)得证.这是因为对上述的\emph{B}和N,我们选出包含2的小参数区
间21\textsuperscript{(}\textsubscript{0}C= {[}00,21由于1 +
\&是按指数增长,只要充分小,总有 \emph{m、}\textgreater{} N,使得 4(a) M
---扌,2 W v W 屿 一 1 \textsc{,q €} 但是氧(赁) \textgreater{}
\textless{}5,其中爲是的左端点.注意到\&,(a) ,a €占。是单调下降
的,所以存在A二得当时,\&,A,f广是一一映射.最 后我们证明指数扩张性.因为'

J

\textbar{}为应(1,\textless{}2)\textbar{} = U(- 2a\&),

y=l

其中\& = 1并且一1 M \& M \textasciitilde{}\^{}*\textsuperscript{v} =
2,\ldots{}W吗---1.从而 结论(3)得证.{]}

下面的引理2.4说明互卧和\&F"是可以比较的.更确切地 讲,M
和為F,的增长速度是相同的,我们已轻知道,

为卧+1 =-\/- 2『为殆,\emph{a\textsubscript{x}F\textsuperscript{a}} = 1

和

平+1 =-\/- 2『"-(尸尸,\emph{3\textsubscript{a}F\textsuperscript{a}} =
0.

因此

\begin{quote}
V---1

M = U(-25), v = 1,2,\ldots{} (2. 3)

:=0
\end{quote}

并归纳地得到

3瑚,=为时多专(1 + 涂)/ = 2,3,\ldots{} (2.4) 显然有

\emph{d\textsubscript{a}F} 一 x\textsuperscript{a}.

\begin{quote}
引理2.4 存在充分小的\^{}\textgreater{}0和㈤V2,\%靠近2,如果 ⑴ 1 一 2N*
\end{quote}

\begin{enumerate}
\def\labelenumi{(\arabic{enumi})}
\setcounter{enumi}{1}
\item
  \% W a V 2;
\item
  \textbar{}为玲-'(\%,a) \textbar{} 2exp\_f'3,j = 8,9,,``", 那么
\end{enumerate}

\begin{quote}
1 {\textbar{}\%Fl(\%,a)\textbar{}}

16、\textbar{}3顼1(知\textless{}2)\textbar{}
\end{quote}

证站 由连续性,如果d充分小,角靠近2,那么对任意%和
a满足条件(1)和(2),容易证明

五弁11 )\textgreater{}丄

\begin{quote}
2。吕「 \textbar{}2aVSa)"" 8
\end{quote}

\includegraphics[width=0.45347in,height=0.43333in]{media/image79.png}\includegraphics[width=0.47361in,height=0.33333in]{media/image80.png}

(2.8)

8

我们仅证第一个不等号,并只就\emph{声}\textgreater{}
°,茶〉\textsuperscript{0}情况给以证明, 其它情况可类似证明.

\begin{quote}
屏2 0,裟 \textgreater{} 0.如果\emph{3\textsubscript{a}Fi} \textgreater{}
0,那么法序2 0.所以1十 \emph{pi o}
\end{quote}

旎戸21---云瀚5・如果%戸\textless{}0,那么勺声\textless{}0.而0/'12

expWJ 于是因为 因

此

\textsubscript{1 +} \_\^{}\_\textgreater{}!\_\_8\_\\
十海尸身\textsuperscript{1} exp",

最后,我们来归纳地证明引理结论.\emph{3} = 7时无需证明,设

j时,有

因为

"许队 I 尸 I

頭吋的\_有丛I「+2辺同

\begin{quote}
fi + 一\emph{沙履\_\_} {]} \textbar{}许\textbar{} h +{屏E}

I十2崩小(E)丿\textbar{}丛\textbar{}丨十料旳")
\end{quote}

因此由(2. 5) --- (2. 9)可证

1 ' {"+W,a)丨} /`` .

行W "珂和7 \$ \textsuperscript{1}

附注2. 5 我们简要地说明一下引理2.1 ---引理2. 4的意
义.在§3主要结果的证明中我们可以看到,執道媚(。)条1将会
反复地进入临界点的小邻域广中.粗略地说,引理2.1和引理2. 2
根据兀的迭代的这种特性,分别讨论了当参数靠近2时,兀进入
\emph{r}前和走出厂后,在广外的扩张性质.引理z.3将作为我们归纳
地得到正Lebesgue参数集丄的基础.我们将F3,a)看成二元函
数,引理2.4说明\emph{F(.x,a)}的迭代对工和a是等度增长的.这一结果
的好处在于对参数或对变量的估算可以相互转化.在§ 3定理3.1
的证明中,我们将看到这些性质所起的重要作用.这些看似简单的
引理的意义还不仅仅如此•对于一般的映射族FGc,a),我们可以
提出与上面基本性质类似的假设.如果FCz,a)满足这些假设,那
么\textsc{F(h,\textless{}0}也具有类似于1 -損的重要结果,细节将在本章的小
结中阐述.

§3 F(x,a)不存在稳定周期轨问塩

在本节里,我们要证明下面的定理:

定理3.1 存在具有正Lebesgue测度的集合 厶C (0,2),使 得对所有a £丄,映射F(
, 屹1 一妃岸色J没有稳定的 題熱凱.

显然,F3,a),a £ (0,2),是S单峰映射.特别,由Singer理 论,当a 6
3,2)血靠近2时,F(z,a)在\emph{J(F)丄}中无稳定不动点
而在中至多有一条稳定周期轨;并且假如x = 0没有
被吸引到一条周期轨,那么F\textless{}x,a)没有稳定周期轨•于是,定理
3.1可以从下面的定理3. 2得到.

定理3.2 存在ZUU(0,2),丄具有正Lebesgue測度,以及
一个正整数蜘,使得对所有Q £丄,

\textbar{}为硏(1,。)\textbar{} 2exp(*3),
\textsubscript{v\textgreater{}yo},

证明这个定理的证明比较长.因此,我们将它分成下面三 个部分.

A我们根据\textgreater{}(0,«)不断返回到/•的特点,讨论返回的形式.
与此同时用自由返回概念,归纳地将参数区间4分类.

由引理2. 3,存在一个最小整数观J〉\emph{N)},使得

\begin{quote}
/1\textsuperscript{!} \emph{a}
\end{quote}

是4,到厂上的1 一 1映射.称旳是映射的首次自由返回的指标,
相应的参数首次分类为

厶1 = (o+l) U /\textsc{'「'(Ach).}

在下一步的讨论中,参数集合耳=4,/4不再考虑,从4中排除
呂是基于如下事实'因为F",(o,A)=尸,因此存在参数角G 4,,
使得0点是FU.oo)的周期为屿的超稳定周期点.为了保证定理
结论成立,将向连同發的一个邻域\&排除掉.请注意,实际上在参
数区间A中,我们全部排除了使兀可能有小于等于观I的稳定周 期轨的参数.

设B是第艮次分类的任意一个构成区间\#21.下面我们定义
第龙+1次自由返回以及相应的分类.

设=7''《•=[如切,\textbar{}``\textbar{} \textgreater{}
0.我们要讨论3在\emph{F} 下的进一步划分,定义非负整数\emph{P}
=少(3),它是使 聞S 一 \emph{Fg©} \textbar{} C鸟辭

成立的最大值.更确切地讲,对所有a £ ®和所有? G (0,c''\textgreater{}),

有伝)一*3,a)\textbar{} M 饗,顶=1,2,・``,夕. (3.1)

但是存在a£s和〃£ \textsc{(O,c``t),}

屁+危)---"+】(")\textbar{}\textgreater{} \sout{J 篇界,.} \textless{}3.
2)

应当注意,对某些\emph{P,\^{}\textsubscript{i+j}}怎)可能返回到厂,我们将它
们记为皿}如,并被称为有界返回.

令,'是最小整数\emph{i\textgreater{}p+l,}使得尸"*+'(0,3)PI广=丄*0.
下面要做的工作是,在任基础上,迸一步从d中排除一些参数集
合,使得在任的剩余集中,不含使\emph{f\textsubscript{a}}有小于等于\^{}\textsubscript{i+1}的稳定周期
轨的参数.不仅如此,相应于剩余集中的参数映射\emph{f\textsubscript{a}}至少在峋
时是指数扩张的.

令\emph{Si}是小区间\emph{ly}的指标集,满足(i)石U聂和(ii)
\textbar{}y\textbar{} M。+
\emph{k.}但是我们在s中排除掉满足条件\&,+:伝)£小,卩\textbar{}
\emph{\textgreater{}a + k +} 1的所有参数(记作瓦,y).

\begin{enumerate}
\def\labelenumi{(\alph{enumi})}
\item
  如果\& = 0,那么我们不做分类而是继续迭代下去.用 爆\textsubscript{1}
  =讯* + \emph{i}表示第一次非本质自由返回的指标,在独1之后,重复
  上述过程,令A表示可能的有界返回指标,小是最小的正整数使
  得它重新返回到\emph{I'}.此时伝)可能再一次出现.此时用\emph{n\textsubscript{ki}
  = \% + Px} + ?i表示第二次非本质自由返回的时刻,并继续这一过程.由
  后面证明中可以得到的映射的指数扩张性,这一过程将在有限步
  后停止(也参见{[}TTY{]}).在根.与昭之间的所有非本质自由返
  回我们用伝*}:=表示.
\item
  若§•尹。,令 f 表示3中映到\emph{1\textsubscript{7}}上的子区间\emph{,7}
  6 5,-
  那么R\textbar{}\textgreater{}«.此时,F"*+''(O,s)未必可以严格地表为区间\&的并.
  令
\end{enumerate}

K = F性+'(0,s)\textbackslash{}(( U 功 U (― c°+砰{]},d+``{]})).

K可能由0,1或2个区间构成.

\begin{enumerate}
\def\labelenumi{(\arabic{enumi})}
\item
  如果对某些由于每个刀毗邻于一个已选定
  的4,此时我们只要稍稍扩展相应的参数区间纳■,使得\^{}*\textsuperscript{+,}'(0,
  \emph{吋)}= ZrU K.这样的构造意味着对恰当的了和符号士有
\end{enumerate}

GUF"*(O,M,7)U0U\&±1・ (3.3)

由此,我们可以定义第\^{} + 1次自由返回.

\begin{enumerate}
\def\labelenumi{(\arabic{enumi})}
\setcounter{enumi}{1}
\item
  如果K中含有区间J,满足七,UJ,M\textbar{} =a,记F\^{}+,(0,
  ")=U糾并考虑它的进一步迭代直至与厂相交.在此基础
  上重复上面的构造.用此方法我们得到由U A*的一不分类已(%
  啲)=(«»:■,\textgreater{}\}■记线=U a(i,y 并定义为+1 为
\end{enumerate}

血+1 = U 练+1O).

a»W勺

附注3.3在构造分类R的过程中,每个参数的小区间

R,都与!'中的一个区间关联.这样的区间分成两类,一类区间是
丄,而另一类区间除包含形如七的区间外,它的端点落入丄毗邻
的区间内(见公式(3. 3)).为了避免引入复杂的记号,在下面的讨
论中,我们只就第一类区间展开讨论,而不考虑第二类区间存在的
可能性.读者可以仿照证明思想,对第二类区间得到同样的结论.

B我们用归纳方法证明,对第\&次自由返回和所有aU瓦,下 面的事实成立.

\begin{enumerate}
\def\labelenumi{(\roman{enumi})}
\item
  如果㈣=吗念)是第\emph{k}次返回的指标成\textgreater{} 1,那么
\end{enumerate}

\textbar{}3\textsubscript{I}F\textsuperscript{n,}4-\textsuperscript{I}(l,a)\textbar{}
2exp(2miP);

\begin{enumerate}
\def\labelenumi{(\roman{enumi})}
\setcounter{enumi}{1}
\item
  如果1 MiM皿(a),那么
\end{enumerate}

5(1,a) I 2expi"3;

(hi) \textgreater{}exp(--- \emph{•/\textasciitilde{}j} ), 1\textless{} j
C m*.

我们注意到对\& = 1,由引理2. 3和叫的定义,显然上面的
⑴,(ii)和(iii)都成立.

下面我们对应+ 1证明(i), (ii)和顷)成立.令3是瓦中一个
小参数区间,并令力=\emph{P3} 网和``如A中所定义.我们也假定旬
中的小区间``映到区间孔,以\textbar{} V产时,归纳步骤成立.下面我们首
先估计P的上界.由中值公式

\begin{quote}
弓(``)-''(\%\textless{}!)--- FPT(l,q) - "一1(1 -
\emph{\textsubscript{a}f,d)} =4矿\&砂一1(1一对2,\textless{}1), (3.4)
\end{quote}

其中 0
\textless{}7\textsuperscript{,}\textless{}7\textless{}c\textsubscript{p}\_\textsubscript{1},由(2.
3)和(3.1)对 我

们有

{庖糜1 一时,a)\textbar{}} 行 {成(1 一妙,4)\textbar{}}

\textbar{}a『(l,a)\textbar{} \textasciitilde{}丛
\textbar{}H(l,a)\textbar{} VW2,(3.5)

其中c = exp(\^{}£日.同理可证

{IM(1 ---硏。)丨}卜丄\\
l"(S `` "2,

容易看到,(3. 5)和(3. 6)对0〈矿〈〃也成立.因此,由(3. 1),(3.
4)和(3.6)我们得到

M亲旧怎)一州(孔©\textbar{} £ 謗V \emph{玄}烏exp(2 E W exp(2 g 只要j
---iWm''那么由(ii),exp(2 /?)Nexp(严)•但是由引理 2.
3和不等式(3.1),当S充分小和向充分靠近2时,我们有

' W 2\#(a + j»+V2¥(ai + 房)

W 157(靛1 + 犊Q W 15-牝 \textless{}". (3.7)

此式也说明,当j增加时不等式\emph{j} M 2故十在\emph{j =
m\textsubscript{k}}之前就破坏掉 了 .因此

PW2M*. (3.8)

令a = \emph{(a,bf.}我们讨论\emph{n =} 的长度的下界.

邛可以表为

\begin{quote}
I糾=\textbar{}E»+P+l(0,4)--- I .

\emph{={[}FX1(釦 s),a)---砂+1 苦% ⑶,占汁 (3.9)}
\end{quote}

因为

I\&*Q) 一扌叫0)\textbar{} =
\textbar{}。一石\textbar{}\textbar{}\%广*(0,廿)丨\\
N g ---们•佥expm¥\$, \emph{a\textless{}a' \textless{}t\textgreater{},\\
}并注意到M对纟和a的导数是同阶的,而从上式可见\emph{\textbackslash{}a-b\textbackslash{}}与\\
I\&Q) --- \&冲)\textbar{}比较是微不足道的(也可参考{[}TTY{]}).因此

\textbar{}糾濠■1\^{}031 \textbar{}\&0) - 轧3)\textbar{} \textbar{}为''0
\textbar{}

貝扣, l/\&F",a)\textbar{}, (3.10)

其中1 一播

由(3. 2)和(3.4),存在 7,r,0\textless{}r
\textless{}7\textless{}\^{}-1\textgreater{} 使得

"\textbar{}小%1 - 寸@) \textbar{} \textgreater{} 备讐暖, (3.11)
利用一致估计(3. 5)和(3. 6)我们有

\begin{quote}
\textbar{}西"(夕@)\textbar{} \textgreater{}-\^{}19\^{}(1
\end{quote}

将(3.11)和上式代入(3.10)

ici \textgreater{} 1 \emph{\textsuperscript{Cfi} \textbackslash{}1
\textbackslash{}} {丨\&+i(Q)丨} g ''、

冋法説右仔』3+安 ③⑵

我们要进一步证明\textbar{}fi\textbar{}\textgreater{}exp(-2z(\textsuperscript{s/8}).事实上,由假设(iii),
底+13)1 2 exP(--- \emph{-/pTi).}利用(3. 8)并注意到 *Na =
log\textsuperscript{J}{[}\textbar{}{]},那么当\$充分小,

\&\textbar{} =exp(---\emph{』p}。)一 exp(--- /V)--- ,

I钉+l(a)\textbar{} \textgreater{}exp{[}-( {]} + 2抒)可,\\
\sout{d =竺p} -厂\^{}\^{}),.

\textsc{\textsuperscript{c}a---}1 exp ' 2 \emph{J卩 \_ {]})}

\begin{quote}
\emph{\textbackslash{}0\textbackslash{} \textgreater{}---} Q +
2+】)・2expG(QN 厂弟,(3.13) 60 V\textgreater{}
\end{quote}

下面我们对\& + I证明⑴ 成立,首先证明对V a E血, la\^{}i+\^{}d.a)!
Nexp(2S* + Q* + 会\#*) • (3.14)

由微分链式法则

=否"厂1(1,侄)(一 \emph{2a(\textsubscript{m} (a)) •}

i

为FP黄叫+1@),a) \textbar{}2a氧(e)为网(紐+心),a) \textbar{}

\begin{quote}
\textgreater{} 2aexp(--- 序)• \&xp(2 \emph{JL} 1) • 彳备沖(-少+ 2
QT)•帰帯.
\end{quote}

因\emph{为''》為埒、}并且\textbar{}弓+1愆)\textbar{}
\textgreater{}exp(--- 上式可表为

\textbar{}球《5(京+心)優)丨鴻温可•

exp(2 --- 1 --- //r --- \emph{Jp} + 1)》exp(+/\textgreater{}*).

(3.15)

由(3. 7),砰》15出从而对所有a £ 3,\\
庖 E+P(l/)\textbar{} \textgreater{} exp{[} \emph{2mf + 十声*)\\
}濠 exp(2O*+ 见专 +

这是因为(戒? +会A (wi* + ",

由分类定义,设对f = h.\^{}+\^{}CO.a)与/•相交.那么

\textbar{}招冲+3+:L(l,a) \textbar{} \textgreater{}exp(2(m* \emph{+
p +} h)b.\\
事实上,由于 \textless{}\#,并且 \textbar{}F《%,a)\textbar{} \emph{R,j
=}

一 2,其中 % = F\^{}+\%1,@),所以由引理 2.1\\
\textbar{}粕-1(從)\textbar{} 2(L9)L.

于是

\begin{quote}
\textbar{}为砂,+Z\textgreater{}K\_l(l,a)\textbar{} \textgreater{}
(1.9/Texp(2S* + *)* + 备疗) N exp(2(m* + 力 + 标)4).
\end{quote}

如果是自由返回,那么记m*+i = m\textsubscript{t} + /\textgreater{} + ;i,
并且对\emph{k} + 1,我们证明了(i),如果是非本质自由返回,由(3.13),
时,+\#+七(0仲)与九,\textbar{}叫〈产相交.重复前面的讨论,并在下一次
返回时,我们有

\begin{quote}
'IV"L(l,a)\textbar{} 2exp(2''£).
\end{quote}

如此继续下去直至到吗时,返回是自由的时侯为止.自由返回在
有限次非本质自由返回后必会出现,这是因为s的象的长度,如同
我们在A中看到的一样,在返回时是非常快地增长的.我们完成了 «)'的证明.

当* + 1时关于W)的证明,我们只对给出证明. 而对时间段\emph{叫〈V 吼}
的证明是类似的.

\begin{quote}
J j+l
\end{quote}

如果i 一皿V内V皿,其中\emph{p = pk}如(3.1)和(3. 2)定义, 那么我们有

=区评-f (最+l(O,a) ,

》exp(2m;〃)\textsc{・2kP(-} 77)1
勿戸T-叫(氧+i(a),a)\textbar{}.

由一致估计(3. 5)和(3.6),归纳假设(ii)和m* \textgreater{}产,

Nexp(2m? --- ■/\}? + (1 --- 1 --- m*)\#) 2 exp\#.

如果 \emph{i\textgreater{}p\textsubscript{k} +
m\textsubscript{k},}由(3.14)

\textbar{}杏尸」1(1")\textbar{}

=\textbar{}\%E`+A(l,a) •布尸-\%---》厂1(8叫+a+{[}(a),a))\textbar{}

2 exp(2g + 力)* + 佥讨)l/FF-l(机+A+1MB 卜. 由引理23,我们得到

低不一叫"L(氧+p,+i(e),a)丨 \textgreater{} 5exp(r(t 一唯一勿一{]})).
注意到

\begin{quote}
2 12 1 \emph{2}
\end{quote}

\emph{2(m\textsubscript{t} + fit)\textsuperscript{3} +
---/\textgreater{}\textsuperscript{3} + 7(I --- 1 --- p\textsubscript{k}
--- mQ} --- log y \textgreater{} 技. 从而(ii)得显.

(iii)的证明.要证当 \emph{m\textsubscript{t} \textless{} j K}
m\textsubscript{i+1} 时\textgreater{} exp(--- /7).由引理2.
3,只要a。充分靠近2,总可适当选取 使得阳\textgreater{}
\emph{2d}注意到\emph{屿}1,我们有

况* 2 \emph{\textasciitilde{}2\^{}\textsuperscript{m}t +} 2 \& + 1 + a.

因此,当时

\textbar{}烏 \textbar{} 2 exp(--- Ja + 点 + 1) \textgreater{} exp (---
\emph{Vm\textsubscript{k})} \textgreater{} exp(--- \emph{R).}
C最后我们来估计,在构造血+1时,从旬中排除的参数集的
测度.更确切地讲,我们要证明

\textbar{}g\textbar{} 2 \textbar{}血丨(1 一9, (3.16)

\begin{quote}
8
\end{quote}

其中0Way 1且U(l-4)\textgreater{}0.如果(3.16)得以证明,那么由

\begin{quote}
*=1
\end{quote}

归纳过程我们有

\begin{quote}
n---1
\end{quote}

14.1 \textgreater{} \textbar{}1 一 J I U I》\ldots{}\textgreater{} U(1
--- G 心 \textbar{}.

注意到4+1 U\&,我们有

Al = I n4t\textbar{}\textgreater{}0.

*=1

并完成了定理3. 2的证明.在证明(3.16)之前,我们暂时假设对所 有%,個G
3UJ、存在与互无关的常数c,使得

\begin{quote}
V {\textbar{}*中(0,代)\textbar{}} 〜
\end{quote}

c ; \textbar{}"E(O,s)\textbar{} f \textsuperscript{C}'

令\textsc{q = u} F"妇(0,3)可以表示为

\textbar{}P«E(O*)\textbar{} = \textbar{}砂5(0\^{}) ---
砂+(0』)\textbar{}

=叫FT(氧+p+i(a)Q -

与推导(3.10)相似,我们得到

\textbar{}尸%1 (0仲)\textbar{}

\textgreater{}号旧『中-叫一1(") \textbar{}
\textbar{}命+小8)-上++(方)I , 其中H在%+a+i(a)和氧+*10)之间.因此得

(1) \textbar{}F'(8,\textless{}z) \textbar{} N 凯丿=0,1,2,•••,/»*+!
\emph{- m\textsubscript{k}- p- 2t}

\emph{(2) C3.}

利用引理2、1

\textbar{}砂中(0,也)丨

\textgreater{} y(l. 9)FF-"\%叫+*(a)-氧+宀0) h

9 j.

=号(1. 9)叫+厂叫一》一1\textbar{}糾\textgreater{}exp(- 2舟).
我们也有

\textbar{}E+i(0,a)\textbar{} = \textbar{}\%FF(0,次)1 I叫, 其中W
£a,于是,我们得到了 a长度的一个下界

\begin{quote}
即\emph{参}.{燮- 2満)}

\textsuperscript{1} ' - .
\end{quote}

另一方面,令刃表示a中可以包含于A+i的小区间全体.根 据四+1定义,记鼠+1 =
w\textbackslash{}a/,则

l\^{}™\textsuperscript{i+1}(O,wt\textsubscript{+1})J \textless{}exp(---
/a + A +1).

而

\textbar{}E+i(O,祐i)\textbar{} = I弟+11,

其中术£添+1,因此

{exp(--- /a + H)\\
}一\textbar{} 讣f(O,a〃)\textbar{} '

利用(3.17),我们可以估计从\&中排除的参数集的测度

\begin{longtable}[]{@{}ll@{}}
\toprule
\endhead
\begin{minipage}[t]{0.47\columnwidth}\raggedright
{丨如一\textbar{}«/ \textbar{}}

1刎 '

因此,有\strut
\end{minipage} & \begin{minipage}[t]{0.47\columnwidth}\raggedright
\strut
\end{minipage}\tabularnewline
\bottomrule
\end{longtable}

I血I - I4t+il ' ...\_exp(--- -/a+T+T)

nr{]} W const r- = «i

\begin{quote}
\textsuperscript{141} exp(-2``E)
\end{quote}

和

\begin{quote}
\textbar{}四+11 (1 --- a*) \textbar{}\textless{}i*
\end{quote}

因为I闵\textless{}a +如所以四£戲,其中

\emph{队}---const exp(2(a + 幻M --- ■/« -pA + 1).

8 8

容易验证IT (I f \&) \textgreater{} 0,因此U (1 - M \textgreater{} 0.

XI \emph{k --- \textbackslash{}}

附注3.4读者可以看到,在证明(3.16)时,我们仅考虑了从
,次自由返回到\emph{k +} 1次自由返回的过程中,没有非本质自由返
回.更明确地说,我们仅■考虑了"=吨+1情况,对于一般情况,读
者可仿照上述过程证明间样的结果.

下面证明(3、17)成立.我们将给出一个形式上更一般的结

论.

.引3 3.5设砂,(0,3)那么有常数q和\&,使

得对任意\emph{a,b£3}和丿 \textless{} 所》+1,下列不等式成立

(3.19)

证明 显然,我们只需证明(3.18)成立.(3,19)式的结果可 从(3.18)和引理2.
4推得.(3.18)式等价于

崗0)\textbar{}

\begin{quote}
l\^{})T \^{}\textsuperscript{C,}
\end{quote}

由归纳的结论⑴,得

22 \textbar{}ET(/)\textbar{} =

因此,利用引理2. 4,得

\textbar{}如 Wdexp(--- 2邳).

因为m》l,

囹'w (1 + 备卩 M (1 + "exp(\_ »d))y 2,\\
于是(W)'在对参数的一致性估计中是不重要的因素.为了证明\\
(3.18),我们只要估计]1 譲勢*因为

吝 \textbar{}£0)\textbar{} ' 丄修川)一\&sin

口吋1 =咬 g\textsuperscript{ln}\textbar{}\textsuperscript{1+}
\textbar{}\^{})\textbar{} ))

b 踞旧。)一\&(4)\textbar{}

W exp応一偽(a) I .

所以我们只要估计

„ I演)一")丨

3 -

令佑}蔼为自由返回或非本质自由返回的指标,并且专V知
如果f\textless{}j,那么S可分段表示为

\begin{quote}
簽芙``W)一膈)1簽V
\end{quote}

\textsc{\textsuperscript{5} = 2j 2j} ---旧商 =匕为.

\begin{quote}
\textgreater{}-1 v=(.
\textsuperscript{\textbar{}C}-\textsuperscript{WI} 斤1
\end{quote}

由有的定义,q e加,并且书=(句伝),句8))u\%.我们将,分 成两部分 ' '

„ I\&3) ---句。)\textbar{} 弓*` \textbar{}5怎)一\&。)\textbar{}

月=* ")1 +『謂 険)1 '

首先估计第一个和式,当卩= \textless{}\textgreater{}
时,\sout{吃:二;件\textsuperscript{1}} M 导;当勺

\emph{+ pj} 时,记 \emph{S = V ---}如显然/ = 1,2,
---,/\textgreater{}\textgreater{},此时

厲(仅)一 "3)丨=\textbar{}玲(匕 S)M)--- F;(\& 9),为)\textbar{}

\begin{quote}
\emph{i I}
\end{quote}

勺 \textbar{}囱\textbar{}.

同时

.\textbar{}\&(a)\textbar{}=\textbar{}F,怎怎)g)\textbar{}

\begin{quote}
\emph{i}
\end{quote}

\textsc{\textgreater{} \textbar{}F,(O,q)\textbar{}} ---
\textbar{}F,(q(Q),a) - FM0,4)L 记\&*)= 〃,则 .因此由(3.1)式,

0a) I \textgreater{} \textbar{}?,(«) I ---法I \textbar{}.

我们得到

成 {0)1}

\emph{q} I 蜘)1

(3. 20) 我们把不等式右边的和式分成两部分

\emph{\% f; f,}

月=力十\emph{i -}

\begin{quote}
5=1 1=1 \$---时+1
\end{quote}

其中\emph{pi} = I Vw.第一部分我们用基本估计

IMI",

\begin{quote}
\emph{.\textbackslash{}W \textbackslash{}} \textgreater{}e-
\end{quote}

第二部分我们利用(3.1),(3. 4)和(3. 5)推导出

\textbar{}割1(1," £ 詞\&(a) ---F,(")\textbar{} M 亦志 lg)\textbar{},
其中咋 g,糾\_\textbar{}).于是

\%\textbar{}丄」旧此一1(1,0)\textbar{} \textbar{}加丨

V戸.

因此

2 =象+套日4也\textsc{mi}吳"十华!{]}

\$=1 sT g + 1 I s=l \emph{气} 〃 + 1 ''

注意到\emph{Pi'} = j Vw和土丨\emph{Ui W*} 我们有 安'{l\%(a) ---4(6)1} V
``顽 {Wl}

因为当勺+ \emph{Pi \textless{}v\textless{}
t\textsubscript{j+1}}时,句(a) n /■ = 0以及

庵詩)一 \emph{\&揷)}I 2号杞此f-"`a) I I物)一晶(6) I,
其中0在5(4)与之间,我们有

\begin{quote}
\textbar{}\&(龄-£0)\textbar{} / 3 \emph{t}
lOl\^{}\textsuperscript{-1}* 驼+«)- \emph{\&揷)}I
一而可一0万\textbar{}的J 1句+,怎)1
\end{quote}

事实上,旧怎)\textbar{} \textgreater{}
\textbar{}弓+[伝)\textbar{},并由引理2.1

\textbar{}古尸;+厂"Sa)\textbar{} \textgreater{} (1.9)',+「",

于是

\textbar{}腿)一伊)I W专鼎``定詩)一鈴件)I.
因此,第二部分可以估计如下:

W {13) --- 60)1}

\emph{\textsubscript{\textless{}}1} 卩O\^{}+L {I句+\&)\_ 句+Q)I}

£如思+J司 呢孩)1

\textsc{\textless{}Ay(io1\textsuperscript{j}} {庇+S ---与璀I}

毛万4【1时 呢,(心

{竄3-队仞1\\
}I氣s -

\begin{quote}
'/+L
\end{quote}

\emph{s} \textless{} const 2 -7=学V

i V\^{}\textgreater{} \textsuperscript{p}'

为完成(3.18)的估计,我们只要估计上面不等式右边和式的
界.完全类似于(3.13)的估计,我们有

\begin{quote}
\textbar{}川+1(\%,次)\textbar{}》exp(\_ 2游), (3.21)
\end{quote}

其中囱£農,切,% =下面我们要证明陶+1! \textgreater{}
5囱I.为此,先证明在有限的时间段 \emph{KKP,}中,\sout{鴛:⑶糾} \emph{la/s
血)[\\
}的一致有界性,其中气,工合£七"3如6欢

\begin{quote}
1---1
\end{quote}

{\textbar{}\%刑(工1,绮)\textbar{}} =(" {巨成 愆",}

\begin{quote}
\emph{V=O}

\emph{I \textsuperscript{a}i\textbackslash{}'}」頌
{\textbar{}Fu(Zi,ai)---狄(丑,企)1}1
\end{quote}

w M丿 \textsuperscript{exp}l \emph{:}---g\&i r

由(3.1),当vW物时

\textbar{}F*"(H,a)---知(\textless{}z) I M \sout{":V 丨},z £ 弓,a 任
\emph{a\textgreater{},}

我们得到

I硏(跖,角)一卧(务血)\textbar{}

M IA(ai)I + \textbar{}\&伝2)+ \textbar{}\^{}(«1) --- \^{}(此)I,

和

\textbar{}殆(处,%)1 \textgreater{}
y\textbar{}\^{}(0,a\textsubscript{2})\textbar{}.

于是

{\textbar{}F"(;C1,勿)一户"(以,\%)丨}

\textbar{}尸6血)[\textsuperscript{\_}

由归纳的结论(ii),引理2. 4,(3. 5)和(3. 6)

\textbar{}"-1(瑚)\textbar{} Vxp(2 4),

IM广(l,a)l N«tP(2罗).

因此

\textbar{}\&(角)---f\textsubscript{M}(a\textsubscript{2}) \textbar{}
--- const\textbar{}\&F*-l(lM) \textbar{} \textbar{}向一此\textbar{}

\begin{quote}
W const \textbar{}9\textsubscript{X}F*-\textsuperscript{1}(1 \textbar{}
\textbar{}角---\emph{a\textsubscript{t} \textbar{}}

C const exp(2 \textsc{a/\^{})} \textbar{}鬲 一% \textbar{},
\end{quote}

以及

\begin{quote}
\textbar{}勺愆)一亀(6)丨
\end{quote}

论 一 K \sout{15-1(丄)\textbar{} -}淄眦 exp(\_ 2痒.\\
注意到\textbar{}嬴饥)\textbar{}濠exp(--- O 和立得\\
頌 屮(时。1)---欢《气血)1 ,

\begin{quote}
\textsc{2j} 77J7 Ti W const,

£ \textbar{}殆(以,角)\textbar{}
\end{quote}

因为囹'是一个有界数,所以

广二 r V const \textsubscript{f}

\begin{quote}
\textbar{}\%尸(工2\& I
\end{quote}

其中册 e \emph{Ip.,} a* a\textgreater{}.

由引理2.1和上述估计,我们得到

\textbar{}bj+"= \textbar{}P\textgreater{}+i(0»a) ---
\textsc{Pj+i(0,Z*)1}

\begin{quote}
=I尸心M,+3+l)(q+3+i(a),a)- F«,zM,+D+I)(q+p,+10)0)\textbar{}

Nconst呢+3+i(a) - q+D+i(6) I r {I"叫叫},

2 const jjT 1。打

\textgreater{}const
0\textbar{}exp(\textbackslash{}/Z)exp(\textasciitilde{} 2才).
(3.22)
\end{quote}

因为\$很小,所以(3. 22)意味着\textbar{}b,+``N5]巩.最后,我们估计 和式

要注意,某些円可能同时位于某个区间L中.当此情况发生时,由 于血+1N
5囱\textbar{},我们有血I \textless{} I''⑴I, J(s)表示这些\emph{。代L}的

最大下标,由此有

\includegraphics[width=0.91319in,height=0.45972in]{media/image81.png}\includegraphics[width=0.65347in,height=0.41319in]{media/image82.png}于是

-s s

\begin{quote}
/ VI 1 \textbar{}bj(\$)丨
\end{quote}

不等式(3. 24)表明和式亩的\emph{杓}不妨是不同的•将和式

V' {1 丨勺丨\\
}叨

分成两部分,令儿是使得

方冊5

的指标集,而儿是剩余指标集.那么

. QO

S 、後 M const,

即上式左端是一致有界的.如果ye则

由(3,22),如果 \emph{k\textgreater{}j}

\textbar{}ct*\textbar{} \textgreater{} const
\textbar{}q\textbar{}exp(y\^{}J) exp( --- )

\begin{itemize}
\item
  \begin{quote}
  const \emph{插}加 \textbar{}exp(V\^{}i exp( 一 2,)
  \end{quote}
\item
  \begin{quote}
  exp( --- 3力).
  \end{quote}
\end{itemize}

但唳U/七由加的定义立得件C 9\#.此不等式说明{的}心,有

限.因此

2声件、声\\
\emph{jWJ*} V \emph{fij \%} 丿€七 V 巧

是一致有界的.于是,我们证明了引理3. 5,从而定理3.2证毕.\textbar{}

\protect\hypertarget{bookmark395}{}{}\textbf{§ 4}分布问题

我们仍用\emph{如}表示点\emph{工}=0的V次迭代,并且令

岡=4立如 (4.1)

在本节中,我们讨论四的渐近性质.更确切地讲,我们将证明下面 的基本定理.

\emph{定}理4.1 对于前面定义的丄中几乎所有的参数,序列
仏广的"、极限产是鲍对连续的:毎=Mr,其中〃,对所
有\emph{P\textless{}2}成立.

附注4.2这个定理表明,测度产不是很特殊的,并且临界点
的轨道关于测度``遍历,为证明这一定理,我们将依照§ 3中所得
到的基本性质分步进行讨论.我们首先讨论在广上的分
布,然后讨论在[---\emph{i,U\textbackslash{}r}上的分布,由后面的讨论我们知道,
\textsc{e-}i,uy 上的分布不难从广上的分布得到.因此,讨论清楚
厂上的性质是重要的.

我们已皎知道,\&8), 丄以三种形式返回到厂,自由返
回,非本质自由返回和有界返回.我们首先证明在r上自由返回
x\textsubscript{B}( = 的分布具有有界密度.其次,用类似的方法可以证明非
本质返回"{£»,}的分布具有有界密度.我们最后证明有界返回{少丿
的分布具有密度g £ \emph{Lf,p\textless{}2.}

定理4,1的证明

1.自由返回的分布问题的讨论.

记4 = 4,不妨设\emph{A}的Lebesgue测度为1,即臨=1.这样,
我们可以把它看成概率测度.在此基础上,我们可以引入期望E和 条件概率等概念.

取定1 MM ••• M V,和間必m,定义

==化 w e \emph{I \%、j} 一 1,2,---,z\},

或者财=\textsc{a}由前面的讨•论我们知道,这种形式的参数集合可以分
成一列子集\# 3),其中每个x (0相应于一条不同的``路径''i,该
路径在第%次自由返回时,通过区间\emph{1\%}我们定义

4, = \{a £ \# \emph{\textbackslash{} x„(a)} 6 \emph{h),}

= \{a \& " I 弓(a) £ \emph{I\textsubscript{u},} x„\textsubscript{+1}(a)
£ 丄,路径 i\},\\
心=1)収\&*).

对每个路径\emph{i,}由(3.13)\emph{丄} 可扩增为l个区间4,(0,其中

\begin{quote}
lAXi) I =乙(用)2 弦卩)=exp(---\textsc{2k*).} (4.2)
\end{quote}

由分类的意义,我们知道珏(D可以分成一些区间孔的并,由前面 的引理3.
5,我们得到

. IZ I

m\%(D M const (4.3)

为完成情形1的证明,我们首先给出两个事实,并把它们归纳 为下面两个引理.

引理4. 3给定\# = 或U 4•有一个与"无关的

常数Q,使得如果对某常数\emph{Q \textgreater{}} c°,下式

mA* M 2Q \textbar{} A, \textbar{} \emph{mA}

对所有的v成立,那么,对所有``我们有

瑚'户 MQIWmA

证明由(4. 3)我们有

mA`` ---必)、\emph{raA\textsubscript{llfl}(i') +}

\emph{v,i} KlMYz)

曰41 S\sout{况n}〉mW).

心\textgreater{}g) 、' ' \emph{k}

因为

亞烏沒心)3山零,

并注意到、心\textbar{}/殓0 V+ 8,所以适当选择g,我们有 S制§m处⑺M糸m\# .

对\emph{,}\textless{}
g,存在一个常数氏,使得\emph{L(.\textsubscript{V},i)}法\emph{L\textsubscript{o}.}于是

N 瑚*) 加 8瑚祯)具争財m\# ,

认Yq\textgreater{} S 3佑 \emph{乙}

其中% 2 (2c/Z\textsubscript{0}).于是

(号+号\textbar{}成冋\#京心\textbar{}m\#. \textbar{}

\begin{quote}
附注4.4引理4.3给岀从V到``的``转移概率''的一个估计. 引理4.
3的一种特殊情况为
\end{quote}

引9 4.5假如

An = \{a w \textsc{AJzrGi)} e \emph{I„\},n} = 1,2,---

那么对所有的如,

mA«Wco\textbar{}A,\textbar{}.

证明我们用归纳法来证明这个引理.当n=l时,由引理 \emph{2.3}可得

M \%!匕 I,\\
如果上述不等式对«时成立,下面证明\\
mA"x+l 具帛//.

事实上,如果我们记x = U/网,那么

\begin{quote}
\textsuperscript{A}» n+l =
\end{quote}

由归纳法假设,

mAnMcolAJ \textless{}Qc\textsubscript{0}\textbar{}Z,,\textbar{}m\^{} ,

其中Q= (m")T.于是,由引理4. 3,我们立得

mAn+1 =庭七 W ■j'QcolAJniNMcolAJ. \textbar{}
下面我们开始估计{\%}的分布函数.令不含0)是长度为
©的任意固定区间,取\#=4,并考虑

\textsc{E(X»(\^{}m))} =--- {[} XsStGz)) da,

J人

其中担,是示性函数,A可以依照不同的路径分解为4, =对每个£,由引理3.
5,為(a)在九中的分布有一致有界 密度.因此

\includegraphics[width=0.37361in,height=0.45972in]{media/image83.png}(4.4)

那么

\textsc{E(Fn)} W const £.

下面我们任意取定一个充分大的正整数五,并考虑

\emph{Eg) =} S N-叶 Xa(气)••%(勺)da.

\begin{quote}
弗``*N 八
\end{quote}

我们要证明

\emph{E\textless{}F£)} \textless{} (const 6)\textsuperscript{h} + Owb.

E(F幻可依下标数组(力,\ldots{},水)分成两部分.因为

CardfOt, ---,j\textsubscript{h}) I min\textbar{}j\textsubscript{ft} ---
j\textsubscript{£}\textbar{} \textless{} -/N} 并注意到 '

丄為,(叼卩\ldots{}么愆``)da W 1,

因此,在(4. 6)中相应的这些项全体为。3遴).E(H)中剩余的 项可做如下估计.取A
\textless{} \emph{k \textless{}}\ldots{}V水W N,满足力+1 -

/N, *=1,・・・,力一1.并考虑一个区间3UA 假设与\textsc{s\&omW}
,,那么我们在引理4.3中选择龙=力以及\emph{i=\{ae} \&吃 e
L,sW2\}.对这样的取法,引理4. 3中的假设条件显然成立.此
时,我们在该引理中取2Q= \textsc{IAJt,}令

4片,=\{a \& 兩 \textbar{}zj,+*(a) £ 丄\}.

由引理4. 3,我们有

\emph{\textless{} Qm\^{}i \textbackslash{}1\^{}\textbackslash{}.}

归纳地应用引理4: 3,在\emph{k = j\textsubscript{l+l} -
h}次迭代后,只要\emph{N}充分大,使
得\textbar{}侦\textgreater{}qT2-E,我们推得

mA件 \textless{} \emph{2\textasciitilde{}\^{}\textsuperscript{{]}} • Q
•} \textsc{•JIM Co\textbar{}/\^{}\textbar{}hi-\^{},.}

因此

于是

\emph{m=h} \textless{} (c°\textbar{}L\textbar{})k

如果我们进一步要求,对每个项``z厶£ = 1,・・・,九,那么由

\begin{quote}
\emph{fti h}
\end{quote}

引理3.5,上述测度的下降比例为M无(参考(4.4)).因此,当 N充分大时,

E(理)M(*)"+O(NT).

于是,对几乎所有4\_

\begin{quote}
\textbar{}\textbar{} limF/le M 勺 e. (4.7)

N-*8
\end{quote}

(4. 7)式意味着自由返回的极限分布在x = 0外有有界密度(界为
c°).由于在x\#:O处密度一致有界,以及对所有的'',为尹0,所以 x =
0处的情况可以不必考虑.

利用非本质返回的定义和性质,类似地可以证明,非本质返回
的极限分布具有有界密度.

2.有界返回的分布问题的讨论.

我们要讨论有界返回⑶?,« \emph{= l,2,-,N,j} = 1,2,-的分
布问题.这个问题的讨论可廿分成二步进行,我们首先讨论对
\emph{J(J}固定)的分布,然后对所有\emph{j\textgreater{}J}迸行分析.

对 令%•表示X-0点第j次返回的时刻,并且记旳(4)

=称此返回是奇(偶)的,如果相应的揉序列(Kneading
sequence)是奇(偶)的.任给定Q和,\textgreater{} 0,并令©(a)是区间,定义 为

a(a)\_ J\textsuperscript{(M}\textgreater{} \emph{---
p\textasciitilde{}l,uj ---} p), 奇返回,

\emph{+ P、*} + p + 2),偶返回. 我们考虑

\emph{破 =H\textsubscript{N}(,a)}=布力九3)(外*)).

\emph{n=\textbackslash{}}

令U是W中的一个参数区间.假设巧S)是0点 的0次迭代,对a €
U,我们考虑马伝)=理,3,a),其中b = (0,
±Ft)二\textgreater{}八.由中值定理和(3. 5)式,我们有

\emph{\textbackslash{}ffj\textbackslash{}} const \textbar{}tf(a)
\textbar{}\textsuperscript{2} exp \emph{qj} \textgreater{} const
\textbar{}ff(a) \textbar{}\textsuperscript{2} expji, (4.8)
这里我们要特别指出,如果{\%}徂i是一个奇(偶)揉序列,那么
的右(左)端点为f,如图6-1所示.

\includegraphics[width=2.4in,height=0.84028in]{media/image84.png}

图6-1

因此,一个点石6孔的第丿次有界返回属于3(a)的``条件概
率'',由引理3.5,满足下面的不等式,

点尸編可'BA'

b, I \textless{} p.

于是,由(4.8)式以及x\textsubscript{B} e \emph{1„,}立得

\begin{quote}
fiCZoifa) (jnXti)) \textless{} const \emph{I
e\textasciitilde{}\textsuperscript{J}'\textsuperscript{3}}
\end{quote}

MM \emph{\textsuperscript{C}P} 其中

"=(g I c; N const \emph{p} eT*\textsuperscript{5}"\} •

显然復\textbar{} I" \textgreater{}同UM.因此

\begin{quote}
\_1 \_ i\_ 4
\end{quote}

\&\%也3)(为,(a)) W const p 刃.

从而得

E(H£(a)) W const \emph{p\textasciitilde{}\^{}l} e"\textsuperscript{-}捉.

我们考虑\emph{(HE*}的期望E((H女a))'').它可以表为下面和的 形式

\emph{N\_h} S E(為sg(a))・``gg(a)))・

如同对自由返回情况的分析,我们将上述和式依照min\textbar{}n\textsubscript{s+1}-n\textsubscript{i}\textbar{}

\begin{quote}
\emph{S}
\end{quote}

\textless{} /N和 厲+1---标 \textgreater{}
加分成两部分.由一个类似的讨论得 到,对充分大的h,我们有

\begin{quote}
J J. I J
\end{quote}

E((H\&(a))A) W (const "〜以`厂,)* + \emph{0\textless{}N-}芝).(4. 9)
由(4. 9)式我们得到,相应于少j,极限分布对几乎所有的a E A有 密度幻,它满足
'

\begin{quote}
{[} g/(a) ds M const Z 厂 M ''厂(4.10)

Jfv(a)
\end{quote}

对J \textgreater{} J,我们要证明相应\emph{j\textgreater{}
J}的全体有界返回的数目(分 布)是小的,记

n ---1

\begin{quote}
\emph{8}
\end{quote}

其中织=2气,而气(a)定义为

丿7+1

\_ J1. 了勻(a)存在,

\begin{quote}
饥8)=[皿 火,\&)不存在.
\end{quote}

如果外6 \emph{WA、}由(3. 8)中有关步的估计推得

\begin{quote}
\emph{8}
\end{quote}

号 gl\^{}w g\textbar{}「

\begin{quote}
J=1
\end{quote}

因此仞M
l``l.由上式还可以得到,只有I川\emph{\textgreater{}J+}1时,们才可能
不为0,从而立得

\begin{quote}
\textless{} const 2

点 \textsc{hJ+1}
\end{quote}

因此E(TQ W eT C 与前面对自由返回情况的讨论相同,我们
考虑ECTQ时分两种情况:minlsi-啪W 加和1\%+1-色\textbar{}

\emph{S}

\emph{\textgreater{} m} 并由此容易证明,对几乎所有的

尽7\textbackslash{}(a) VeTQ

\begin{quote}
N-*w
\end{quote}

总结上述的结论,我们可以将;•上的分布情况写成下面的定理.

定理4.6对几乎所有的\emph{aEA,}自由返回和非本质自由返回
的极限分布具有一致有界的密度,而有界返回极限分布具有密度 gU).
2满足不等式

gGr) Wcon就[习厂第侦旳---工)+习e「第*甄工---\emph{uj)J,\\
}\%odd 勺even

(4.11)

其中

\begin{quote}
•z V0.
\end{quote}

3.在厂外的分布问题的讨论.

到目前为止,我们仅在r内讨论了分布问题.令{岡愆))丄】
表示轨道{5(a)}:=i回到广的点.那么(4.1)的冶,可以表示为

1 a 1 \textsuperscript{M n} \textsubscript{d} , "

\emph{上}=4力%时=时\%)=吏沼),\\
v=l \emph{v=\textbackslash{}} 卜 1 \emph{u} i=0

(4.12)

其中求和号Z'表示,如果F'Jq \emph{\& r,j= l,2,-,k,}那么

\begin{quote}
V
\end{quote}

\^{}(\textsubscript{Uv})属于该和式・

\begin{quote}
令产表示(他住】的``•-极限.我们已经证明 皿聞=川厂£ '', P V 2.

,---*8 J
\end{quote}

下面我们讨论{成>}丄1.由于风W ",卩=1,\ldots{},AC所以\\
supp /4° U {[}1 ---泌2,{]}丄

给定小区间3U 口一渺S1{]},那么F-4由广中两个对称小区间
。和一0组成.由(4.12)我们有

\begin{quote}
\&3)=必>3)=\emph{成3} U (-。)).
\end{quote}

因此

\emph{ju(co)} < f A(x) di,

\begin{quote}
8 expl --- •\textbar{}\textsuperscript{\_}搭)
\end{quote}

其中ft(x) = const 2 .做变量替换y = FS),那么

\begin{quote}
\textbf{I \& ---z\textbar{}E}
\end{quote}

上式的右端具形式

\textbf{, (.2}

Hz) dx = 2>(F尸Cr))(F「> dz ,

JnU(-fl) J® ;±7

其中{FQ捻是FT的两个分支,我们证明

\begin{quote}
2
\end{quote}

J\textgreater{}(F尸(z))(F湼),(Z)£ 〃,/\textgreater{} V 2.

事实上

IL* "一3)的F-3I风= j:\textbar{}y \textbar{},云帀血\\
const \textbar{}\textsubscript{o} \textbar{} A (x) \textbar{} *•
\emph{Ax} \textless{} const \textgreater{} £. \&

因为

\begin{quote}
\includegraphics[width=1.65972in,height=0.52639in]{media/image85.png}L
\emph{P}
\end{quote}

\& + G 、, 卩=[击+1
其中E和G为常数.记上式右端的两项为£和S''为证明(4.13)
的收敛性,我们只要估计\&和£的阶.利用\textbar{}旳\textbar{}
2e\textasciitilde{}",我们 有\textbar{}x-\textsubscript{M}J
\textgreater{}齢財,Y W]口 E丄.因此得S的估计式

\begin{quote}
7
\end{quote}

S] W const]号七一伊'\textbar{}心1*2丄和祯W const e(邱T")・

\begin{quote}
\emph{y=\textbackslash{}}
\end{quote}

用积分估计式,我们容易证明

Sgf const 、厂亍\\
T/勺+1 V

7 1丄

\begin{quote}
V const (1 +戸來)厂野七
\end{quote}

由上述对\&和另的估计立得(4.13)的收敛性.因此,V \emph{P\textless{}2.}

闵(1-折,1) £ (4.14)

现在令a是(一 1,1 - 中的一个区间,那么

\emph{n}

为(a) =亀〉3)・ (4.15)

*=2

我们已经知道对于z£ (1 ---抻,D,如果FXQ冬广`` =1,2,
\emph{-,k-l,}那么存在A\textgreater{} 1,使得

\textsc{\textbar{}3hF*(z)\textbar{}} \textgreater{}A*. (4.16)

用此事实以及完全类似前面的方法,我们有下面的估计

W j g*(z)dx, \emph{h =} 2,3,---

其中kill? \textless{} const A\textasciitilde{}\textbackslash{}
2,由此推得

\begin{quote}
L8
\end{quote}

ZJgAO)\&,

\emph{k=2}

8 , '

其中习由此我们证明了定理4.1. \textbar{}

*=2

小話4.7以上我们已经详细地讨论了 F(z,a) = 1 ---心2在 \emph{a =}
2附近的动力学行为.证明过程也说明,[Yo]中阐述的关于当
前如何研究非双曲系统的看法的实用性.下面我们简单地介绍,关
于一般单峰睞射族兀S)的新结果,以此作为对上面特例的总结.

定义4.8设\#是一个参数空间\emph{.Y艾 口},设兀(工)是区 间1= [一
1,口上的映射.称{兀}住"是正则映射族,如果

(i) 是关于Gc,a)的 C,映射;

(ii) c\textsubscript{0} = 0是九的唯一临界点,兀在[一 1,0)单调上升而在
(0,口单调下降\emph{,A(oxo\textless{}f\textsubscript{a}(o),fl(o)KJl(o),}并且对所有
X 6 (--- 1,0)匕(z) \textgreater{} X\textbar{} •"

(iii)存在正数4: ,A; ,C*和r\^{}2,使得对所有a任\#和所 有愆,少G \emph{I}

A; \textbar{}硏7三庖兀愆)丨\textbar{}x\textbar{}\textsuperscript{r}-i,

以及

I \emph{aja}S) I v { p. \emph{X} \textsubscript{1}1 \textbackslash{}\\
TV\^{})T\textsuperscript{Cexp}l\textsuperscript{c}
7\textasciitilde{}\textsuperscript{1}I)-

附注4. 9请读者分析F(H,a)是否是正则映射族.

记{c„(a)l-\textsubscript{=1}表示临界点c° = 0的轨道.下面我们凫义可扰
动参数的概念•

定义4.10设{兀}抵``是正则映射族.一个参数a .称为可扰
动參數,如果它满足如下条件:

\textless{}M)存在L \textgreater{}Q,y 并且兀.没有稳定

周期轨;

(CE。)存在l \textgreater{}0,使得对每个de\textsc{(0,l)}和''2、如果
\emph{H £1} 满足 n.(釦 \$ 并且尺 *3) e \emph{(-S,3),i = 0,l,}

•••,* --- 1,那么 (X)\textbar{} \textgreater{} e"\textsubscript{(}

\begin{quote}
M (qd))
\end{quote}

\textsuperscript{(T)}战叫(")) \textsuperscript{=Q}'\^{}°-

附注4.11 (M)条件即通常所说的Misiurewicz条件,它要求
临界点勺=0是非回复的;面九.没有稳定周期轨可以保证在临界
点的任意邻域外的任何足够长南轨道段的指数增长.(CEQ条件
是证明过程中技术上的要求,它保证轨道的增长指数(比方说A)
与临界点邻域的选取无关.(T)条件是橫截条件,它保证九橫截 地穿过0 .对兀愆)=1
一技回言,丄=2是可扰的.可以证明 对此映射

定义4.12 参数队称为一个Borel集JJ £或的Lebesgue正 91点,如果

,-{\textbar{}。D(4. --- e,a* + e) \textbar{}\\
}T I\# n (a. --- e\textgreater{}a. + e) \textbar{}

其中 冋表示集合。的"besgue測度.特别,如果该极限值为1,
则称a•,为Lebesgue全?I点.

附注4.13显然全稠性意味着正稠性.在对兀=1 一山的 讨论中,我们仅证明了。. =
2是正稠点■-实际上将证明稲加改动, 可以证明。.=2是全稠点.

对一般的具可扰参数山的正则族{兀槌A我们有

定理4.14令{兀}亦/是正则族.对每一个可扰参数a* E
行,存在正常数a和用使得%是满足以下诸条件的参数a构 成的集合"的全稠点.

(NS)兀无稳定周期轨;

(ER) V « \textgreater{} 1. \textbar{}/S(0) \textbar{} \textgreater{} e
exp(--- na)\textsubscript{f}

(CE1) V ''20,\textbar{}勺務(兀(0))\textbar{} \textgreater{} A exp («A);

(CE2)对V«\textgreater{}1,如果x6 [-1.1]满足充愆)尹0对于 \emph{k} ==
0,1,--- ,n --- 1 成立,并且務(£)= 0,那么 \textbar{} \textgreater{}

\emph{k} exp(n人).

由于对每个a E \emph{D,f\textsubscript{a}}满足条件(NS),(CE1)和(CE2),利用
CNS]的结果,立得对\emph{f\textsubscript{a}}存在一个关于Lebesgue测度绝对连续的
不变概率测度产.

近年来,在非双曲系统研究中不断有较深刻的结果出现.我们
在本章中不准备介绍了.但有一点要指出,我们在这里介绍的
Benedic屁和Carleson的证明思想(简称BC方法),近几年来在这个
方向的研究中起着十分关键的作用.事实上,无论是对一般区间映
射族还是对平面间宿相切(例如Henon映射),鞍结分岔甚至高维
的同宿分岔现象所得到的主要结果的证明方法,基本上是BC方
法,或者是从这一方法中派生出来的.

即使我们只考虑R"上的向量场.在讨论分岔的余维数时,也
需要在全体光滑向量场所成的(无穷维)空间中考虑某些子流形的
余维数,以及光滑映射与这个子流形的横截相交性等.为此,我们
在附录A-C中介绍一些有关微分流形与微分拓扑的概念、名词
和重要结果,以便使读者减少査找参考书的麻烦.对这些结果本
身感兴趣的读者,可以参考有关的文献,例如

{[}Al{]},和{[}Zg{]}等.

附录\textbf{A Banach}流形和流形间的映射

微分流形是欧氏空间中光滑曲面概念的抽象和推广.它的基
本思想是,先在这个对象的局部通过与Banach空间的一个开集建
立微分同胚而引进相应的代数和拓扑结构,然后再把这些局部结
构光滑地粘接起来.因此,我们可以把Banach空间中的运算推广 到微分流形上.

微分流形的定义

定义A. 1 设M是一个连通的Hausdorff空间,B是一个
Banach空间.假设U是M的开子集怦是从U到8中开子集\emph{构)}
的同胚映射,则祢(U瘁)是\emph{M}的一个坐标卡.

\emph{M}的两个坐标卡(U,时,\emph{(V,\^{}}祢为是\emph{C}相容的,如果当\emph{u}
n V尹。时,映射

,。广 1叫而,p(C7 fl V) - \emph{\textless{}P(JJ} f\textbar{} V)

是8上的仃微分同胚,

\emph{M}的坐标卡集\emph{必}=\{(亿洪)\textbar{}a£ A\}
\emph{(A}是一个指标集)称 为一个L坐标系,如果

(1) \emph{\{U\textsubscript{a}\textbackslash{}aeA\}}是M的一个开覆盖;

(2) \emph{庭}中的任意两个坐标卡都是r相容的.

\emph{M}的两个\emph{C}坐标系\#冃庭勢为等价的,如果也U M还是
M的(7坐标系.M上(7坐标系的一个等价类£称作M的一个仃 微分结构.
2内所有坐标系的并集\#=U \{方1\^{}65\}称作M
的一个极大仃坐标系,而(V,时6丿称作一个容许坐标卡.

如果在M上给定了一个<7■微分结构\&删称S= (M,£)是一
个仃微分流形.一般常把M与f等同,简称M为微分流形,为了标
明卡映射\emph{甲}的取值空间B,可称M为装备在\emph{B}上的Banach流孵
(注意,由\emph{M}的连通性和坐标卡的相容性易知\emph{,B}与坐标卡的选取
无关).特别,当B为Hilbert空间时,称M为Hilbert流形湎当B
为有限维空间(例如R")时■,称M为3维)微分流形.

附注A.2应该指出,一旦给定了診的一个坐标系行,就可
以把与M等价的全部坐标系合起来而得到一个极大坐标系,从而
生成\emph{M}上的一个微分结构.因此,只须给定\emph{M}上一个特定的坐标
系,就可以决定这个微分流形.

附注A.3如果把定义A.1中的仃全部换成C",则我们相
应地得到b微分流形.C"微分流形也称为光滑流形.

例A.4 (1)设B是一个Banach空间,则可取坐标卡(B,id),
(id是B的恒同映射),这个坐标卡显然就构成B上的一个b坐标
系.因此,任何Banach空间都是装备在它自身上的一个光滑 Banach 流形.

⑵S" = \{工£ R"\textsuperscript{+1}1 U\textbar{}\textbar{} =
1)是一个m维流形.事实上,记 \emph{N =} \{1,O,-,O\},S=
\{-1,0,•••,()\}分别是S''的北极与南极,取
坐标卡(S"\textbackslash{}\{N\}, " (S»\textbackslash{}\{S\}, ft),其中

"S"\textbackslash{}\{N\}fRK,啊=(\sout{]三僅}, 卻
S\textsuperscript{n}\textbackslash{}\{S\} ---
R",你如,\ldots{},命+1)=(\sout{].务}, 容易得出

\^{}?T\textsuperscript{,!} R\textsuperscript{n}\textbackslash{}\{0\}
-\textgreater{}R"\textbackslash{}\{0\},件矿伝)=赤
是C\textasciitilde{}微分同胚.

流形间的映射

定义A.5设M,N分别是装备在Banach空间A,B±的役'
流形.\emph{ftM\^{}N}称为是(/映射,如果\textsc{VhEM,}及N上任一容许
坐标卡\textless{}V,Q J3)6 V, 3必中的容许坐标卡(U,机工G \emph{U, f(U)
U} V,使得映射(叫作\emph{f}的局部表示)

\emph{f\^{} = \textless{}!\textgreater{}\textsuperscript{a} f -f'*}
中(U) U A --- \emph{\textless{}p(V)} U B 是仃的.

容易证明下面的两个定理.

定理A.6设\emph{f*MfN}是流形间的连续映射,则/■是顷的,
当且仅当对M与N上任意取定的坐标系而言,相关的局部表示都 是仃的.\textbar{}

定理A.7设\emph{M,N,P}都是仃流形都
是尸映射,则复合映射也是(7的.\textbar{}

定义A.8设是\textless{}7■微分流形,称須,MfN为仃微分
同胚,如果它是一一的仃映射,并且其逆映射M也是C5
的.如果两个流形间存在一个r微分同胚,则称这两个流形C微 分同胚.

子流形与积流形

类似于向量空间的子空间与乘积空间,微分流形也存在子流

形与积流形.

设N是微分流形M的一个开子集,则把M上的微分结构限
制到N上,就自然得到\emph{N}上的一个微分结构,从而使\emph{N}成为一个
与M同维数的流珍这时,称N是M的一个开子流形.俱是M前
亠个闭子集,例为R''+i中的闭子集3"也可以成为一个流形(见例
A.4).注鬻,此时程R"+i中存在坐标卡〔U,乎),使祁J (AS'')成为
R"的子集.这引出如下的一般定义

定义A.9设M是裝备在Banach流形理上的微分流形.M的
一个子集\emph{N称为M}的子流形.如果\emph{戏 £N,3M}的容许坐标卡
(。疗),和\emph{B}的宜和分解B = 8】击0,使得戸£ U,并且

卩《7\textbar{}"\textbar{}村)=中《7) 0(4 X {0}). (A.1)

可对这个定义给出直观的几何解释\emph{N NC} 容许坐标

卡(U,时,使邻域U fl
N经甲作用展平在\emph{B}的子空间\emph{B\textsubscript{t}}上.

定理A.UJ设N是微分流形M的子流形,则N自身也是一
个微分流形,并且它的微分结构可由下面的坐标系生成:

{(C/r\textbar{}N, pBnM):(U,P)是M的容许坐标卡,

它满足条件(A.1)}. \textbar{}

在附录B中,我们将介绍通过浸入或浸盖来构造(或鉴别)子 流形的方法.

定义A.11设是微分流形,它们分别有坐标系{(。心
"\textbar{}a£4}和{V"侏B},删集\emph{合MXN}在由坐标卡{(亿 X皿,饱X
sMu°x*)\textbar{}a £ A,尸务B }生成的微分结构下作成微
分流形,称它为肱与N的积流形,仍记为\emph{MXN.}

附录B切丛与切映射,向量场及其流,漫入与浸盖

向量丛是积流形的推广,而流形上的向量场,则是作为一个特
殊的向量丛(即切丛的截面)而定义的.利用切映射,我们可以把

Banach空间中的隐函数(反函数)定理以及局部浸入和浸盖定理
推广到流形上,从而得到判断(构造)子流形的有效手段.下面先 定义向量丛.

向■丛

定义B.1设M是仃微分流形,B是\&皿ch空间\emph{\textsubscript{t}UCM}是
开集,称17X3为局部向量丛,并称U为底空间,它可以等同于(7 X
\{0\},后者称为同部向量丛的\#截面.V«6 u,称3\} X B为过
«的鉀维,它可以众\emph{B}获取Banach结构.\emph{Y u£U,bW B,}由R(``,
方)="所定义的映射\emph{\^{}UXB\^{}U}称为投影..

注意,过``的纤维就是而U X召为积流形\emph{MXB}的 一个开子流形.

定义B. 2设\emph{UXB}和5 X B都是局部向量丛.如果映射 \emph{"U 7}和%:E7f
)都是(7的,则由

\emph{强u,b)=}(阳(``),¥\%(``)• 6)

所定义的映射\emph{fUXB-\^{}U'} x \emph{B'}称为r局部向量丛映射.如果
这个映射还是一一的,则称它为仃局部向量丛同构.

注意,一个局部向量丛映射把纤维S\} X五线性地映到纤维 («(»)\} XB*.
一个局部的向量丛映射是局部向量丛同构,当且仅 当VuGU,
\%(«)是Banach空间器与否之间的同构映射.

类似于微分流形的定义,我们可以把局部向量丛粘接起来,得
出整体的向量丛结构.

定义B.3设S是集合.称(W,时是S的一个局都向量丛卡, 如果IV U
S,甲是从W到一个<7局部向量丛\emph{UXB}的一一映射
(U,B可能与p有关).称这样的卡集留=\{(W\^{},\%)\textbar{}a£7i\}是S
的b向■丛坐标系,如果

(1) \emph{\{W\textsubscript{a}\textbackslash{}aeA)}覆盖 S;

(2) 若("》,代),(w``啊)e第,吼,n巧声尹。,则%(叽n
巫\#)是局部向量丛,且阳°缶'\^{}(W\textsubscript{O}AW\textsuperscript{,}\textsubscript{\textbar{})})是b局部向量丛同构.
称S的两个向量丛坐标系幽1与绥将价,如果躋1U漆必是
一个向量丛坐标系.向量丛坐标系的一个等价类称为S上的一个
C°■向量丛结构.称\emph{Ef} )是一个b向暈丛,这里S是一个集
合,戸是S上的一个r向量丛结构.与微分流形类似,通常也把E
与S等同,并把源中的任意局部向量丛卡称为一个容许向■丛卡.
向量丛有如下性质'

(1) 仃向量丛\emph{E}是一个侦微分流形.

(2) 对向量丛E,定义其專截面(或称为E的底空间)

\emph{M \{p e E\textbackslash{} 3} 容许向量丛卡使 p = 它是E的子流形.

⑶ 设怦)是一个容许向量丛卡\emph{,弈WfU X} 设pF

W,满足\emph{甲}(/>) = 3,0),令集合

\emph{Eg} = ?"'(\{``\} X BQ,

它可以通过中从B诱导Banach结构,并以\emph{p}为零元素.如果(Wi,
ft)和(巫``啊)是两个容许向量丛卡,P G W\^{}i Cl W2,则\emph{Eg} 与
E',%线性拓扑同构(作为线性空网是同构的,作为拓扑空间*同
胚的).在这个意义下,可以认为Eg与甲无关,可记为\emph{E\textsubscript{P}.}

(4)v«eE,3唯一的\emph{\textsc{p}}e
\textsc{m}(底空间),使得ee\emph{\textsc{e\textsubscript{p}.}}由此,
可定义投彩

\emph{E f M, = p,}

它是仃的满映射,并且旷 W)=耳称为过\emph{peM}的纤维,它具
有的Banach结构称为纤维型.在有些书上,把上面的性质作为向
量丛的定义.我们有时也把向量丛记为或wEf M,以 标明投影映射m和底空间Af.

从几何上粗略地说,以流形M为底空间的向量丛,就是在\emph{M}
上的每一点``附着''一个以该点为零元素的Banach空间,而不同点
上附着的Banach空间是彼此同构的.例如,在S?上,我们可以用下

列两种方式构造出不同的向量丛:V,€ S2,取过力的法线为纤维
E,;或取过\emph{p}的切平面为\emph{Ep.}前者的纤维型是畔,而后者是\emph{R\textsuperscript{s}.}

切空间与切丛

我们可以通过坐标卡把Banach空间中曲线相切的概念诱导
到流形上,从而建立切空间与切丛.

定义B.4设M是仃微分流形设\$\textgreater{}0,开区间Z =(一畧3).称。映射cJf
M为M上的一条曲线「如果''0)=
P,则称曲线以\emph{P}为基点.设上,电是以\emph{P}为基点的两条曲线,并且
(U怦)是一容许坐标卡\emph{\textsc{,p}} e \emph{U.}如果

D(p。C])(0) --- D(p。@)(0)

(即Banach空间中的曲线处q与中。勺在。点相切),则称M上的
曲线勺与6在p点相切,

注意,利用流形M上坐标卡的相容性,%与位在声点的相切性
与容许坐标卡的选取无关,事实上,设0,對),(叫决)是两个坐
标卡,/\textgreater{} 6 n \emph{u\textsubscript{2}.}设 D3。2(0) =
D(p。C2)(O).由于•

矽。G = (©。p\textsuperscript{\_1})-(中。c;), i = 1,2,

所以

D(© ° \textless{}:i)(0) = D(0 » 亍')(卩3)) • D(p- cD(0)

=D@ -伊f)(p (/\textgreater{})) - D(p» c\textsubscript{z})(0) =
D(。。C2)(0).

这样,我们在同基点的曲线之间规定了一个等价关系m〜◎
Qq与§在*点相切.记c在戸的等价类为称它为流形M在 P点的一个切向■

定义B.5设M是U微分流形(r\textgreater{}l),/\textgreater{}eAf.在》点的全
体切向量之集合

T»W) = 是M上以?为基点的曲线}
称为流形M在p点的切空间.并称\emph{M}上全体切空间的集合

为M的切丛.

定SB.6切空间是一个Banach空间,而切丛TM在 投影

\emph{jrtTMM (jr\^{}cjp --- p)}

之下作成一个向量丛

证明先证明\emph{T?(M)}是一个Banach空间.取包含7■点的一
个坐标卡■(!/,甲),则以/■为基点的曲线c有局部表示D3u)(0)W 0
(R,Span?07)).记旦=(7?,Spanp(\{7)),它是一个 Banach 空间•这样.v
Ed\textsubscript{f} er\textsubscript{?}(A/),有'' \textsc{=
d(\textgreater{}c)(o)}与之对应. 反之,\textsc{Vp} € 耳,令 c«)=广(敏'')+
拔),则 p\textsuperscript{o} c(r)=\emph{秩i\textgreater{}) + vt,}
从而D("Q(O)=a,从c可得于是,得到TpW)到品的一
一对应.这样,就可把\emph{B\textsubscript{r}}的Banach空间结构通过局部表示
D0\textgreater{}。c)(0)而迁移到\emph{T政M)}上.

再来证明是向量丛,其中投影\emph{mTM f M}通过
巩[員捞=/•规定.取仃微分流形M的一个坐标系==\emph{\{(U\textsubscript{a},}
物)\textbar{}a£A\},则由下文的定理B.10可知\emph{,\^{}=\{(TU\textsubscript{a},T\textless{}p\textsubscript{a})\textbackslash{}\textsubscript{a}eA\textbackslash{}}
就构成了 TM上一个(7-\textsuperscript{1}向量丛坐标系,其中7Va =
\emph{U\^{}Tp(M),}而丁削丁(我UQ)是如下定义的切映射.

切映射

设是微分流形J:A/f N是仃映射5法1).设q宀是
财上在"点相切的两条曲线,则/。勺和 E 是N上在/(/.)点
相切的曲线.事实上,设(U,时,(V\textsubscript{)S}i)分别是\emph{M.N}上的容许坐标
卡\emph{,p € u,fip')e} v,y(u)uv,则 \textsc{d}(少勺``。)=D(wy)(o). 注意

\emph{\textless{}l\textgreater{}° f ° c,---} 3。y。p-') ° (p ° C,),
\emph{i} --- 1,2, 则由Banach空间中导算子的链式法则可得

D(0 ° \emph{f} - c,)(0) = D(° " \emph{f}。广)(中(0)) -
D(中■\textgreater{} c;)(0),

(B. 1)

从而

\begin{quote}
D@ \emph{* f。}勺)(0) = D((J。\emph{f} ° c\textsubscript{z})(0).
\end{quote}

由此,我们给出如下定义

\begin{quote}
定义臥7如果是流形间的C\textsuperscript{1}映射,则称映射 T/: \emph{TM
-\textgreater{} TN,} 7y({[}c3 = {[}/。理3
\end{quote}

\emph{为f}的切映射.有时也把7Y记为d/或兀.

如果(U建),(y,Q是上而所说的坐标卡,则由(B.1)式可知, 序有如下表示式

\emph{Tfi} 3,a) 1 (須(/\textgreater{}), D3 -
\emph{f}。广)(?\textless{}/\textgreater{})) • 0),

(B.2) 其中p表示D(p。c) (0).由此,可从Banach空间中导算子的性质,
得出切映射的如下性质,

(1)若\emph{f:MfN}是仃的,则\emph{TfWN}是(7T的.

⑵ T"M) - Tp(N)是线性的.

\begin{enumerate}
\def\labelenumi{(\arabic{enumi})}
\setcounter{enumi}{2}
\item
  若\emph{AMf,} K是流形间的仃映射,则
\end{enumerate}

\emph{T(g,f)=TgF}

是\emph{TM\^{}TK}的(7T映射.

\begin{enumerate}
\def\labelenumi{(\arabic{enumi})}
\setcounter{enumi}{3}
\item
  若五:是恒同映射,则\emph{Th:TMfTM}也是恒同映 射.
\item
  若BMf N是微分同胚,则\emph{Tf'TMfTN}是单、满映 射,且
\end{enumerate}

引理B.8 设W是Banach空间B的开子集(从而是 b
Banach流形B的开子流形),则7W同构于局部向量丛W X
\emph{尊}(因此,我们在下文中把加与"X B等同).

证明设c是W上以在为基点的曲线,则存在唯一的i\&B, 使得由

\emph{Cp,b(.t) --- p 1b}
所定义的曲线在\emph{P}点与\emph{c}褂切.事实上,以(0)是名\emph{(R,B)}中唯一
的线性映射,使得与曲线,在在点相切的另一曲线具有形 式

\emph{g(t) = p +} Dc(0) • \emph{t.}

令g = \textsc{Cm,}则\emph{b} = DC(0) - 1是唯一存在的.定义映射

A' IV X B-► \emph{TW, h\{p,b\} --- {[}Cp,b\}\textsubscript{P},}

则上而的结论表明\emph{h}是一一的.我们可以在\emph{TW}上建立局部向量
丛结构.例如,取\emph{K\{p\}XB)}为过\emph{pEW}的纤维,则儿恰是7W与
\emph{WXB}的局部向量丛同构映射.\textbar{}

引理B.9设W和W\textsuperscript{7}分别是Banach空间B和3的开子集,
\emph{f,W\^{}W'}是仃微分同胚,则\emph{TfiWxB\^{}W' XB'}是局部向量
丛同构映射.

证明因为

773,8)=(須(力),D\_f。)E)

(见(B. 2)式,取务少为恒同映射),所以\emph{Ff是 L 局部}向量丛映
射(见定义B. 2).又因为『是仃微分同胚,因此\emph{(T* = TH}
也是一个<7一'局部向量丛映射,从而"是一向量丛同构映射.\textbar{}
现在可以证明

定理B.10设财是</微分流形\&梁1),\#=\{(儿用)楂£
A\}是M的一个仃坐标系,则\emph{T庭}=((T(U\textsubscript{a}) ,T皿))S £⑴是
切丛\emph{TM}的C-\textsuperscript{1}向量丛坐标系,从而TM是(7T向量丛.'此外,如
果M是"维流形,则\emph{TM}是\emph{2n}维流形.

证明 (U T«j\textsubscript{a}) \textbar{}« e A\} Z) 7W.
T«J\textsubscript{O}) n

\begin{quote}
A.则由引理B.8,
\end{quote}

\textsc{t}物 cr(s)n \textsc{t(uq)=t(\textsubscript{p}}n T(a(s))

是局部向量丛,并且

\emph{T\%} °
(Tfj,)\textsuperscript{-1}1码(7\textbackslash{}\%)(1崇``)=T(師。缶')

(见定义.B. 7下的性质(3)和(5)).注意驿。传'是\%(C/\textsubscript{Q})所在的

Banach空间E中的开子集你T(S) f\textbar{} \emph{TW)}到自身的CT映射.
由引理B.
9立得,(T"。(亍铅厂是C-\textsuperscript{1}向量丛同构映射.因此,
帛曳口.9申的柔侔U)和⑵坞成立.有关维数的论断,是引理B.8
的自然推论(注意做S)是它所在Banach空间的开子集).\textbar{}

附注B. 11前面已经提到,流形间的切映射是Banach空间
中导算子的推广•因此,有时把它称为导映射•设M和N是充分光
滑的微分流形,则定理B. 10说明,TM和TN是向量丛,从而是微
分流形(见向量丛的性质⑴),并且 TM因此,可以继

续求二阶导映射\emph{T(Tf)
=T\textsuperscript{z}f\textgreater{}T(TM)\^{}T(TN),}并可递推地定
义高阶导映射.

向量场及其流

定义B.12 设(E,心M)是一个向量丛.如果映射

满足?rM = idM,则称它为向量丛的一■个截面M称为是(7连续的,
如果它作为Banach流形间的映射Af-£是(7的・

定义B. 13设M是一个C\textsuperscript{5}微分流形,其切丛上
的一个C7rMsW8)截面\emph{XiM\^{}TM,}称为M上的一个仃向
量场.\emph{M}上一切(7向量场的集合记为缶飞M).

附注B.M设\emph{X'M\^{}TM}为一向量场,按定义QX = idM, 即V \emph{P
\&M,}有X3)€ \emph{TpM.}换句话说,M上的向量场就是把M
上的每一点赋予一个(在该点切空间内的)切向量,从而形成一个
M上的切向量场.

现在设(―】,1)\textasciitilde{}M是M上的一条§曲线(sWr),
我们要规定曲线上每一点的切向量•注蠢由映射\emph{f M}可得切 映射

\emph{Tai} 77 = / X R--- \emph{TM.}

取切丛77的一个截面A-/-TI, A(t)-a,D,Vt\&7,则复合映 射

\emph{Ta -} A\textgreater{} 7-\textgreater{}TAf
是上的一条C\textsuperscript{1}"\textsuperscript{1}曲线.记廿=•
Ta。人,称\emph{QTM\^{}M}
上的曲线a在p点的切向董.如果取/上的坐标卡(W,id)和M上 的坐标卡(。并)(£ E
\emph{W,\textsubscript{a}(t) EU),}则由公式(B. 2)可知,映射/ 有局部表示

\begin{quote}
*(f) = Ta(t,l) = (a(t), D(0。a)(r)). (B. 3)
\end{quote}

注意,上面的1是R'中的单位映射.

定义B.1S 设f £亥'(M)JURi是开区间,0 e \emph{I.}称a:/ f
M是S的过A6M的积分曲线或流,如果\emph{Yt\&r,}

\begin{quote}
J \textless{}/(f) =£(a«)),

1 a(0) \emph{= p„.}
\end{quote}

称上述流是极大的,如果£过A,的任一流俄J U Ri f M (J是包
含0的开区间),都有\emph{JUI,}且a\textbar{}j = \&

取M上的坐标卡\emph{(U,\^{},\textsubscript{fio}eU\textsubscript{,}}设向量场TM有局部
表示其中\emph{sU---B} (Banach空间).与上面/(c)的局部
表示相对照,就得到Banach空间中关于3。a)的微分方程初值问 题

\emph{j} D(p ° a) «) --- \emph{v(a(t)})=为。少-i ((中。a) (f)),

I(0。a)(0) --- \emph{\textless{}p 3°).}

利用坐标卡的c\textsuperscript{1}■相容性,此方程与坐标卡的选取无关.就是说,如
果取另一坐标卡\emph{(V仲HpM} V,则只须把上面微分方程和初值条
件中的\emph{\textless{}P°} a换成0。a即可.

利用Banach空间中微分方程初值问题解的存在和唯一性定 理,可以得到

定理B. 16 设?€ l.jgjf过力的极大流

是存在且唯一的. I

由附注B. 14可知,求? \& 过户的流,就是在M
上找一条过*的曲线,使它在每一点的切向量刚好与S在该点给
出的切向量相吻合,这与欧氏空间中向量场的流的概念相一致.

附注B. 17 如果Af = U是Banach空间B的一个开子集,则

u上的向量场就是一个映射x:i/f a x 8,它具有如下形式

X(H)=(工 VGc)).

我们称V是X的主部(principal part).显然,我们可以把X与V等
同,简单地认为向量场X就是映射们\emph{U\^{}B.}此时,如果曲线吋)
满足微分方程

Da(f) =U(g),

则它就是\emph{X}的流(在定义B. 15中仲=id).

如果M是一个U微分流形,(U,p)是它的「个坐标卡,\emph{啊U}
fiy'UB,厕M上的向量场X诱导出8上的一个向量场.

\begin{quote}
云=7V・X(?ri(z)),
\end{quote}

它称为X的局部表示.

设是一曲线,。是亲的主部,如果

D5(r) =V(«(0),

其中\emph{三=罕5、}则a就是M上的积分曲线.

特别地,当B = R''时,京的主部0具有形式(叭。),\ldots{},
吃(``)),z£R".设曲线a(£)的局部表示为(勺《),\ldots{},a''Q)),则
它成为积分曲线的条件是

因此可以说,对于«维流形上的向量场而言,其积分曲线的局部表
示满足R"中的微分方程组.

沒入与浸盖

隐函数定理是微分学中最重要的定理之一•设f R''是 a'的.当\emph{m =
n}并且导算于\textsc{D/(x\textsubscript{d})}是单、满映射时J给出x\textsubscript{0}的邻
域到K(女)的某邻域之间的微分同胚.,如果D/(x\textsubscript{0})仅是单射(设
\emph{m
<}沥,则存在R''中/(x\textsubscript{0})附近的局部微分同胚幻使得在%点
附近有

\emph{g ° f\textsuperscript{:}}(命,...,办)H*
(气,\ldots{},%,。,•••,()),

这在微分几何中称为1局部)正则浸入.另一方面,若W(他)仅是
一个满射(设\emph{m>n),}则存在R"中归点附近的局部微分同胚矿使
在玄点附近有

\emph{f°\textsuperscript{h}-}(气,\ldots{},弓,%+1,.``,工京 i
(而,\ldots{},不), 这称为(局部)正则浸盖(投影).

容易把隐函数定理推广到Banach空间.现在我们把上面有关
技入和複盖的结果就Banach空间的一般情形给岀证明,然后再通
过坐标卡推广到微分流形上,并由此得到构造(或判断)子流形的
方法.对于无穷维的Banach空间,仅有D/(x)的单射(或满射)> '
件是不够的,要附加适当的可裂性条件.首先给出下面的..代"「

定义B. 18设E是Banach空间,田是E的闭子空间.妬棄容
在\emph{E}的团子空间码,使E = g £爲,则称瓦是可裂的(split).

附注B.19如果E是有限维的,则它的任何子空间都是闭且
可裂的;如果日是Hilbert空间,G匚E是闭子空间,则E = G®
G丄,因此Hilbert空间的闭子空间都是可裂的.但存在无限维
Banach空间,它有不可裂的闭子空间.一般而言,E的闭子空间\emph{F}
可裂的充要条件是;寻\emph{P3} (E,E),使= 且「=脸£
\emph{E\textbackslash{}Pe=\^{}e\}.}

定理B.20 (局部浸入:定理) 设E,F是Banach空间\emph{,八UU
EfF}是仃映射\emph{,r\^{}l,u\textsubscript{a}EU.}设''g),Ef
F是单射,并且 D/(«o)的值域码=烫是闭且可裂的.从而存在闭子空 间
\emph{F\textsubscript{1} U F,F} =呂 £ \textsc{Fr.}(当 E = R'',F =
R"时,只须设 rank(D/■(阳))=m.)则存在开集 VCF (/(«„) € V)和 归
UE W码,以及仃微分同胚?> V-> W,使得(卩。\emph{f)(e)} = (e,0),Y e

£ V (1 (E X \{0\}) UE.

证明 由条件可知,线性映射D/(«\textsubscript{o} )是£到码= 阕(''(%))
UF的代数与拓扑同构.令

g: U X 码 U E X 卩2 f F = F1 ㊉码,g('',t\textgreater{}) = /(a) + v,
其中\emph{\textsubscript{u}eu,ve F\textsubscript{s}.}注意

\begin{quote}
D/(u\textsubscript{0}) 0 '
\end{quote}

DggO) = „ ,

(J

是\emph{U
XF\textsubscript{2}\^{}F\textsubscript{t}®F\textsubscript{t}\^{}F}的U微分同胚'由隐函数定理,存在开
集卩,归,使(\#。,0)e 及仃微分

同胚 \emph{f\textgreater{}:V} f W,P\textsuperscript{\_,} =
\emph{g\textbackslash{}\textsubscript{w}.}因此,v (e,0) e v, (p • y)(e)=
3。g)(e,O) --- (e,0). I

定理B.21 (局部浸盖定理)设E,F为Banach空间\emph{,f\textgreater{}U(Z} Ef
F为(7映射,r\textgreater{}l,«\textsubscript{0} e
tA设\textsc{D/(m\textsubscript{0})}是满射,并且战=
ker(D/(\textsubscript{Ko}))是可裂的,E = \& \$ Ez(当 E = R-,F =
R"时,只须 设rank(D/m))=如),则存在开子集U和V,\% £17, UC7,VU
F£E,以及(7微分同胚\emph{岫 f邛}使得(«, u) e V.

证明定义映射

g: t7 C £j ® E\textsubscript{2} -* F® E\textsubscript{2}, g(``i,``z)---
(/(a\textsubscript{l},tt\textsubscript{a}),w\textsubscript{z}). 因此

(Dj/Cwo) D\textsubscript{2}/(a\textsubscript{0})'

Dg(uo,O)= ,

\begin{quote}
。 \textsuperscript{j£}2
\end{quote}

故Dg(\textsubscript{Mo},0)eGX(£,K®E\textsubscript{a}),由隐函数定理,存在开集V,

成UC7,VUF㊉旦,以及仃微分同胚SV-*U,,使得\textless{}!\textgreater{}-'
\emph{---g\textbackslash{}v-}因此

a,勿=(g。,)(心力=\textsc{(/w«tv),\^{}\textsubscript{3}(w,v)),}

其中。■山X处,即间(心a) = 5 \emph{e Q} 0)(14加)=*・I

现在把上面的结果推广到微分流形之间的映射\emph{f'M f N.}
注意及\emph{\textsc{T\textsubscript{m}(N)}}都是Banach空间,因此下面的可裂性条
件是合理的.

定义B.22设M和N是Banach流形\emph{J、MfN}是C7映射.
称/■在\emph{PEM\^{}C-}局部浸入,如果Tp/■是单射,而且它的象集在
\emph{T\textsubscript{\}W}N}中是闭的可裂集.称;'在\emph{P\&M}是b局部浸盖,如果
是满射,而且ker(Tyf)作为丁双 的闭子空间是可裂的.在流形
\emph{M}上每一点都是C局部浸入(局部浸盖)的映射称为\emph{C}浸入(浸 盖).

定理B.23设是Banach流形,\emph{f:M\^{}N}是仃映射(r \textgreater{}
1),则下列三种陈述是等价的:

\begin{enumerate}
\def\labelenumi{(\arabic{enumi})}
\item
  /在,EM是局部浸入;
\item
  存在坐标卡\emph{(U,G兩,P £ U ,fUV冷U fU、 5Vfir} x u',且K3)=0,使得
  o。須。0T:。'---成 x V' 是 包含映射Pi 3,。);
\end{enumerate}

⑶ 存在少的邻域U,使得是N中的子流形,且flu是 U到/W)的微分同胚.

证明(1)与(2)的等价性可由定理B.20得到.(2)与(3)的
等价性可由子流形定义A.9得到,注意V是N中的开集.\textbar{}

附注B.24定理B.23说明,若在力是局部浸入,则存在/\textgreater{}
的邻域U,使\emph{f(U)}是\emph{N}中的子流形.要特别注意这个结论的局部
性.卽使須在M的每一点都是局部接入,也不能断言/'(M)是N
的子流形.事实上,浸入的局部单射性不能保证整体的单射性.
例如,由平面极坐标方程r = COS2W定义的映射f IT是一个
浸入,但不是单射,且/(5\textsuperscript{1})不是W的子流形(见图1).即使浸入
\emph{『MfN}在整体上是单射,仍不足以保证須(M)是N的子流形.
一个反例是由极坐标方程\emph{r} = sin田定义的映射/■ (0,2\^{}) U R f
R2(见图2).上面两例中的问题都出在W原点附近的邻域内、

\includegraphics[width=2.07361in,height=0.9in]{media/image86.png}\includegraphics[width=0.11319in,height=1in]{media/image87.png}\includegraphics[width=0.29306in,height=0.22014in]{media/image88.png}\includegraphics[width=0.33333in,height=0.96667in]{media/image89.png}

图L 图2

利用須从M诱导的拓扑与7XM)作为晔的子集而获得的拓
扑是不同的.在前一种拓扑下,/"(M)作成一微分流形,因而有时
称单射溪入的象集\_f€M)是一个漫入子流形;但是在后一种拓扑
下,它可能不是\emph{N}的子流形.此时M与\emph{HMK}作为\emph{N}的子空间)
不同胚.这就引出了下面的

定义B.25设\emph{北M-N}是寝入,并且是M到/'(M)(在N的
相关拓扑下)的同胚,则称它是一个嵌入.此时,称是N的 嵌入子流孵.

容易证明下面的

定SB. 26 设\emph{戶MfN}是单值的浸入,若它是\emph{M}到\emph{f(M)}
开映射(或闭映射),则■/是一个嵌入.I

定理B. 27 设是微分流形\emph{,f\^{}M -\textgreater{} N}是仃映射\emph{,q €
N,S} =广W).若■/■在S上每一点都是局部浸盖,则S是M中的 一个仃子流形.

证明 由定理B.21及子流形的定义即得.\textbf{I}

附注B.28 设是仃映射.点\emph{qgN}称为f的正则
值,如果\emph{Y作广5项}是满射且其核在与M中是可裂的.记
为■为•/的一切正则值的集合,则定理B. 27有如下等价的陈述:

"定理B.29 若则广】(q)是M中的子流形.\textbar{}

在讨论一个映射的水平集时,这种说法是方便的.

附录\textbf{C Thom}横截定理

在寻找结构不稳定向量场的普适开折时,有时要确定开折中 的通有族(generic
family),或称为一般族,非退化族.为此,需要利
用Thom横截定理(它是从著名的Sard定理导出的),事实上,为了
把无穷维问题简化到有穷维,我们需要的是jet形式的Thom横截
定理.因此,我们先建立映射空间中的拓扑,再引进成流形,最后
介绍Thom橫截定理.

映射空间的拓扑 '

记(7(M,N)为\emph{C}微分流形M与N之间的(7映射所成的集
合.我们在\emph{C\textbackslash{}M,N)}中引迸拓扑,使它成为拓扑空间.

定义C.1 (弱拓扑,即compact-open拓扑) 设/' £ C\emph{(.M, N),}
(t/瘁)和(丫仲)分别是M和N的容许坐标卡;令\emph{KUU}是紧
集,使\emph{f(K) U} 丫;令e为正实数.定义弱子基邻域

彳(m7,9\textgreater{}),W,0),K,e) = \{g £ (7(5/0)
\textbar{}g(K) U U,

sup \textbar{}\textbar{}。(。° \emph{f}。pT)(z) ---。(4\&。\emph{g
°} \textless{} e\}.

C7M,N)中的U弱拓扑就是由这种形式的集合所生成的.任何
包含有限个这种集合交集的集合都是\emph{f}的一个邻城.所得的拓扑 空间记为

定义C.2 (强拓扑,或fine拓扑,WHiney拓扑). 令* =
\{(亿,铀)\textbar{}a任4\}是M的一个局部有限坐标系,即\emph{M}的每一点都
有一个邻域,它只与有限个相交;记火=A\},K.U 亿是M上的紧集;图=\{(皿*)愣£
B\}是N的坐标系.对任一 正数集合£ = (\&a\textbar{}a£ A\},定义

,(儿岳,Q - \{g£(T(Af,N)\textbar{}g(KQU*(小

sup ID0S)° \emph{f} \textsuperscript{0} 妃)(Z)

\emph{---成由3 ° g "} Q(Z)II \textless{}£\textsubscript{a}. V a e
\emph{A\},
\textsc{cxm,n)}}上的r强拓扑就是以上面类型的集合为拓扑基(开集)
所生成的.记所得的拓扑空间为\emph{CslM'N).}

附注C3上面我们假设r\textless{}8.为了定义C\$(Af,N)(或
C;W,N)),只须把包含映射 \emph{\textsc{c8(m,ns(m,n)(}}或
\emph{N机}诱导的拓扑对所有有限的r合起来即得.

附注C.4 q(Af,N)有很好的性质.例如,它可以赋予完备
的度量,并有可数基'当\emph{M}为紧流形时\emph{与皿M,N)}
一致,它们是Banach空间.当肱不紧时,弱拓扑不能很好地控制
U映射在``无穷远''的性质,强拓扑就成为需要的了'注意,在M
非紧时,强拓扑在任一点都没有可数基,因而它不可度量化.但
是,在强拓扑下応(M,N)的任一弱闭子空间都是Baim空间(剩余
集在其中稠密).这对研究通有性质是很重要的.如无特别声明,
下文中都取强拓扑,并把応(M,N)简记为C7M,N).

下面几个定理反映了 \emph{CXM'N)}中函数类的性质.

定3 C.5 设为U微分流形,1 M s M+ 8, ICrCs.dimM \textless{}+oo.则
G\textbackslash{}M,N )在 )中稠密,其中

\emph{U CXM,N)}表示如下任何一个映射类:

微分同胚、嵌入、闭嵌入、浸入、浸盖、真映射(即紧集的原象是 紧集). I

定理C.6 G)设则任何U镰分流形U微分同 胚于一个C"■微分流形'

(2)设lWrVsW+8,若两个C•微分流形(7微分同胚,则
它们C\textsuperscript{1}微分同胚.\textbar{}

定理C.7 (Whitney定理)设lW『W+8,则任何n维的C
微分流形都仃微分同胚于RE'的一个闭子流形.I

附注G8 从定理C6可知,如果把所论流形的光滑性提高
到C",并不是一个很严重的事情.今后我们将经常作这样的假定.
虽然》维微分流形是我维欧氏空间非常一般的推广,但定理C 7说
明,它反过来又可作为嵌入子流形放到R\textsuperscript{2}"\textsuperscript{+1}维欧氏空间中.

射式W)流形

设是U微分流形.我们把三元组\emph{愆,了,U)}的等价类母,
\emph{f,U\textbackslash{}}称为从肱到\emph{N}的一个r-jet,其中。U
\emph{M}是开集,h 6 \emph{U ,f} 6
\emph{C\textbackslash{}U、N机}等价关系\emph{(.x,f,U\textbackslash{}}〜S'
\emph{,f ,U'\textbackslash{}}是指注=/ ,并
存在M与N中的容许坐标卡(W,仞和\emph{(V,\textless{}!\textgreater{}-),}使WUU
(\textasciitilde{}1 t? UM,并且局部表示

\emph{\textsc{\textgreater{}p} - £ * p'} 与。。尹。广:Wf /(iy)n
\emph{f}(W) 在z的前r阶(包括0阶)导数均相同.

显然,上面的定义与坐标卡的选取无关,从而也与。的选取无 关.因此记

称\emph{徵}为映射r\emph{在工}点的r*,并称Z为起点,■/■»)为终点.

记\emph{尸(M,N)}为全体从肱到\emph{N}的r-jet之集合'我们把\emph{J\textsuperscript{r}(M,}
N)中起点在x的子集记为终点在y的子集记为
\emph{J試M,N)*}而把再)与\emph{J\^{}M,N})的交集记为

现在考虑一个特殊情形,M = R\textsuperscript{m}\textsubscript{(}N =
R".此时简记

\begin{quote}
J(R\textsuperscript{m},R\textsuperscript{n}) =
\end{quote}

设uuRm是开集,/er(u,R\textsuperscript{B}),则f在点的广阶丁卽辰多项
式就给出/■在\#点的。jet 一种自然的表示.这个从R"劉R"的多
项式映射可由『在纟点直到r阶(包括。阶)导算子所唯一决定.这
些导算子合起来,属于向量空间

FGng) = R\textsuperscript{n} X \%R七R*),

*=1

其中Sf
*(R\textsuperscript{ra},R")表示从R™到RK的龙重对称线性映射所成的向
量空间.这说明,对\textsc{PGmm)}中的每一个元素,都有且仅有一个
中的元素与之对应,从而有下面的等同关系,

--- P'S,''),

= R™ X PXwig).

因此,当r有限时,尸(成是有限维的向量空间'如果UUR七V U
R"都是开集,则尸(U,V)是按S,'')的开子集.

\begin{quote}
现.在设\emph{M, N}分别是皿维和''维流形.若(U ,會),(V, °)分别是
上的容许坐标卡,则
\end{quote}

E \&/■ h* 疗Cr)(少。/"。广)

就给出了尸(U,V) f尸(戦。),択丫))的单、满映射,从上面的讨
论中得知是jr(m,'')中的开集,而jr(m两 与欧
氏空间同构'所以(8,尸(U,矿))就是尸3f,N)的一个坐标卡'这
样,就可以从的tT坐枚系导出\emph{J\textsuperscript{r}(.M,N)}的C。坐标系'当

是\emph{C\textsuperscript{+i}}流形时,尸(M,M)是一个\emph{甘}微分流形,称为从\emph{M}到
N的射式(jet)流形.

Thom横截定理

定义C9线性空间Z的两个线性子空间X与Y称为是横栽
的,如果它们之和是整个空间\emph{,L = X + Y.}

由于微分流形的切空间是线性空间,而切映射是切空间之间
的映射,所以可以定义两个子流形的横截性和流形间的映射与象
空间中某子流形的横截性.

定义C. 10 设\emph{A,B}是光滑Banach流形\emph{M}的两个光滑子流
形.称于流形A与8横截,如果A PI B = 0,或者V \emph{pEAdB,
T\textsubscript{P}A}与\emph{T\textsubscript{p}B}在\emph{TpM}中(在定义C.
9的意义下)横截.

定义C.11设M,N是光滑的Banach流形M是N的光滑子
流形.称C"(r\textgreater{}l)映射\emph{f:MfN在}点p £ M与A横截,如果
\emph{面 \& A,}或者/(/\textgreater{}) 6 A,并且满足下面的条件

• (1) \emph{(Tpf'XTpM) + T\textsubscript{fw}A = Tfs\textgreater{}N,而}且

(2) \emph{T\textsubscript{fw}A}
的原象\emph{.(T\textsubscript{f}\^{}A)}在 \emph{T\textsubscript{P}M}
中可裂. 在每一点都与A横截的映射f,称为与子流彩A■檎截.

注意,当M是Hilbert流形(特别地,是有限维流形)时,可裂
性条件(2)是自然成立的(见附注B. 20).

例C.12 (1)如果定JtC.il中子流形4上的每一点都是y 的正则值(见附注B.
28),则\emph{f与A}横截.

(2)如果 Af,N 是有穷维的,且 dim(M) 4-dini(\^{})\textless{}dimN,
则;■与A横截意味着\emph{f(.M) CIA =
0.}例如,把R\textsuperscript{1}嵌入R\textsuperscript{5}的映
射与R中的曲线\emph{7}横截K嵌入曲线与7在It,中无公共点'

由于横截相交性与坐标卡的选取无关,所以当\emph{M与N}都为有
穷维流形时,下面的结果显然成立.

定理C. 13设M和N分别为林维和孙维的C"微分流形, \emph{f
N}是可微映射,A是N的余维为5的\emph{C\textasciitilde{}}子流形.设\emph{M}在
血点附近有局部坐标扌,\ldots{},漬,\emph{N在fg} 点附近有局部坐标
护,\ldots{},俨.如果在『3。)的某邻域U内,点集A Ci U有表达式尸
=\ldots{}=尸=0,则映射『与子流形A横截的充分必要条件是 (尊\textbar{}
在⑶点的秩为s. I

下面设M和N是有限维光滑微分流形,且是第二可数的,人 是N的光滑子流形,记

\begin{quote}
= j/er(A/,N)\textbar{}r\textgreater{}l,/ 与 A 横截},
\end{quote}

虫Sard定理和附注C, 4,可以推出下面的结果.

定理C. 14 (Thom横截定理)\emph{W(M,N;A})是\emph{C\^{}M'N})中
的剩余集(即可数个开調集的交集),从而在\emph{UVM'N)}中稠密.如

果厶是闭子流形,则它还是开的.I

证明可参考[Hir,pp74 - 77丄这个定理说明,若厶是M的闭
子流形,则中与\emph{A}不横截的映射可以在任意小扰动下
成为横截,而原来横截的映射仍可保持横截,因此,利用橫截性可
以得出CTW'N)中的通有族,又称为一般族.

Thom横截定理可以推广到jet形式.这样就可把无限维空间
中的通有性问题,转化到有限维空间\emph{mM,N)}中的横 截问题.

定理C. 15 (Thom横截定理的jet形式) 设\emph{M,N是无}边界
的有限维光滑流形,A是的L子流形,1Mr Vs W 8. 则映射集合

\emph{事=}\{/eCCA/.JOI//与 A 横截\}

在C\textbackslash{}M,N)中是剩余集,从而是稠密集.如果4是团的,则質在
中还是开的.\textbar{}

证明可见[Hir,\textsubscript{PP}80 一 811下面的定理对于确定子流形的余
维数是重要的.

定理C.16设\emph{f'MfN}是5映射,A是N的子流形.如果
f与A横截,则尸'(4)是M的子流形.如果A在N中有有限余 维,则

codim(yTG4)) = codim(A). \textbar{}

特别地,当是浸盖时,条件勺'与4横截''总是溝足 的;而投影是特殊的浸盖.

参考文献

{[}Al{]} Arnold V I. Geometric Methods in the Theory of Ordinary
Differential Equations. Second Edition. New York: Springer-Verlag, 1983

{[}A2{]} Arnold V L Ten problems. Adv. in Sov. Math. 1990(1): 1---8*
数学 译林,1993(4):257-261

CAAIS{]}Afraimovich V \emph{S,} Arnold V I, H如shenko Yu
S\textsubscript{f} Sil\^{}iikov L P・ Bi­furcation Theoryj Dynamical
Systems V . New York: Springer-Ver­lag, 1988

{[}ALGM{]} Andronov A A, Leontovich E A, Gordon I I\textsubscript{t}
Maier A G. Theory of Bifurcations of Dynamic Systems on a Plane- New
York( Israel Program for Sci. Transl. , Wiley« 1973

{[}AMR{]} Abraham R, Marsden J E, Ratiu T. Manifolds, Tensor Analysis,
and Applications. London Amsterdam --- Tokyo; Addison-Wesley Publishing
Company, Inc. , 1983

{[}Ba{]} Bautin N N. On the number of limit cycles which appear with
variation of the coefficients from an equilibrium position of focus or
center type. Math. Sb. 1952, 30: 181 --- 196 (in Russian)i AMS Trans.
Series 1962, 5: 396-413 (in English)

{[}BC1{]} Benedicks M, Carleson L. On iteration of 1 --- «
x\textsuperscript{2} on ( --- 1,1). Ann. Math. 1985, 122: 1-25

{[}BC2{]} . The dynamics of the Hfenon map. Ann Math. 1991» 133: 73 ---

169

{[}Be{]} EeJIHHUKHft r P・ HopMaJILHHe gpMbh I HBapMaHMH H jTOWIbHblC
OTOGpaJKOHMfi- \textsc{Khcbi} Haykosa AyMKa, 1979

{[}BL{]} Bonin G, Legault J. Comparision de la methode des eonstantes de
Lia­punov et la bifurcation de Hopf. Canad. Math. Bulk 1988, 31(2): 200
-209

{[}Bol{]} Budanov R T. Versal deformations of a singular point of a
vector field

on the plane in the case of zero eigenvalues. Trudy Sem. Petrovsk. 1976,
2: 37---65 (in Russian) \emph{i} Sei. Math. Sov. 1981, 1\emph{1}
389---421 (in English)

{[}Bo2{]} ・ Bifurcations of the limit cycle of a family of plane vector
fields.

\begin{quote}
Trudy Sem. Petrovsk. 1976, 2; 23 --- 35 Gn Russain)i Sei. Math. Sov.
1981, \emph{l\textsubscript{t}} 373-387 (in English)
\end{quote}

\begin{enumerate}
\def\labelenumi{\Roman{enumi}.}
\setcounter{enumi}{99}
\item
  陈翔炎.含参数微分方程的周期僻与极限环.教学学报,1963, 13(4): 607 ---
  609
\end{enumerate}

{[}Cyl{]}曹永罗.关于非双曲奇异吸引子.北京大学博士论文,1994

{[}Cy2{]} Cao Yongluo. Strange attractor of H\^{}non map and its basin.
Scientia Sinica (Series A), 1995, 3839-35

{[}CH{]} Chow Shui-Nee, Hale J K. Methods of Bifurcation Theory. New
York: Springer-Verlag, 1982

{[}CLW{]} Chow Shui-Nee\textless{} Li Chengzhi, Wang Duo. Normal Forms
and Bifur­cation of Planar Vector Fields. New Yorkt Cambridge University
Press, 1994

{[}CM{]}蔡燧林,马晖.广义Lizard方程的奇点的中心焦点判定问题.浙江大
学学报.1991, 25(4): 562-589

{[}Cs{]}蔡燧林.二次系统研究近况.数学遂展,1989, 18(1): 5-21

{[}CS{]} Cushman R, Sanders J. A codimension two bifurcation with a
third or­der Picard-Fuchs equation. J・Diff・Eq. 1985, 59: 243 --- 256

{[}CW{]}陈兰嬴,王明二次彼分系统极限芥的相对位置和数目.败学学 报,1979,
22(6): 751-758

{[}CY{]}陈兰莉,叶奮谦.方程组等=-¥ + \&r + ZH,+ zy + p:,\$=z的
极限环的唯一性.数学学报,1975, 18. 219-222

ECZ{]}蔡燧林,张平光.二次系统极限环的唯一性.高校应用敛学学报, 1991,
6(3): 450-461

\begin{enumerate}
\def\labelenumi{\Alph{enumi}.}
\setcounter{enumi}{3}
\item
  Dulac H・ Sur ks cycles limites. Bull. Soc・ Math. Fr. 1923 ・ 511 45
  --- 188
\end{enumerate}

{[}DER{]} Dumortier F • El Morsalani M, Rousseau C. Hilbert\% 16th
problem for quadratic systems and cyclicity of elementary graphics, to
appear in

Nonlinearity

{[}DGZ{]} Drachman B\textsubscript{t} van Gils S A, Zhang Zhi-Fen.
Abelian integrals for quadratic vector fields. J. Reine Angew. Math.
1987, 3821 165---180

EDL{]} 丁同仁,李承治.常微分方程教程.北京:高等教育出版社,1991

{[}DLZ{]} Dumortier F, Li Chengzhi• Zhang Zhi-Fen. Unfolding of a
quadratic integrable system with two centers and two unbounded
hetroclinic loops. Preprint \textsubscript{f} 1996

{[}DRR1{]} Dumortier F\textsubscript{f} Roussarie R, Rousseau C.
Hilbert\% 16th problem for quadratic vector fields, J. Diff・ Eq・ 1994,
1101 86---133

{[}DRR2{]} ・ Elementary graphics of cyclicity one and two.
Nonlinearity,

1994, \emph{h} 86-133

{[}DRS1{]} Dumortier F\textsubscript{f} Roussarie R, Sotomayor J.
Generic 3-parameter family of vector fields on the plane, unfolding a
singularity with nilpotent linear part. The cusp case of codimension 3.
Ergodic Theory and Dy­namical Systems 1987, 7: 375---413

{[}DRS2{]} ・ Generic 3-parameter family o£ planar vector fields
\textsubscript{T} unfolding of

\begin{quote}
saddle, focus and elliptic singularities with nilpotent linear parts.
Lec­ture Notes in Math. 1991* 1480: 1---164
\end{quote}

{[}DZ{]}杜乃林,曾宪穢・计算焦点畳的一类通推公式.科学通报,1994, 39
(19): 1742-1744

{[}E {]} Ecalle E J・ Finitude des cycles limites et acc616ro-sommation
de \^{}application de re tour. Lecture Notes in Math. 1990・ 14551 74
--- 159

{[}EZ{]} Edmunds D E \textsubscript{t} Zheng Z. On the stable periodic
orbits of regular maps on a completely ordered invariant set. Preprint

{[}F{]}泻贝叶.临界情况下奇环的稿定性.数学学报,1990, 33(1): 113-134
{[}FLLL{]} Farr W W, Li Chengzhi\textsubscript{9} Labouriau I
S\textsubscript{t} Langford W F. Degenerete

\begin{quote}
Hopf bifurcation formulas and Hilbert七 16th problem. SIAM Math. Anal.
1989, 20: 13-30
\end{quote}

{[}GH{]} Guckenheimer J•, Holmes P. Nonlinear Oscillations, Dynamical
Systems and Bifurcations of Vector Fields. New York: Springer-Verlag
\textsubscript{v} 1983

{[}GaHo{]} Gavrilov L« Horoaov E・ Limit cycles and zero of Abelian
integrals satisfying third order Picard-Fuchs equations. Lecture Notes
in

Math. 1990, 1455: 160-196

{[}Go{]} \textsc{Pomosob} E IL 9KBMBaJieHTHocTE ceMeAcTB
4n44»eoMOp\^{}x*MOB KaHeMHoro K/iacca rjtaflKocTw- BecTH. XapbKOB.
yw-Ta- Cep. Max. -MaT, 1976, 134(41)、95--- 104

{[}Gw{]}高维新.僻析理论讲义(北京大学数学系用).

\begin{enumerate}
\def\labelenumi{\Alph{enumi}.}
\setcounter{enumi}{7}
\item
  Hilbert H D. M\&thmatische Probleme (lecture). The second
  Internation­al Congress of Mathematicians, Paris 1900» Gottinger
  Nachrichtenj 1900: 253-297
\end{enumerate}

{[}Ha{]} Hayashi S. On the solution of C\textsuperscript{l} stability
conjecture for flows. Preprint. {[}HI{]} Horozov E, Uiev I D. On
saddle-loop bifurcations of limit cycles in per­turbations o£ quadratic
Hamiltonian systems. J. Diff・ Eq. 1994, 113 (1): 84-105

{[}Hir{]} Hirsch M W. Differential Topology. New York Heidelberg
Berlin: Springer-V\^{}rlag, 1976

{[}HLZ{]}韩茂安,罗定军,朱幫明.奇闭轨分支出极限环的唯一性(I
\emph{)、5).} 数学学报,1992, 35(4): 541-548、35(5), 673-684

{[}Hm{]}韩茂安.周期扰动系统的不变环面与亚调和解的分支.中国科学(A
辑),1994(11), 1152-1160

{[}Ho{]} Horozov E. Versal deformations of equivariant vector fields in
the case of symmetry of order 2 and 3. Trans, of Petrov ski Seminar
1979, 5\emph{1} 163 \_192 (in Russian)

{[}Hw{]} Huang Wenzao. The bifurcation theory for nonlinear equations*
Lecture Notes in Pure and Applied Mathematics. VoL 109♦ New York, Marcel
Dekber. INC, 1987, 249---260.

{[}HWW{]} Huang Qichang\textsubscript{t} Wei Junjie. Wu Ji\&ngbong. Hopf
bifurcations oi some second-order FDEs with infinite delay and their
applications. Chinese Science Bulletin, 1995, 40(4):

{[}Hu{]} Hu Sen. A. proof of the C stability conjecture for
3-dimensional flows. Trans. AMS. 1994.

{[}HZ{]}韩茂安.朱德明.微分方程分支理论.北京:煤炭工业出版社,1994

\begin{enumerate}
\def\labelenumi{\Alph{enumi}.}
\setcounter{enumi}{8}
\item
  Il\textbackslash{}eshenko Yu S. Finiteness theorems for limit cycles.
  Russian Math. Surveys 1990, 40\textsubscript{1} 143-200
\end{enumerate}

{[}IL{]} UVashenko Yu S, Li Weigu. Nolocal Bifurcation\textsubscript{f}
to be published by AMS

{[}IY{]} UVashenko Yu S, Yakovenko S. Finitely smooth normal forms of
local families diffeomorphisms and vector fields. Russian Math. Surveys
1991, 46: 1-43

{[}Jajjakobson M V. Absolutely continuous invariant measure for
one-parame­ter family of one-dimensional map- Comm. Math. Phys. 1981*
81: 39 --- \emph{88}

{[}Jo{]} Joyal P. Generalized Hopf bifurcation and its dual generalized
hoinuclinic bifurcation. SIAM J. Math. 1988. 48: 481---496

{[}K{]} Khovansky A G・ Real analytic manifolds with finiteness
properties and complex Abelian integrals. Funct. Anal. Appl. 1984, 18:
119---128

{[}LI{]}廖山涛.微分动力系统的建性理论.北京,科学出版社,1992

{[}L2{]} Liao Shantao. Obstruction sets, Minimal rambling sets and their
applica­tions. In: Oiinese Mathematics into 21st Century. Peking
University Press, 1992.

{[}Lc{]}李承治.关于平面二次系统的两个向風 中国科学(A1»)・1982Q2),
1087-1096

{[}LbZ{]} Li Bao-Yi・ Zhang Zhi-Fen. A note on a result of G・ S・
Petrov about the weakened 16th Hilbert problem. JMAA 1995, 190:
489\^{}516

{[}LH{]}李继彬,黄其明.平面三次微分系统的极限环复眼分支.数学年刊(B
辑),1987, 8; 391-403

{[}LHZ{]}罗定军,韩茂安.朱德明.奇闭轨分支出极限环的唯一性(I ).数学
学报,1992, \emph{35(3)\textsubscript{;} 407-417}

{[}LR1{]} Li Chengzhi, Rousseau C. A system with three limit cycles
appearing in a Hopf bifurcation and dying in a homoclinic bifurcation
\textsubscript{t} the cusp of order 4. J. Diff. Eq. 1989, 79: 132-167

{[}LR2{]} ・ Codimension \emph{2} symmetric homoclinic bifurcation. Cam
J.

Math. 1990, 42: 191 一212

{[}Lw{]} Li Weigu. The bifurcation of "eight figure" of separatrix of
saddle with zero saddle value in the plane» Preprint of Peking
University \textsubscript{f} Research Report \textsubscript{f} No 46,
1995

{[}Lz{]}梁肇军.多项式律分系统全简分析导引.武汉,华中师范大学出版社. 1989

{[}LZ{]} Li Chengzhi, Zhang Zhi-Fen. A criterion for determing the
monotonici­ty of the ratio of two Abelian integrals. J. Diff. Eq・,1996,
124\$ 407--- 424

{[}M{]} MaM R・ A proof of the C\textsuperscript{1} stability
conjecture. Inst. Hautes. Sci・ Publ. Math. 1987- 66, 161-210

{[}Maj Mardesic P. The Number of limit cycles of polynomial deformations
of a Hamiltonian vector field. Ergod. Th・ \& Dynam. Sys. , 1990, 10;
523 \_529

{[}Mo{]} Mourtada A. Degenerate and non-trivial hyperbolic polycycles
with two vertices. J・ Diff. Eq. , 1994, 113: 68---83

{[}MV{]} Mora L, Viana L・ Abundance of strange attractor. Acta. Math.
1993, 170, 1-63

{[}Ma{]}马知恩.种群生态学的数学建模与研究.合肥,安律教育出版社,1996
{[}NS{]} Nowicki T, Strien V S. Absolutely continuous invariant measures
forC* unimodal maps satisfying the Collet-Eckmann condition- Inv・ Math.
1988, 93: 619---635

{[}Pl{]} Petrov G S・ Number of zeros of complete elliptic integrah.
Funct. Anal. AppL 1988. 18, 148---149

{[}P2{]} ・ The Chebyshev property of elliptic integrals. Funct. Anah
AppL

1988. 22: 72-73

{[}Pa{]} Palis J, A proof of the \emph{Q-} stability conjecture. Inst,
Hautes. Sci. Pubh Math. 1987, 66, 211-218

{[}Pon{]} nOHTTJITHH JI Q O flMHlMHTeCKHX CHCTeMaX, \textsc{GjIMSKIDC} K
EMHJTVTOHODbJM. \^{}ypHUi SKcnepHMeHTLrrbHoA \textsc{h} TeopeTHwecKofi
\textless{}t)HamcK, 1934* 4: 883 --- 885

{[}PT{]} Palis J, Taken,F. Hyperbolicity and sensitive chaotic dynamics
at ho­moclinic bifurcations. Cambridge University Press« 1992

{[}Q{]}秦元«h律分方程所定义的积分曲线,上、下册,北京*科学出版社,
1956C959

{[}QL{]}秦元勛,刘尊全.律分方程公式的机器推导(■)・科学通报,1981,7:
388-391

{[}Rl{]} Roussarie R. Weak and continuous equivahnces for families of
line dif- feomorphisms. In \^{}Dynamical Systems and Bifurcation
Theory*, Cama­cho, Pacifico ed.\emph{,}Longman, Scientific and
Technical, Pitman Research Notes in Math. Series 160, 1987* 377-385

{[}R2{]} ■ On the number of limit cycles which appear by perturbation of

\begin{quote}
separatrix loop of phnar vector fields, BoL Soc・ Bras. Mat. 1986, 17:
67-101
\end{quote}

{[}Rc{]} Rousseau C- Universal unfolding of a singularity of a symmetric
vector field with 7-jet C™- equivalent to + (± J:\textsuperscript{3} ±
\emph{冨・} Lecture Notes in Math. 1989, 1455\$ 334---354

{[}RS{]} Rousseau C; Schlotniuk D. Generalized Hopf bifurcations and
applica­tions to planar quadratic systems. Ann. Polon・ Math. 1988, 49*
1 --- 16

{[}RT{]} Ruelle D, Tokens F. On the nature of turbulence. Common. Math.
Phys. 1971. 20: 167-192

{[}S{]} Singer D・ Stable orbits and bifurcation of maps of the
interval. SIAM Ap- pL Math. 1978, 35: 260-267

{[}Sh{]} Shoshitaishvili A N・ Bifurcation of topok\^{}ical type of
singular points of parameterized vector fields. Funct. Anal. AppL
1972(2): 169 --- 170

{[}Si{]} Sibirskii K S. On the number of limit cycles in a neighborhood
of singular points. Diff. Eq. 1965, 11 36---47 (in Russian)

{[}Sij{]} Sijbrand J. Properties of center manifold亀 Trans. AMS 1985,
289; 431 -469

{[}Sill{]} Sil\^{}iikov L P. On a Poincart-Birkhoff problem. Math USSR
Sb. 3 i 353-371 .

{[}Sil2{]} . On the generation of a periodic motion from a trajectory
doubly

\begin{quote}
asymptotic to an eqilibrium state of saddle type. Math USSR Sb 1968, 6:
428-438
\end{quote}

{[}Sil 3{]}--- ・ A contribution to the problem of the structure of an
extended neighborhood of a rough equilibrium state of saddh-focus type.
Math USSR Sb 1970. 10, 91-102

{[}SJ{]} Shen Jaqi, Jing Zujun. A new detecting method of conditions for
the ex-

istence of Hopf bifurcation. In "Dynamical Systems* Nankai Series in
Pure, Apl. Math, and Theor・ Phy. , Vol 4, eds. S*T Liao,T-R Ding and
Y-Q Ye, World Scientific Publishing, Singapore» 1993\$ 188---203
{[}Sm{]} Smale S. Dynamics retrospective, great problems, attempts that
failed.

\begin{quote}
Nonlinear Science, The Next Decade.见"数学译林'',动力系统学的回
顾;宣大问题■失败的尝试.1993(4),262-269''
\end{quote}

{[}\%{]}史松龄.平面二次系统存在四个极限环的具体例子.中国科学,1979

(11), 1051-1056

{[}T{]} Takens F. Forced oscillations and bifurcations: Applications of
global analysis I . In Commun. Math. Vol 3 Inst. Rtjksuniv. Utrecht. ,
1974 {[}TTY{]} Thieullen P, Tresser C, Young L-S・ Positive Liapunov
exponent for gereric one-parameter families of unimodal maps. Preprint

\begin{enumerate}
\def\labelenumi{\Roman{enumi}.}
\setcounter{enumi}{4}
\item
  Vanderbauwhede A. Center manifold私 normal forms and elementary
  bi­furcations. Tn Dynamics Reported, Vol 2, ed\& U. Kirchgraber and
  O・ Walther, New York. Wiley, 1989, 89-169
\end{enumerate}

{[}V\&{]} Varchenko A N. Estimate of the number of zeros of an Abelian
integral depending on a parameter and limt cycles. Funct. AnaL AppL
1984, 18: 98-108

{[}W{]} Wen Lan, On the C stability conjecture for flows. {]}. Diff. Eq-
1996, 129(2): 334---357

{[}Wd{]} Wang Duo- An introduction to the normal form theory of odinary
dif­ferential equations. Advances in Math. 1990, 19(1)\$ 38 --- 71

{[}Wil{]} Wiggins S・ Introduction to Applied Nonlinear Dynamic成
Systems and Chaos. New York Berlin Heidelberg: Springer-Verlag,1990

{[}Wi2{]} , Global Bifurcations and Chaos \textsubscript{T} Analytical
Methods. New York

Berlin Heidelberg ■ Springer-Verlag»198 8

{[}Wl{]}王兰宇.多峰映射的动力学.北京大学博士论文.1996

\begin{enumerate}
\def\labelenumi{\Roman{enumi}.}
\setcounter{enumi}{9}
\item
  肖冬梅.一类余维3鞍点型平面向員场的分支,中国科学(A辑).1993
\end{enumerate}

(3); 252------262

LY1{]}叶彦谦等.极限环论■第二版.上海:上海科学技术出版社,1984
{[}Y2{]}叶彦3L多项式微分系统定性理论.上海,上梅科学技术出版社,1995
\textsc{{[}Yq{]} Yoccoz} J C. Recent developments in dynamics. Plenary
Address of the

TCM 勺4

{[}YY{]}杨信安,叶彦谦.方程糸=-】 + \&■ + \#+玲+陌,咨=工的
极限环的唯一性.福州大学学报,1978(2): 122-127

{[}Z{]} Zheng Zhiming. On the abundance of chaotic behavior for generic
one-pa­rameter families of maps. Acta Math- Sinica, 1996(12); 398---412

{[}Zdl{]} Zhu Deming, Melnikov vector and heteroclinic manifolds.
Science tn China, Ser.A, 1994. 37(6), 673-682.

{[}Zd2{]} ・ Melnikov-type vectors and principal normals. Science in
China,

Se``A, 1994, 37(7); 814-822.

{[}Zd3{]} . Transversal hetroclinic orbits in general degenerate cases.
Science

in China, Sen A, 1996, 39(2): U2-12L

EZDHD{]}张芷芬.丁同仁,黄文灶.董镇喜.微分方程定性理论.北京:科学
出版社.1985

:Zg{]}张恭庆.临界点理论及其应用.上海,上海科学技术出版社.1986

{[}Zj{]}张嫦炎.常微分方程几何理论与分支问庖(修订本).北京,北京大学出
版社,1987

{[}ZJ{]} Zeng Xianwu, Jing Zujun, Monotonicity and critical points of
period.

Prepress in Natural Science, 1996, 6(4) t 401 --- 407

{[}ZQ{]}张锦炎,钱敏.微分动力系统导引.北京,北京大学出版社.1991

{[}Zol{]} Zoladek H・ On the versality of certain family o£ vector
fields on the plane. Math. USSR Sb. 1984, 48: 463-492

{[}Zo2{]} ・ Bifurcations of certain family of planar vector fields
tangent to

axes. J. Diff, Eq. 1987 ・ 671 1 --- 55

{[}N{]}涅梅斯基B B.四十年来的苏联数学(1917-1957),常做分方程部分.
饶生忠译.北京:科学出版社,I960

{[}Zzfl{]} Zhang Zhi-fen・ On the uniqueness of the limit cycles of some
nonlinear oscillation equations. Doki. Acad. Nauk SSSR» 1958, 119\$ 659
--- 662 (in Russian )

{[}Zzf2{]} . Proof of the uniqueness theorem of limit cycles of
generalized

Li\&nard equations. Applicable Analysis, 1986, 23\$ 63---67.

囱晩{]}张筑生.微分动力系统原理.北京,科学出版社,1987

中文词条按首字的笔画排列,西文开头的词条按字母顺序排列.

\begin{longtable}[]{@{}llll@{}}
\toprule
\endhead
\begin{minipage}[t]{0.22\columnwidth}\raggedright
\begin{quote}
{---}
\end{quote}\strut
\end{minipage} & \begin{minipage}[t]{0.22\columnwidth}\raggedright
画\strut
\end{minipage} & \begin{minipage}[t]{0.22\columnwidth}\raggedright
\begin{quote}
五
\end{quote}\strut
\end{minipage} & \begin{minipage}[t]{0.22\columnwidth}\raggedright
\strut
\end{minipage}\tabularnewline
子流形 & 7,46-50,286 & 正规形 & 34,40\tabularnewline
\begin{minipage}[t]{0.22\columnwidth}\raggedright
马蹄映射\strut
\end{minipage} & \begin{minipage}[t]{0.22\columnwidth}\raggedright
\begin{quote}
170
\end{quote}\strut
\end{minipage} & \begin{minipage}[t]{0.22\columnwidth}\raggedright
正则奇点\strut
\end{minipage} & \begin{minipage}[t]{0.22\columnwidth}\raggedright
117,118\strut
\end{minipage}\tabularnewline
\begin{minipage}[t]{0.22\columnwidth}\raggedright
马蹄存在定理\strut
\end{minipage} & \begin{minipage}[t]{0.22\columnwidth}\raggedright
\begin{quote}
173
\end{quote}\strut
\end{minipage} & \begin{minipage}[t]{0.22\columnwidth}\raggedright
正则映射族\strut
\end{minipage} & \begin{minipage}[t]{0.22\columnwidth}\raggedright
281\strut
\end{minipage}\tabularnewline
\begin{minipage}[t]{0.22\columnwidth}\raggedright
\begin{quote}
四
\end{quote}\strut
\end{minipage} & \begin{minipage}[t]{0.22\columnwidth}\raggedright
19\strut
\end{minipage} & \begin{minipage}[t]{0.22\columnwidth}\raggedright
对参数一致的Hopf分岔定理 82\strut
\end{minipage} & \begin{minipage}[t]{0.22\columnwidth}\raggedright
\strut
\end{minipage}\tabularnewline
\begin{minipage}[t]{0.22\columnwidth}\raggedright
双曲奇点\strut
\end{minipage} & \begin{minipage}[t]{0.22\columnwidth}\raggedright
\begin{quote}
4
\end{quote}\strut
\end{minipage} & \begin{minipage}[t]{0.22\columnwidth}\raggedright
可扰动参数\strut
\end{minipage} & \begin{minipage}[t]{0.22\columnwidth}\raggedright
282\strut
\end{minipage}\tabularnewline
\begin{minipage}[t]{0.22\columnwidth}\raggedright
双曲闭轨\strut
\end{minipage} & \begin{minipage}[t]{0.22\columnwidth}\raggedright
\begin{quote}
4
\end{quote}\strut
\end{minipage} & \begin{minipage}[t]{0.22\columnwidth}\raggedright
可裂\strut
\end{minipage} & \begin{minipage}[t]{0.22\columnwidth}\raggedright
297\strut
\end{minipage}\tabularnewline
\begin{minipage}[t]{0.22\columnwidth}\raggedright
双曲不动点定理\strut
\end{minipage} & \begin{minipage}[t]{0.22\columnwidth}\raggedright
\begin{quote}
160
\end{quote}\strut
\end{minipage} & \begin{minipage}[t]{0.22\columnwidth}\raggedright
边界的水平部分\strut
\end{minipage} & \begin{minipage}[t]{0.22\columnwidth}\raggedright
159,198,200\strut
\end{minipage}\tabularnewline
\begin{minipage}[t]{0.22\columnwidth}\raggedright
分岔\strut
\end{minipage} & \begin{minipage}[t]{0.22\columnwidth}\raggedright
\begin{quote}
9
\end{quote}\strut
\end{minipage} & \begin{minipage}[t]{0.22\columnwidth}\raggedright
边界的垂直部分\strut
\end{minipage} & \begin{minipage}[t]{0.22\columnwidth}\raggedright
159,198,200\strut
\end{minipage}\tabularnewline
\begin{minipage}[t]{0.22\columnwidth}\raggedright
分岔集\strut
\end{minipage} & \begin{minipage}[t]{0.22\columnwidth}\raggedright
\begin{quote}
9
\end{quote}\strut
\end{minipage} & \begin{minipage}[t]{0.22\columnwidth}\raggedright
边界条件\strut
\end{minipage} & \begin{minipage}[t]{0.22\columnwidth}\raggedright
160\strut
\end{minipage}\tabularnewline
\begin{minipage}[t]{0.22\columnwidth}\raggedright
分宿值\strut
\end{minipage} & \begin{minipage}[t]{0.22\columnwidth}\raggedright
\begin{quote}
9
\end{quote}\strut
\end{minipage} & \begin{minipage}[t]{0.22\columnwidth}\raggedright
主稳定方向\strut
\end{minipage} & \begin{minipage}[t]{0.22\columnwidth}\raggedright
191\strut
\end{minipage}\tabularnewline
\begin{minipage}[t]{0.22\columnwidth}\raggedright
分岔图\strut
\end{minipage} & \begin{minipage}[t]{0.22\columnwidth}\raggedright
19,47\strut
\end{minipage} & \begin{minipage}[t]{0.22\columnwidth}\raggedright
\begin{quote}
六
\end{quote}\strut
\end{minipage} & \begin{minipage}[t]{0.22\columnwidth}\raggedright
19\strut
\end{minipage}\tabularnewline
\begin{minipage}[t]{0.22\columnwidth}\raggedright
分岔曲线\strut
\end{minipage} & \begin{minipage}[t]{0.22\columnwidth}\raggedright
\begin{quote}
19
\end{quote}\strut
\end{minipage} & \begin{minipage}[t]{0.22\columnwidth}\raggedright
轨道\strut
\end{minipage} & \begin{minipage}[t]{0.22\columnwidth}\raggedright
2,3.168\strut
\end{minipage}\tabularnewline
分宿方程 & 62,63 & 同宿轨 & 10\tabularnewline
分岔函数 & 60,63 & 同宿点 & 168\tabularnewline
\begin{minipage}[t]{0.22\columnwidth}\raggedright
分岔的余维\strut
\end{minipage} & \begin{minipage}[t]{0.22\columnwidth}\raggedright
\begin{quote}
46
\end{quote}\strut
\end{minipage} & \begin{minipage}[t]{0.22\columnwidth}\raggedright
同宿分岔\strut
\end{minipage} & \begin{minipage}[t]{0.22\columnwidth}\raggedright
15,85\strut
\end{minipage}\tabularnewline
\begin{minipage}[t]{0.22\columnwidth}\raggedright
无穷阶非共振\strut
\end{minipage} & \begin{minipage}[t]{0.22\columnwidth}\raggedright
\begin{quote}
41
\end{quote}\strut
\end{minipage} & \begin{minipage}[t]{0.22\columnwidth}\raggedright
异宿轨\strut
\end{minipage} & \begin{minipage}[t]{0.22\columnwidth}\raggedright
10,23,114\strut
\end{minipage}\tabularnewline
\begin{minipage}[t]{0.22\columnwidth}\raggedright
无限C水平曲线\strut
\end{minipage} & \begin{minipage}[t]{0.22\columnwidth}\raggedright
\begin{quote}
159
\end{quote}\strut
\end{minipage} & \begin{minipage}[t]{0.22\columnwidth}\raggedright
异宿点\strut
\end{minipage} & \begin{minipage}[t]{0.22\columnwidth}\raggedright
168\strut
\end{minipage}\tabularnewline
\begin{minipage}[t]{0.22\columnwidth}\raggedright
无限C垂直曲线\strut
\end{minipage} & \begin{minipage}[t]{0.22\columnwidth}\raggedright
\begin{quote}
159
\end{quote}\strut
\end{minipage} & \begin{minipage}[t]{0.22\columnwidth}\raggedright
异宿分念\strut
\end{minipage} & \begin{minipage}[t]{0.22\columnwidth}\raggedright
146\strut
\end{minipage}\tabularnewline
\begin{minipage}[t]{0.22\columnwidth}\raggedright
开折\strut
\end{minipage} & \begin{minipage}[t]{0.22\columnwidth}\raggedright
\begin{quote}
44
\end{quote}\strut
\end{minipage} & \begin{minipage}[t]{0.22\columnwidth}\raggedright
共振\strut
\end{minipage} & \begin{minipage}[t]{0.22\columnwidth}\raggedright
35.40\strut
\end{minipage}\tabularnewline
\begin{minipage}[t]{0.22\columnwidth}\raggedright
不徳定流形\strut
\end{minipage} & \begin{minipage}[t]{0.22\columnwidth}\raggedright
\begin{quote}
7
\end{quote}\strut
\end{minipage} & \begin{minipage}[t]{0.22\columnwidth}\raggedright
共振多项式\strut
\end{minipage} & \begin{minipage}[t]{0.22\columnwidth}\raggedright
36,40\strut
\end{minipage}\tabularnewline
\begin{minipage}[t]{0.22\columnwidth}\raggedright
切向童\strut
\end{minipage} & \begin{minipage}[t]{0.22\columnwidth}\raggedright
\begin{quote}
290
\end{quote}\strut
\end{minipage} & \begin{minipage}[t]{0.22\columnwidth}\raggedright
闭轨分岔\strut
\end{minipage} & \begin{minipage}[t]{0.22\columnwidth}\raggedright
64\strut
\end{minipage}\tabularnewline
\begin{minipage}[t]{0.22\columnwidth}\raggedright
切空间\strut
\end{minipage} & \begin{minipage}[t]{0.22\columnwidth}\raggedright
\begin{quote}
290
\end{quote}\strut
\end{minipage} & \begin{minipage}[t]{0.22\columnwidth}\raggedright
曲线坐标\strut
\end{minipage} & \begin{minipage}[t]{0.22\columnwidth}\raggedright
66\strut
\end{minipage}\tabularnewline
\begin{minipage}[t]{0.22\columnwidth}\raggedright
切丛\strut
\end{minipage} & \begin{minipage}[t]{0.22\columnwidth}\raggedright
\begin{quote}
290
\end{quote}\strut
\end{minipage} & \begin{minipage}[t]{0.22\columnwidth}\raggedright
多重极限辞\strut
\end{minipage} & \begin{minipage}[t]{0.22\columnwidth}\raggedright
68\strut
\end{minipage}\tabularnewline
切映射 & 291,292 & 后继函数 & 67,76,95\tabularnewline
\bottomrule
\end{longtable}

\begin{longtable}[]{@{}llll@{}}
\toprule
\endhead
自由返回 & 254 & 典型族 & 228\tabularnewline
\begin{minipage}[t]{0.22\columnwidth}\raggedright
有界返回\strut
\end{minipage} & \begin{minipage}[t]{0.22\columnwidth}\raggedright
. 255\strut
\end{minipage} & \begin{minipage}[t]{0.22\columnwidth}\raggedright
\begin{quote}
九画
\end{quote}\strut
\end{minipage} & \begin{minipage}[t]{0.22\columnwidth}\raggedright
\strut
\end{minipage}\tabularnewline
全局中心藏形 & 24 & 结构稳定 & 6,8\tabularnewline
全局中心流形定理 & 23 & 矩阵表示法 & 38\tabularnewline
全局分岔 & 10 & 相交条件 & 160\tabularnewline
向量丛 & 288,289 & 复合映射的双曲性 & 178\tabularnewline
\begin{minipage}[t]{0.22\columnwidth}\raggedright
纤维型\strut
\end{minipage} & \begin{minipage}[t]{0.22\columnwidth}\raggedright
289\strut
\end{minipage} & \begin{minipage}[t]{0.22\columnwidth}\raggedright
\begin{quote}
十画
\end{quote}\strut
\end{minipage} & \begin{minipage}[t]{0.22\columnwidth}\raggedright
\strut
\end{minipage}\tabularnewline
\begin{minipage}[t]{0.22\columnwidth}\raggedright
\begin{quote}
七百
\end{quote}\strut
\end{minipage} & \begin{minipage}[t]{0.22\columnwidth}\raggedright
\strut
\end{minipage} & \begin{minipage}[t]{0.22\columnwidth}\raggedright
弱Hilbert第16问题\strut
\end{minipage} & \begin{minipage}[t]{0.22\columnwidth}\raggedright
98---100,109\strut
\end{minipage}\tabularnewline
局部分岔 & 10 & 弱等价 & 45\tabularnewline
局部中心流形 & 25 & 通有族 & \emph{47}\tabularnewline
局部中心流形定理 & 25 & 倍周期分岔 & 16\tabularnewline
局部族 & 44 & 高阶Melnikov函数 & \emph{97}\tabularnewline
局部表示 & 286 & 积检形 & \emph{287}\tabularnewline
更簪法 & 60 & 浸入 & 7.297\tabularnewline
投影 & 288,289 & 浸盖 & 49,297\tabularnewline
\begin{minipage}[t]{0.22\columnwidth}\raggedright
极限环\strut
\end{minipage} & \begin{minipage}[t]{0.22\columnwidth}\raggedright
13,67\strut
\end{minipage} & \begin{minipage}[t]{0.22\columnwidth}\raggedright
\begin{quote}
+\_画
\end{quote}\strut
\end{minipage} & \begin{minipage}[t]{0.22\columnwidth}\raggedright
\strut
\end{minipage}\tabularnewline
芽 & \emph{44} & &\tabularnewline
& & 移位映射 & 167\tabularnewline
坐标卡 & 284 & &\tabularnewline
\begin{minipage}[t]{0.22\columnwidth}\raggedright
\begin{quote}
八画
\end{quote}\strut
\end{minipage} & \begin{minipage}[t]{0.22\columnwidth}\raggedright
\strut
\end{minipage} & \begin{minipage}[t]{0.22\columnwidth}\raggedright
符号动力系统\strut
\end{minipage} & \begin{minipage}[t]{0.22\columnwidth}\raggedright
165\strut
\end{minipage}\tabularnewline
\begin{minipage}[t]{0.22\columnwidth}\raggedright
\strut
\end{minipage} & \begin{minipage}[t]{0.22\columnwidth}\raggedright
\strut
\end{minipage} & \begin{minipage}[t]{0.22\columnwidth}\raggedright
\begin{quote}
+二画
\end{quote}\strut
\end{minipage} & \begin{minipage}[t]{0.22\columnwidth}\raggedright
\strut
\end{minipage}\tabularnewline
周期点 & 4,168 & &\tabularnewline
周期執 & 168 & 普适开折 & 19.45\tabularnewline
拓扑執道等价 & 5 & 焦点量 & 78\tabularnewline
环 & 216 & 游荡环 & 217.227\tabularnewline
非游荡集 & 4 & 超稳定周期轨道 & 243\tabularnewline
非游荡环 & 217 & 遍历 & 271\tabularnewline
非共振 & 41 & 揉序列 & 276\tabularnewline
非本质自由返回 & 255 & 嵌入 & 300\tabularnewline
\begin{minipage}[t]{0.22\columnwidth}\raggedright
单重极限环\strut
\end{minipage} & \begin{minipage}[t]{0.22\columnwidth}\raggedright
68\strut
\end{minipage} & \begin{minipage}[t]{0.22\columnwidth}\raggedright
\begin{quote}
+三画
\end{quote}\strut
\end{minipage} & \begin{minipage}[t]{0.22\columnwidth}\raggedright
\strut
\end{minipage}\tabularnewline
单峰映射 & 243 & 临界元 & 4\tabularnewline
奇点分岔 & 58.64 & 强等价 & 45\tabularnewline
细焦点 & \emph{73} & 稠密轨道 & 169\tabularnewline
细鞍点 & \emph{93} & 辐角原理 & 124\tabularnewline
\bottomrule
\end{longtable}

\begin{longtable}[]{@{}llll@{}}
\toprule
\endhead
微分流形 & 285 & ,次正规形 & 34\tabularnewline
微分结构 & 285 & \emph{i-jet} & 49\tabularnewline
微分同胚 & 286 & \emph{i}阶非共振 & 41\tabularnewline
零截面 & 288,289 & 为参数开折 & 44\tabularnewline
\begin{minipage}[t]{0.22\columnwidth}\raggedright
\begin{quote}
十四画
\end{quote}\strut
\end{minipage} & \begin{minipage}[t]{0.22\columnwidth}\raggedright
\strut
\end{minipage} & \begin{minipage}[t]{0.22\columnwidth}\raggedright
人阶Hopf分岔\strut
\end{minipage} & \begin{minipage}[t]{0.22\columnwidth}\raggedright
73\strut
\end{minipage}\tabularnewline
稳定流形 & 7 & \&阶细焦点 & 73\tabularnewline
稳定周期轨道 & 243 & Lebesgue 测度 & \emph{242}\tabularnewline
端点 & 158 & Lebesgue正桐点 & 282\tabularnewline
\begin{minipage}[t]{0.22\columnwidth}\raggedright
\begin{quote}
十五画
\end{quote}\strut
\end{minipage} & \begin{minipage}[t]{0.22\columnwidth}\raggedright
\strut
\end{minipage} & \begin{minipage}[t]{0.22\columnwidth}\raggedright
Lebesgue全稠点\strut
\end{minipage} & \begin{minipage}[t]{0.22\columnwidth}\raggedright
282\strut
\end{minipage}\tabularnewline
鞍结点分岔 & 11 & Liapunov系数法 & 78.79\tabularnewline
鞍点量 & 190,200 & Liapunov-Schmidt 方法 & 60.62,63\tabularnewline
鞍焦点 & 43 & Maigrartge 定理 & 18\tabularnewline
鞍焦环 & 216 & Melnikov 函数 & 90,97,109\tabularnewline
横截 & 46,304 & Misiurewicz 条件 & 282\tabularnewline
& & Morse-Smale 向■场 & 8\tabularnewline
Abel 积分 99,101,106-109 & Pichfork 分岔 & 11 &\tabularnewline
Banach流形 & 284 & Picard-Fuchs 方程 & 111\tabularnewline
Bogdanov-Takens \emph{系统} & 47,109.130 & Pioneer\^{} 映射 &
\emph{5}\tabularnewline
Birkhoff-Smale 定理 & 184 & Pioncare 分岔 & 94\tabularnewline
C水平曲线 & 158 & Hiss约化原理 & 27\tabularnewline
C垂直曲线 & 158 & 序单峰映射 & 244\tabularnewline
C水平带域 & 159 & Smale马歸 & 170\tabularnewline
C垂直帯城 & 159 & Schwartz 导数 & 243\tabularnewline
6线性化 & 41,43 & Thom横检定理 & 305,306\tabularnewline
germ & 44 & (/\%,㈤)矩形 & 159\tabularnewline
Fuchs 31 方程 & 118 & (相■四)矩形的高 & 162,163\tabularnewline
Hartman-Grobman 定理 & 6 & \emph{g心}矩形的宽 & 162,163\tabularnewline
Hilbert第16问题 & 99 & (如代)锥形条件 & 159\tabularnewline
Hilbert-Arnold 问题 & 98 & 。爆炸 & 228\tabularnewline
Hopf分岔 & 13 & &\tabularnewline
\bottomrule
\end{longtable}

责任编辑杨芝馨 封面设计季思九 责任绘囲都林

版式设计李承治

责任印制王彦

(京\textbf{)112}号

全书分为六章,各章内容分别是:基本槪念和准备知识,常见的局部与菲局都分
岔,几类余维2的平面同曩场分岔,双曲不动点及马蹄存在定理,空间中双曲戦点的同
宿分岔,实二次单峰映射族的吸引子.在訶三章的每章之后,部配备了一定致量的习
题.

本书可作为高等学校數学专业高年级本科生的选修课敎材,或相关专业研究生的
基BB课教材;也可供希望了解分岔理佬这门学科的学生、教师或科技人员作为参考书.

图书在版编目\textbf{(CIP)}数据

向量场的分岔理论基础/张芷芬等编.-北京:高等教

育出版社,1997

ISBN 7-04-006216-X

I.向\ldots{} 口.张\ldots{}皿.矢量场 0413.3

中国版本图书馆CIP数据核字(97)第22634号

高等教育出版社出版\\
北京沙摊后街55号\\
邮政编码:100009 传真\textsubscript{:}64014048 电话:64054588

新华书店总店北京发行所发行

国防工业出版社印刷广印刷

*■

开 \^{} 850X 1168 1/32 印张 10.375 字敷 250 000

.1997年10月第I版 1997年10月第I次印刷

印数 0 001-1 715\\
定价10.20元

凡购买高等敎育出版社的图书,如有缺页、倒页、脱页等\\
质量问题者,请与当地图书箫售部门联系调换

版权所有.不得間印

\includegraphics[width=2.33333in,height=3.7in]{media/image90.jpeg}

word版下载:\href{https://www.ixueshu.com/api/search/info/a3581eaaa81c186f5dde17aced68889d318947a18e7f9386.html?from=pdf\&ck=PTW}{{http://www.ixueshu.com}}

\textbf{阅读此文的还阅读了:}

\href{https://www.ixueshu.com/api/search/info/e4859208ffe87b819e1ea67c33098024318947a18e7f9386.html?from=pdf}{成人教育基础理论》}

\begin{enumerate}
\def\labelenumi{\arabic{enumi}.}
\setcounter{enumi}{1}
\item
  \href{https://www.ixueshu.com/api/search/info/16c147a8ae4f4038a7b5481c9869a2e9318947a18e7f9386.html?from=pdf}{{2.基础理论}}
\item
  \href{https://www.ixueshu.com/api/search/info/b182fab31af30eb17f8df8c9bbd94686318947a18e7f9386.html?from=pdf}{{基于分岔理论的ERP系统实施研究}}
\item
  \href{https://www.ixueshu.com/api/search/info/4e0dce557d53fd6c62e6d7200d0a1b1b318947a18e7f9386.html?from=pdf}{{2.基础理论}}
\item
  \href{https://www.ixueshu.com/api/search/info/9e34d7aacc0ae6f1bdcd58f11f2323ed318947a18e7f9386.html?from=pdf}{{论档案利用的理论基础}}
\item
  \href{https://www.ixueshu.com/api/search/info/4e0dce557d53fd6c92ad11030a046b8e318947a18e7f9386.html?from=pdf}{{2.基础理论}}
\item
  \href{https://www.ixueshu.com/api/search/info/c52eb7b044f9bfffebc5f9849106cd8c318947a18e7f9386.html?from=pdf}{{2.基础理论}}
\item
  \href{https://www.ixueshu.com/api/search/info/cbdda1caac6215c2ac042a3a6ba190b8318947a18e7f9386.html?from=pdf}{{一类平面五次向量场的奇点分岔}}
\item
  \href{https://www.ixueshu.com/api/search/info/16c147a8ae4f403810e5176e67e23e41318947a18e7f9386.html?from=pdf}{{2.基础理论}}
\item
  \href{https://www.ixueshu.com/api/search/info/3310b82f4f180d58a9f4f7a2f01bcb7a318947a18e7f9386.html?from=pdf}{{太极拳的理论基础}}
\item
  \href{https://www.ixueshu.com/api/search/info/850e6dc2463dc579569d00af3647e668318947a18e7f9386.html?from=pdf}{{论无因管理的理论基础}}
\item
  \href{https://www.ixueshu.com/api/search/info/886e55bb4cce3afe356fb1aa078fbb30318947a18e7f9386.html?from=pdf}{{浅论PBL的理论基础}}
\end{enumerate}

\href{https://www.ixueshu.com/api/search/info/e27ba01d21ad780feb915770f9e138c2318947a18e7f9386.html?from=pdf}{{13.2
.基础理论}}

\begin{enumerate}
\def\labelenumi{\arabic{enumi}.}
\setcounter{enumi}{13}
\item
  \href{https://www.ixueshu.com/api/search/info/586c8b8cb957527f4217ce50d230be75318947a18e7f9386.html?from=pdf}{{大跨分岔隧道分岔段施工方法研究}}
\item
  \href{https://www.ixueshu.com/api/search/info/c52eb7b044f9bfffd7431665a395e8fe318947a18e7f9386.html?from=pdf}{{色彩基础理论研究}}
\item
  \href{https://www.ixueshu.com/api/search/info/f21b7d05519c591128263b67cacd362b318947a18e7f9386.html?from=pdf}{{造林的理论基础探究}}
\item
  \href{https://www.ixueshu.com/api/search/info/f4c5973e4e895581689cb14f1ffbe82d318947a18e7f9386.html?from=pdf}{{具有二重零特征根的平面向量场分岔分析}}
\item
  \href{https://www.ixueshu.com/api/search/info/16c147a8ae4f4038470ba2d9bf5ba883318947a18e7f9386.html?from=pdf}{{2.基础理论}}
\item
  \href{https://www.ixueshu.com/api/search/info/707a828b3050b9d8fd0046410e8122da318947a18e7f9386.html?from=pdf}{{论审计的理论基础}}
\item
  \href{https://www.ixueshu.com/api/search/info/93bfd537f66e154e277926f0b671e673318947a18e7f9386.html?from=pdf}{{医事刑法的基础理论}}
\item
  \href{https://www.ixueshu.com/api/search/info/c672b6fc6944f3b6cde7b4de16698ffa318947a18e7f9386.html?from=pdf}{{论目录基础理论(下)}}
\item
  \href{https://www.ixueshu.com/api/search/info/c52eb7b044f9bfffbc2f70f14dcf0071318947a18e7f9386.html?from=pdf}{{2.基础理论}}
\item
  \href{https://www.ixueshu.com/api/search/info/af5a20653d8fe20f07f5b7915c1420f8318947a18e7f9386.html?from=pdf}{{时间永远分岔通向无数的将来一一《小径分岔的花园》导读}}
\item
  \href{https://www.ixueshu.com/api/search/info/9ccb7b1c7a71e8ffdab3cbaa5a2b8ddc318947a18e7f9386.html?from=pdf}{{法学基础理论}}
\item
  \href{https://www.ixueshu.com/api/search/info/a0d7bb67273064f3c86d800301c2d25b318947a18e7f9386.html?from=pdf}{{2.基础理论}}
\item
  \href{https://www.ixueshu.com/api/search/info/f21b7d05519c5911fe94471af2c9af50318947a18e7f9386.html?from=pdf}{{治脾法的理论基础}}
\item
  \href{https://www.ixueshu.com/api/search/info/e9d0c5b740fa83064c3efcd3a0e64b76318947a18e7f9386.html?from=pdf}{{单一税的理论基础}}
\item
  \href{https://www.ixueshu.com/api/search/info/6880d4cbb106d1b5118132ff7262483a318947a18e7f9386.html?from=pdf}{{浅析法治的理论基础}}
\item
  \href{https://www.ixueshu.com/api/search/info/be6c862ae782c14116db0ecbc89e9363318947a18e7f9386.html?from=pdf}{{推进实践基础上的理论创新}}
\item
  \href{https://www.ixueshu.com/api/search/info/9ccb7b1c7a71e8ff196b6afc955132aa318947a18e7f9386.html?from=pdf}{{论刑罚的理论基础}}
\item
  \href{https://www.ixueshu.com/api/search/info/466acba9d1aca3e9dbfaafda5ab043e5318947a18e7f9386.html?from=pdf}{{林业碳汇基础理论}}
\item
  \href{https://www.ixueshu.com/api/search/info/16c147a8ae4f40389de819c7de14e6c1318947a18e7f9386.html?from=pdf}{{2.基础理论}}
\item
  \href{https://www.ixueshu.com/api/search/info/16c147a8ae4f4038dea0003605e9c285318947a18e7f9386.html?from=pdf}{{2.基础理论}}
\item
  \href{https://www.ixueshu.com/api/search/info/16c147a8ae4f4038de5b09912d765bc6318947a18e7f9386.html?from=pdf}{{2.基础理论}}
\item
  \href{https://www.ixueshu.com/api/search/info/a1a89c49bc6bff80c053408afcfe304f318947a18e7f9386.html?from=pdf}{{五说中医基础理论}}
\item
  \href{https://www.ixueshu.com/api/search/info/f21b7d05519c5911d4ede80fbfca742f318947a18e7f9386.html?from=pdf}{{基于分岔理论的大型并网光伏模型探究}}
\item
  \href{https://www.ixueshu.com/api/search/info/0d3f00e8fa9f0eac475c5786c9cc24cc318947a18e7f9386.html?from=pdf}{{时间:分岔与循环一一读《小径分岔的花园》}}
\item
  \href{https://www.ixueshu.com/api/search/info/07ab6bb454441f20625d627589b464f8318947a18e7f9386.html?from=pdf}{{司法审查的理论基础}}
\item
  \href{https://www.ixueshu.com/api/search/info/6f6ebb7bc153a6a8fbff1aa1b4487f78318947a18e7f9386.html?from=pdf}{{股权激励的理论基础}}
\item
  \href{https://www.ixueshu.com/api/search/info/c52eb7b044f9bfff47a4173522984669318947a18e7f9386.html?from=pdf}{{2.基础理论}}
\item
  \href{https://www.ixueshu.com/api/search/info/a0d7bb67273064f3c9b51cf1a4080b65318947a18e7f9386.html?from=pdf}{{压合基础理论}}
\item
  \href{https://www.ixueshu.com/api/search/info/c2ce2a60dbe1de5d7506eab85b64a851318947a18e7f9386.html?from=pdf}{{论审判监督的理论基础}}
\item
  \href{https://www.ixueshu.com/api/search/info/214c9856c46110074aa829a3e2793c5b318947a18e7f9386.html?from=pdf}{{计税基础的理论分析}}
\item
  \href{https://www.ixueshu.com/api/search/info/16c147a8ae4f40380dca8f358f391cf7318947a18e7f9386.html?from=pdf}{{2.基础理论}}
\item
  \href{https://www.ixueshu.com/api/search/info/6880d4cbb106d1b5f8886f61ce516b76318947a18e7f9386.html?from=pdf}{{现场管理的理论基础}}
\item
  \href{https://www.ixueshu.com/api/search/info/2748d45327f762db44a3b899707ad492318947a18e7f9386.html?from=pdf}{{论快易网球的理论基础}}
\item
  \href{https://www.ixueshu.com/api/search/info/16c147a8ae4f4038987f902320cbe9d6318947a18e7f9386.html?from=pdf}{{分层教学的理论基础}}
\item
  \href{https://www.ixueshu.com/api/search/info/df79d37cc467dbb9e85944743c450306318947a18e7f9386.html?from=pdf}{{一维向量场含时变参数的非完全分岔}}
\item
  \href{https://www.ixueshu.com/api/search/info/1b154ecb637fc9ea14bfe24b4944e15a318947a18e7f9386.html?from=pdf}{{具有二重零特征根的平面向量场的分岔分析}}
\item
  \href{https://www.ixueshu.com/api/search/info/99b122264be0bceee2cabdce0bd5f19d318947a18e7f9386.html?from=pdf}{{电子证据的基础理论}}
\end{enumerate}
\end{document}