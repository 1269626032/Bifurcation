\section{中心流形定理}
在考虑一个向量场的分岔问题之前,一般要进行简化处理。
一方面,希望在不改变力学性质的前提下把相空间维数尽可能降低;
另一方面,力求在等价的意义下把微分方程的形式尽可能化简。
前者要用到中心流形定理,将在本节介绍,后者要用正规形理论,将在下节介绍。
这两节中讨论主要对$\mathbf { R } ^ { n }$中向量场进行,不难把结果推广到微分同胚的情形(例如,参见\ref{Wi}).
观察上节图\ref{pic1-6}可以发现,二维空间上的分岔其实主要是由动力系统在一维不变流形\(y=0\) 上结构变化所决定(参照图\ref{pic1-5}),而在这不变流形之外的轨线,无非是向者不变流形的压缩。
出现这种规律并不是偶然的,系统\ref{equ2.3}在\((0,0)\) 点的线性部分矩阵为
\[
	\left(
	\begin{array} { c c }
			{ 0 } & { 0 }   \\
			{ 0 } & { - 1 }
		\end{array}
	\right)
\]
它分别以0和-1为特征值,它的非双曲部分是一维的。
由此猜想:
当微分方程右端在某一奇点的线性部分矩阵有\(n_1\) 个零实部特征根,\(n_2\) 个非零实部特征根时,可以把分岔现象的研究,在奇点附近限制在某一个\(n_1\) 维的不变流形上,i从而使问题的难度得以降低。

\subsection{线性情形}
先考虑线性方程

\begin{equation}
	\frac {\mathrm { d } x } { \mathrm { d } t } = \boldsymbol { A } x,
	\label{eq:1.3.1}
\end{equation}

其中\(x \in\mathbf{R} ^ { n }\) ,\(A\) 为n阶实数矩阵。
我们知道,方程\ref{equ3.1}的解

\begin{equation}
	\varphi _ { t } ( x ) = e ^ { tA} x
	\label{eq1.3.2}
\end{equation}

的性态完全被矩阵A的特征值的性质所决定。

设矩阵A的特征值的集合为$\sigma = \sigma ( \boldsymbol { A } )$,则

\[
	\sigma=\sigma_{s} \cup\sigma_ {u}\cup\sigma_{ c },
\]

其中
\begin{equation}
	\begin{aligned}
		\sigma _ { s } \stackrel { \mathrm { d } } { \longrightarrow } \{ \lambda \in \sigma | \operatorname { Re } \lambda < 0 \} \\
		\sigma _ { u } = \{ \lambda \in \sigma | \operatorname { Re } \lambda > 0 \}                                               \\
		\sigma _ { c } = \{ \lambda \in \sigma | \operatorname { Re } \lambda = 0 \}
	\end{aligned}
	\label{eq:1.3.3}
\end{equation}

记$E ^ { s }$为$\mathbf {
		R} ^ { \boldsymbol { n} }$中相应于$\lambda \in \sigma _ { s }$的那些特征值的广义特征向量所张成的子空间;
并可类似定义\(E^u\) 和\(E^c\),则有直和分解。

\begin{equation}
	\mathbf { R } ^ { n } = \boldsymbol { E } ^ { s }\oplus E ^ { u } \oplus E ^ { c },
	\label{eq:1.3.4}
\end{equation}

和相应的投影
\[
	\pi_{s} : \mathbf{R}^ { n } \rightarrow E ^ { s },
	\pi _ { u } : \mathbf { R } ^ { n } \rightarrow E ^ { u },
	\pi _ { c } : \mathbf { R } ^ { n } \rightarrow E ^ {  c  }.
\]
bb
这些投影映射的零空间分别为
\begin{align}
	\operatorname { ker } \left( x _ { s } \right) = E ^ { cu } \stackrel { d } { \longrightarrow } E ^ { c } \oplus E ^ { u }, \\
	\operatorname { ker } \left( x _ { u } \right) = E ^ { cs } \stackrel { d } { \longrightarrow } E ^ { c } \oplus E ^ {s },  \\
	\operatorname { ker } \left( x _ { c } \right) = E ^ { h } \stackrel { d } { \longrightarrow } E ^ { s } \oplus E ^ { w },
\end{align}
上述投影都与A可交换,故$E ^ { s } , E ^ { u } , E ^ { \mathrm { c } }$都是\ref{equ3.1}的不变子空间。


当$t \rightarrow + \infty$时,从非奇点出发的轨道在\(E^s\) 中是指数型“压缩”的,而在\(E^u\) 中则是指数型“增长的”($t \rightarrow - \infty$时情况相反)。
所有对$t \leftarrow + \infty$有界的轨道(特别地,所有奇点,闭轨)都停留在\(E^c\)内。
由于这些性质,通常称\(E^s\)为\textbf{稳定子空间},\(E^u\)为\textbf{不稳定子空间},\(E^c\)为\textbf{中心子空间},$E ^ { h} = E ^ { s } \oplus E ^ { u}$为\textbf{双曲子空间},并记投影$\pi _ { h } : R ^ { n } \rightarrow E^h$.
轨道的动力学行为在双曲子空间内是单纯的,而复杂现象发生在中心子空间\(E^c\)内。

\subsection{非线性情形}

现在考虑非线性方程

\begin{equation}
	\frac { \mathrm { d } x } { \mathrm { d } t } = \mathbf { A } \boldsymbol { x } + f ( x ),
	\label{eq:1.3.5}
\end{equation}

其中$f \in C ^ { k } \left( \mathbf{R} ^ { n } , \mathbf{R} ^ { n } \right) , k \geqslant 1 , f ( 0 ) = 0$且$\operatorname { Df } ( 0 ) = 0$。

问题是:
方程\ref{equ3.5}的轨道结构是否任然具有方程\(equ3.1\)的上述规律?
下面的结果表明,线性方程\ref{equ3.1}的中心子空间\(E^c\)推广为非线性方程\ref{equ3.5}的中心流形\(W^c\).
虽然在较强条件下\(W^c\)可以整体存在,但通常实用的还是在奇点\(x=0\)的局部。
那些“复杂现象”(特别地,所有奇点、闭轨、同宿轨、异宿轨等)都发生在\(W^c\)上:
在一定条件下,\(W^c\)外的解指数型地趋于\(W^c\)上的解;
且流形\(W^c\)上解的性质,可通过对\(E^c\)上诱导的方程的研究而得到。
本节的内容主要参考了[V]和[CLW].
我们先称述整体的结果。
\begin{defination}
	设X , Y为Banach空间, $ k \in \mathbf { Z } ^ { + }$ ,定义映射空间
	\begin{equation*}
		C_ { b } ^ { k } ( X , Y )
		\left\{ g \in C ^ { k } ( X , Y )|g%的C^k模有界
		\right\}
	\end{equation*}
	在本节中,记 $\tilde { x } ( t , x )$为\ref{equ3.5}的满足初值条件$\tilde { x } ( 0 , x ) = x$ 的解。
	并记 $\| \mathrm { Dg } \| = \sup _ { x \in \mathbf { X } } | \operatorname { Dg } ( x ) |$ .
\end{defination}

\begin{theorem}[全局中心流形定理]
	对于系统\ref{equ3.5},存在与矩阵A和数k有关的正数\(\delta_k\),
	如果$f \in C _ { b } ^ { k } \left( \mathbf { R } ^ { n } , \mathbf { R } ^ { n } \right)$,且$\left| \mathrm { D } f \left\| < \delta _ { k }\right. \right.$,则有下列结论:
	
	
	(1)集合
	% \begin{equation}
	%                 W ^ { c } = \left\{ x \in \mathbf { R } ^ { n }
	%                 |\sup _ { \theta \in \mathbb { R } }|\pi _ { h }\mathbb { x } ( t , x )< \infty\}
	% \end{equation}
	?????????????????????????????????
	是\ref{equ3.5}的不变集,它是\(\mathbb{R}^n\)的\(C^k\)子流形,
	既存在唯一的\(\varphi \in C _ { b } ^ { k } \left( E ^ { \mathrm { c } } , E ^ { k } \right)\),使
	\begin{equation}
		W ^ { c } = \left\{ x _ { \mathrm { c } } + \varphi \left( x _ { \mathrm { c } } \right) | x _ { c } \in E ^ { c } \right \};
		\label{eq:1.3.7}
	\end{equation}
	
	
	(2)如果有$\psi \in C _ { b } ^ {  0 } \left( E ^ { c } , E ^ { h } \right)$,使集合
	\[
		M ^ { c } = \left\{ x _ { c } + \psi \left( x _ { c } \right) | x _ { c } \in E ^ { c } \right\}
	\]
	是\ref{equ3.5}的不变集,则$M ^ { c } = W ^ { c }$,且$\phi = \varphi$;
	
	
	(3)如果$\boldsymbol { y } \in W ^ { c }$,令$x _ { c } ( t ) = \pi _ { c } \widetilde { x } ( t , y )$,则$x _ { c } ( t )$满足方程
	\begin{equation}
		\frac { \mathrm { d } x _ { \mathrm { c } } } { \mathrm { d } t } = \mathbf { A } x _ { c } + \pi _ { \mathrm { c } } f \left( x _ { c } + \varphi \left( x _ { c } \right) \right),
		x _ { c } \in E ^ { \mathrm { c } }.
		\label{eq:1.3.8}
	\end{equation}
	\label{thm1.3.2}
\end{theorem}

\begin{defination}
	定理\ref{thm3.2}中的不变集\(W^c\)称为\ref{equ3.5}的\textbf{全局中心流形}。
\end{defination}

\begin{corollary}
	定理\ref{thm3.2}中有关全局中心流形唯一性的结论(2)指出,若\(W^c\)在\ref{equ3.5}的流下不变,则$\varphi \in C _ { b } ^ { 0 } \left( E ^ { c } , E ^ { h } \right)$是唯一确定的。
	如果改为$\varphi \in C ^ { 0 } \left( E ^ { c } , E ^ { h } \right)$,则结论一般不对,见\ref{sij}.
	
	定理\ref{thm3.2}中的条件$\| \mathbf { D } f \| < \delta _ { k }$是很强的,它使得该定理实际上很难应用。
	由于$f ( 0 ) = 0 , \operatorname { Df } ( 0 ) = 0$,故在奇点\(x=0\) 附近这个条件却是自然成立的。
	因此,采用截断(cut-off)函数从定理\ref{thm3.2}得出的局部结果更自然,从而更实用。
	取截断函数$\eta ( x ) \in C ^ { \infty } \left( \mathbb { R } ^ { n } , \mathbb { R } \right)$,
	满足$ 0 \leqslant \eta ( x ) \leqslant 1$,且
	\[
		\eta ( x ) =
		\left\{
		\begin{array} { l l }
			{ 1 , } & { text{当 }\| x \| \leqslant 1} , \\
			{ 0 , } & { text{当}\| x \| \geqslant 2 .}
		\end{array}
		\right.
	\]
	
	再令
	\begin{equation}
		f _ { \rho } ( x ) = f ( x ) \eta \left( \frac { x } { \rho } \right) , \quad \forall \mathbf{x} \in \mathbb { R } ^ { n }
		\label{eq:1.3.9}
	\end{equation}
	此时,为了研究方程\ref{equ3.5}在\(x=0\) 附近的中心流形,我们考虑方程
	\begin{equation}
		\frac { \mathrm { d } x } { \mathrm { d } t } = \boldsymbol { A } x + f _ { \rho } ( \boldsymbol { x } )
		\label{eq1.3.10}
	\end{equation}
	显然,当\(\|x\| \leqslant \rho \) 时,$f ( x ) \equiv f _ { \rho } ( x )$,且不难证明
	\begin{equation}
		\left\| \operatorname { D } f _ { \rho } ( x ) \right\| \rightarrow 0,
		\text{当}\rho \rightarrow 0.
		\label{eq1.3.11}
	\end{equation}
\end{corollary}

\begin{theorem}[局部中心流形定理]
	设$f \in C ^ { k } \left( \mathbb { R } ^ { n } , \mathbb { R } ^ { n } \right) , k \geqslant 1$,
	$f ( 0 ) = 0 , \operatorname{ D } f ( 0 ) = 0$,
	则$\exists \varphi \in C ^ { k } \left( E ^ { c } , E ^ { h } \right)$在\(\mathbb{R}^n\) 中的开领域U,使得
	
	(1)流形
	\begin{equation}
		W _ { \varphi } = \left\{ x _ { c } + \varphi \left( x _ { c } \right) \right) | x _ { c } \in E ^ { c } \}
		\label{eq:1.3.12}
	\end{equation}
	对\ref{equ3.5}的流\textbf{局部不变},既
	\[
		\tilde { x } ( t , x ) \in W _ { \varphi } , \forall x \in W _ { \varphi } \cap U , \forall t \in J _ { U } ( x ).
	\]
	这里\(\tilde{x}(t,x)\)为\ref{equ3.5}的满足$\widetilde { x } ( 0 , x ) = x$的解,
	\(J_U(x)\)为x在U内极大流相应时间的时间区间;
	
	(2)$\varphi ( 0 ) = 0 , \operatorname{D }\varphi ( 0 ) = 0$;
	
	(3)如果$x \in U$,且$J _ { U } ( \boldsymbol { x } ) = \mathbb{ R }$,则$x \in W_\varphi$.
	
\end{theorem}

\begin{proof}
	设与\(\operatorname{A}\)相关的\(\delta_k\)已经确定(见定理\ref{thm3.2}的条件),则由\ref{equ3.9}和\ref{equ3.11},可取\(\rho>0\)使$f _ { \rho} ( x ) \in C _ { b } ^ { k } \left( \mathbb { R } ^ { n } , \mathbb { R } ^ { n } \right)$,
	且$\left\| \operatorname { D } f _ { \rho } \right\| < \delta_k$.
	对系统\ref{equ3.10}应用定理\ref{thm3.2}可知,存在由\ref{equ3.7}给出的\(C^k\) 子流形\(W^c\),其中
	$\varphi \in C _ { b } ^ { k } \left( E ^ { c } , E ^ { h } \right)$,且
	$\varphi ( 0 ) = 0 , \operatorname{D} \varphi ( 0 ) = 0$。
	
	另一方面,由\ref{equ3.9}可知,
	若取$U \stackrel { \mathrm { d } } { \longrightarrow } \left\{ x \in \mathbb { R } ^ { n } |\ | x \| < \rho \right\}$,
	则系统\ref{equ3.10}与系统\ref{equ3.5}在U中完全相同,故结论(1),(2)成立。
	限制在U内,此处的\(W_\varphi\)就是\ref{equ3.6}或\ref{equ3.7}式定义的\(W^c\).
	
	现设$\boldsymbol { x } \in U , J _ { U } ( \boldsymbol { x } ) = \mathbb { R }$。
	则$ \forall t \in \mathbb { R } , \tilde { \boldsymbol { x } } ( t , \boldsymbol { x } ) \equiv \tilde { \boldsymbol { x } } _ { \rho } ( t , \boldsymbol { x } ) \subset U$,
	从而$\sup _ { a \in \mathbb { R } } \left| \pi _ { h } \tilde { x } ( t , x ) \right| < \infty$,由\ref{equ3.6}式,$\boldsymbol { x } \in \mathbf { W } ^ { c }$。
	限制在U内,也就是$x \in W _ { \varphi }$。
	故结论(3)成立。
\end{proof}

\begin{defination}
	如果$\psi \in C ^ { k } \left( E ^ { c } , E ^ { h } \right) , k \geqslant 1 , \psi ( 0 ) = 0 , \operatorname{D} \varphi ( 0 ) = 0$,
	使
	$W_\psi \underbrace { \mathrm { d } }\left\{ x _ { c } + \psi \left( x _ { c } \right) | x _ {c } \in E ^ { c } \right\}$
	在\ref{equ3.5}的流下局部不变,则称\(W_\psi\)为\ref{equ3.5}的一个\(k\)\textbf{局部中心流形}。
\end{defination}


\begin{corollary}
	显然,可以对\ref{equ3.5}取不同的截断函数,而得到不同的局部中心流形(尽管对每一个截断函数而言,\ref{equ3.10}的全局中心流形是唯一的)。
	例如,在图\ref{tu1-6}$( \mu = 0 )$中的原点附近,取右半平面上与x轴相切的任一轨线,在拼接上坐标原点及负x轴,都构成一个局部中心流形。
	但从定理\ref{rhm3.5}的结论(3)可知,
	\ref{equ3.5}保持在U内的任何有界轨道(包括奇点、周期轨、同宿轨、异宿轨等)都出现在\ref{thm3.5}的任一局部中心流形上。
	因此,对于研究分岔现象而言,局部中心流形的不唯一性不是一个重要的问题。
	还要指出,虽然\(f\)的\(C^k\)光滑性保证了\(W_\varphi\)的\(C^k\)光滑性,一般来说,\(f\)的\(C^\infty\)光滑性(甚至解析性)却不足以保证\(W_\varphi\)是\(C^\infty\)的。
	事实上,从定理的证明中可以看出,U是以\(\rho\)为半径的球形领域,而\(\rho\)的选取要保证$\left\| \operatorname { D } f _ { \rho } \right\| < \delta _ { k }$.
	一般来说,当$k \rightarrow \infty$,$\delta _ { k } \rightarrow 0$,这可能导致$\rho\rightarrow 0$。
	
	现在我们把局部中心流形、稳定、不稳定流形的结果合写成下面的定理。
\end{corollary}

\begin{theorem}
	对于方程\ref{equ3.5},设$f \in C ^ { k } \left( \mathbb { R } ^ { n } , \mathbb { R } ^ { n } \right) , f ( 0 ) = 0$,
	$\operatorname{D}f ( 0 ) = 0$;
	相对于\(\operatorname{A}\),
	由如上所述的子空间\(E^s\),\(E^u\)和\(E^c\).
	则在\(\mathbb{R}^n\)中\(x=0\)附近存在开领域U,
	和U中的\(C^k\)流形
	\(W^s\),\(W^u\)和\(W^c\),它们的维数分别与这三个子空间相同,
	在x=0点分别与\(E^s\),\(E^u\)和\(E^c\)相切,并且在U内是方程\ref{equ3,5}的不变流形;
	\(W^s\)和\(W^u\)有定义\ref{def1.12}所表示的形式,
	\(W^c\)有表达式\ref{equ3.12},其中$\varphi \in C ^ { k } \left( E ^ { \mathrm { c } } , E ^ { \mathrm{k} } \right) , \varphi ( 0 )= 0 , \operatorname{D} \varphi ( 0 ) = 0$.
\end{theorem}

类似于中心流形的讨论,可以定义并讨论\ref{equ3.5}的中心稳定流形\(W^{cs}\)和中心不稳定流形\(W^{cu}\),这里不再详述。
上面已经说道,系统在奇点附近的“复杂现象”发生在它的任一局部中心流形上,下面的两个定理说明了中心流形的其他重要作用。

\begin{theorem}[渐进性质定理]
	设$f \in C ^ { 1 } \left( \mathbb { R } ^ { n } , \mathbb { R } ^ { n } \right) , f ( 0 ) = 0,\operatorname { D } f ( 0 ) = 0$,
	且对矩阵\(\operatorname{A}\),$\delta _ { x } = \varnothing$,
	令\(W_\varphi\)为\ref{equ3.5}的一个\(C^2\)局部中心流形,
	则可在\(\mathbb{R}^n\)中找到O的一个领域V和正数\(\gamma\),
	如果\(x \in V\),
	且 \( \{ \tilde{x} ( t , x ) | t \geqslant 0 \} \) 的闭包含在V内,
	则$\exists t _ { 0 } \geqslant 0 , M > 0$和\(y \in W _ { \varphi } \cap V\),使得
	\begin{equation}
		\left| \tilde { x } ( t , x ) - \tilde { x } \left( t - t _ { 0 } , y \right) \right| \leqslant M e ^ { - \gamma _ { t } } , \quad \forall t \geqslant t _ { 0 }
		\label{eq1.3.13}
	\end{equation}
	\label{thm1.3.9}
\end{theorem}

\begin{theorem}[Pliss约化原理]
	在定理\ref{thm3.9}的条件下,设
	\(y \in W _ { \varphi } \cap V\),
	且$\{ \tilde { x } ( t , x ) | t \geqslant 0 \}$的闭包含在于V中。
	则$\tilde { x } ( t , y )$作为\ref{equ3.5}的解是稳定的(渐进稳定,或不稳定)的,当且仅当\(x_c(t)\)作为\ref{equ3.8}的解是稳定的(渐进稳定,或不稳定)的。
	\label{thm1.3.9}
\end{theorem}

定理\ref{thm3.9}说明,在一定条件下,在奇点O的一个小领域V内,中心流形外的解可以指数型地趋于中心流形上的某一解
(当\(t\to +\infty\),如果\(\sigma_u=\varnothing\);
或当\(t\to -\infty\),如果\(\sigma_s=\varnothing\)).
而定理\ref{thm3.10}说明,在类似条件下,为了得到局部中心流形上O点附近的轨道结构,只需要对它在线性子空间\(E^c\)上诱导的方程\ref{equ3.8}来研究即可。
事实上,\ref{equ3.8}的轨道是\ref{equ3.5}在\(W^c\)上的真实轨道向\(E^c\)的投影,见图\ref{pic1-14}.
一般而言,从原方程\ref{equ3.5}得到诱导方程\ref{3.8}不是容易的,需要先知道\(\phi\),为此我们给出下面的定理。


\begin{theorem}
	设$f \in C ^ { k } \left( \mathbb{ R } ^ { n } , \mathbb{ R } ^ { n } \right) , k \geqslant 1 , f ( 0 ) = 0 , \operatorname { D } f ( 0 ) = 0$;
	$\varphi \in C ^ { 1 } \left( E ^ { c } , E ^ { h } \right) , \varphi ( 0 ) = 0 , \operatorname{D} \varphi ( 0 ) = 0$.
	则
	$W _ { \varphi } \stackrel { d } { \longrightarrow } \left\{ x _ { c } + \varphi \left( x _ { c } \right) x _ { c } \in E ^ { c }\right\}$是\ref{equ3.5}的一个局部中心流形,当且仅当存在\(E^c\)中原点的开领域\(\Omega\),使得$\forall x _ { c } \in \Omega$,有
	\begin{equation}
		\begin{aligned}
			\operatorname{D} \varphi \left( x _ { c } \right) \pi _ { c } \left( \operatorname{A} x _ { c } + f \left( x _ { c } + \varphi \left( x _ { c } \right) \right)\right. \\
			= \pi _ { h } \left( \operatorname{A} \varphi \left( x _ { c } \right) + f \left( x _ { c } + \varphi \left( x _ { c } \right) \right)\right.
		\end{aligned}
		\label{eq:1.3.14}
	\end{equation}
\end{theorem}

在很多情形下,我们并不需要知道\(\varphi(x_c)\)的确切表达式,而是利用\ref{equ3.14}式算出它的Taylor展开的前几项。

为了简单,先把\ref{equ3.5}化成如下的标准形式:

\begin{equation}
	\left\{
	\begin{array} { l }
		{ \frac { \mathrm { d } x } { \mathrm { d } t } = \operatorname { B } x + f ( x , y ) } \\
		{ \frac { \mathrm { d } y } { \mathrm { d } t } = \operatorname { C }y + g ( x , y ) }
	\end{array}
	\right.
	\label{eq:1.3.15}
\end{equation}

其中$x \in \mathbb { R } ^ { n } , y \in \mathbb { R } ^ { m } , f , g = O \left( | x , y | ^ { 2 } \right)$;
\(\mathbf{B}\) 的特征根的实部为零,而

\[
	\boldsymbol { C } =
	\left[
		\begin{array} { l l }
			{ \boldsymbol { C } _ { 1 } } & { \mathbf { 0 } }             \\
			{ \mathbf { 0 } }             & { \boldsymbol { C } _ { 2 } }
		\end{array}
		\right]
	,
\]

上面的\(\operatorname{C_1}\) 与 \(\operatorname{C_2}\) 的特征根实部分别为负数与正数。
因此 \(  E ^ { c} = \{ ( x,0)\} ,E ^ { h } = \{ ( 0 , y ) \} , W ^ { c } = \{ ( x , \varphi ( x ) ) | x \in \mathbb { R } ^ { n } \}\).下面尝试寻找\(y=\varphi(x)\) 的展式,
$x \in \mathbb { R } ^ { n }$,
\ref{equ3.14}现在成为
\[
	\operatorname { D} \varphi  ( x ) \left[ \operatorname { B }  { x } + f ( x , \varphi ( x ) ) \right] = \operatorname{C} \varphi ( x ) + g ( x , \varphi ( x ) )
\]
既
\begin{equation}
	\operatorname { D } \varphi ( x ) [ \operatorname { B } x + f ( x , \varphi ( x ) ) ] - \operatorname { C } \varphi ( x ) - g ( x , \varphi ( \boldsymbol { x } ) ) = 0
	\label{eq:1.3.16}
\end{equation}
以及条件
\[
	\varphi ( 0 ) = 0 , \operatorname { D } \varphi ( 0 ) = 0.
\]
利用待定系数法,可逐项计算\(\varphi(x)\).

\begin{example}
	考虑二维方程
	\begin{equation}
		\left\{
		\begin{array} { l }
			{ \frac { \mathrm { d } x } { \mathrm { d } t } = y } \\
			{ \frac { \mathrm { d } y } { \mathrm { d } t } = \beta y + x ^ { 2 } + x y }
		\end{array}
		\right.
		\label{eq1.3.17}
	\end{equation}
	
	经过扰动在在奇点\((0,0)\)附近可能发生的分岔现象,
	其中\(\beta \neq 0\).
	注意\ref{equ3.17}在\((0,0)\)点的线性部分矩阵为$\left(
		\begin{array} { l l }
				{ 0 } & { 1 }     \\
				{ 0 } & { \beta }
			\end{array}
		\right)$,
	因此当$\beta \neq 0$时,它有且只有一个零根,中心流形是一维的。
	我们首先设法找出方程\ref{equ3.17}在\(E^c\)上诱导出的方程,由此推断在中心流形上轨道的结构。
	为此,先把\ref{equ3.17}化为\ref{3.15}的形式。
	令
	\[
		\left(
		\begin{array}  { l }
				{ u } \\ { v }
			\end{array}
		\right)
		= \left(
		\begin{array} { l l }
				{ \beta } & { - 1 } \\
				{ 0 }     & { 1 }
			\end{array}
		\right)
		\left(
		\begin{array} { l }
				{ x } \\
				{ y }
			\end{array}
		\right)
	\]
	,既
	\[
		\left(
		\begin{array} { l }
				{ x } \\
				{ y }
			\end{array}
		\right)
		= \left(
		\begin{array} { l l }
				{ \frac { 1 } { \beta } } & { \frac { 1 } { \beta } } \\
				{ 0 }                     & { 1 }
			\end{array}
		\right)
		\left(
		\begin{array} { l }
				{ u } \\
				{ v }
			\end{array}
		\right)
	\]
	则\ref{equ3.17}化为
	\begin{equation}
		\left\{
		\begin{array} { l }
			{ \frac { \mathrm { d } u } { \mathrm { d } t } = - \frac { 1 } { \beta ^ { 2 } } ( u + v ) ^ { 2 } - \frac { 1 } { \beta } ( u + v ) v }, \\
			{ \frac { \mathrm { d } v } { \mathrm { d } t } = \beta v + \frac { 1 } { \beta ^ { 2 } } ( u + v ) ^ { 2 } + \frac { 1 } { \beta } ( u + v ) v }.
		\end{array}
		\right.
		\label{eq:1.3.18}
	\end{equation}
	
	注意到\ref{3.16}式下面的条件,我们可以设\ref{equ3.18}的中心流形函数
	\[
		v = \varphi ( u ) = a u ^ { 2 } + b u ^ { 3 } + \dots
	\]
	所表达。
	对方程\ref{equ3.18}应用\ref{equ3.16}式,并把$\varphi ( u )$的如上表达式代入,由待定系数法不难得到
	\[
		\varphi (u ) = - \frac { 1 } { \beta ^ { \alpha } } u ^ { 2 } + \cdots
	\]
	再把上式代入\ref{equ3.18}的第一个方程,
	得出\ref{equ3.18}在\(E^c\)上诱导的方程为
	\[
		\frac { \mathrm { d } u } { \mathrm { d } t } = - \frac { 1 } { \beta ^ { 2 } } u ^ { 2 } + \dots
		\quad
		\beta \neq 0.
	\]
	由定理\ref{thm3.10},附注\ref{fuzhu2.13}和例\ref{exam2.7}可知,系统经扰动在中心流形上发生鞍结点分岔。
	当\(\beta<0\)时,它的拓扑结构与图\ref{pic1-6}相同。
\end{example}

最后,我们考考虑系统\ref{equ3.5}依赖于参数的情形。
设
\begin{equation}
	\frac { \mathrm { d } x } { \mathrm { d } t } = \operatorname{A} x + f ( x , \mu ),
	\label{eq:1.3.19}
\end{equation}

这里对矩阵A的假设同上,\(E^u,E^s\)和\(E^c\)的维数分别是
\(n^+,n^-\)和\(n^0\);
$f \in C ^ { r } \left( \mathbb{ R } ^ { n } \times \mathbb { R } ^ { k } , \mathbb{ R } ^ { n } \right) , r \geqslant 1 , f (\boldsymbol{ x} , 0 ) = \boldsymbol { O } \left( \| \boldsymbol{ x} \| ^ { 2 } \right)$.
为简单起见,还设\(f(0,\mu)=0\),
既\(x=0\)总是\ref{equ3.19}的一个奇点。

\begin{theorem}
	在上面的假设下,\ref{equ3.19}拓扑轨道等价于如下系统
	\begin{equation}
		\left\{
		\begin{array}{l}
			\frac { d \xi } { d t } = g ( \xi , \mu ) ,
			\quad
			\xi \in \mathbb { R } ^ { n _ { 0 } } , \mu \in \mathbb { R } ^ { k }, \\
			\frac { d \eta } { d t } = - \eta ,
			\quad
			\eta \in\mathbb{ R} ^ { n _-}                                          \\
			\frac { \mathrm { d } \xi } { \mathrm { d } t } = \xi ,
			\quad
			\zeta \in \mathbb{ R } ^ { n_+ } .
		\end{array}
		\right.
		\label{eq:1.3.20}
	\end{equation}
\end{theorem}


这个定理的证明可参考\ref{sh}.
实际上,它的第一个方程就是\ref{equ3.8}.

\begin{corollary}
	为了研究\ref{equ3.19}的中心流形,我们把\(\mu\)也视作(空间变量),考虑
	\begin{equation}
		\frac { \mathrm { d } x } { \mathrm { d } t } = \operatorname { A } x + f ( x , \mu ) ,
		\quad
		\frac { \mathrm { d } \mu } { \mathrm { d } t } = 0.
		\label{eq:1.3.21}
	\end{equation}
	则在奇点
	$( \mathbf{x} ,\mathbf{ \mu} ) = (\mathbf{ 0},\mathbf{0} )$
	附近,稳定与不稳定子空间任为\(E^s\)与\(E^u\),而中心子空间为
	$E^{c} \times \mathbb{R}^{k}$。
	故\ref{equ3.21}的局部稳定流形与不稳定流形\(W^s\)与\(W^u\)的结构与$\mu \equiv 0$时类似。
	此时中心流形
	$W ^ { c } = \left\{ \left( x _ { c } , \mu \right)+ \varphi \left( x _ { c } , \mu \right) \left| \left( x _ { c } , \mu \right) \in E ^ { \mathrm { c } } \times \mathbb { R } ^ { k }\right.\right\}$,
	其中$\varphi \left( x _ { c } , \mu \right) \in C ^ { r } \left( E ^ { c } \times \mathbb{ R } ^ { k },E^h\right)$。
	由\ref{equ3.21}的第二个方程易知,
	$\{ ( x , \mu ) | \mu =常数\}$是\ref{equ3.21}的不变集,从而对固定的\(\mu\),\(W^c|_{\mu=常数}\)是\ref{equ3.21}的不变流形。
	注意$( x , \mu ) = ( 0,0 )$是\ref{equ3.21}的非双曲奇点。
	一般而言,对于不同的\(\mu\),\(W^c|_{\mu=常数}\)上的轨道结构可能不同,见下例。
\end{corollary}

\begin{example}
	设$x , y , \mu \in \mathbb { R }$,考虑光滑系统
	\begin{equation}
		\ label{eq:1.3.22}
		\left\{
		\begin{array}{l}
			{\dxdt= \mu x - x ^ { 3 } + f ( x , y , \mu ) }, \\
			{\dydt= - y + g ( x , y , \mu ) } ,              \\
			{ \frac{\mathrm{d}\mu}{\mathrm{d} t } = 0 },
		\end{array}
		\right.
	\end{equation}
	
	其中$f = O \left( | x , y | ^ { 4 } \right) , g = O \left( | x , y | ^ { 2 } \right)$
	且$g ( x , 0 , \mu ) = 0$。
	由附注\ref{fuzhu3.14}和例\ref{exam 2.6}可知,在$\left( x , y , { \mu } \right) = ( 0,0,0 )$的小领域内,中心流形\(W^c\)如图\ref{pic1-15}所示。
	显然,当$\mu \leqslant 0$或$\mu > 0$时,\(W^c|_\mu=常数\)有不同的结构。
	利用定理\ref{thm3.13}及简单计算可知,系统\ref{equ3.22}拓扑轨道等价于
	\[
		\frac { d \xi } { d t } = \mu \xi  - \xi ^ { 3 } ,
		\frac { d \eta } { d t } = - \eta ,
		\frac { d \mu } { d t } = 0
	\]
	
	它在$( \xi , \eta , \mu ) = ( 0,0,0 )$附近的中心流形如图\ref{tu1-167}所示,它不过是把图\ref{pic1-15}中的中心流形“摊平”在$( \xi , \mu )$空间而已。
\end{example}

在下文对局部分岔的讨论中,我们大都假定已经把问题化归到它的中心流形上,既对所讨论方程的线性部分矩阵\(\operatorname{A}\)而言,$\sigma ( \mathbf { A } )= \sigma _ { c }$(既$\sigma _ { u } , a _ { s } = \varnothing$)。
