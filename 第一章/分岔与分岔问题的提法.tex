\section{分岔与分岔问题的提法}

从上节定义\ref{def1.7}可知,若\(X\in \mathscr{X}^r(M)\)不是结构稳定的,则在X的任何一个\(C^1\)领域内,都可以找到与X轨道结构不同的向量场。
研究结构不稳定向量场在"扰动"下轨道结构的变化规律,是分岔理论的核心内容。

\subsection{分岔的概念}

\begin{defination}
设M是光滑流形,\(\Sigma^r(M)\)是\(\mathscr{X}^r(M)\)中结构稳定向量场的集合,则集合$\Lambda ^ { r } ( M ) = \mathscr { B } ^ { r } ( M ) \backslash \Sigma ^ { r } ( M )$称为 \textbf{分岔集}。
\end{defination}

\begin{defination}设
  $\epsilon \in \mathbf { R } ^ { k } , X ( \varepsilon ) \in \mathscr { E } \mathscr { X } ^ { r } ( M )$。
  若$X \left( \varepsilon _ { 0 } \right) \in \Lambda ^ { r } ( M )$,则称\(\varepsilon_0\)为族\(X(\varepsilon)\)的\textbf{分岔值}。
  当参数\(\varepsilon\)通过分岔值时,在相空间中向量场$X ( \varepsilon )$所发生的轨道拓扑分类的变化称为\textbf{分岔}。
\end{defination}
对于离散动力系统,也可以给出类似的定义。
由定理\ref{thm1.9}可知,如果向量场在奇点附近发生分岔现象,则该奇点必是非双曲的。


当M是紧致二维定向流形时,由定理\ref{thm1.13}可知,分岔集$\boldsymbol { \Lambda} ^r( M )$所包含的向量场具有如下结构:
它有非双曲的奇点或闭轨;
或者它的奇点和闭轨是双曲的,但它们的稳定流形和不稳定流形部横截相交(这时出现下面定义的同宿轨或异宿轨);
或者它的非游荡点集含有无限多个临界元。


由定义\ref{def1.7}易知,结构稳定集$\boldsymbol { \Sigma } ^ { r } ( M )$是$\mathscr{ X } ^ { r } ( M )$中的开集。
当M为二维紧致定向流形时,再由定理\ref{thm1.14}知,$\boldsymbol { \Sigma } ^ { r } ( M )$是开稠集,从而分岔集$\boldsymbol { \Lambda} ^r( M )$是这个开稠集的“边界”,但在高纬流形的情形,$\boldsymbol { \Lambda} ^r( M )$可以有更复杂的结构(见附注\ref{fuzhi1.15}).

\begin{defination}
    \label{def1.2.3}
  向量场的相轨线称为奇点(或闭轨)的\textbf{同宿(homoclinic)轨},如果这轨线不是奇点(或闭轨)本身,而且它的\(\alpha\)极限集与\(\omega\)极限集都与这奇点(或闭轨)一致。
  相轨线称为\textbf{异宿(heteroclinic)轨},如果它的\(\alpha\)极限集和\(\omega\)极限集是不同的奇点或者闭轨。
\end{defination}

\begin{defination}
  如果发生在奇点(或闭轨)的小领域内,并且与它的双曲性破坏相联系的分岔称为\textbf{局部分岔}.
  发生在有限个同宿轨或异宿轨的小领域内的分岔称为\textbf{半局部分岔}。
所有其余的分岔称为\textbf{全局分岔}.
\end{defination}
上面的定义2.1-2.4取自\ref{AAIS}.
我们将在后面看到,在研究局部分岔时,可能会伴随出现半局部分岔;
而研究半局部分岔时,也可能伴随出现全局分岔。
\begin{corollary}
  如上所述,分岔集$\Lambda^ { r } ( M ) = \mathscr { X} ^ { \prime } ( M ) \backslash \Sigma ^ { r } ( M )$是$\mathscr { X} ^ { r } ( M )$中的闭子集。
  如果限制在$\Lambda^ { r } ( M )$上,任可以考虑结构稳定集$\Sigma _ { 1 } ^r ( M )$及其余集$\Lambda _ { 1 } ^ { r } ( M ) = \Lambda^ { r } ( M )\backslash  \Sigma _ { 1 } ^ { r } ( M )$ .
  $ \Lambda _ { 1 } r ( M )$是$ \Lambda  r ( M )$中具有更多分岔现象的闭子集。
  我们可以继续对$ \Lambda _ { 1 } r ( M )$进行各种剖分,得到各种层次的分岔集。
\end{corollary}

\begin{example}
  考虑$\RR^{1}$上的向量场
  \begin{equation}
    \frac { \mathrm { d } x } { \mathrm { d } t } = \mu x - x ^ { 3 } = x \left( \mu - x ^ { 2 } \right).
    \label{equ2.1}
\end{equation}
显然,
$ x=0 $
总是方程 \ref{equ2.1}的一个奇点。
当$\mu < 0$时,它是\ref{equ2.1}的唯一奇点,并且是双曲的(定义见\ref{1.1}).
当\(\mu=0\)时,\(x=0\)任是\ref{equ2.1}的唯一奇点,但它是非双曲的。
当\(\mu>0\)时,除了\(x=0\)之外,方程\ref{equ2.1}还有两个奇点$x = - \sqrt { \mu } $和$ x = \sqrt { \mu }$,它的三个奇点都是双曲的。
图\ref{tu1-2}给出了\ref{equ2.1}的奇点分布对参数\(\mu\)的依赖关系。
由此可以看出,当参数\(\mu\)变化值通过\(\mu=0\)时,\ref{equ2.1}的奇点个数发生了突变,从而表示奇点个数的图形在\(\mu=0\)处发生了分岔。
习惯上把这种分岔现象称为\textbf{叉(pichfork)分岔}。
图\ref{ru1-3}给出$u < 0 , \mu= 0 , \mu > 0$三种情形下,\ref{equ2.1}的相图,由此可以看出,相对于不同的\(\mu\),\ref{equ2.1}的轨道拓扑结构所发生的变化。
显然,\(\mu=0\)是唯一的分岔值。
\end{example}

\begin{example}
  考虑$\mathbf { R } ^ { 1 }$上的向量场
  \begin{equation}
    \frac { \mathrm { d } x } { \mathrm { d } t } = \mu - x ^ { 2 },
    \label{equ2.2}
  \end{equation}
  与上例同样的分析可知,$\mu = 0$是\ref{equ2.2}的唯一分岔值:
  \(\mu<0\)时向量场\ref{equ2.2}无奇点;
  当\(\mu>0\)时,\ref{equ2.2}有两个双曲奇点;
  当\(\mu=0\)时\ref{equ2.2}有唯一奇点\(x=0\),它是非双曲的,称为\textbf{鞍结点(saddle-node)},并把这种分岔现象称为\textbf{鞍结点分岔}。
  与上例相似,可分别做出图\ref{tu1-4}与图\ref{tu1-5}.
  从下面的例中,我们可以对这一名称有更直观的理解。
\end{example}

\begin{example}
  二维的鞍结点分岔。考虑$\mathbf { R } ^ { 2 }$上的向量场
  \begin{ode}
        \dxdt = \mu - x ^ { 2 }  ,\\
        \dydt = - y .
  \end{ode}
  不难得知\(\mu=0\)是唯一的分岔值,并可做出轨道拓扑分类图(二维鞍结点分岔),见图\ref{1-6}.
\end{example}

\begin{example}
  考虑$\mathbb{ R } ^ { 2 }$上(0,0)点附近的单参数系统族
\begin{ode}
\label{equ2.4}
\dxdt = \mu x - y - x \left( x ^ { 2 } + y ^ { 2 } \right) , \\
\dydt = x + \mu y - y \left( x ^ { 2 } + y ^ { 2 } \right) ,
\end{ode}
它的线性部分矩阵以$\mu \pm i$为特征值。
在极坐标交换下,方程\ref{equ2.4}变形为
\begin{ode}
      \dxdt= r \left( \mu - r ^ { 2 } \right)  ,\\
     \dydt = 1 ,
\end{ode}
当$\mu \leqslant 0$时,原点是稳定的焦点(\(\mu=0\)时非双曲);
当$\mu > 0$时,原点为不稳定焦点,并有唯一闭轨$r = \sqrt{ \mu }$,它是稳定的\textbf{极限环}\(\Gamma_\mu\)见图
% \ref{tu1-7}.
?????????????????????????????????
这里,我们把平面上的孤立的闭轨称为极限环;
它附近的轨道以它为\(\omega\)极限集,因此称它为稳定的极限环。
注意,当$\mu \rightarrow 0$时,\(\Gamma_\mu\)趋于奇点$r = 0$。
显然,\(\mu=0\)是一个分岔值。
我们可以这样描述这个分岔现象:
当\(\mu\)的取值从小到大通过0时,奇点$( x , y ) = ( 0,0 )$改变其稳定性,并从此奇点“冒出”一个极限环。
这称为\textbf{Hopf分岔}。
\end{example}

\begin{example}[DL]
  首先考虑一个平面系统
\begin{ode}
\label{equ2.5}
\dxdt= y , \\
\dydt= - x \left( 1 - \frac { 3 } { 2 } x \right)  .
\end{ode}
  它有两个奇点:$A \left( \frac { 2 } { 3 } , 0 \right)$为鞍点,  $O ( 0,0 )$为中心。
  容易从\ref{equ2.5}消去t,得到首次积分
  \begin{equation}
    F ( x , y ) \stackrel { d } { \longrightarrow } y ^ { 2 } + x ^ { 2 } - x ^ { 2 } = C,
    \label{equ2.6}
  \end{equation}
  
  
  其中C为任意常数。
  为了下文中的方便,记
  \begin{equation}
    C = \frac { 4 } { 27 } - \mu,
    \label{equ2.7}
  \end{equation}
  则方程\ref{equ2.5}的轨道依\(\mu\)的不同选取,有如下分布:


  (1)当\(\mu=0\)时,由\ref{equ2.6}得到鞍点分界线$\Gamma _ { 0 } \cup H _ {  0  }$,其中\(\Gamma_0\)是鞍点A的同宿轨道,而\(H_0\)把相平面分成左右两部分(见图\ref{tu1-8}).


  (2)当$\mu < 0$时,由\ref{equ2.6}可确定一条(无界)轨道\(\gamma_\mu\),它位于\(\Gamma_0\)所围有界区域之外和\(H_0\)的左侧;当$\mu \rightarrow 0$,\(\gamma_\mu\)收缩到$\Gamma_ { 0 } \cup H _ { 0 }$。


(3)当$\mu > 0$时,由\ref{equ2.6}可确定两条轨道,其中一条为比轨道\(
  \Gamma_\mu \),它位于\(\Gamma_0\)所围有界区域的内部;
  另一条(无界)轨道\(H_\mu\)在\(H_0\)右侧。
  令$\mu \rightarrow 0 ^ { + }$时,$\Gamma _ { \mu } \rightarrow \Gamma _ { 0 } , H _ { \mu } \rightarrow H _ { 0 }$。

  令函数
  \[f ( x , y , \mu ) = F ( x , y ) - \left( \frac { 4 } { 27 } - \mu \right)\]
  其中的$F ( x , y )$由\ref{equ2.6}式定义。
  
  利用此函数构造向量场族$X _ { \mu }$如下
\begin{ode}
\dxdt= y , \\
\dydt= - x \left( 1 - \frac { 3 } { 2 } x \right) + f ( x , y , \mu ) y .
\label{equ2.8}
\end{ode}
  容易算出
  \[
  \left. \frac { \mathrm { d } F } { \mathrm { d } t } \right| _ { ( 2,8 ) }
    = \left. \left( \frac { \partial F } { \partial x } \frac { d x } { d t } + \frac { \partial F } { \partial y } \frac { d y } { d t } \right) \right| _ { ( 2.8 ) }
    = 2 f ( x , y , \mu ) y ^ { 2 },
    \]
  由此可知,当$| \mu | \ll 1$时,系统\ref{equ2.8}的轨道分布如图\ref{tu1-9}所示。(所用的论据,类似于Lyapunov函数判断奇点的稳定性),并且易知,当$0 < \mu \ll 1$时,\ref{equ2.8}的唯一闭轨就是\ref{equ2.5}的闭轨\(\Gamma_\mu\),当$\mu \rightarrow 0 ^ { + }$时,它趋于\(X_0\)的同宿轨\(\Gamma_0\).
\end{example}
在例\ref{exam2.6}-例\ref{2.9}中,分岔现象都是由于奇点的非双曲性而发生的,属于局部分岔。
本例则不同,分岔现象是由同宿轨(对应于\(\mu=0\))在扰动下(\(\mu \neq 0\))破裂而发生的,称为\textbf{同宿分岔},它是一种版局部分岔。


\begin{example}
  考虑映射
  $\boldsymbol { F } : \mathbf { R } ^ { 1 } \rightarrow \mathbf { R } ^ { 1 }$,
  它的线性部分以-1为特征根。
  由下文中\ref{1.4}例\ref{exam4.15}可知,无妨设
  
  \begin{equation}
    F ( x ) = - x + a x ^ { 3 } + O \left( | x | ^ { 5 } \right),
    \label{equ2.9} 
  \end{equation}
  
    其中\(\alpha \neq 0\).
    在\(x=0\)附近的一个领域内,映射\(F\)以\(x=0\)为唯一不动点。
    注意,由隐函数定理可知,\ref{equ2.9}的任一扰动在\(x=0\)附近任有唯一的不动点,所以我们不妨取它的扰动系统保持\(x=0\)为不动点,且具有下面的形式
    
    \[
      F _ { \mu } ( x ) = - ( 1 + \mu ) x + a _ { 2 } ( \mu ) x ^ { 2 } + a _ { 3 } ( \mu ) x ^ { 3 } + O \left( | x | ^ { 4 } \right),
    \]
    其中$\mu \in \mathbf { R }^1$,光滑函数$a _ { 2 } ( \mu ) , a _ { 3 } ( \mu )$满足$a _ { 2 } ( 0 ) = 0 , a _ { 3 } ( 0 ) = a$。考虑\(F_\mu\)的两次叠代映射,得到
    \[F _ { \mu } ^ { 3 } ( x ) = ( 1 + \mu ) ^ { 2 } x + O \left( \mu ^ { 2 } \right) x ^ { 2 } - ( 2 + O ( \mu ) ) a x ^ { 3 } + O \left( | x | ^ { 4 } \right),\]
    从而$F _ { \mu } ^ { 2 } ( x ) - x$可表示为
    
    \[
    x \left[ \mu ( 2 + \mu ) + O \left( \mu ^ { 2 } \right) x - ( 2 + O ( \mu ) ) a x ^ { 3 } + O \left( | x | ^ { 3 } \right) \right].
  \]
  
    因此,利用\(\alpha \neq 0\)及下文中的Malgrange定理(定理\ref{thm2.12})可知,当$\mu a > 0$且$| \mu | \ll 1$时,$F _ { \mu } ^ { 2 }$除了有不动点\(x=0\)(它是\(F_\mu\)的不动点)之外,又有两个新的不动点,它们是\(F_\mu\)的2周期点(图\ref{tu1-10(a)}相应于\(a>0\)的情形)。
    如果把上面的映射\(F_\mu\)看成一向量场的流的\textbf{Poincare映射},则当\(\mu\)的值从负到正的瞬间(设\(a>0\)),原有的稳定闭轨变为不稳定的闭轨\(\gamma_1\),而在它的领域内又产生了一个稳定的几乎两倍与原周期的闭轨\(\gamma_2\),这种分岔现象称为\textbf{倍周期分岔},它发生在一个\(Mobius\)带上(见图\ref{tu1-10(b)}),\(\gamma_1\)位于这带的轴线,而\(\gamma_2\)是这带的边界。
  \end{example}
  
  
 \subsection{分岔问题的提法}
  从实际中产生的分岔问题,常常是带参数的向量场族。
  例如
  
  \begin{equation}
    \frac { \mathrm { d } x } { \mathrm { d } t } = f ( x , \mu )
    \label{equ2.10} 
  \end{equation}
  
  其中$f \in C ^ { r } \left( \mathbf { R } ^ { n } \times \mathbf { R } ^ { k } , \mathbf { R } ^ { n } \right)$。
  当\(\mu=0\)时,相应系统
  
  \begin{equation}
    \frac { d x } { d t } = f ( x , 0 ) = f _ { 0 } ( x )
    \label{equ2.11}
  \end{equation}
  
  是结构不稳定的。
  当$| \mu | \ll 1$时,常把\ref{equ2.10}称为\ref{equ2.11}的一个  \textbf{\(C^r\)开折(unfolding)}.我们要研究的是:


  \textbf{问题A}  能否找到\(\mu=0\)在$\mathbf { R } ^ { \boldsymbol { k } }$中的一个领域V,使得当\(\mu\)在V中变动时,弄清系统\ref{equ2.10}的轨道结构如何变化?


  在较简单的情况下还可以考虑:能否把V分成若干子集,它们对应\ref{equ2.10}的轨道拓扑结构的不同等价类?
  在例\ref{exam2.6}中,\(\mu=0\)把参数空间$\mu \in \mathbf { R } ^ { 1 }$分成两部分,其中\(\mu <0\)和\(\mu > 0\)相应于系统\(equ2.1\)两种不同的且分别结构稳定的轨道拓扑类型。


  进一步的问题是:

  \textbf{问题B}  能否找到\ref{equ2.11}的开折,它“包含”了\ref{equ2.11}的任一开折所能出现的轨道结构?

  
  \textbf{问题C}  能否找到\ref{equ2.11}的开折,它满足问题B的要求,并且含有“最少”的参数?


  对上述问题的一些名词的确切含义进行澄清,将会引导到“普适开折”和“分岔的余维”这样一些深刻的概念,我们将在\ref{1.5}中介绍,本节先对这些概念给出直观的描述。
  这里需要指出的是,由于向量场分岔集合可以是任意复杂结构的,一般而言,对上述问题额回答是非常苦难的。
  在多数情形下,只能对问题A作出问题的回答。
  但当系统\ref{2.11}的相空间维数较低且它的"退化程度"不高时,目前对问题B和C已有一些完整的结果。


  下面以\(\mathbf{R}^1\)上的系统
  
  \begin{equation}
\frac { d x } { d t } = - x ^ { 3 }
\end{equation}

  为例,回答上面的问题。
  例\ref{exam2.6}中系统\ref{equ2.1}是\ref{equ2.12}的一个开折。
  我们要证明,\ref{2.1}不是满足问题B要求的那种开折。
  现在考虑\ref{equ2.12}的任意一个\(C^\infty\)开折
  
  \begin{equation}
    \frac { \mathbf { d } x } { \mathrm { d } t } = f ( x , \mu ) , \quad f ( x , 0 ) = - x ^ { 3 }
    \label{equ2.13}
  \end{equation}
  
  里这$f ( x , \mu ) $在$( 0,0 ) \in \mathbf { R } \times \mathbf { R } ^ { m }$的一个小领域中定义,m是一个足够大的正整数。
  在进一步讨论之前,我们需要下面的结果,它可以看成是隐函数定理的推广。

  \begin{theorem}[Malgrange定理]
    设$U \subset \mathbf { R } \times \mathbf { R } ^ { n }$
    是包含原点的开集,
    $f \in C ^ { \infty } ( U , R )$
    并且满足
    $f ( t , 0 ) = t ^ { \mathbf { h } } \mathbf { g } ( t )$,
    其中
    $k \in \mathbf { Z } ^ { + } , \boldsymbol { g }$在$t = 0$
    附近是光滑函数,且
    $g ( 0 ) \neq 0$。
    则存在$\mathbf { R } \times \mathbf { R } ^ { n }$中\((0,0)\)附近的领域\(V\subset U\)和光滑函数 \(q(t,x)\),以及\(\mathbf{R}^n\)中O点附近某领域中的光滑函数$a _ { 0 } ( x ) , \cdots , a _ { k - 1 } ( x )$,满足
    $q ( 0,0 ) \neq 0 , a_{ 0 } ( 0 ) = \dots a_{ k - 1 } ( 0 ) = 0,$以及
    
    \begin{equation}
f ( t , x ) = q ( t , x ) \left[ t ^ { k } + \sum _ { i = 0 } ^ { k-1 } \alpha _ { i } ( x ) t ^ { i } \right] , \quad \forall ( t , x ) \in V.
\end{equation}

\end{theorem}

    证明可见[CH,pp43-45].定理中的\(C^\infty\)光滑条件后来减弱到有限光滑性。
    由上面的定理可知,对于开折\ref{equ2.13}存在\(\mathbf{R}^{m+1}\)中零点的领域V,以及光滑函数$q ( x , \mu ) , a ( \mu ) , b ( \mu )$和$c ( \mu )$,满足
    
    \begin{equation}
      q ( 0,0 ) \neq 0 , a ( 0 ) - b ( 0 ) = c ( 0 ) = 0
      \label{equ2.13}
    \end{equation}
    
    以及
    
    \[
      f ( x , \mu ) = q ( x , \mu ) \left[ a ( \mu ) + b ( \mu ) x + c ( \mu ) x ^ { 3 } - x ^ { 3 } \right]
    \]
    
    因此,当$( x , \mu )$在$\mathbf { R } \times \mathbf { R } ^ { m }$中\((0,0)\)点附近的一个小领域中取值时,系统\ref{equ2.13}与系统
    
    \begin{equation}
      \frac { \mathrm { d } x } { \mathrm { d } t } = a ( \mu ) + b ( \mu ) x + c ( \mu ) x ^ { 2 } - x ^ { 3 }
      \label{equ2.15}
    \end{equation}
    
    有相同的轨道结构。
    注意上式右端可改写为$\lambda _ { 1 } ( \mu ) +\lambda _ {2 } ( \mu ) \left( x - \frac { c ( \mu ) } { 3 } \right) - \left( x - \frac { c ( \mu ) } { 3 } \right) ^ { 3 }$,
    其中$\lambda _ { 1 } ( \mu ) = a ( \mu ) + \frac { b ( \mu ) c ( \mu ) } { 3 }+ \frac { 2 ( c ( \mu ) ) ^ { 3 } } { 27 } , \lambda _ { 2 } ( \mu ) = b ( \mu ) + \frac { ( c ( \mu ) ) ^ { 2 } } { 3 }$。
    令$y = x - \frac { c ( \mu ) } { 3 }$,然后把y改写成x,则系统\ref{equ2.15}变为
    
    \begin{equation}
      \frac { \mathrm { d } x } { \mathrm { d } t } = \lambda _ { 1 } ( \mu ) + \lambda _ { 2} ( \mu ) x - x ^ { 3 }
      \label{equ2.16}
    \end{equation}
    
    显然,\ref{equ2.16}所能出现的轨道拓扑类型不超出系统
    
    \begin{equation}
      \frac { d x } { d t } = \lambda _ { 1 } + \lambda _ { 2 } x - x ^ { 3 },
      \label{equ2.17}
    \end{equation}
    
    所能出现的轨道拓扑类型,
    其中$\lambda _ { 1 } , \lambda _ { \mathrm { 2 } } \in \mathbf { R } ^ { 1 }$
    是独立的参数。
    至此我们证明了\ref{equ2.12}的任一开折(可以含有任意多个参数)所能出现的轨道拓扑结构,都含于双参数开折\ref{equ2.17}的轨道拓扑结构类型中,既\ref{equ2.17}就是问题B中所要求的开折。
    这时我们称\ref{equ2.17}是\ref{equ2.12}的一个
    \textbf{普适开折}。
    对\ref{equ2.17}进行定性分析不难得知,在\(\lambda=(\lambda_1,\lambda_2)\)平面上使对应的系统\ref{equ2.17}成为结构不稳定的点的集合为原点\(O()\)和由
    
    \[
\left\{ \lambda | \lambda _ { 1 } = \pm 2 \left( \frac { \lambda _ { 2 } } { 3 } \right) ^ { \frac { 3 } { 2 } } , \quad \lambda _ { 2 } > 0\right \}
    \]

    给出的两条\textbf{分岔曲线}$\Gamma _ { + } , \Gamma _ { - }$,它们在O点相切,并形成尖点,见\textbf{分岔图}\ref{tu1-11},相应于不同的\(\lambda\),开折\ref{equ2.17}的5种拓扑轨道分类中有2种是结构稳定的(分别相应于
    \(\Gamma_+\cup \Gamma_-\)把平面分成两个区域),有三种是结构不稳定的。显然,例\ref{equ2.6}中开折\ref{equ2.1}所能出现的轨道拓扑类型都含于其中。
    图\ref{tu1-12}给出了方程\ref{equ2.17}的奇点个数对\(\lambda\)的依赖关系,图中的“奇点曲面”方程为
    $\lambda _ { 1 } + \lambda _ { 2 } x - x ^ { 3 } = 0$;
    沿x轴方向穿越此曲面的次数给出系统的奇点个数。
    曲面折叠部分的“边缘”向\((\lambda_1,\lambda_2)\)平面投影,就得到图\ref{tu1-11}中分岔曲线\(\Gamma_+\)和\(\Gamma_-\).


    最后我们来证明\ref{equ2.12}的任一单参数开折都不具有\ref{equ2.17}的如上性质,从而\ref{equ2.17}还满足问题C的要求,此时称由\ref{equ2.12}引出的分岔为\textbf{余维2}的。
    事实上,由于\ref{equ2.12}的任一开折都可等份地转化成\ref{equ2.15}的形式,所以我们只需在三维空间\((a,b,c)\)的原点附近的某领域U中考虑即可。
    $\forall \mu$,当$| \mu | \ll 1$时,向量场\ref{equ2.15}对应U中一点。
    设$\Delta$为U的一个子集,其中的点对应的向量场与\ref{equ2.12}有“相同的奇异性”,既
    
    \[
      \Delta = \left\{
        ( a , b , c ) \left| \exists x _ { 0 }\right.,使a + b x + c x ^ { 3 } - x ^ { 2 } = - \left( x - x _ { 0 } \right) ^ { 3 }
      \right\}.
    \]
    
    故\(\Delta\)中的点\((a,b,c)\)满足
    
    \[
a =x _ { 0 } ^ { 3 } , b = - 3 x _ { 0 } ^ { 2 } , c = 3 x _ { 0 }.
\]

由此不难得知,\(\Delta\)是U中的一条曲线(余维2子流形)。
所以在U中至少二维曲面(相应于\ref{equ2.12}的二参数开折)才能与\(\Delta\)横截相交,见图\ref{pic1-13},而\ref{equ2.12}的单参数开折所对应的U中的曲线虽然也可以在O点与\(\Delta\)相交,
但在任意小的扰动下,它都可以与\(\Delta\)分离,换句话说,至少在二参数的开折中,像\ref{equ2.12}这样的分岔现象才是"不可去"的。


注意到\((a,b)\)坐标平面在原点与\(\Delta\)横截,所以\ref{equ2.17}同时具有问题B、C所要求的性质。


\begin{corollary}
  利用同样的推理不难证明:
  当\(a\neq 0\)时,$\mathbf { R } ^ { 1 }$上的系统
  $\frac { d x } { d t} = a x ^ { k + 1 } + O \left( | x | ^ { k + 2 } \right)$
  是余维k的,它的一个普适开折可取为
  $\frac { d x } { d t } = \mu _ { 1 } + \mu _ { 2 } x + \dots + \mu _ { k } x ^ { k - 1 } + a x ^ { k + 1 }$。
  特别地,例\ref{equ2.7}中的系统\ref{equ2.2}是$\frac { \mathbf { d } _ { x } } { d t } = - x ^ { 2 }$的一个普适开折,但$\frac { d x } { d t } = \mu x - x ^ { z }$则不是。
  我们将在\ref{1.5}中把这里的讨论精确化、一般化。
\end{corollary}
