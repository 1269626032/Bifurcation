\section{正规形}
正规形(normal form)理论的基本思想,是在奇点(或不动点)附近经过光滑变换把向量场(或微分同胚)化成(在一定意义下)尽可能简单的形式,以便于研究。
这是源于Poincare时代的一个课题。
由于近年来分岔理论的发展,正规形的应用更加广泛,因而重新引起人们对它的重视,并得出若干计算正规形的新方法。
众所周知,经非退化线性变换$x = \boldsymbol { T } y$,线性微分方程
\[
\frac { \mathrm { d } x } { \mathrm { d } t } = \boldsymbol { A } x
\]
变换为
\[
  \frac { \mathrm { d } y } { \mathrm { d } t } =
  \left( \operatorname{T} ^ { - 1 } \operatorname{A}  \operatorname{T}  \right) y
\]
这里$x , y \in \mathbb { R } ^ { n }$,$\mathbf{A}$和$\mathbf{T}$为$n \times n$矩阵,且
$\operatorname { det } x \neq 0$。
因此,在讨论线性系统的轨道结构时,我们无妨假设\(\mathbf{A}\)
为Jordan标准形,除了Jordan块的排列顺序外,它是唯一确定的。
接下去的问题自然是:对非线性部分是否可以做类似的简化?
在一定意义下,答案是肯定的,但一般不在有唯一性。
无论如何,这种简化对进一步研究分岔问题是很有用的。
\subsection{常微分方程在奇点附近的正规形}
考虑以\(x=0\) 为奇点的\(C^r\) 微分方程$( r \geqslant 3 )$,它在\(x=0\) 附近可表示为
\begin{equation}
\frac { d \boldsymbol{x} } { d t } = \mathbf{A} \boldsymbol{x} + f ( \boldsymbol{x}) + \dots + f ^ { r - 1 } ( \boldsymbol{x} ) + O \left( | \boldsymbol{x}| ^ { r } \right),
\label{eq:1.4.1}
\end{equation}
其中$\boldsymbol{x} \in \mathbb { R } ^ { n }$(或 \(\mathbb{C}^n\)),
\(\mathbf{A}\)是线性部分矩阵,$\boldsymbol{f}^ { k } ( \boldsymbol{x} ) \in H _ { m } ^ { k } $,$ H _ { n } ^ { k }$为n元n维k次齐次向量多项式所成的空间,$k = 2 , \cdots  , r-1$.
先进行变换
\begin{equation}
x = y + h ^ { 2 } ( y ),
  \label{eq:1.4.2}
\end{equation}
其中$h ^ { 2 } ( y ) \in H _ { n } ^ { 2 }$待定,以使变换后的方程具有简单的形式,把变换\ref{equ4.2}代入\ref{equ4.1},并注意
\[
\left[\boldsymbol  I + \operatorname D h ^ { 2 } ( y ) \right] ^ { - 1 } =  \boldsymbol I -\operatorname D h ^ { 2 } ( y ) +\boldsymbol O\left( | y | ^ { 2 } \right),
\]
其中\(\boldsymbol I\) 是 \(n \times n\) 单位矩阵,Jacobi矩阵$\operatorname D h ^ { 2 } ( y )$的每一元素都是$O ( | y | )$,
而$\mathscr O \left( | y | ^ { 2 } \right)$表示\(n \times n\) 矩阵,它的每一个元素都是$O \left( | y | ^ { 2 } \right)$,由此把方程\ref{equ4.1}化为
\begin{equation}
  \frac { d y } { d t } =
  A y + \left\{ f ^ { 2 } ( y ) -
    \left[ D h ^ { 2 } ( y ) A _ { y } - A h ^ { 2 } ( y ) \right] \right\}
  + \overline { f } ^ { 3 } ( y ) +\dots + \overline { f } ^ { r - 1 } ( y ) + O \left( | y | ^ { x } \right),
  \label{eq:1.4.3}
\end{equation}
其中\(f^2\) 与\ref{equ4.1}中的相同,而\(\overline{f}^k\) 是经过运算得到的新的k次齐次多项式。

引入算子$a d _ { A } ^ { 2 } : H _ { n } ^ { 2 } \rightarrow H _ { n } ^ { 2 }$,

\begin{equation}
\operatorname { ad } _ { \operatorname A } ^ { 2 } \left[ h ^ { 2 } ( y ) \right] = \operatorname { D } h ^ { 2 } ( y ) \operatorname A { y } - \operatorname A h ^ { 2 } ( y ),
\label{eq:1.4.4}
\end{equation}

则\ref{equ4.3}变成

\begin{equation}
\frac { d y } { d t } = A y + \left[ f ^ { 2 } ( y ) - \operatorname { ad } _{\operatorname A}^2 \left( h ^ { 2 } ( y ) \right) \right] + \overline { f } ^ { 3 } ( y ) +\dots + f ^ { r - 1 } ( y ) + O \left( | y | ^ { r } \right),
  \label{eq:1.4.5}
\end{equation}

记$\mathscr { R } ^ { 2 }$为算子$\operatorname { ad } _ { \operatorname { A } } ^ { 2 }$在$H _ { n } ^ { 2 }$中的值域,
而$\mathscr{S}^2$是\(\mathscr{R}^2\) 在\(H_n^2\) 中的一个补空间,既
\[
H _ { n } ^ { 2 } = \mathscr { R } ^ { 2 } \oplus \mathscr { S }^2.
\]

如果$f ^ { 2 } ( y ) \in \mathscr { R } ^ { 2 }$,
则存在$h ^ { 2 } ( y ) \in H _ { n } ^ { 2 }$,
使$f ^ { 2 } ( y ) = \operatorname { ad } _ { \operatorname A } ^ { 2 } \left( h ^ { 2 } ( y ) \right)$,
既\ref{equ4.5}中的二次项可以消去;
否则,只能找到
$h^{2}(y) \in H_{n}^{2}$,
使
\begin{equation}
\label{eq:1.4.6}
f^{2}(y) - \operatorname { ad } _ { A } ^ { 3 } \left( h ^ { 2 } ( y ) \right) \in \mathscr{S}^2.
\end{equation}
这样,我们把\ref{equ4.1}化为
\begin{equation}
  \frac { d y } { d t } = A y + g ^ { 2 } ( y ) + \overline { f } ^ { 2 } ( y ) + \cdots + \overline { f } ^ { r - 1 } ( y ) + O \left( | y | ^ { r } \right),
  \label{eq:1.4.7}
\end{equation}
其中\(g^2(y) \in \mathscr{S}\).
其次,再考虑变换,
\begin{equation}
y = x + h ^ { 3 } ( x ),
\label{eq:1.4.8}
\end{equation}
其中$\mathbf { h } ^ { 3 } ( \mathbf { z } ) \in \boldsymbol { H } _ { \boldsymbol { n } } ^ { \mathbf { 3 } }$.
重复上面的推理,并注意这个变换不影响线性项与二次项,容易得知\ref{equ4.7}变为
\begin{equation}
  \frac { d z } { d t } = A z + g ^ { 2 } ( z ) +
  \left[ \overline { f } ^ { 2 } ( z ) - a d _ { A } ^ { 3 } h ^ { 3 } ( z ) \right] + \tilde { f } ^ { 4 } ( x ) +
  \cdots + \tilde { f } ^ { y - 1 } ( z ) + O ( | z |^ r ),
  \label{eq:1.4.9}
\end{equation}
其中\(g^2,\overline{f}^2\)与\ref{equ4.7}中的相同,而$\widetilde { f } ^ { 4 } , \cdots , \tilde { f } ^ { r - 1 }$是经计算得到的,而且
\[
\operatorname { ad } _ { A } ^ { 3 }:H _ { n } ^ { 3 } \rightarrow H _ { n } ^ { 3 },h ^ { 3 } ( z ) \mapsto - D h ^ { 3 } ( z ) A z - A h ^ { 3 } ( z ).
\]
记$\mathscr { R } ^ { 3 } = \mathscr { R } \left( \mathrm { ad } _ { A } ^ { 3 } \right) , \mathscr { S } ^ { 3 }$,\(\mathscr{S}^3\)为\(\mathscr{R}^3\)在\(H_n^3\) 中的补空间,既
\[
  H _ { n } ^ { 3 } = \mathscr { R } ^ { 3 } \oplus \mathscr S ^ { 3 }
\]

则当$\overline { f} ^ { 2 } ( z )  \in \mathscr { R } ^ { 3 }$时,存在
\(h^3(z)\),使得经过变换\ref{equ4.8}可消去\ref{equ4.7}中的三次项;
否则,只能找到$h ^ { 3 } ( z ) \in H _ { n } ^ { 3 }$,使\ref{equ4.7}变为
\[
\frac { d z } { d t } = A z + g ^ { 2 } ( z ) + g ^ { 3 } ( z ) + \tilde { f } ^ { 4 } ( z ) +\cdots + \tilde { f } ^ { r - 1 } ( z ) + O \left( | z | ^ { r } \right),
\]

其中$g ^ { h } ( z ) \in \mathscr { S } ^ { k } , k = 2,3$。由此递推下去,就可得到

\begin{theorem}
  设$X \in \mathscr { X } ^ { r } \left( \mathbf { R } ^ { n } \right)$
  (或$\mathscr { X } ^ { r } \left( \mathbf { C } ^ { n } \right) $) ,$ X ( 0 ) = 0 , \mathbf { D } X ( 0 )=A$,并且X有表达式\ref{4.1},则在原点附近的领域内存在一系列变换
  \begin{equation}
x = y + h ^ { k } ( y ) , \quad k = 2 , \dots , r - 1,
    \label{eq:1.4.10}
  \end{equation}
  其中,$h ^ { k } ( y ) \in H _ { n } ^ { k }$,经过这一系列变换
  (每次变换后把y换回x),可把\ref{equ4.1}变为如下形式
  \begin{equation}
\frac { d x } { d t } = A x + g ^ { 2 } ( x ) + \cdots + g ^ { r - 1 } ( x ) + O \left( | x | ^ { r } \right)
    \label{eq:1.4.11}
  \end{equation}
  其中,$g ^ { k } ( x ) \in \mathscr { S } ^ { k } , \mathscr { S }$是
  $\mathscr { R } ^ { k  }: = \mathscr { R } \left( \operatorname { ad } _ { A } ^ { k } \right)$在\(H_n^k\) 中的补空间,
  算子$\mathrm { ad } _ { \boldsymbol { A } } ^ { k }$由下式定义
  \begin{equation}
        \label{eq:1.4.12}
    a d _ { A } ^ { k }:H _ { \vec { n } } ^ { k } \rightarrow H _ { \mathbf { n } } ^ { k },
    \operatorname { ad } _ { A } ^ { k } \left( h ^ { k } ( x ) \right) = \operatorname { Dh } ^ { k } ( x ) A x - A h ^ { k } ( x )
  \end{equation}
  $k = 2 , \dots , r - 1$
\end{theorem}

\begin{defination}
  微分方程\ref{equ4.11}的j次截取式$( 2 \leqslant j \leqslant r - 1 )$
  \begin{equation}
\frac { d x } { d t } = A x + g ^ { 2 } ( x ) + \cdots + g ^ { j } ( x )
    \label{eq:1.4.13}
  \end{equation}
  其中,$g ^ { i } ( x ) \in \mathscr{S} ^ { i } , i = 2 , \cdots , j$,称为方程\ref{equ4.1}的\textbf{j次正规形}.
\end{defination}

\begin{corollary}
    \label{corollary:1.4.3}
  上面进行的只是有限步运算.
  当\ref{equ4.1}右端可以展开成收敛的幂级数时,这种步骤原则上可以无限地进行下去,问题在于变化本身和所得到的(形式)正规形是否收敛。
  在一定条件下是肯定的,这是Poincaré-Dulac定理,见\ref{al},或\ref{clw}.
\end{corollary}
\subsection{共振与非共振}
在定理\ref{equ4.1}中,我们希望进一步确定,\ref{equ4.11}式中哪些
\(g^k(x)=0\),也就是说,\ref{4.1}中哪些\(f^k(x)\)属于\(\mathscr{R}^k\)?
为此,我们假定A已化为它的Jordan标准形,并引入共振的概念。

\begin{defination}
  特征值$\lambda = \left( \lambda _ { 1 } , \cdots , \lambda _ { n } \right)$称为\textbf{共振}的,如果存在自然数$s ( 1 \leqslant s \leqslant n )$和整数组$m = \left( m _ { 1 } , \cdots , m _ { n } \right)$,其中$m _ { i } \geqslant 0$并且$| m |$$\frac { d } { i = 1 } ^ { n } m _ { i } \geq 2$,使得
  \begin{equation}
\lambda_{ s } = ( m , \lambda ) \stackrel { d } { \longrightarrow } \sum _ { i = 1 } ^ { n } m _ { i } \lambda _ { i }
    \label{eq:1.4.14}
  \end{equation}
  正数 \(|m|\)称为\textbf{共振的阶}。
  \label{def1.4.11}
\end{defination}
例如$\lambda _ { 1 } = 2 \lambda _ { 2 }$是2阶共振的;
$2 \lambda _ { 1 } = 3 \lambda _ { 1 }$不是共振的;
$\lambda _ { 1 } + \lambda _ { 2 } =0$是3阶共振的,因为它可以改写为
$\lambda _ { 1 } = 2 \lambda _ { 1 } + \lambda _ { 2 }$。
考察\ref{4.11}中哪些\(g^k(x)\) 不出现,就是要考察\textbf{同伦方程}
\begin{equation}
  a d _ { A } ^ { k } \left( h ^ { k } ( x ) \right)
  \equiv
  \mathrm { D } h ^ { k } ( x ) \boldsymbol { A } \boldsymbol { x } - \boldsymbol { A } h ^ { k } ( x ) = f ^ { k } ( x )
  \label{eq:1.4.15}
\end{equation}
对给定的$f ^ { k } ( x ) \in H _ { n } ^ { k }$,是否有解$h ^ { k } ( x ) \in H _ { \mathrm { n } } ^ { k }$。
设A是对角矩阵,特征值\(\lambda_i\) 互不相同,
\(e_i\) 是A相应于\(\lambda_i\) 的特征向量,
则$\left( e _ { 1 } , \cdots , e _ { n } \right)$构成一组基。
设$\left( x _ { 1 } , \cdots , x _ { n } \right)$是相对于
$\left( e _ { 1 } , \cdots , e _ { n } \right)$的坐标,则
\begin{equation}
  x ^ { m } \Longrightarrow x _ { 1 } ^ { m _ { 1 } \ldots x _ { n } ^ { m } } , \quad | m | = k
\end{equation}

就是\(H_n^k\) 中元素某一分量中的最简形式。

令$h ^ { k } ( x ) = x ^ { m } \varepsilon _ { s } , | m | = k$,
则$\operatorname { Dh } ^ { k } ( x ) A x$中只有第s个分量非零,
并且他等于
\[
  \frac { \partial x ^ { m } } { \partial x } A x
  =\left( m _ { 1 } x _ { 1 } ^ { m - 1 } \dots x _ { n } ^ { m_n } , \dots , m _ { n } x _ { 1 } ^ { m_1 } \dots x _ { n } ^ { m_n - 1 } \right)
  \left(
    \begin{array} { c c c }
      { \lambda _ { 1 } } & { } & { } \\
      { } & { \ddots } & { } \\
      { } & { } & { \lambda _ { n } }
    \end{array}
  \right)
  \left(
    \begin{array} { l }
      { x _ { 1 } } \\
      { \vdots } \\
      { x _ { n } }
    \end{array}
  \right)
  =( m , \lambda ) x ^ { m }
\]

另一方面,由于$\boldsymbol { e } _ { \mathbf { s } }$是A相应于\(\lambda_s\)的特征向量,因此
\[
A x ^ { m } e _ { s } = \lambda _ { s } x ^ { m } e _ { s }.
\]

把上面的结果带入\ref{equ4.15}的左端,得到
\begin{equation}
a d _ { A } ^ { k } x ^ { m } e _ { s } = \left[ ( m , \lambda ) - \lambda _ { s } \right] x ^ { m } e _ { s },
  \label{eq:1.4.17}
\end{equation}

这说明\(ad_A^k\) 也是对角的,并且它的所有特征值具有
$\left[ ( m , \lambda ) - \lambda _ { s } \right]$的形式。
由此可知,当A的特征值非共振时,\(ad_A^k\) 的所有特征值均非零,
故算子\(ad_A^k\) 可逆,同伦方程\ref{equ4.15}可解。

当A有重特征值时,A的Jordan标准形是上三角矩阵。
此时\(ad_A^k\) 也有相应的Jordan块,
并且\(ad_A^k\) 的特征值任具有$\left[ ( m , \lambda ) - \lambda _ { s } \right]$的形式。

\begin{defination}
  向量值多项式$x ^ { m } e _ { s }$称为\textbf{共振多项式},如果
  \[
\lambda _ { s } = ( m , \lambda ) , \quad | m | \geqslant 2,
\]
其中\(\lambda,e_s,m\),和\(|m|\) 的意义同前。
  \label{def:1.4.5}
\end{defination}

利用上面的讨论,我们可以得到下面

\begin{theorem}
  设A是上三角的Jordan标准形,则可适当选取变换\ref{equ4.10},
  使\ref{equ4.11}右端的诸\(g^i(x)\) 仅由共振多项式组成。

\end{theorem}

\subsection{正规形的计算}

\begin{example}
  求$\frac { d x } { d t } = A x + f ^ { 2 } ( x ) + \dots$的二次正规形,其中\(x\in \mathbb{R}^2\),
  $A = \left\{
    \begin{array} { l l }
      { 0 } & { 1 } \\
      { 0 } & { 0 }
    \end{array}
  \right)$
\end{example}

\begin{maxima}[]
diff(sin(x^2),x)
\end{maxima}

\begin{solve}
  先在\(H_2^2\)中取一组基
  \[
    \varepsilon _ { 1 } = \left(
      \begin{array} { l }
        { 0 } \\
        { x _ { 1 } ^ { 2 } }
      \end{array}
    \right) ,
    \varepsilon _ { 2 } = \left(
      \begin{array} { c }
        { 0 } \\
        { x _ { 1 } x _ { 2 } }
      \end{array}
    \right) ,
    \varepsilon _ { 3 } = \left(
      \begin{array} { l }
        { 0 } \\
        { x _ { 3 } ^ { 2 } }
      \end{array}
    \right)
  \]
  \[
    \varepsilon _ { 4 } = \left(
      \begin{array} { l }
        { x _ { 1 } ^ { 2 } } \\
        { 0 }
      \end{array}
    \right) ,
    \varepsilon _ { \mathrm { s } } = \left(
      \begin{array} { c }
        { \boldsymbol { x } _ { 1 } x _ { 2 } } \\
        { 0 }
      \end{array}
    \right) ,
    \varepsilon _ { 6 } = \left(
      \begin{array} { l }
        { x _ { 2 } ^ { 3 } } \\
        { 0 }
      \end{array}
    \right)
  \]
  设\(ad_A^2\) 在这组基下的矩阵为\(L_A^2\),既
  \[
    \operatorname { ad } _ { A } ^ { 2 }
    \left( \varepsilon _ { 1 } , \cdots , \varepsilon _ { 6 } \right)
=   \left( \varepsilon _ { 1 } , \cdots , \varepsilon _ { 6 } \right)
L_A^2,
\]
容易算出
\begin{equation}
  \boldsymbol { L } _ { \hat { 3 } } ^ { 2 } = \left(
    \begin{array} { r r r r r r }
      { 0 } & { 0 } & { 0 } & { 0 } & { 0 } & { 0 } \\
      { 2 } & { 0 } & { 0 } & { 0 } & { 0 } & { 0 } \\
      { 0 } & { 1 } & { 0 } & { 0 } & { 0 } & { 0 } \\
      { 1 } & { 0 } & { 0 } & { 0 } & { 0 } & { 0 } \\
      { 0 } & { - 1 } & { 0 } & { 2 } & { 0 } & { 0 } \\
      { 0 } & { 0 } & { - 1 } & { 0 } & { 1 } & { 0 }
    \end{array}
  \right).
\label{eq:1.4.18}
\end{equation}
现将空间\(H_2^2\) 与\(\mathbb{R}^6\) 等同:
$\varepsilon _ { i } = e _ { i }$,
其中$\varepsilon_1,dots,varepsilon_6$为\(\mathbb{R}^6\) 中的标准正交基,由Fredholm定理可知
\[
\mathbb { R } ^ { 6 } = \mathscr { R } \left( \boldsymbol { L } _ { A } ^ { 2 } \right) + \mathscr { N } \left( L _ { A } ^ { 2 } \right),
\]
其中$\mathscr { N } \left( L _ { A } ^ { 2 } \right)$表示$L _ { A } ^ { 2 }$零空间。
由\ref{equ4.18}容易看出,
\[
\mathscr { N } \left( L _ { A } ^ { 2 } \right) = \operatorname { Span } \left( e _ { 1 } , e _ { 2 } + 2 e _ { 4 } \right\}
\]
由此得到
\[
H _ { 2 } ^ { 2 } = \mathscr { R } ( ad_A^2 ) \oplus \mathscr {  S} ^ { 2 }
\]

其中$\mathscr { S } ^ { 2 } = \operatorname { Span } \left\{ \boldsymbol { \varepsilon } _ { 1 } , \boldsymbol { \varepsilon } _ { 2 } + 2 \boldsymbol { \varepsilon } _ { \mathbf { 4 } } \right\}$,
由此得出二次正规形为
\[
  \frac { d } { d z } \left(
    \begin{array} { l }
      { x _ { 1 } } \\
      { x _ { 2 } }
    \end{array}
  \right)
  = \left(
    \begin{array} { l l }
      { 0 } & { 1 } \\
      { 0 } & { 0 }
    \end{array}
  \right)
  \left(
    \begin{array} { l }
      { x _ { 1 } } \\
      { x _ { 2 } }
    \end{array}
  \right)
  + a \left(
    \begin{array} { l }
      { 0 } \\
      { x _ { 1 } ^ { 3 } }
    \end{array}
  \right)
  + b \left(
    \begin{array} { c }
      { 2 x _ { 1 } ^ { 2 } } \\
      { x _ { 1 } x _ { 2} }
    \end{array}
  \right)
\]

或化成等价形式
\[
  \left\{
    \begin{array} { l }
      { \frac { d x _ { 1 } } { d t } = x _ { 2 } + 2 b x _ { 1 } ^ { 2 } } \\
      { \frac { d x _ { 2 } } { d t } = a x _ { 1 } ^ { 2 } + b x _ { 1 } x _ { 2 } }
    \end{array}
  \right.
\]
\end{solve}

注意,$\operatorname { Span } \left\{ \varepsilon _ { 1 } , \varepsilon _ { 2 } \right\}$同样构成$\mathscr { R } \left( \mathrm { ad } _ { A } ^ { 2 } \right)$的一个补空间。
事实上,取
$v _ { 1 } = \varepsilon _ { 1 } , v _ { 2 } = \varepsilon _ { 2 } + 2 \varepsilon _ { 4 } , w _ { 1 } = \varepsilon _ { 1 } , w _ { 2 } = \varepsilon _ { 2 }$,
则内积
\[
  \left\langle
    \boldsymbol { v } _ { i } , \boldsymbol { w } _ { j }
  \right\rangle
  = \delta _ { i j } =
  \left\{
    \begin{array} { l }
      { 1 },当i=j \\
      { 0 },当i\neq j.
    \end{array}
  \right.
\]
因此,可取$\operatorname { Span } \left\{ \varepsilon _ { 1 } , \varepsilon _ { 3 } \right\}$为
$\mathscr { R } \left( \operatorname { ad } _ { A } ^ { 2 } \right)$的另一个补空间,
而相应的二次正规形为
\begin{equation}
  \left\{
    \begin{array} { l }
      { \frac { d x _ { 1 } } { d t } = x _ { 2 } } \\
      { \frac { d x _ { 2 } } { d t } = a x _ { 1 } ^ { 2 } + b x _ { 1 } x _ { 2 } }
    \end{array}
  \right.
  \label{eq:1.4.19}
\end{equation}
用类似的方法,可以算出k次$( k \geqslant 2 )$正规形为

\begin{equation}
  \label{eq:1.4.20}
  \left\{
    \begin{array}{ l }
      { \frac { \mathrm { d } x _ { 1 } } { \mathrm { d } t } = x _ {2 } } \\
      { \frac { \mathrm { d } x _ { 2 } } { \mathrm { d } t } = a x _ { 1 } ^ { 2 } \left( 1 + P \left( x _ { 1 } \right) \right) + b x _ { 1 } x _ { 2 } \left( 1 + Q \left( x _ { 1 } \right) \right) }
    \end{array}
  \right.
\end{equation}

其中\(P(x_1)\)与\(Q(x_1)\)是\(x_1\)的\(k-2\)次多项式,
并且$P ( 0 ) = Q ( 0 )=0$,当$a b \neq 0$,可经过尺度变换把\ref{equ4.20}化为
\begin{equation}
  \left\{
    \begin{array} { l }
      { \frac { \mathrm { d } x _ { 1 } } { \mathrm { d } t } = x _ { 2 } } \\
      { \frac { \mathrm { d } x _ { 2 } } { \mathrm { d } t } = x _ { 1 } ^ { 2 } \left( 1 + P \left( x _ { 1 } \right) \right) + \eta x _ { 1 } x _ { 2 } \left( 1 + \mathrm { Q } \left( x _ { 1 } \right) \right) }
    \end{array}
  \right.
  \label{eq:1.4.21}
\end{equation}

其中$\eta = \operatorname { sign } ( a b )$(+1或-1).

\begin{corollary}
  由于$\mathscr { C R } \left( \mathrm { ad } _ { \mathrm { A } } ^ { k } \right)$的补空间不唯一,因此正规形也不是唯一的,从上面的例子中已经看出了这一点。
  但当取定一个补空间后,正规形中的系数就唯一确定了。
  在例\ref{equ4.7}中,求正规形的方法称为\textbf{矩阵表示法}。
  由于
  $
  \operatorname { dim } H _ { k } ^ { k }
  = n  \left(
    \begin{array}
      { c } { x + k - 1 } \\
      { n - 1 }
    \end{array}
  \right)
  $,矩阵表示法的计算量随着n或k的增大而迅速增大。
  近年来又发现了计算正规形的共轭算子法和群表示论法等,见王铎的综述文章\ref{Wd}及其所引的文献。
\end{corollary}


\begin{example}
  考虑复方程
  \[
    \frac { d } { d t }
    \left(
      \begin{array} { l }
        { z } \\
        { \overline{z} }
      \end{array}
    \right)
    =    A \left(
      \begin{array} { l }
        { z } \\
        { \overline{z} }
      \end{array}
    \right)
  +\mathbf{O}\left(|z|^2\right),
\]
其中,
\[
  A = \left(
    \begin{array} { c c }
      { i \omega } & { 0 } \\
      { 0 } & { - i \omega }
    \end{array}
  \right),
  \omega > 0.
\]
既A有一对共轭纯虚特征根,求它的(形式)正规形。
\end{example}

\begin{solve}
  我们用共振原理求解。
  记$\lambda _ { 1 } = i \omega , \lambda _ { 2 } = - i \omega _ { n }$,
  则共振条件是
  \[
\lambda _ { 1 } = ( k + 1 ) \lambda _ { 1 } + k \lambda _ { 2 } , \quad k = 1,2 , \dots
\]

\begin{center}
(或$\lambda _ { 2 } = k \lambda _ { 1 } + ( k + 1 ) \lambda _ { 2 } , \quad k = 1,2 , \cdots$)
\end{center}
由定理\ref{thm4.6}可知,复正规形为
\begin{equation}
\frac { d z } { d t } = i \omega z + c _ { 1 } | z | ^ { 2 } z + \cdots + c _ { k } | z | ^ { 2k } z + \cdots
  \label{eq:1.4.22}
\end{equation}
(第二个方程与之共轭,故略去不写)。
\end{solve}

\subsection{映射在不动点附近的正规形}
考虑以\(x=0\)为不动点的\(C^r\)映射$F ( r \geq 3 )$,它在\(x=-0\)附近可表示为
\begin{equation}
F ( x ) = A x + f ^ { 2 } ( x ) + \dots + f ^ { r - 1 } ( x ) + O \left( | x | ^ { r } \right),
  \label{eq:1.4.23}
\end{equation}

其中\(x\in \mathbb{R}^n\)(或\(\mathbb{C}^n\)),A是线性映射(我们把它在某组基下的矩阵仍记为A),$f ^ { k } ( x ) \in H _ { n  } ^ { k } , H _ {  n  } ^ { k }$为n元n维齐次向量多项式所成的空间,
$k = 2 , \dots , r - 1$。

考虑变换
\begin{equation}
x = H ( y ) \stackrel { \mathrm { d } } { \longrightarrow } y + h ^ { k } ( y ),
\label{eq:1.4.24}
\end{equation}

其中$h ^ { k } ( y ) \in H _ { n } ^ { k } , I + h ^ { k } ( \cdot )$在原点附近可逆,
则\ref{equ4.24}有如下的逆变换
\[
y = x - h ^ { k } ( x ) + O \left( | x | ^ { k + 1 } \right) , \quad | x | \ll 1.
\]
令
\[
G ( y ) = H ^ { - 1 } \circ F \circ H ( y ),
\]
则可把\ref{equ4.23}化为  
\[
  G ( y ) = A y + f ^ { 2 } ( y ) + \cdots + f ^ { k - 1 } ( y ) +
  \left[ f ^ { k } ( y ) - \left( h ^ { k } ( A y ) - A h ^ { k } ( y ) \right) \right] + O \left( | y | ^ { k + 1 } \right),
\]

$| y | \ll 1$.因此,与定理\ref{thm4.1}平行,可得到


\begin{theorem}
  设$F \in D i f f ^ { r } \left(\mathbb{ R} ^ { n } \right)$
  (或$Diff^ { r } \left( \mathbb { C } ^ { n } \right)$),
  并且有表达式\ref{equ4.23},则在原点附近的领域内存在一系列变换
  \begin{equation}
x = y + h ^ { k } ( y ) , \quad k = 2 , \cdots , r - 1,
\label{eq:1.4.25}
  \end{equation}

  其中$h ^ { k } ( y ) \in H _ { n } ^ { k }$,经过这一系列变换(每次变换后把y换回x),可把\ref{equ4.23}变成如下形式
  \begin{equation}
G ( x ) = A x + g ^ { 2 } ( x ) + \dots + g ^ { r - 1 } ( x ) + O \left( | x | ^ { r } \right),
    \label{eq:1.4.26}
\end{equation}

其中$g ^ { k } ( x ) \in \mathscr { S } ^ { k } , \mathscr{S}^k$是
$\mathscr { R } ^ { k } = \mathscr { R } \left( L _ { A } ^ { k } \right)$在\(H_n^k\)中的补空间,算子\(L_A^k\)由下式定义
\[
  L _ { A } ^ { k } : H _ { n } ^ { k } \rightarrow H _ { n } ^ { k },
  \quad
  L _ { A } ^ { k } \left( h ^ { k } ( x ) \right) = h ^ { k } ( A x ) - A h ^ { k } ( x ),
\]
$k = 2 , \cdots , r - 1$.
\end{theorem}

\begin{defination}

  映射\ref{equ4.26}的j次截取式$( 2 \leqslant j \leqslant r - 1 )$
  \[
A x + g ^ { 2 } ( x ) + \cdots + g ^ { j } ( x )
\]
其中$g ^ { i } ( x ) \in \mathscr { S } ^ { i } , i = 2 , \dots , j$,
称为映射\ref{equ4.23}的\textbf{j次正规形}.
\end{defination}

\begin{defination}
  Jordan型矩阵A的特征值
  $\lambda= \left( \lambda _ { 1 } , \cdots , \lambda _ { n } \right) \subset \mathbb{C} ^ { n }$称为\textbf{共振}的,
  如果存在$s ( 1 \leqslant s \leqslant n )$
  和整数组$m = \left( m _ { 1 } , \cdots , m _ { n } \right) , m _ { i }\gg 0$,$| m | = \sum _ { i = 1 } ^ { n } m _ { i } \geqslant 2$,使得
  \begin{equation}
    \lambda _ { s } = \lambda ^ { m } = \lambda _ { 1 } ^ { m _ { 1 } } \dots \lambda _ { n } ^ { m _n}.
        \label{eq:1.4.27}
  \end{equation}
  正数\(|m|\)称为\textbf{共振的阶}.
\end{defination}

\begin{defination}
  向量值多项式\(x^me_s\)称为\textbf{共振多项式},如果m和s满足共振条件\ref{equ4.27},
  其中$\left\{ e _ { 1 } , \dots , e _ { n } \right\}$是\(\mathbb{C}^n\)的一组基,
  \((x_1,\dots,x_n)\)为相对于这组基的坐标,
  并且矩阵A在此基下的Jordan标准形以$\left\{ \lambda _ { 1 } , \dots \infty , \lambda _ { n } \right\}$为对角元素。
  \label{def:1.4.13}
\end{defination}

类似于定理\ref{thm4.6},可以得到下面的

\begin{theorem}
  设$\boldsymbol { A } = \operatorname { Diag } \left\{ \lambda _ { 1 } , \cdots , \lambda _ { n } \right\}$,则可以选取适当的变换\ref{equ4.25},使得\ref{equ4.26}右端的诸\(g^i(x)\)仅由共振多项式组成.  
\end{theorem}

\begin{example}
  求以\(x=0\)为不动点的一维映射
  \(F(x)=-x+\dots\)
  的六次正规形。
\end{example}

\begin{solve}
  $\lambda = - 1$是唯一的特征值,所以共振条件为
  \[
\lambda ^ { k } - \lambda = 0 , \quad k \geqslant 2,
  \]
  也就是
  \[
( - 1 ) ^ { k - 1 } = 1 , \quad k \geqslant 2.
\]
所以共振多项式为$x ^ { 3 } , x ^ { 5 } \cdot x ^ { 7 } , \dots$
由定理\ref{4.14},六次正规形为
\[
G ( x ) = - x + a x ^ { 3 } + b x ^ { 5 }
\]
其中a,b为常数。
\end{solve}

\subsection{光滑线性化}
\begin{defination}
  设\(k \geqslant 2\)是一个自然数,称光滑向量场(或微分同胚)的双曲奇点(或双曲不动点)为\textbf{k阶非共振}的,如果它的特征根不满足所有\(\geqslant k\)阶的共振关系。
  如果一个奇点(或不动点)是任意有限阶非共振的,则陈它为\textbf{无穷阶非共振}的,或简称\textbf{非共振}。
  \label{def:1.4.16}
\end{defination}

从前面的讨论可以看出,一个k阶非共振奇点(或不动点)的k次正规形是线性的。
换句话说,在奇点(或不动点)的领域里可以找到一个多项式的坐标变换,使得在新坐标系下系统可以表示为一个线性部分加上一个k阶小量。
一个自然的问题是,进一步可以通过什么样的坐标变换能把这个k阶小量去掉。

\begin{defination}
  设k是一个自然数或\(k=\infty\).
  称\(\mathbb{R}^n\)的光滑向量场(或微分同胚)在它的奇点(或不动点)O处可以\(C^k\)\textbf{线性化},如果存在点O的领域U和\(C^k\)微分同胚$H : U \rightarrow \mathbb { R } ^ { n } , H ( O ) = O$,使得经过坐标变换$x \mapsto H ( x )$后,系统在O点领域内变为线性的。
\end{defination}

\begin{theorem}[IY]
  设k是一个自然数或\(k=\infty\),A是一个n阶方程,则存在一个依赖于k和A的数\(k^{\prime}\),满足
  \[
    k ^ { \prime } =
    \left\{
      \begin{array} { l }
        { n \in \mathbb { Z } } ,当k < \infty\\
        { \infty },当k= \infty,
      \end{array}
    \right.
  \]
  使得如果原点是\(C^\infty\)微分方程
  
\begin{equation}
\frac { d x } { d t } = A x + \dots \quad x \in \mathbb{R}^{n}
\label{eq:1.4.28}
\end{equation}

  或微分同胚
  \begin{equation}
x \mapsto A x + \cdots \quad x \in \mathbb { R } ^ { n }
    \label{eq:1.4.29}
  \end{equation}
  
  的\(k^\prime\)阶非共振双曲奇点(或非共振双曲不动点),
  则系统\ref{equ4.28}(或系统\ref{4.29})可以\(C^k\)线性化。  
\end{theorem}

\begin{note}
  由于$k ^ { \prime } = k ^ { \prime } ( k , A )$的表达式比较复杂,此处没有给出。
  当\(k\to \infty\)时,\(k^\prime\)线性地依赖于k的增长。
\end{note}

\begin{example}
  考虑\(\mathbb{R}\)上的\(C^\infty\)光滑微分方程
  \[
  \frac { d x } { d t } = a x + \dots \therefore \quad a \neq 0;
\]
或微分同胚
\[
x \mapsto \mu x + \cdots \quad | \mu | \neq 0,1.
\]
因为特征根\(\lambda=a\)(或\(\lambda=\mu\))不满足任意阶的共振关系,
故由定理\ref{thm4.18},系统在原点处可以\(C^\infty\)线性化。
既\(\mathbb{R}^1\)上的\(C^\infty\)向量场(或微分同胚)在它们的双曲奇点(或双曲不动点)处可以\(C^\infty\)线性化。
\end{example}

\begin{example}
  由于\(\mathbb{R}^2\)上的向量场在双曲集点的特征根为$\lambda \pm \mathrm { i } \omega$,\(\lambda \neq 0,\omega \neq 0\).
  它不满足任意阶的共振关系,故\(\mathbb{R}^2\)上的\(C^\infty\)向量场在它的双曲集点处可以\(C^\infty\)线性化。  
\end{example}
在考虑分岔问题时,我们常常只需要\(C^1\)线性化。
对此,有下面较为简单的结果。

\begin{theorem}
  设点O是\(\mathbb{R}^n\)中一个\(C^\infty\)光滑向量场(或微分同胚)X的双曲奇点(或双曲不动点)。
  如果X在点O的线性部分算子的特征根
  $\lambda_{ 1 } , \lambda_{ 2 } , \dots , \lambda_{ n }$满足下列不等式
  \begin{equation}
    \mathbf{ R } e \lambda_{ i } \neq \mathbf{ R } e \lambda _ { j } + \mathbf{ R } e \lambda _ { k }
  \end{equation}或

  \begin{equation}
    \left| \lambda _ { i } \right| \neq \left| \lambda _ { j } \right| \cdot \left| \lambda_ { k } \right|
    \label{eq:1.4.30}
  \end{equation}
  
  $\forall i , j , k \in \{ 1,2 , \cdots , n \}$,
  则系统在点O处可以\(C^1\)线性化。
\end{theorem}


上述定理的证明对向量场和映射的情况分别由\ref{be}和\ref{go}给出。

\begin{example}
平面上的\(C^\infty\)光滑向量场(或微分同胚)在它的双曲鞍点(或双曲鞍不动点)处可以\(C^1\)线性化。
\end{example}

\begin{example}
  如果\(\mathbb{R}^3\)中的向量场的一个双曲奇点有一对复特征根$\lambda \pm \mathrm { i } \omega$和一个实根\(\mu\),满足$\omega \neq 0 , \lambda \mu < 0$,则称奇点为\textbf{鞍焦点}.
  鞍焦点的特征根显然满足\ref{equ4.30}.
  故\(\mathbb{R}^3\)中\(C^\infty\)向量场在它的鞍焦点处可以\(C^1\)线性化。
\end{example}
\begin{example}
如果\(\mathbb{R}^3\)中\(C^\infty\)向量场在它的奇点O处的特征根满足
$ { \lambda } _ { 2 } < \lambda _ { 1 } < 0 < \mu$,并且
\[
\lambda _ { 2 } + \mu \neq \lambda _ { 1 } , \quad \lambda _ { 2 } \neq 2 \lambda _ { 1 },
\]
则它在该点可以\(C^1\)线性化。
\end{example}

在第五章讨论非局部分岔时,遇到的向量场都是依赖与参数的。
因此,下面我们讨论带参数的向量场或者映射的线性化问题。

\begin{defination}
设k是一个自然数或\(k=\infty\),$X _ { \varepsilon } \left( \varepsilon \in \mathbb { R } ^ { m } \right)$
是\(\mathbb{R}^n\)上的一个依赖于参数的向量场(或微分同胚)族。
设
\(\varepsilon=\varepsilon_0\)时,
\(X_{\varepsilon_{0}}\)
有一个双曲奇点(或双曲不动点)O。
称族\(X_\varepsilon\)\textbf{在点(O,\(\varepsilon_0\)附近可以\(C^k\)线性化)},如果存在参数空间\(\varepsilon=\varepsilon_0\)的领域V,相空间\(\mathbb{R}^n\)中点O的领域U,以及一个\(C^k\)映射
$H , U \times V \rightarrow \mathbb { R } ^ { n }$,满足

(1)$H \left( O , \varepsilon _ { 0 } \right) = O _ { 1 }$;

(2)对每一个参数$\varepsilon \in V , H ( \bullet , \varepsilon ) : U \rightarrow \mathbb { R } ^ { n }$是一个微分同胚,
使得通过依赖于参数\(\varepsilon\)的坐标变换$x \mapsto H ( x , \varepsilon )$后,系统在O点的领域变成一个线性系统。
\end{defination}

\begin{theorem}[IY]
  设\(X_\varepsilon\)是\(\mathbb{R}^n\)中的\(C^\infty\)向量场(或微分同胚)族,
  且$\varepsilon = \varepsilon _ { 0 }$时点O是系统$X _ { \varepsilon _ { 0 } }$的非共振双曲奇点(或双曲不动点),则对任意自然数k,
  \(X_\varepsilon\)可以在点$\left( O , \varepsilon _ { 0 } \right)$附近  \(C^k\)线性化。
\end{theorem}

\begin{note}
  由上述定理的结论并不能推出\(X_\varepsilon\)可以在点$\left( O , \epsilon _ { 0 } \right)$附近\(C^\infty\)线性化。
  因为随着k的增加,可以实现线性化的点O的领域U和\(\varepsilon\)的领域V可能不断减小。
\end{note}

本小节给出的定理是我们在第四章和第五章中讨论问题的基础。
